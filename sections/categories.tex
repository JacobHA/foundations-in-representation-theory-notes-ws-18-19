\section{Categories}

\begin{definition}
  A \emph{category}~$\Ccat$\index{category} consists of the following data:
  \begin{itemize}
    \item
      A class~$\Ob(\Ccat)$.
      The elements are the \emph{objects}\index{object} of~$\Ccat$.
    \item
      For any two objects~$X, Y \in \Ob(\Ccat)$ a set~$\Ccat(X,Y)$.% dont let footnote see this line break
      \footnote{Other common notations are~$\Hom_{\Ccat}(X,Y)$ or~$\operatorname{Mor}_{\Ccat}(X,Y)$.}
      The elements of~$\Ccat(X, Y)$ are the \emph{morphisms}\index{morphism!in a category} from~$X$ to~$Y$.
      That~$f \in \Ccat(X,Y)$ is denoted by~$f \colon X \to Y$ or~$X \xto{f} Y$.
    \item
      For any three objects~$X, Y, Z \in \Ob(\Ccat)$ a map
      \[
                \Ccat(Y,Z) \times \Ccat(X,Y)
        \to     \Ccat(X,Z) \,,
        \quad   (g, f)
        \mapsto g \circ f \,.
      \]
      For any two morphisms~$f \colon X \to Y$ and~$g \colon Y \to Z$ the morphism~$g \circ f \colon X \to Z$ is the \emph{composition} of~$g$ and~$f$.
  \end{itemize}
  These data are subject to the following conditions:
  \begin{enumerate}[label=(C\arabic*)]
    \item
      The composition of morphisms is associative\index{associativity}:
      For all objects~$X, Y, Z, W \in \Ob(\Ccat)$ and morphisms~$f \colon X \to Y$,~$g \colon Y \to Z$ and~$h \colon Z \to W$ it holds that
      \[
          (h \circ g) \circ f
        = h \circ (g \circ f) \,.
      \]
    \item
      There exists for every object~$X \in \Ob(\Ccat)$ an \emph{identity morphism}\index{identity!morphism}~$\id_X \colon X \to X$ such that
      \[
        f \circ {\id_X} = f
        \quad\text{and}\quad
        {\id_X} \circ g = g
      \]
      for all morphisms~$f \colon X \to Y$ and~$g \colon Y \to X$ in~$\Ccat$.
  \end{enumerate}
\end{definition}


\begin{remark}
  Let~$\Ccat$ be a category.
  \begin{enumerate}
    \item
      It could happen for objects~$X, Y \in \Ccat$ that~$\Ccat(X,Y) = \emptyset$, i.e.\ that there exists no morphism from~$X$ to~$Y$ in~$\Ccat$.
    \item
      For every object~$X \in \Ccat$ the identity morphism~$\id_X$ is unique.
      If~$\id'_X$ is another identity morphism of~$X$ then
      \[
          \id_X
        = \id_X \id'_X
        = \id'_X \,.
      \]
  \end{enumerate}
\end{remark}


\begin{remark}
  We sometimes want to consider categories whose objects are all sets (which satisfy certain conditions).
  This can lead to set theoretic problems (also known as \emph{set theoretic difficulties}\index{set theoretic difficulties}).
  One way out of this predicament are \emph{universes}\index{universe}.
  (See \cite[I.6]{Working} and \cite[3.2]{Schubert} for more details on this.)
  We will always fix a universe~$U$ and say that
  \begin{itemize}
    \item
      $X$ is a set if~$X \in U$, and that
    \item
      $X$ is a class if~$X \subseteq U$.
  \end{itemize}
\end{remark}





\lecturend{4}




\begin{example}
  \leavevmode
  \begin{enumerate}
    \item
      The category~$\Set$\index{category!of sets} of sets:
      The objects of~$\Set$ are given by
      \[
          \Ob(\Set)
        = \{
            \text{sets (which are elements of the fixed universe)}
          \} \,,
      \]
      and for any two sets~$X$ and~$Y$ the morphism set~$\Set(X,Y)$ is given by
      \[
          \Set(X,Y)
        = \{
            \text{maps~$f \colon X \to Y$}
          \} \,.
      \]
      The composition of morphisms in~$\Set$ is the usual composition of maps.
    \item
      The category~$\Group$\index{category!of groups} of groups:
      The objects of~$\Group$ are given by
      \[
          \Ob(\Group)
        = \{
            \text{groups}
          \} \,,
      \]
      and for any two groups~$G$ and~$H$ the morphism set~$\Group(G,H)$ is given by
      \[
          \Group(G,H)
        = \{
            \text{group homomorphisms~$f \colon G \to H$}
          \} \,.
      \]
      The composition of morphisms in~$\Group$ is the usual composition of group homomorphisms.
    \item
      The category~$\kAlg$\index{category!of $k$-algebras} of~{\kalgs}:%
      \footnote{This example was not given in the lecture, but will be used \cref{examples for functors}.}
      The objects of~$\kAlg$ are given by
      \[
          \Ob(\kAlg)
        = \{
            \text{{\kalgs}}
          \} \,,
      \]
      and for any two {\kalgs}~$A$ and~$B$ the morphism set~$(\kAlg)(A,B)$ is given by
      \[
          \kAlg(A,B)
        = \{
            \text{{\kalg} homomorphisms~$f \colon A \to B$}
          \} \,.
      \]
      The composition of morphisms in~$\kAlg$ is the usual composition of~{\kalg} homomorphisms.
    \item
      For a~{\kalg}~$A$ the category~$\Modl{A}$\index{category!of $A$-modules} of left~{\modules{$A$}}:
      The objects of~$\Modl{A}$ are given by
      \[
          \Ob(\Modl{A})
        = \{
            \text{left~{\modules{$A$}}}
          \} \,,
      \]
      and for any two left~{\modules{$A$}}~$M$ and~$N$ the morphism set~$(\Modl{A})(M,N)$ is given by
      \[
          (\Modl{A})(M,N)
        = \left\{
            \text{homomorphisms of left~{\modules{$A$}}~$f \colon M \to N$}
          \right\} \,.
      \]
      The composition of morphisms in~$\Modl{A}$ is the usual composition of module homomorphisms.
      
      The category~$\Modr{A}$ of right~{\modules{$A$}} is defined analogous.
    \item
      The category~$\Top$\index{category!of topological spaces} of topological spaces:
      The objects of~$\Top$ are given by
      \[
          \Ob(\Top)
        = \{
            \text{topological spaces}
          \} \,,
      \]
      and for any two topological spaces~$X$ and~$Y$ the morphism set~$\Top(X,Y)$ is given by
      \[
          \Top(X,Y)
        = \{
            \text{continuous maps~$f \colon X \to Y$}
          \} \,.
      \]
      The composition of morphisms in~$\Top$ is the usual composition of continuous maps.
    \item
      Let~$G$ be a group.
      We can then define a category~$\Gcat$ which consists of a single object~$\Ob(\Gcat) = \{ \ast \}$, the single morphism set~$\Gcat(\ast,\ast) = G$, and for which the composition of morphisms is the multiplication of~$G$, i.e.\ the composition is given as
      \[
                  g \circ h
        \defined  gh
      \]
      for all~$g, h \in \Gcat(\ast,\ast) = G$.
      One may pictorially represent the category~$\Gcat$ as follows:
      \[
        \begin{tikzcd}
          \ast
          \arrow[out=30,in=110,loop]{}[above right, near start]{G}
          \arrow[out=150,in=230,loop,looseness=6.5]
          \arrow[out=270,in=350,loop]
        \end{tikzcd}
      \]
    \item
      Let~$Q$ be a quiver.
      Its \emph{category of paths}\index{category!of paths}~$\Path(Q)$, is defined as follows:%
      \footnote{In the lecture, the notation~$Q_*$ is used for the category of paths in~$Q$.}
      The objects of~$\Path(Q)$ are the vertices of~$Q$, hence
      \[
          \Ob(\Path(Q))
        = Q_0 \,,
      \]
      and for any two vertices~$i$ and~$j$ in~$Q$ the morphism set~$\Path(Q)(i,j)$ is given by
      \[
          \Path(Q)(i,j)
        = \{
            \text{paths from~$i$ to~$j$ in~$Q$}
          \} \,.
      \]
      The composition of morphisms in~$\Path(Q)$ is given by concatenation of paths.
      This gives a category because concatenation of paths is associative, and for every~$i \in Q_0$ the lazy path~$\varepsilon_i \in \Path(Q)(i,i)$ acts as the identity of the object~$i$.
  \end{enumerate}
\end{example}


\begin{definition}
  Let~$\Ccat$ be a category.
  The \emph{opposite category}\index{opposite!category}~$\Ccat^\op$ results from~$\Ccat$ by formally reversing the direction of all morphisms:
  The objects of~$\Ccat^\op$ are given by
  \[
      \Ob(\Ccat^\op)
    = \Ob(\Ccat) \,,
  \]
  for any two objects~$X, Y \in \Ob(\Ccat^\op)$ the morphisms set~$\Ccat^\op(X,Y)$ is given by
  \[
      \Ccat^\op(X,Y)
    = \Ccat(Y,X) \,.
  \]
  The composition of any two morphisms~$f \colon X \to Y$ and~$g \colon Y \to Z$ in~$\Ccat^\op$ is given by
  \[
      g \circ_{\Ccat^\op} f
    = f \circ_{\Ccat} g \,.
  \]
\end{definition}


\begin{definition*}
  Let~$\Ccat$ be a category.
  \begin{enumerate}
    \item 
      A \emph{subcategory}\index{subcategory}~$\Scat$ of~$\Ccat$ consists of
      \begin{itemize}
        \item
          a subclass~$\Ob(\Scat) \subseteq \Ob(\Ccat)$, and
        \item
          for any two objects~$X, Y \in \Ob(\Scat)$ a subset~$\Scat(X,Y) \subseteq \Ccat(X,Y)$,
      \end{itemize}
      that are subject to the following conditions:
      \begin{enumerate}[label=(S\arabic*)]
        \item
          It holds that~$\id_X \in \Scat(X,X)$ for every object~$X \in \Ob(\Scat)$.
        \item
          It holds that~$g \circ f \in \Scat(X,Z)$ for all objects~$X, Y, Z \in \Ob(\Scat)$ and all morphisms~$f \in \Scat(X,Y)$ and~$g \in \Scat(Y,Z)$.
      \end{enumerate}
    \item
      A subcategory~$\Scat$ of~$\Ccat$ is \emph{full} if~$\Scat(X,Y) = \Ccat(X,Y)$ for all objects~$X, Y \in \Ob(\Scat)$.
  \end{enumerate}
  A subcategory~$\Scat$ of~$\Dcat$.
\end{definition*}


\begin{remark*}
  Let~$\Ccat$ be a category.
  \begin{enumerate}
    \item
      If~$\Scat$ is a subcategory of~$\Ccat$ then~$\Scat$ inherits from~$\Ccat$ the structure of a category.
    \item
      Any full subcategory~$\Scat$ of~$\Ccat$ is uniquely determined by its class of objects~$\Ob(\Scat)$.
      Converseley, every subclass of~$\Ob(\Ccat)$ defines a full subcategory.
  \end{enumerate}
\end{remark*}





