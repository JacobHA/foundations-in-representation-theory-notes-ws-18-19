\section{Kernels and Cokernels}


\begin{definition}
  Let~$\Ccat$ be a category that has a zero object, or that is preaddive.
  Let~$f \colon X \to Y$ be a morphism in~$\Ccat$.
  \begin{enumerate}
      \item
        A \emph{kernel}\index{kernel} of~$f$ is a pair~$(K,k)$ consisting of an object~$K \in \Ob(\Ccat)$ and a morphism~$k \colon K \to X$ with~$f \circ k = 0$, such that for every morphism~$\ell \colon L \to X$ in~$\Ccat$ with~$f \circ \ell = 0$ there exists a unique morphism~$\lambda \colon L \to K$ which makes the following diagram commute:
        \[
          \begin{tikzcd}[row sep = small, column sep = large]
              K
              \arrow{dr}[above right]{k}
              \arrow[bend left]{drr}[above right]{0}
            & {}
            & {}
            \\
              {}
            & X
              \arrow{r}[above, near start]{f}
            & Y
            \\
              L
              \arrow{ur}[below right]{\ell}
              \arrow[bend right]{urr}[below right]{0}
              \arrow[dashed]{uu}[left]{\lambda}
            & {}
            & {}
          \end{tikzcd}
        \]
      \item
        A \emph{cokernel}\index{cokernel} of~$f$ is a pair~$(C,c)$ consisting of an object~$C \in \Ob(\Ccat)$ and a morphism~$c \colon Y \to C$ with~$c \circ f = 0$, such that for every morphism~$d \colon Y \to D$ in~$\Ccat$ with~$d \circ f = 0$ there exists a unique morphism~$\mu \colon C \to D$ which makes the following diagram commute:
        \[
          \begin{tikzcd}[row sep = small, column sep = large]
              {}
            & {}
            & C
              \arrow[dashed]{dd}[right]{\mu}
            \\
              X
              \arrow{r}[above]{f}
              \arrow[bend left]{urr}[above left]{0}
              \arrow[bend right]{drr}[below left]{0}
            & Y
              \arrow{ur}[above left]{c}
              \arrow{dr}[below left]{d}
            & {}
            \\
              {}
            & {}
            & D
          \end{tikzcd}
        \]
  \end{enumerate}
\end{definition}


\begin{remark}
  Let~$\Ccat$ be a category that has a zero object, or that is preaddive.
  Let~$f \colon X \to Y$ be a morphism in~$\Ccat$.
  \begin{enumerate}
    \item
      A pair~$(K,k)$ is a kernel of~$f$ in~$\Ccat$ if and only if it is a cokernel of~$f$ in~$\Ccat^\op$.
    \item
      Kernels and cokernels are unique up to unique isomorphism.
    \item
      If every morphism in~$\Ccat$ has a kernel (resp.\ a cokernel) then the category~$\Ccat$ \emph{has kernels}\index{category!has!kernels}\index{has!kernels} (resp.\ \emph{has cokernels}\index{category!has!cokernels}\index{has!cokernels}).
    \item
      The kernel of~$f$ is denoted by~$\ker(f) \to X$, and the cokernel of~$f$ is denoted by~$Y \to \coker(f)$.
  \end{enumerate}
\end{remark}





\lecturend{10}





\begin{notation*}
  \leavevmode
  \begin{enumerate}
    \item
      Let~$\Ccat$ be a category that has a zero object and let~$f \colon X \to Y$ be a morphism in~$\Ccat$.
      Then we write~$\ker(f) = 0$ to mean that the zero morphism~$0 \to X$ is a kernel of~$f$.
      We dually write~$\coker(f) = 0$ to mean that the zero morphism~$Y \to 0$ is a cokernel of~$f$.
    \item
      Let~$\Ccat$ be a category that is preadditive, or that has a zero object, and let~$f \colon X \to Y$ be a morphism in~$\Ccat$.
      If~$g \colon Y \to Z$ is another morphism in~$\Ccat$ then we write that~$\ker(f) = \ker(gf)$ to mean that a morphism~$k \colon K \to X$ in~$\Ccat$ is a kernel of~$f$ if and only if it is a kernel of~$gf$.
      (This entails in particular that a kernel for~$f$ exists if and only if a kernel for~$gf$ exist.)
      
      Dually, if~$h \colon W \to X$ is a morphism in~$\Ccat$, then we write that~$\coker(f) = \coker(fh)$ to mean that a morphism~$c \colon Y \to C$ in~$\Ccat$ is a cokernel of~$f$ if and only if it is a cokernel of~$fh$.
      (This entails in particular that a cokernel for~$f$ exists if and only if a cokernel for~$fh$ exists.)
  \end{enumerate}
\end{notation*}


\begin{remark*}
  Let~$\Ccat$ be a category that has a zero object and let~$f \colon X \to Y$ be a morphism in~$\Ccat$.
  Then~$\ker(f) = 0$ if and only if it follows for every morphism~$u \colon W \to X$ in~$\Ccat$ with~$f \circ u = 0$ that already~$u = 0$.
  
  Indeed, the composition~$0 \to X \xto{f} Y$ is the zero morphism.
  That~$\ker(f) = 0$ therefore means that every morphism~$u \colon W \to X$ with~$f \circ u = 0$ factors uniquely trough the zero morphism~$0 \to X$, i.e.\ that there exists a unique morphism~$W \to 0$ that makes the triangle
  \[
    \begin{tikzcd}[sep = large]
        W
        \arrow[dashed]{d}
        \arrow{dr}[above right]{u}
        \arrow[bend left]{drr}[above right]{0}
      & {}
      & {}
      \\
        0
        \arrow{r}
      & X
        \arrow{r}[above]{f}
      & Y
    \end{tikzcd}
  \]
  commute.
  That~$u$ factors trough the zero morphism~$0 \to X$ is equivalent to~$u = 0$, and this factorization is necessarily unique because there exist only one morphism~$W \to 0$.
  
  It holds dually that~$\coker(f) = 0$ if and only if it follows for every morphism~$v \colon Y \to Z$ in~$\Ccat$ with~$v \circ f = 0$ that already~$v = 0$.
\end{remark*}


\begin{lemma}
  Let~$\Ccat$ be a category that is preadditive, or that has a zero object.
  Let~$f \colon X \to Y$ be a morphism in~$\Ccat$.
  \begin{enumerate}
    \item
      If~$k \colon \ker(f) \to X$ is a kernel of~$f$ then the morphism~$k$ is a monomorphism.
      Dually, if~$c \colon Y \to \coker(f)$ is a cokernel of~$f$ then the morphism~$c$ is an epimorphism.
    \item
      Suppose that~$\Ccat$ is both preadditive and has a zero object (e.g.~$\Ccat$ is additive).
      If~$\ker(f) = 0$ then~$f$ is a monomorphism.
      Dually, if~$\coker(f) = 0$ then~$f$ is an epimorphism.
    \item
      Suppose that~$\Ccat$ has a zero object.
      If~$f$ is a monomorphism then~$\ker(f) = 0$, and if~$f$ is an epimorphism then~$\coker(f) = 0$.
    \item
      If~$u \colon Y \to Z$ is a monomorphism in~$\Ccat$ then~$\ker(f) = \ker(uf)$.
      Dually, if~$p \colon W \to X$ is an epimorphism in~$\Ccat$ then~$\coker(f) = \coker(fp)$.
  \end{enumerate}
\end{lemma}


\begin{proof}
  \leavevmode
  \begin{enumerate}
    \item
      Let~$u, v \colon W \to \ker(f)$ be two parallel morphisms with~$k \circ u = k \circ v$, and denote this composition by~$w \colon W \to X$.
      Then
      \[
          f \circ w
        = f \circ k \circ u
        = 0 \circ u
        = 0 \,.
      \]
      It follows from the universal property of the kernel~$k \colon \ker(f) \to X$ that there exist a unique morphism~$W \to \ker(f)$ which makes the triangle
      \[
        \begin{tikzcd}[sep = large]
            W
            \arrow[dashed]{d}
            \arrow{dr}[above right]{w}
          & {}
          \\
            \ker(f)
            \arrow{r}[above, near start]{k}
          & X
        \end{tikzcd}
      \]
      commute.
      Both~$u$ and~$v$ make this triangle commute, and so it follows that~$u = v$.
      
      That the cokernel~$c \colon Y \to \coker(f)$ is an epimorphism can be shown dually.
    \item
      Let~$u,v \colon W \to X$ be two morphisms with~$f \circ u = f \circ v$.
      Then
      \[
          f \circ (u-v)
        = f \circ u - f \circ v
        = 0 \,,
      \]
      and it follows from the universal property of the kernel~$\ker(f)$ that the difference~$u-v$ factors through~$\ker(f) = 0$, which results in the following commutative triangle:
      \[
        \begin{tikzcd}[sep = large]
            W
            \arrow[dashed]{d}
            \arrow{dr}[above right]{u-v}
          & {}
          \\
            0
            \arrow{r}
          & X
        \end{tikzcd}
      \]
      The morphisms~$W \to 0$ and~$0 \to X$ are necessarily the zero morphisms, so it follows that~$u-v = 0 \circ 0 = 0$, and hence~$u = v$.
      
      That~$f$ is an epimorphism if~$\coker(f) = 0$ can be shown dually.
    \item
      If~$u \colon W \to X$ is any morphism with~$f \circ u = 0$ then
      \[
          f \circ u
        = 0
        = f \circ 0
      \]
      and hence~$u = 0$.
      
      That~$\coker(f) = 0$ if~$f$ is an epimorphism can be shown dually.
    \item
      It holds for every morphism~$v \colon W \to X$ that
      \[
              f \circ v = 0
        \iff  uf \circ v = u \circ 0
        \iff  uf \circ v = 0 \,,
      \]
      where the first equivalence holds because~$u$ is a monomorphism.
      We thus find for every morphism~$k \colon K \to X$ in~$\Ccat$ that
      \begin{align*}
            {}& \text{$k$ is a kernel of~$f$} \\
        \iff{}& \text{every morphism~$v \colon W \to X$ with~$f \circ v = 0$ factors uniquely through~$k$}  \\
        \iff{}& \text{every morphism~$v \colon W \to X$ with~$uf \circ v = 0$ factors uniquely through~$k$} \\
        \iff{}& \text{$k$ is a kernel of~$uf$} \,.
      \end{align*}
      
      That~$\coker(f) = \coker(fp)$ can be shown dually.
    \qedhere
  \end{enumerate}
\end{proof}


\begin{notation*}
  Let~$\Ccat$ be a category that is preadditive, or that has a zero object.
  We say that a morphism~$k \colon K \to X$ in~$\Ccat$ \emph{is a kernel}\index{is a!kernel} if it is a kernel for some morphism~$f \colon X \to Y$.
  Dually, we say that a morphism~$c \colon Y \to C$ in~$\Ccat$ \emph{is a cokernel}\index{is a!cokernel} if it is a cokernel for some morphism~$f \colon X \to Y$.
\end{notation*}


\begin{example}
  \leavevmode
  \begin{enumerate}
    \item
      Let~$A$ be a~{\kalg} and consider the module category~$\Modl{A}$.
      Let~$f \colon M \to N$ be a homomorphism of~{\modules{$A$}}.
      Then the submodule~$\ker(f) = f^{-1}(0)$ of~$M$ together with the inclusion~$k \colon \ker(f) \to M$ is a kernel of~$f$.
      The quotient moduel~$\coker(f) = N/f(M)$ together with the canonical projection~$c \colon N \to \coker(f)$ is a cokernel of~$f$.
      
      Note that in the category~$\Modl{A}$, a morphism~$k$ is a monomorphism if and only if~$k$ is a kernel, and similarly that a morphism~$c$ is an epimorphism if and only if~$c$ is a cokernel.
      (Recall that the monomorphisms in~$\Modl{A}$ are preciely the injective module homomorphisms, and that the epimorphisms in~$\Modl{A}$ are precisely the surjective module homomorphisms.)
    \item
      Consider the category~$\Group$ and let~$f \colon G \to H$ be a group homorphism.
      
      Then the subgroup~$\ker(f) = f^{-1}(1)$ together with the inclusion~$k \colon \ker(f) \to G$ is a kernel of~$f$.
      Note that~$\ker(f)$ is always a normal subgroup of~$G$, while for every subgroup~$K' \subseteq G$ the inclusion~$k' \colon K' \to G$ is a monomorphism.
      So in~$\Group$ not every monomorphism is a kernel.
      
      A cokernel of~$f$ is given by the quotient group~$\coker(f) = H/\closure{f(G)}$ together with the canonical projection~$c \colon H \to \coker(f)$, where
      \[
                  \closure{f(G)}
        \defined  \gen{
                    h f h^{-1}
                  \suchthat
                    h \in H, g \in G
                  }
      \]
      is the normal subgroup of~$H$ generated by~$f(G)$, i.e.\ the normal closure of~$f(G)$ in~$H$.
      If~$p \colon H \to C$ is any epimorphism in~$\Group$, then~$p$ is surjective, and hence
      \[
              C
        \cong H/\ker(p)
        =     \coker(\ker(p) \to H)
      \]
      This shows that in~$\Group$ every epimorphism is a cokernel.
    \item
      Let~$\Set_*$ be the category of pointed sets\index{category!of pointed sets}:
      
      The objects of~$\Set_*$ are pairs~$(X,x_0)$ consisting of a set~$X$ and a base point~$x_0 \in X$.
      A morphism~$f \colon (X,x_0) \to (Y,y_0)$ in~$\Set_*$ is a map~$f \colon X \to Y$ with~$f(x_0) = y_0$.
      (Such maps are also known as \emph{pointed maps}\index{pointed map}.)%
      \footnote{One may think about pointed sets as somewhat similar to~\dash{$\Finite_1$}{vector spaces}.}
      The category~$\Set_*$ has the singleton~$(\{\ast\}, \ast)$ as a zero object.
      The monomorphisms in~$\Set_*$ are precisely the injective pointed maps, and the epimorphisms are the surjective pointed maps.
      
      Let~$f \colon (X,x_0) \to (Y,y_0)$ be a morphism in~$\Set_*$.
      A kernel for~$f$ is given by~$\ker(f) = (f^{-1}(y_0), x_0)$ together with the inclusion~$k \colon \ker(f) \to (X,x_0)$.
      A cokernel for~$f$ is given by~$\coker(f) = (Y/{\sim}, \class{y_0})$ together with the canonical projection~$c \colon (Y, y_0) \to \coker(f)$, where~$\sim$ is the equivalence relation on~$Y$ generated by~$y \sim y'$ for all~$y, y' \in f(X)$.
      More explicitely, we have for any two~$y, y' \in Y$ that
      \[
              y \sim y'
        \iff (\text{$y = y'$ or~$y, y' \in f(X)$}) \,.
      \]
      
      Every monomorphism in~$\Set_*$ is a kernel (namely that of its cokernel).
      But not every epimorphism in~$\Set_*$ is a cokernel, because it holds for every cokernel~$c \colon (Y,y_0) \to (C,c_0)$ that every element~$z \in C$ with~$z \neq c_0$ has precisely one preimage under~$c$.
      (This follows from the above explicit description of the cokernel.)
%     TODO: Prove the above claims.
    \item
      The ring~$A \defined \Integer[t_1, t_2, t_3, \dotsc]$ is not noetherian because the ideal~$I \defined \genideal{t_1, t_2, t_3, \dotsc}$ of~$A$ is not finitely generated.
      Consider the category~$\Modlfg{A}$ of finitely generated~{\modules{$A$}}.
      This category is additive.
      
      Every morphism~$f \colon M \to N$ in~$\Modlfg{A}$ has a cokernel in~$\Modlfg{A}$ because the cokernel of~$f$ in~$\Modl{A}$ is already contained in~$\Modlfg{A}$ (and the zero morphisms in~$\Modlfg{A}$ coincides with the one in~$\Modl{A}$).
      
      Let~$f \colon A \to A/I$ be the canonical projection;
      note that~$A/I$ is finitely generated and hence contained in~$\Modlfg{A}$ (even though the~{\modules{$A$}}~$I$ is not contained in~$\Modl{A}$).
      The morphism~$f$ has no kernel in~$\Modl{A}^\fg$:
      
      Morally speaking, the problem is that the kernel of~$f$ in~$\Modl{A}$, which is~$I$, is not contained in~$\Modlfg{A}$.
      But this is not yet a proper proof because a kernel of~$f$ in~$\Modlfg{A}$ does not necessarily have to also be a kernel of~$f$ in~$\Modl{A}$.
      
      So instead, assume that there exists a kernel~$k \colon K \to A$ of~$f$ in~$\Modlfg{A}$.
      Then~$f \circ k = 0$ and hence~$k(K) \subseteq I$.
      Then for every~$i \in I$ the morphism
      \[
                \ell_i
        \colon  A
        \to     A \,,
        \quad   a
        \mapsto t_i a
      \]
      satisfies~$\ell_i(A) \subseteq I$ and hence~$f \circ \ell_i = 0$.
      It follows that the morphism~$\ell_i$ factors uniquely trough the kernel~$k$, i.e.\ that there exists a unique morphism~$\lambda_i \colon A \to K$ which makes the triangle
      \[
        \begin{tikzcd}[sep = large]
            A
            \arrow[dashed]{d}[left]{\lambda_i}
            \arrow{dr}[above right]{\ell_i}
          & {}
          \\
            K
            \arrow{r}[below]{k}
          & A
        \end{tikzcd}
      \]
      commute.
      It follows for every~$i \in I$ that
      \[
            t_i
        =   \ell_i(1)
        =   (k \circ \lambda_i)(1)
        =   k(\lambda_i(1))
        \in k(K)
      \]
      and hence~$t_i \in k(K)$.
      This shows that also~$I \subseteq k(K)$, and hence~$I = k(K)$.
      But it follows from~$K$ being finitely generated that~$k(K)$ is also finitely generated, which contradicts~$I$ not being finitely generated.
  \end{enumerate}
\end{example}


% TODO: Add example: divisible abelian groups.


\begin{definition}
  Let~$\Ccat$ be a category that is preadditive, or that has a zero object, and which has kernels and cokernels.
  Let~$f \colon X \to Y$ be a morphism in~$\Ccat$.
  \begin{enumerate}
    \item
      An \emph{image}\index{image} of~$f$ is a kernel of a cokernel of~$f$, and is denotedy by~$\im(f) \to Y$.
    \item
      A \emph{coimage}\index{coimage} of~$f$ is a cokernel of a kernel of~$f$, and is denoted by~$X \to \coim(f)$.
  \end{enumerate}
\end{definition}


\begin{example*}
  Let~$A$ be a~{\kalg} and let~$f \colon M \to N$ be a morphism in~$\Modl{A}$ (or~$\Modr{A}$).
  Then an image of~$f$ is given by the submodule~$\im(f) = f(M)$ of~$N$ together with the inclusion~$\im(f) \to N$.
  A coimage of~$f$ is given by the quotient module~$\coim(f) = M/\ker(f)$ together with the canonical projection~$M \to \coim(f)$.
\end{example*}


\begin{remarknonum}
  Images and coimages are unique up to unique isomorphisms.
\end{remarknonum}


\begin{lemma}[Canonical factorization]
  \index{canonical factorization}
  \label{canonical factorization}
  Let~$\Ccat$ be a category that is preadditive, or that has a zero object, and that has kernels and cokernels.
  Let~$f \colon X \to Y$ be a morphism in~$\Ccat$.
  \begin{enumerate}
    \item
      \label{restriction to image}
      There exists a unique morphism~$f' \colon X \to \im(f)$ in~$\Ccat$ which makes the following triangle commute:
      \[
        \begin{tikzcd}
            X
            \arrow{r}[above]{f}
            \arrow[dashed]{dr}[below left]{f'}
          & Y
          \\
            {}
          & \im(f)
            \arrow{u}
        \end{tikzcd}
      \]
    \item
      There exists a unique morphism~$\bar{f} \colon \coim(f) \to Y$ in~$\Ccat$ which makes the following triangle commute:
      \[
        \begin{tikzcd}
            X
            \arrow{r}[above]{f}
            \arrow{d}
          & Y
          \\
            \coim(f)
            \arrow[dashed]{ur}[below right]{\bar{f}}
          & {}
        \end{tikzcd}
      \]
    \item
      There exists a unique morphism~$\tilde{f} \colon \coim(f) \to \im(f)$ in~$\Ccat$ which makes the following square commute:
      \begin{equation}
        \label{canonical morphism from coim to im}
        \begin{tikzcd}
            X
            \arrow{r}[above]{f}
            \arrow{d}
          & Y
          \\
            \coim(f)
            \arrow[dashed]{r}[above]{\tilde{f}}
          & \im(f)
            \arrow{u}
        \end{tikzcd}
      \end{equation}
    \item
      The morphism~$\tilde{f}$ is compatible with the morphisms~$f'$ and~$\bar{f}$ in the sense that the diagrams
      \begin{equation}
        \label{compatibility of induced morphisms}
        \begin{tikzcd}[column sep = large, row sep = huge]
            X
            \arrow{r}[above]{f}
            \arrow{dr}[above right]{f'}
            \arrow{d}
          & Y
          \\
            \coim(f)
            \arrow{r}[above]{\tilde{f}}
          & \im(f)
            \arrow{u}
        \end{tikzcd}
        \qquad\text{and}\qquad
        \begin{tikzcd}[column sep = large, row sep = huge]
            X
            \arrow{r}[above]{f}
            \arrow{d}
          & Y
          \\
            \coim(f)
            \arrow{r}[above]{\tilde{f}}
            \arrow{ur}[above left]{\bar{f}}
          & \im(f)
            \arrow{u}
        \end{tikzcd}
      \end{equation}
      commute.
  \end{enumerate}
\end{lemma}


\begin{proof}
  We denote the various kernels and cokernels by
  \[
    k \colon \ker(f) \to X  \,,
    \quad
    c \colon Y \to \coker(f)  \,,
    \quad
    i \colon \im(f) \to Y \,,
    \quad
    p \colon X \to \coim(f) \,.
  \]
  \begin{enumerate}
    \item
      The morphism~$i \colon \im(f) \to Y$ is a kernel of the cokernel~$c \colon Y \to \coker(f)$, hence it follows from~$c \circ f = 0$ that there exist a unique morphism~$f' \colon X \to \im(f)$ which makes the following diagram commute:
      \[
        \begin{tikzcd}[row sep = large]
            {}
          & \coker(f)
          \\
            X
            \arrow{ur}[above left]{0}
            \arrow{r}[above]{f}
            \arrow[dashed]{dr}[below left]{f'}
          & Y
            \arrow{u}[right]{c}
          \\
            {}
          & \im(f)
            \arrow{u}[right]{i}
        \end{tikzcd}
%         \begin{tikzcd}
%             X 
%             \arrow{r}[above]{f}
%             \arrow[bend left = 40]{rr}[above left]{0}
%             \arrow[dashed]{dr}[below left]{f'}
%           & Y
%             \arrow{r}[above]{c}
%           & \coker(f)
%           \\
%             {}
%           & \im(f)
%             \arrow{u}[right]{i}
%           & {}
%         \end{tikzcd}
      \]
    \item
      This can be shown dually to part~\ref*{restriction to image}.
    \item
      We construct~$\tilde{f}$ by using the already constructed morphism~$f' \colon X \to \im(f)$:
      The morphism~$p \colon X \to \coim(f)$ is a cokernel of the the kernel~$k \colon \ker(f) \to X$.
      It holds that
      \[
          i \circ f' \circ  k
        = f \circ k
        = 0
        = i \circ 0 \,,
      \]
      and hence~$f' \circ k = 0$ because~$i$ is a monomorphism.
      It follows from the universal property of the cokernel~$p \colon X \to \coim(f)$ that there exist a unique morphism~$\tilde{f} \colon \coim(f) \to \im(f)$ which makes the following diagram commute:
      \[
        \begin{tikzcd}[column sep = 5em, row sep = huge]
            \ker(f)
            \arrow{d}[left]{k}
            \arrow[bend left]{dr}[above right]{0}
          & {}
          \\
            X
            \arrow{r}[above, near start]{f}
            \arrow{dr}[above right]{f'}
            \arrow{d}[left]{p}
          & Y
          \\
            \coim(f)
            \arrow[dashed]{r}[below]{\tilde{f}}
          & \im(f)
            \arrow{u}[right]{i}
            \arrow[dashed, from=uul, bend left = 15, crossing over, near start, "0"]
        \end{tikzcd}
      \]
      This shows the existence of~$\tilde{f}$.
      Suppose that~$\tilde{\tilde{f}} \colon \coim(f) \to \im(f)$ is another morphism which makes the diagram~\eqref{canonical morphism from coim to im} commute.
      Then
      \[
          i \circ \tilde{\tilde{f}} \circ p
        = f
        = i \circ \tilde{f} \circ p \,.
      \]
      It follows from~$i$ being a monomorphism that
      \[
          \tilde{\tilde{f}} \circ p
        = \tilde{f} \circ p \,,
      \]
      and it then further follows from~$p$ being a epimorphism that
      \[
        \tilde{\tilde{f}} = \tilde{f} \,.
      \]
      This shows the desired uniqueness of the morphism~$\tilde{f}$.
    \item
      It follows from the above construction of~$\tilde{f}$ that of the two diagrams in~\eqref{compatibility of induced morphisms} the left one commutes.
      We could have dually constructed~$\tilde{f}$ by using the morphism~$\bar{f}$ instead of~$f'$, which would then give us that the right diagram commutes.
      We can alternatively check the commutativity of the right hand diagram by hand:
      It holds that
      \[
          i \circ \tilde{f} \circ p
        = f
        = \bar{f} \circ p
      \]
      and hence~$i \circ \tilde{f} = \bar{f}$ because~$p$ is an epimorphism.
    \qedhere
  \end{enumerate}
\end{proof}


\begin{example*}
  Let~$A$ be a~{\kalg} and consider the category~$\Modl{A}$.
  Let~$f \colon M \to N$ be a homomorphism of left~{\modules{$A$}}.
  Then the morphism~$f' \colon \im(f) \to N$ is the inclusion~$n \mapsto n$, the morphism~$\tilde{f} \colon M \to \coker(f) = M/\ker(f)$ is the canonical projection~$m \mapsto \class{m}$, and the morphism~$\bar{f} \colon M/\ker(f) \to \im(f)$ is the isomorphism~$\class{m} \mapsto f(m)$.
\end{example*}




