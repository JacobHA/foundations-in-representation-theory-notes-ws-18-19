\section{Algebras}


\begin{conventions}
  Throughout this lecture~$\kf$ denotes a commutative ring, which we will often additionally assume to be a field.
\end{conventions}


\begin{definition}
  A~\emph{{\kalg}}\index{k-algebra@{\kalg}} is a ring~$A$ together with the structure of a~{\module{$\kf$}} on~$A$, such that the ring multiplication and the scalar multiplication are compatible in the sense that
  \begin{equation}
    \label{compatibility of multiplications}
      (\lambda a) b
    = \lambda (ab)
    = a (\lambda b)
  \end{equation}
  for all~$\lambda \in \kf$ and all~$a, b \in A$.
\end{definition}


\begin{definition}
  For~{\kalgs}~$A$ and~$B$ a map~$f \colon A \to B$ is a \emph{homomorphism of~{\kalgs}}\index{homomorphism!of k-algebras@homomorphism of~{\kalgs}}\index{k-algebra@{\kalg}!homomorphism of} if it is a ring homomorphism which is also~{\klin}.
\end{definition}


\begin{remark}
  Let~$A$ be a ring.
  Then
  \[
              \ringcenter(A)
    \defined  \{
                a \in A
              \suchthat
                \text{$ab = ba$ for every~$b \in A$}
              \}
  \]
  is a commutative subring of~$A$, the \emph{center}\index{center} of~$A$.
\end{remark}


\begin{remark}
  Let~$A$ be a ring.
  To give a~{\kalg} structure on~$A$ is the same as giving a ring homomorphism~$\varphi \colon \kf \to \ringcenter(A)$.
  More precisely:
  \begin{enumerate}
    \item
      If~$A$ is a~{\kalg} then we get a ring homomorphism~$\tilde{\varphi} \colon \kf \to A$ given by~$\lambda \mapsto \lambda \cdot 1_A$.
      This ring homomorphism satisfies~$\im(\tilde{\varphi}) \subseteq \ringcenter(A)$ and therefore restricts to a ring homomorphism~$\varphi \colon A \to \ringcenter(A)$.
    \item
      Let~$\varphi \colon \kf \to \ringcenter(A)$ be a ring homomorphism.
      Define~$\lambda \cdot a = \varphi(\lambda) \cdot a$ for all~$\lambda \in \kf$ and all~$a \in A$.
      This gives~$A$ the structure of a~{\module{$\kf$}}.
      The compatibility condition~\eqref{compatibility of multiplications} is satisfied because~$\im(\varphi) \subseteq \ringcenter(A)$.
  \end{enumerate}
  These two constructions are mutually inverse.
  \begin{enumerate}[resume]
    \item
      Let~$A$ and~$B$ be~{\kalgs} and let~$f \colon A \to B$ be a homomorphisms of rings.
      Then~$f$ is a homomorphism of~{\kalgs} if and only if it is compatible with the ring homomorphisms~$\kf \to \ringcenter(A)$ and~$\kf \to \ringcenter(B)$ in the sense that the following diagram commutes:
      \[
        \begin{tikzcd}[column sep = small]
            A
            \arrow{rr}[above]{f}
          & {}
          & B
          \\
            \ringcenter(A)
            \arrow[hook]{u}
          & {}
          & \ringcenter(B)
            \arrow[hook]{u}
          \\
            {}
          & \kf
            \arrow{ul}
            \arrow{ur}
          & {}
        \end{tikzcd}
      \]
  \end{enumerate}
\end{remark}


\begin{example}
  \leavevmode
  \begin{enumerate}
    \item
      Let~$V$ be a~{\module{$\kf$}} and consider ~$\End_\kf(V)$ with the multiplication
      \[
                \End_\kf(V) \times \End_\kf(V)
        \to     \End_\kf(V) \,,
        \quad   (f, g)
        \mapsto f \circ g \,.
      \]
      Then~$\End_\kf(V)$ is a ring and becomes a~{\kalg} via the ring homomorphim
      \[
                \tilde{\varphi}
        \colon  \kf
        \to     \End_\kf(V) \,,
        \quad   \lambda
        \mapsto \lambda \cdot \id_V \,,
      \]
      for which~$\im(\tilde{\varphi}) \subseteq \End_\kf(V)$.
      (If~$\kf$ is a field then~$\ringcenter(\End_\kf(V)) = \kf \cdot \id_V \cong \kf$.)
    \item
      Take~$V = \kf^n$ (the free~{\module{$\kf$}} of rank~$n$).
      Then
      \begin{gather*}
              \End_\kf(V)
        \cong \mat{n}{\kf} \,,
      \shortintertext{and}
                  \triag{n}{\kf}
        \defined  \left\{
                    M \in \mat{n}{\kf}
                  \suchthat
                    \text{$M$ is upper triangular}
                  \right\}
      \end{gather*}
      is a subalgebra\index{subalgebra} of~$\mat{n}{\kf}$, i.e.\ it is both a subring and a {\submodule{$\kf$}} of~$\mat{n}{\kf}$.
    \item
      Let~$G$ be a group.
      We define the \emph{group algebra}\index{group algebra}\index{algebra!group}~$\kf[G]$ as follows:
      As a~{\module{$\kf$}} we have that
      \begin{align*}
                  \kf[G]
         \defined \kf^{(G)}
        &\defined \text{free~{\module{$\kf$}} with basis~$G$}  \\
        &=        \left\{
                    \sum_{g \in G} a_g g
                  \suchthat*
                    \begin{tabular}{@{}c@{}}
                      $a_g \in \kf$ for every~$g \in G$, \\
                      all but finitely many~$a_g$ vanish
                    \end{tabular}
                  \right\} \,.
      \end{align*}
      The multiplication of two elements~$x, y \in \kf[G]$ that are given by linear combinations~$x = \sum_{g \in G} a_g g$ and~$y = \sum_{g \in G} b_g g$ is given by
      \[
          x \cdot y
        = \sum_{g, h \in G} a_g b_h (gh)
        = \sum_{g \in G}
          \left(
            \sum_{\substack{h, h' \in G \\ h h' = g}} a_{h} b_{h'}
          \right)
          g \,.
      \]
      This multiplication is associative and {\kbilin}, and the unit of~$\kf[G]$ is given by~$1_{\kf[G]} = e$ (where~$e$ denotes the neutral element of~$G$).
  \end{enumerate}
\end{example}


\begin{definition}
  A \emph{quiver}\index{quiver}~$Q$ is a directed graph.
  Formally~$Q$ is a quadruple~$(Q_0, Q_1, s, t)$ consisting of two sets~$Q_0$,~$Q_1$ and two functions~$s, t \colon Q_1 \to Q_0$.
  \begin{itemize}
    \item
      The elements~$i \in Q_0$ are the \emph{vertices}\index{vertex} of~$Q$.
    \item
      The lements~$\alpha \in Q_1$ are the \emph{arrows}\index{arrow} of~$Q$.
    \item
      For an arrow~$\alpha \in Q_1$ the vertex~$s(\alpha)$ is the \emph{source}\index{source!of an arrow} of~$\alpha$.
    \item
      For an arrow~$\alpha \in Q_1$ the vertex~$t(\alpha)$ is the \emph{target}\index{target!of an arrow} of~$\alpha$.
  \end{itemize}
  An arrow~$\alpha \in Q_1$ can pictorially be represented as follows:
  \[
    \begin{tikzcd}
        s(\alpha)
        \arrow{r}[above]{\alpha}
      & t(\alpha)
    \end{tikzcd}
  \]
\end{definition}


\begin{example}
  \label{quivers first examples}
  \leavevmode
  \begin{enumerate}
    \item
      The quiver~$Q = (\{1\}, \emptyset, \emptyset, \emptyset)$ is given by a single vertex, labeled~$1$, and no arrows:% dont let footnote see this line break
      \footnote{The third and fourth elements of the quadrupel~$(\{1\}, \emptyset, \emptyset, \emptyset)$ refer to the empty function~$\emptyset \to \{1\}$.}
      \[
        \begin{tikzcd}
          1
        \end{tikzcd}
      \]
    \item
      The quiver~$Q = (\{1\}, \{\alpha\}, s, t)$ with (necessarily)~$s(\alpha) = t(\alpha) = 1$ is given by a single vertex, labeled~$1$, together with a single arrow, labeled~$\alpha$:
      \[
        \begin{tikzcd}
          1
          \arrow[loop]{r}[above]{\alpha}
        \end{tikzcd}
      \]
    \item
      The quiver~$Q = (\{1, 2\}, \{\alpha, \beta\}, s, t)$ with~$s(\alpha) = s(\beta) = 1$ and~$t(\alpha) = t(\beta) = 2$ is given by two vertices, labeled~$1$ and~$2$, which are connected via two parallel arrows going from~$1$ to~$2$, that are labeled~$\alpha$ and~$\beta$:
      \[
        \begin{tikzcd}
            1
            \arrow[shift left = 0.3em]{r}[above]{\alpha}
            \arrow[shift right = 0.3em]{r}[below]{\beta}
          & 2
        \end{tikzcd}
      \]
  \end{enumerate}
\end{example}


\begin{definition}
  Let~$Q$ be a quiver, assumed to be finite (i.e.\ both~$Q_0$ and~$Q_1$ are finite sets)\index{quiver!finite}.
  \begin{enumerate}
    \item
      Let~$\ell \in \Integer_{\geq 1}$.
      A \emph{path}\index{path!of length~$\geq 1$} in~$Q$ of length\index{length}~$\ell$ is a sequence~$\alpha_\ell \dotsm \alpha_1$ of arrows in~$Q$ such that~$t(\alpha_i) = s(\alpha_{i+1})$ for every~$i$.
      This may pictorially be represented as follows:
      \[
        \begin{tikzcd}
            \bullet
            \arrow{r}[above]{\alpha_1}
          & \bullet
            \arrow{r}[above]{\alpha_2}
          & \bullet
            \arrow{r}
          & \cdots
            \arrow{r}[above]{\alpha_\ell}
          & \bullet
        \end{tikzcd}
      \]
      The set of all paths of length~$\ell$ in~$Q$ is denoted by~$Q_\ell$.
      For a path~$p = \alpha_\ell \dotsm \alpha_1 \in Q_\ell$ its \emph{source}\index{source!of a path} is given by~$s(p) \defined s(\alpha_1)$ and its \emph{target}\index{target!of a path} is given by~$t(p) \defined t(\alpha_\ell)$.
      
      The set of paths of length~$0$\index{path!of length~$0$} in~$Q$ is formally defined as~$Q_0$, i.e.\ there exists for every~$i \in Q_0$ a unique path~$\varepsilon_i$ of length~$0$, and every path of length~$0$ is of this form.
      The path~$\varepsilon_i$ is the \enquote{lazy path}\index{path!lazy} at~$i$.
      We set~$s(\varepsilon_i) = i$,~$t(\varepsilon_i) = i$.
    \item
      Let~$p = \alpha_\ell \dotsm \alpha_1$ and~$q = \beta_k \dotsm \beta_1$ be paths in~$Q$ of lengths~$\ell, k \geq 1$.
      If~$t(p) = s(q)$ then the \emph{composition}\index{composition of paths} of~$p$ and~$q$ is defined as
      \[
          p \circ q
        = \beta_k \dotsm \beta_1 \alpha_\ell \dotsm \alpha_1 \,.
      \]
      The path~$p \circ q$ can pictorially be represented as follows:
      \[
        \begin{tikzcd}
            \bullet
            \arrow{r}[above]{\alpha_1}
          & \bullet
            \arrow{r}[above]{\alpha_2}
          & \cdots
            \arrow{r}[above]{\alpha_\ell}
          & \bullet
            \arrow[equal]{r}
          & \bullet
            \arrow{r}[above]{\beta_1}
          & \bullet
            \arrow{r}[above]{\beta_2}
          & \cdots
            \arrow{r}
            \arrow{r}[above]{\beta_k}
          & \bullet
        \end{tikzcd}
      \]
      % TODO: Add underbraces under the parts belonging to p and q.
      
      If~$p$ is a path in~$Q$ of length~$\ell \geq 0$ and~$i \in Q_0$ is a vertex then we set~$\varepsilon_i \circ p = p$ if~$t(p) = i$, as well as~$p \circ \varepsilon_i = p$ if~$s(p) = i$.
      
      In all other cases the composition of paths is not defined.
      
    \item
      We define~$Q_*$ to be the set of all paths in~$Q$, that is~$Q_* \defined \coprod_{\ell \geq 0} Q_\ell$.
      The \emph{path algebra}\index{path!algebra}\index{algebra!path}~$\kf Q$ of~$Q$ is defined as follows:
      As a~{\module{$\kf$}} we have that
      \[
                  \kf Q
        \defined  \kf^{(Q_*)}
        =         \left\{
                    \sum_{p \in Q_*} a_p p
                  \suchthat*
                    \begin{tabular}{@{}c@{}}
                      $a_p \in \kf$ for every~$p \in Q_*$ \\
                      all but finitely many~$a_p$ vanish
                    \end{tabular}
                  \right\} \,.
      \]
      The multiplication of two elements~$x, y \in \kf Q$ which are given by linear combinations~$x = \sum_{p \in Q_*} a_p p$ and~$y = \sum_{p \in Q_*} b_p p$ is given by
      \begin{gather*}
                  x \cdot y
        \defined \sum_{p, q \in Q_*} a_p b_q (p \cdot q) \,,
      \shortintertext{where}
                  p \cdot q
        \defined  \begin{cases}
                    p \circ q & \text{if~$s(p) = t(q)$},  \\
                    0         & \text{else} \,,
                  \end{cases}
      \end{gather*}
      for all paths~$p, q \in Q_*$.
      This multiplication is associative because composition of paths is associative and {\kbilin}, and the unit of~$\kf Q$ is given by~$1_{\kf Q} = \sum_{i \in Q_0} \varepsilon_i$.
  \end{enumerate}
\end{definition}


\begin{example}
  We determine the path algebras of the quivers from \cref{quivers first examples}.
  \begin{enumerate}
    \item
      For the quiver
      \[
        Q :
        \begin{tikzcd}
          1
        \end{tikzcd}
      \]
      its path algebra is given by~$\kf Q = k$, as it is \dash{one}{dimensional}.
    \item
      For the quiver
      \[
        Q :
        \begin{tikzcd}
            1
            \arrow[loop]{r}[above]{\alpha}
        \end{tikzcd}
      \]
      % TODO: Fix the vertical placement of Q in the above.
      its set of paths is given by~$Q_* = \{ \alpha^n \suchthat n \geq 0 \}$ with multiplication given by~$\alpha^n \cdot \alpha^m = \alpha^n \circ \alpha^m = \alpha^{n+m}$ for all~$n, m \geq 0$.
      It follows that~$\kf Q \cong \kf[t]$.
    \item
      For the quiver
      \[
        Q :
        \begin{tikzcd}
            1
            \arrow[shift left = 0.3em]{r}[above]{\alpha}
            \arrow[shift right = 0.3em]{r}[below]{\beta}
          & 2
        \end{tikzcd}
      \]
      its set of paths is given by~$Q_* = \{ \varepsilon_1, \varepsilon_2, \alpha, \beta \}$.
      Its path algebra is given by
      \[
          \kf Q
        = \kf \varepsilon_1 \dsum \kf \varepsilon_2 \dsum \kf \alpha \dsum \kf \beta \,,
      \]
      with multiplication being given on the basis elements given as follows:
      \[
        \begin{array}{r|cccc}
            \text{row $\cdot$ column}
          & \varepsilon_1
          & \varepsilon_2
          & \alpha
          & \beta
          \\
          \hline
            \varepsilon_1
          & \varepsilon_1
          & 0
          & 0
          & 0
          \\
            \varepsilon_2
          & 0
          & \varepsilon_2
          & \alpha
          & \beta
          \\
            \alpha
          & \alpha
          & 0
          & 0
          & 0
          \\
            \beta
          & \beta
          & 0
          & 0
          & 0
        \end{array}
      \]

  \end{enumerate}
\end{example}


\begin{lemma}
  Let~$\kf$ be a field and let~$A$ be a {\fd}~{\kalg}.
  Then there exists an injective homomorphism of~{\kalgs}~$\varphi \colon A \to \mat{n}{\kf}$ for~$n \defined \dim_\kf(A)$.
\end{lemma}


\begin{proof}
  Consider the~{\kalg}~$\End_\kf(A)$.
  It follows from choosing a \kdash{basis} of~$A$ that~$\End_\kf(A) \cong \mat{n}{\kf}$.
  It therefore sufficies to construct an injective homomorphism of~{\kalgs}~$\varphi \colon A \to \End_\kf(A)$.
  Let~$a \in A$.
  Then the map
  \[
            \varphi(a)
    \colon  A
    \to     A \,,
    \quad   b
    \mapsto ab
  \]
  is {\klin} by~\eqref{compatibility of multiplications}.
  The map~$\varphi \colon A \to \End_\kf(A)$ is the desired injective homomorphism of~{\kalgs}:
  \begin{itemize}
    \item
      The map~$\varphi$ is~{\klin} by~\eqref{compatibility of multiplications} and by the distributivity of the multiplication of~$A$.
    \item
      It holds for all~$a, a', b \in A$ that
      \[
          \varphi(a a')(b)
        = (a a') b
        = a (a' b)
        = \varphi(a)( \varphi(a')(b) )
        = (\varphi(a) \varphi(a'))(b) \,,
      \]
      which shows that~$\varphi(a a') = \varphi(a) \varphi(a')$, i.e.\ that~$\varphi$ is multiplicative.
    \item
      It holds that~$\varphi(1_A) = \id_A = 1_{\End_\kf(A)}$.
    \item
      It holds for~$a \in \ker(\varphi)$ that~$\varphi(a) = 0$ and therefore that
      \[
          0
        = \varphi(a)(1)
        = a \cdot 1
        = a \,,
      \]
      which shows that~$\varphi$ is injective.
  \end{itemize}
  This altogether shows claim.
\end{proof}





\lecturend{1}





\begin{definition}
  Let~$A$ be a~{\kalg}.
  The \emph{opposite algebra}\index{opposite!algebra}\index{algebra!opposite}~$A^\op$ of~$A$ has the same underlying~{\module{$\kf$}} as~$A$ but with multiplication given by
  \[
              a \cdot_{A^\op} b
    \defined  b \cdot_A a
  \]
  for all~$a, b \in A^\op$.
\end{definition}


\begin{example}
  Let~$Q$ be a quiver.
  The \emph{opposite quiver}\index{opposite!quiver}\index{quiver!opposite}~$Q^\op$ of~$Q$ results from~$Q$ by reversing the direction of all its arrows.
  More formally, if $Q = (Q_0, Q_1, s, t)$ then the opposite quiver~$Q^\op$ is given by~$Q^\op = (Q_0, Q_1, s^\op, t^\op)$ with
  \[
      s^\op(\alpha)
    = t(\alpha)
    \quad\text{and}\quad
      t^\op(\alpha)
    = s(\alpha) \,.
  \]
  This may also be expressed as~$(Q_0, Q_1, s, t)^\op = (Q_0, Q_1, t, s)$.
  It then holds that
  \[
    (\kf Q)^\op = \kf (Q^\op) \,.
  \]
\end{example}




