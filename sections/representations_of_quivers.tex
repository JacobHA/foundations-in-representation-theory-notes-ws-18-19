\section{Representations of Quivers}


\begin{remarkdefinition}
  Let~$Q$ be a quiver.
  \begin{enumerate}
    \item
      A \emph{representation}~$X$\index{representation!of a quiver}\index{quiver!representation} of~$Q$ consists of the following data:
      \begin{itemize}
        \item
          A~{\module{$\kf$}}~$X_i$ for every vertex~$i \in Q_0$.
        \item
          For every arrow~$\alpha \in Q_1$ a~{\klin} map~$X_\alpha \colon X_{s(\alpha)} \to X_{t(\alpha)}$.
      \end{itemize}
  \end{enumerate}
  Let~$X$,~$Y$ and~$Z$ be representations of~$Q$.
  \begin{enumerate}[resume]
    \item
      A \emph{homomorphism}\index{homomorphism!of quiver representations}~$f \colon X \to Y$ is a tupel~$(f_i)_{i \in Q_0}$ of~{\klin} maps~$f_i \colon X_i \to Y_i$ such that the square
      \[
        \begin{tikzcd}
            X_{s(\alpha)}
            \arrow{r}[above]{X_\alpha}
            \arrow{d}[left]{f_{s(\alpha)}}
          & X_{t(\alpha)}
            \arrow{d}[right]{f_{t(\alpha)}}
          \\
            Y_{s(\alpha)}
            \arrow{r}[above]{Y_\alpha}
          & Y_{t(\alpha)}
        \end{tikzcd}
      \]
      commutes for every arrow~$\alpha \in Q_1$.
      
      If~$f \colon X \to Y$ and~$g \colon Y \to Z$ are homomorphisms of representations then the \emph{composition}~$g \circ f$ is the homomorphism~$X \to Z$ with component~$(g \circ f)_i = g_i \circ f_i$ for every~$i \in Q_0$.
      
      The \emph{identity homomorphism}\index{identity!homomorphism}\index{homomorphism!identity} of the representation~$X$ is the homomorphism of representations~$\id_X \colon X \to X$ with~$(\id_X)_i = \id_{X_i}$ for every~$i \in Q_0$.
      It holds for every homomorphism of representations~$f \colon X \to Y$ that~${\id_Y} \circ f = f$ and~$f \circ {\id_X} = f$.
    \item
      A homomorphism~$f \colon X \to Y$ of representations is an~\emph{isomorphism}\index{isomorphism!of quiver representations} if there exists a homomorphism of representations~$g \colon Y \to X$ with~$g \circ f = \id_X$ and~$f \circ g = \id_Y$.
      The homomorphism~$f$ is an isomorphism if and only if the component~$f_i$ is an isomorphism for every~$i \in Q_0$.
  \end{enumerate}
\end{remarkdefinition}


\begin{example}
  \leavevmode
  \begin{enumerate}
    \item
      For the quiver~$Q = (\bullet)$ a representation of~$Q$ is the same a~{\module{$\kf$}}~$V$.
      For two such representations~$V$ and~$W$, a homomorphism of representations~$V \to W$ is just a {\klin} map.
    \item
      For the quiver~$Q = (\begin{tikzcd} \bullet \arrow[loop right] \end{tikzcd})$ a representation of~$Q$ is the same as a pair~$(V,\varphi)$ consisting of a~{\module{$\kf$}}~$V$ together with a~{\klin} endomorphism~$\varphi \colon V \to V$.
      
      Given two such representations~$(V, \varphi)$ and~$(W,\psi)$, a homomorphism of representations~$f \colon (V,\varphi) \to (W,\psi)$ is the same as a~{\klin} map~$f \colon V \to W$ with~$f \circ \varphi = \psi \circ f$.
    \item
      For the quiver~$Q = (\begin{tikzcd} 1 \arrow[shift left]{r}[above]{\alpha} \arrow[shift right]{r}[below]{\beta} & 2 \end{tikzcd})$ a representation of~$Q$ is the same as a quadruple~$(V_1,V_2,A_1,A_2)$ consisting of two~{\modules{$\kf$}}~$V_1$ and~$V_2$ and two~{\klin} maps~$A_1, A_2 \colon V_1 \to V_2$.
      
      Given two such representations~$(V_1, V_2, A_1, B_1)$ and~$(W_1, W_2, A_2, B_2)$, a homomorphism of representations~$f \colon (V_1, V_2, A_1, B_1) \to (W_1, W_2, A_2, B_2)$ is the same as a pair~$(f_1, f_2)$ of~{\klin} maps~$f_1 \colon V_1 \to W_1$ and~$f \colon V_2 \to W_2$ with~$f_2 A_1 = A_2 f_1$ and~$f_2 B_1 = B_2 f_1$.
  \end{enumerate}
\end{example}





\lecturend{2}




\begin{remark}
  For a finite quiver~$Q$ we can consider its representations over~$\kf$ as well as modules over its path algebra~$\kf Q$.
  It turns out that both concepts are equivalent.
  \[
    \begin{tikzcd}
        {}
      & \begin{tabular}{c} $Q$ a quiver, \\ $\kf$ a commutative ring\end{tabular}
        \arrow[squiggly]{dl}
        \arrow[squiggly]{dr}
      & {}
      \\
        \begin{tabular}{c} left modules \\ over~$\kf Q$ \end{tabular}
        \arrow[<->]{rr}
      & {}
      & \begin{tabular}{c} representations \\ of~$Q$ over~$\kf$ \end{tabular}
    \end{tikzcd}
  \]
  \begin{enumerate}
    \item
      Let~$X$ be a representation of~$Q$ over~$\kf$.
      We associate to~$X$ a left~{\module{$\kf Q$}} module~$M = F(X)$ as follows:
      
      As a~{\module{$\kf$}} let~$M = \bigdsum_{i \in Q_0} X_i$.
      Define an action of~$\kf Q$ on~$M$ by actions of the paths~$p \in Q_*$:
      Let~$p$ be a path of length~$\geq 1$ with~$p = \alpha_\ell \dotsm \alpha_1$ for arrows~$\alpha_\ell, \dotsc, \alpha_1 \in Q_1$.
      Define a~{\klin} map~$X_{s(p)} \to X_{t(p)}$ as~$X_p \defined X_{\alpha_\ell} \dotsm X_{\alpha_1}$.
      Also define a~{\klin} map~$\tilde{X}_p \colon M \to M$ as the composition
      \[
          \tilde{X}_p
        \colon
          M
        =
          \bigdsum_{i \in Q_0} X_i
        \xlongto{\pi_{s(p)}}
          X_{s(p)}
        \xlongto{X_p}
          X_{t(p)}
        \xlongto{\iota_{t(p)}}
          \bigdsum_{i \in Q_0} X_i \,.
      \]
      By using these endomorphisms we define on~$M$ the structure of a~{\module{$\kf Q$}} via
      \begin{align*}
              \kf Q \times M
        &\to  M \,,
        \\
                  \left(
                    a = \sum_{p \in Q_*} \lambda_p p,
                    x = (x_i)_{i \in Q_0}
                  \right)
        &\mapsto  ax
         =        \sum_{p \in Q_*} \lambda_p \tilde{X}_p(x)
         =        \sum_{p \in Q_*} \lambda_p \iota_{t(p)} X_p( x_{s(p)} ) \,.
      \end{align*}
      We have to check that this action satisfies the module axioms.
      We will check the axiom~\ref{module associative} as an example:
      We need to show that
      \[
          a(bx)
        = (ab)x
      \]
      for all~$a, b \in \kf Q$ and~$x \in M$.
      Both expressions are~{\kbilin} in~$(a,b)$, so it sufficies to show this equality for the case that~$a$ and~$b$ are basis elements of~$\kf Q$, i.e.\ paths~$p$ and~$q$ in~$Q_*$.
      It then holds that
      \begin{align*}
          p \cdot (q \cdot x)
        = \tilde{X}_p \tilde{X}_q(x)
        &= \iota_{t(p)} X_p
          \underbrace{ \pi_{s(p)} \iota_{t(q)} }_{\mathclap{
            = \left\{
                \begingroup
                \renewcommand{\thickspace}{\kern 0.4em} % column distance
                \begin{smallmatrix*}[l]
                  {\id} & \text{if~$s(q) = t(p)$},  \\
                  0     & \text{otherwise},
                \end{smallmatrix*}
                \endgroup
              \right.
          }}
          X_q \iota_{s(q)}
        \\
        &= \begin{cases}
            \iota_{t(p)} X_p X_q(x_{s(q)})  & \text{if~$t(q) = s(p)$}, \\
            0                               & \text{otherwise},
          \end{cases}
      \end{align*}
      as well as
      \[
          (
          \underbrace{ p \cdot q }_{\mathclap{
              = \left\{
                  \begingroup
                  \renewcommand{\thickspace}{\kern 0.4em} % column distance
                  \begin{smallmatrix*}[l]
                    p \circ q & \text{if~$s(q) = t(p)$},  \\
                    0         & \text{otherwise},
                  \end{smallmatrix*}
                  \endgroup
                \right.
            }}
            )
            \cdot x
        =   \begin{cases}
              \tilde{X}{(p \circ q)}(x) & \text{if~$s(q) = t(p)$},  \\
              0                         & \text{otherwise}.
            \end{cases}
      \]
      It holds in the case~$t(q) = s(p)$ that
      \[
          \tilde{X}_{p \circ q}
        = \iota_{t(p \circ q)} X_{p \circ q} \pi_{s(p \circ q)}
        = \iota_{t(p)} X_p X_q \pi_{s(q)} \,,
      \]
      which shows that the two expressions~$p \cdot (q \cdot x)$ and~$(p \cdot q) \cdot x$ coincide.
      
      The construction~$F$ is functorial:
      If~$X$ and~$Y$ are representations of~$Q$ over~$\kf$ and~$f \colon X \to Y$ is a homomorphism of representations then we get an induced homomorphism of left~{\modules{$\kf Q$}}~$F(f) \colon F(X) \to F(Y)$ given by
      \[
          F(f)\left( (x_i)_{i \in Q_0} \right)
        = ( f_i(x_i) )_{i \in Q_0}
      \]
      for every~$(x_i)_{i \in Q_0} \in \bigdsum_{i \in Q_0} X_i = F(X)$.
      
    \item
      Let~$M$ be a left~{\module{$\kf Q$}}.
      We associate to~$M$ a representation~$X = G(M)$ of~$Q$ over~$\kf$ as follows:
      
      We set~$X_i \defined \varepsilon_i M$ for every~$i \in Q_0$, which is a~{\submodule{$\kf$}} of~$M$ (but in general not a~{\submodule{$\kf Q$}}).
      For every arrow~$\alpha$ of~$Q$ we define a~{\klin} map~$X_\alpha \colon X_{s(\alpha)} \to X_{t(\alpha)}$ by
      \[
                  X_\alpha( x )
        \defined  \alpha x \,.
      \]
      for every~$x \in X_{s(\alpha)}$.
      This map is~{\welldef} because it holds for every~$x \in X_{s(\alpha)}$ (and more generally every~$x \in M$) that
      \[
            \alpha x
        =   (\varepsilon_{t(\alpha)} \alpha) x
        =   \varepsilon_{t(\alpha)} \alpha x
        \in \varepsilon_{t(\alpha)} M
        =   X_{t(\alpha)} \,.
      \]
      
      This construction is again functorial:
      Let~$M$ and~$N$ be left~{\modules{$\kf Q$}} and let~$g \colon M \to N$ be a homomorphism of left~{\modules{$\kf Q$}}.
      Let~$X \defined G(M)$ and~$Y \defined G(N)$.
      For every~$i \in Q_0$ the homomorphism~$g$ restricts to a~{\klin} map
      \[
                G(g)_i
        \colon  X_i
        \to     Y_i \,,
        \quad   x
        \mapsto g(x) \,.
      \]
      This restriction is {\welldef} because
      \[
                  g(X_i)
        =         g(\varepsilon_i M)
        =         \varepsilon_i g(M)
        \subseteq \varepsilon_i N
        =         Y_i \,.
      \]
      To show that~$G(g)$ is a homomorphism of representations we need to show that for every arrow~$\alpha$ in~$Q$ the following square commutes:
      \[
        \begin{tikzcd}
            X_{s(\alpha)}
            \arrow{r}[above]{X_\alpha}
            \arrow{d}[left]{G(g)_{s(\alpha)}}
          & X_{t(\alpha)}
            \arrow{d}[right]{G(g)_{t(\alpha)}}
          \\
            Y_{s(\alpha)}
            \arrow{r}[above]{Y_\alpha}
          & Y_{t(\alpha)}
        \end{tikzcd}
      \]
      It holds for every~$x \in X_{s(\alpha)}$ that
      \[
          G(g)_{t(\alpha)}( X_\alpha( x ) )
        = g(\alpha x)
        = \alpha g(x)
        = Y_\alpha( G(g)_{s(\alpha)}(x) ) \,,
      \]
      which shows the commutativity of the above square.
  \end{enumerate}
\end{remark}


\begin{theorem}
  \label{quiver rep are modules}
  Let~$Q$ be a finite quiver.
  \begin{enumerate}
    \item
      If~$M$ is a left~{\module{$\kf Q$}} then~$FG(M) \cong M$ as left~{\modules{$\kf Q$}}.
    \item
      If~$X$ is a representation of~$Q$ over~$\kf$ then~$GF(X) \cong X$ as representations of~$Q$.
  \end{enumerate}
\end{theorem}


\begin{proof}
  \leavevmode
  \begin{enumerate}
    \item
      Let~$X \defined\ G(M)$.
      Then~$F(X) = \bigdsum_{i \in Q_0} X_i = \bigdsum_{i \in Q_0} \varepsilon_i M$ as~{\modules{$\kf$}}.
      The~{\module{$\kf Q$}}~$M$ is actually the internal direct sum of its~{\submodules{$\kf$}}~$\varepsilon_i M$:
      \begin{itemize}
        \item
          It follows from $1_{\kf Q} = \sum_{i \in Q_0} \varepsilon_i$ that every~$x \in M$ can be expressed as
          \[
                x
            =   1 \cdot x
            =   \sum_{i \in Q_0} \varepsilon_i x
            \in \sum_{i \in Q_0} \varepsilon_i M \,.
          \]
          This shows that~$M = \sum_{i \in Q_0} \varepsilon_i M$.
        \item
          It holds for all~$i, j \in Q_0$ with~$i \neq j$ that~$\varepsilon_i \varepsilon_j = 0$.
          It follows for~$i \in Q_0$ and~$x \in \varepsilon_i M \cap \sum_{j \neq i} \varepsilon_j M$ with~$x = \varepsilon_i x_i$ for some~$x_i \in M$ and~$x = \sum_{j \neq i} \varepsilon_j x_j$ for some~$x_j \in M$ that
          \[
              x
            = \varepsilon_i x_i
            = \varepsilon_i \varepsilon_i x_i
            = \varepsilon_i x
            = \varepsilon_i \sum_{j \neq i} \varepsilon_j x_j
            = \sum_{j \neq i} \underbrace{\varepsilon_i \varepsilon_j}_{=0} x_j
            = 0 \,.
          \]
          This shows that the sum~$\sum_{i \in Q_0} \varepsilon_i M$ is direct.
      \end{itemize}
      
      Together this shows that the~{\klin} map
      \[
                \varphi
        \colon  F(X)
        =       \bigdsum_{i \in Q_0} \varepsilon_i M
        \to     M \,,
        \quad   (x_i)_{i \in Q_0}
        \mapsto \sum_{i \in Q_0} x_i
      \]
      is an isomorphism of~{\modules{$\kf$}}.
      We claim that~$\varphi$ is already an isomorphism of left~{\modules{$\kf Q$}}.
      For this we need to show that~$\varphi(ax) = a \varphi(x)$ for all~$a \in \kf Q$ and all~$x \in M$.
      It sufficies to show this equality in the cases that~$a = p$ is a path~$p \in Q_*$.
      It then holds for every element~$x = (x_i)_{i \in Q_0} \in F(X)$ that
      \[
        \begingroup
        \arraycolsep=1.4pt
        \renewcommand\arraystretch{1.5}
        \begin{array}{rcl}
          \varphi(px)
        & =
        & \varphi( \iota_{t(p)} X_p( x_{s(p)} ) )
        \\
          {}
        & \underset{ \text{def.~$X_p$} }{=}
        & \varphi( \iota_{t(p)}( p \cdot x_{s(p)} ) )
        \\
          {}
        & =
        & \varphi( p \cdot x_{s(p)} )
        \\
          {}
        & =
        & p \cdot x_{s(p)}
        \end{array}
        \endgroup
      \]
      as well as
      \[
          p \cdot \varphi(x)
        = p \cdot \sum_{i \in Q_0} \varepsilon_i x_i
        = \sum_{i \in Q_0} p \varepsilon_i x_i
        = p \cdot x_{s(p)} \,.
      \]
    \item
      Let~$M \defined F(X)$.
      Then
      \[
              G(M)_i
        =     \varepsilon_i G(M)
        =     \varepsilon_i \bigdsum_{j \in Q_0} X_j
        \cong X_i
      \]
      for every vertex~$i \in Q_0$, and
      \[
          G(M)_\alpha( x_{s(\alpha)} )
        = \alpha x_{s(\alpha)}
        = X_\alpha( x_{s(\alpha)} ) \,.
      \]
      for every arrow~$\alpha \in Q_1$.
    \qedhere
  \end{enumerate}
\end{proof}


\begin{remark}
  Let~$Q$ be a finite quiver.
  \begin{enumerate}
    \item
      If~$M$ is a left~{\module{$\kf Q$}} and~{$\kf$} is a field then~$\dim_k(M) = \sum_{i \in Q_0} \dim_k(X_i)$ for~$X \defined F(M)$.
    \item
      The path algebra~$\kf Q$ has finite rank as a~{\module{$\kf$}} if and only if~$Q$ contains no oriented cycles (where an \emph{oriented cycle}\index{oriented cycle} is a path~$p$ of length~$\geq 1$ with~$s(p) = t(p)$).
    \item
      If a~{\kalg}~$A$ is finitely generated as a~{\module{$\kf$}} then a left~{\module{$A$}}~$M$ is finitely generated if and only if it is finitely generated as a~{\module{$\kf$}}:
      If~$M$ is finitely generated as a~{\module{$\kf$}} then every finite~\dash{$\kf$}{generating} set of~$M$ is also a finite \dash{$A$}{generating} set for~$M$.
      If on the other hand~$M$ is finitely generated as a left~{\module{$A$}} then there exists a surjective homomorphism of left~{\modules{$A$}}~$A^n \to M$;
      the~{\module{$A$}}~$A^n$ is again finitely generated and so~$M$ is finitely generated.
      
      It follows in particular that if~$Q$ has no oriented cycles then a~{\module{$\kf Q$}}~$M$ is finitely generated as an~{\module{$\kf Q$}} if and only if it is is finitely generated as a~{\module{$\kf$}}.
      If~$\kf$ is additionally a field, then this means that~$M$ is finitely generated as a~{\module{$\kf Q$}} if and only if it is {\fd}.
    \item
      If~$X$ is a representation of~$Q$ over~$\kf$ then a \emph{subrepresentation}\index{sub-!representation} of~$X$ is a tupel~$(Y_i)_{i \in Q_0}$ of~{\submodules{$\kf$}}~$Y_i \subseteq X_i$ such that~$X_\alpha( Y_{s(\alpha)} ) \subseteq Y_{t(\alpha)}$ for every arrow~$\alpha \in Q_1$.
      The subrepresentations of~$X$ correspond under~$F$ bijectively to the left {\submodules{$\kf Q$}} of~$F(X)$.
    \item
      If~$(X^j)_{j \in J}$ is a family of representations~$X^j$ of~$Q$ over~$\kf$ then their \emph{direct sum}\index{direct sum of representations} is the representations~$\bigdsum_{j \in J} X^j$ of~$Q$ over~$\kf$ that is given on vertices by
      \[
          \left( \bigdsum_{j \in J} X^j \right)_i
        = \bigdsum_{j \in J} X^j_i
      \]
      for every~$i \in I$, and on arrows by
      \[
                \left( \bigdsum_{j \in J} X^j \right)_\alpha
        =       \bigdsum_{j \in J} X^j_\alpha
        \colon  \bigdsum_{j \in J} X^j_{s(\alpha)}
        \longto \bigdsum_{j \in J} X^j_{t(\alpha)}
      \]
      for every~$\alpha \in Q_1$.
      Under~$F$, the direct sum of representations corresponds to the direct sum of left~{\modules{$\kf Q$}}.
  \end{enumerate}
\end{remark}




