\section{Bimodules and Tensor Products}


\begin{remark*}
  The author recommends \cite[10.4]{DummitFoote2004} for an introduction to tensor products.
\end{remark*}


\begin{definition}
  Let~$A$ and~$B$ be~{\kalgs}.
  An~{\module{$A$}[$B$]}\index{bimodule}\index{module!bi-|see {bimodule}} is a~{\module{$\kf$}}~$M$ together with two~{\kbilin} multiplications
  \begin{gather*}
            A \times M
    \to     M \,,
    \quad   (a,m)
    \mapsto am \,,
  \shortintertext{and}
            M \times B
    \to     M \,,
    \quad   (m,b)
    \mapsto mb
  \end{gather*}
  such that
  \begin{enumerate}[label = (B\arabic*)]
    \item
      $M$ becomes a left~{\module{$A$}},
    \item
      $M$ becomes a right~{\module{$B$}}, and
    \item
      $(am)b = a(mb)$ for all~$a \in A$,~$b \in B$ and  all~$m \in M$.
  \end{enumerate}
  That~$M$ is an~{\module{$A$}[$B$]} is denoted by~$\indmodule[A]{M}[B]$.
\end{definition}


\begin{remark}
  \leavevmode
  \begin{enumerate}
    \item
      If~$\indmodule[A]{M}[B]$ and~$\indmodule[A]{N}[B]$ are~{\modules{$A$}[$B$]} then a map~$f \colon M \to N$ is a \emph{homomorphisms of bimodules}\index{homomorphism!of bimodules}\index{bimodule!homomorphism of} if it is both a homomorphism of left~{\modules{$A$}} and a homomorphism of right~{\modules{$B$}}.
      
      If~$\indmodule[A]{M}[B]$ is an~{\module{$A$}[$B$]} then a subset~$N \subseteq M$ is a bisubmodule if is both a left~{\submodule{$A$}} and a right~{\submodule{$B$}}.
      The~{\module{$A$}[$B$]} structure of~$M$ then restricts to an~{\module{$A$}[$B$]} module structure on~$N$, which is the unique one that makes the inclusion~$N \to M$ a homomorphis of~{\modules{$A$}[$B$]}.
    \item
      If~$A$ is a~{\kalg} then~$A$ carries the structure of an~{\module{$A$}[$A$]} by letting~$A$ act on itself both via left multiplication and right multiplication.
  \end{enumerate}
\end{remark}


\begin{lemma}
  Let~$A$,~$B$ and~$C$ be~{\kalgs}, and let~$\indmodule[A]{M}[B]$ and~$\indmodule[A]{N}[C]$ be bimodules.
  Then~$\Hom_A(M,N)$ becomes a~{\module{$B$}[$C$]} via the multiplications
  \begin{align*}
          B \times \Hom_A(M,N)
    &\to  \Hom_A(M,N) \,,
    \\
              (b,f)
    &\mapsto  \bigl[
                        bf
                \colon  M
                \to     N,
                \;      m
                \mapsto f(mb)
              \bigr] \,,
  \shortintertext{and}
          \Hom_A(M,N) \times C
    &\to  \Hom_A(M,N) \,,
    \\
              (g,c)
    &\mapsto  \bigl[
                        gc
                \colon  M
                \to     N,
                \;      m
                \mapsto f(m)c
              \bigr] \,.
  \end{align*}
\end{lemma}


\begin{proof}
  We start by showing that for every~$f \in \Hom_A(M,N)$ the~{\klin} maps~$bf$ and~$fc$ are again homomorphisms of left~{\module{$A$}}.
  This holds because
  \[
      (bf)(am)
    = f(amb)
    = a f(mb)
    = a ((bf)(m))
  \]
  for all~$a \in A$ and all~$m \in M$, and
  \[
      (gc)(am)
    = g(am) c
    = a g(m) c
    = a ((gc)(m))
  \]
  for all~$a \in A$ and all~$m \in M$.
  
  To show that~$\Hom_A(M,N)$ becomes a left~{\module{$B$}} and right~{\module{$C$}} we need to verify the various module axioms.
  As an example, we check the axiom~\ref{module associative} for the left~{\module{$B$}} structure via the calculation
  \[
      ((b b')(f))(m)
    = f(m(bb'))
    = f((mb)b')
    = (b'f)(mb)
    = (b(b'f))(m)
  \]
  for all~$f \in \Hom_A(M,N)$, all~$b, b' \in B$ and all~$m \in M$.
  
  The compatibility of the left~{\module{$B$}} structure and right~{\module{$C$}} structure of~$M$ follow from the calculation
  \[
    ((bf)c)(m)
    = (bf)(m)c
    = f(mb)c
    = (fc)(mb)
    = (b(fc))(m)
  \]
  for all~$f \in \Hom_A(M,N)$, all~$b \in B$, all~$c \in C$ and all~$m \in M$.
\end{proof}





\lecturend{3}





\begin{definition}
  Let~$A$ be a~{\kalg}, let~$\indmodule{M}[A]$ be right~{\module{$A$}} and let~$\indmodule[A]{N}$ be a left~{\module{$A$}}.
  \begin{enumerate}
    \item
      If~$P$ is a~{\module{$\kf$}} then a map~$\varphi \colon M \times N \to P$ is~\emph{\balanced{$A$}}\index{balanced map} if it is~{\kbilin} and satisfies
      \[
          \varphi(xa, y)
        = \varphi(x, ay)
      \]
      for all~$x \in M$, all~$y \in N$, and all~$a \in A$.
    \item
      A pair~$(T,\tau)$ consisting of a~{\module{$\kf$}}~$T$ and an~{\balanced{$A$}} map~$\tau \colon M \times N \to T$ is a \emph{tensor product}\index{tensor product!of modules} of~$M$ with~$N$ over~$A$ if it satisifies the following universal property:
      There exists for every~{\module{$\kf$}}~$P$ and every~{\balanced{$A$}} map~$\varphi \colon M \times N \to P$ a unique~{\klin} map~$f \colon T \to P$ with~$f \circ \tau = \varphi$, i.e.\ such that the following triangle commutes:
      \[
        \begin{tikzcd}[row sep = large]
            M \times N
            \arrow{r}[above]{\varphi}
            \arrow{d}[left]{\tau}
          & P
          \\
            T
            \arrow[dashed]{ur}[below right]{f}
          & {}
        \end{tikzcd}
      \]
  \end{enumerate}
\end{definition}


\begin{lemma}

Let~$A$ be a~{\kalg}, let~$\indmodule{M}[A]$ be a right~{\module{$A$}} and let~$\indmodule[A]{N}$ be a left~{\module{$A$}}.
  \begin{enumerate}
    \item
      There exists a tensor product~$(T,\tau)$ of~$M$ with~$N$ over~$A$.
    \item
      The tensor product of~$M$ with~$N$ over~$A$ is unique up to unique isomorphism:
      If~$(T',\tau')$ is another such tensor product then there exists a unique homomorphism of~{\modules{$\kf$}}~$f \colon T \to T'$ with~$f \circ \tau = \tau'$, i.e.\ such that the triangle
      \[
        \begin{tikzcd}
            {}
          & M \times N
            \arrow{dl}[above left]{\tau}
            \arrow{dr}[above right]{\tau'}
          & {}
          \\
            T
            \arrow[dashed]{rr}[above]{f}
          & {}
          & T'
        \end{tikzcd}
      \]
      commutes, and the homomorphism~$f$ is an isomorphism.
  \end{enumerate}
\end{lemma}


\begin{proof}
  We start by showing the claimed uniqueness of the tensor product.
  It follows from the universal property of the tensor product~$(T,\tau)$ applied to the~{\balanced{$A$}} map~$\tau' \colon M \times N \to T'$ that there exists a unique~{\klin} map~$f \colon T \to T'$ with~$f \circ \tau = \tau'$.
  It also follows from the universal property of the tensor product~$(T',\tau')$ applied to the~{\balanced{$A$}} map $\tau \colon M \times N \to T$ that there exists a unique~{\klin} map~$g \colon T' \to T$ with~$g \circ \tau' = \tau$.
  Together this means that the following two triangles commute:
  \[
    \begin{tikzcd}
        {}
      & M \times N
        \arrow{dl}[above left]{\tau}
        \arrow{dr}[above right]{\tau'}
      & {}
      \\
        T
        \arrow[dashed]{rr}{f}
      & {}
      & T'
    \end{tikzcd}
    \qquad\qquad
    \begin{tikzcd}
        {}
      & M \times N
        \arrow{dl}[above left]{\tau'}
        \arrow{dr}[above right]{\tau}
      & {}
      \\
        T'
        \arrow[dashed]{rr}{g}
      & {}
      & T
    \end{tikzcd}
  \]
  These two triangles team up to form the following commutative diagram:
  \[
    \begin{tikzcd}[sep = large]
        {}
      & M \times N
        \arrow{dl}[above left]{\tau}
        \arrow{d}[left]{\tau'}
        \arrow{dr}[above right]{\tau}
      & {}
      \\
        T
        \arrow{r}[above, near end]{f}
        \arrow[dashed, bend right]{rr}[below]{g \circ f}
      & T'
        \arrow{r}[above, near start]{g}
      & T
    \end{tikzcd}
  \]
  It follows that~$g \circ f$ is the unique homomorphism~$T \to T$ that makes the triangle
  \[
    \begin{tikzcd}
        {}
      & M \times N
        \arrow{dl}[above left]{\tau}
        \arrow{dr}[above right]{\tau}
      & {}
      \\
        T
        \arrow[dashed]{rr}
      & {}
      & T
    \end{tikzcd}
  \]
  commute.
  This shows that~$g \circ f = \id_T$ because the identity~$\id_T \colon T \to T$ also makes this triangle commute.
  We find in the same way that also~$f \circ g = \id_{T'}$.
  Together this shows that the homomorphism~$f$ is an isomorphism (with inverse~$g$).
  
  Now we show the existence of the tensor product:
  Let~$F$ be the free~{\module{$\kf$}} on the set~$M \times N$ and let~$U \subseteq F$ be the~{\submodule{$\kf$}} that is generated by the elements
  \begin{gather*}
    (x_1 + x_2, y) - (x_1, y) - (x_2, y) \,,
    \qquad
    (x, y_1 + y_2) - (x, y_1) - (x, y_2) \,,
    \\
    (\lambda x, y) - \lambda (x,y) \,,
    \qquad
    (x, \lambda y) - \lambda (x,y) \,,
    \qquad
    (xa, y) - (x, ay)
  \end{gather*}
  with~$x, x_1, x_2 \in M$,~$y, y_1, y_2 \in N$,~$\lambda \in \kf$,~$a \in A$.
  We define~$M \tensor_A N$ to be the~{\module{$\kf$}}
  \[
              M \tensor_A N
    \defined  F/U
  \]
  and define the required map~$\tau \colon M \times N \to M \tensor_A N$ for all~$x \in M$ and~$y \in N$ by
  \[
              \tau(x,y)
    \defined  x \tensor y 
    \defined  \class{(x,y)} \,.
  \]
  
  We need to check that~$(M \tensor_A N, \tau)$ satisfies the universal property of the tensor product:
  Let~$P$ be a~{\module{$\kf$}} and let~$\varphi \colon M \times N \to P$ be an~{\balanced{$A$}} map.
  It follows from the universal property of the free~{\module{$\kf$}}~$F$ that there exists a unique~{\klin} map~$\widehat{\varphi} \colon F \to P$ that makes the triangle
  \[
    \begin{tikzcd}
        M \times N
        \arrow{r}[above]{\varphi}
        \arrow{d}[left]{\text{incl.}}
      & P
      \\
        F
        \arrow[dashed]{ur}[below right]{\widehat{\varphi}}
      & {}
    \end{tikzcd}
  \]
  commute.
  That~$\varphi$ is~{\balanced{$A$}} means precisely that~$U \subseteq \ker(\widehat{\varphi})$.
  It follows that~$\widehat{\varphi}$ factors uniquely through a~{\klin} map~$\induced{\varphi} \colon F/U \to P$, which then makes the following diagram commute:
  \[
    \begin{tikzcd}[sep = large]
        M \times N
        \arrow{r}[above]{\varphi}
        \arrow{d}[left]{\text{incl.}}
        \arrow[bend right = 60]{dd}[left]{\tau}
      & P
      \\
        F
        \arrow{ur}[below right]{\widehat{\varphi}}
        \arrow{d}[left]{\text{can.}}
      & {}
      \\
        F/U
        \arrow[dashed, bend right]{uur}[below right]{\induced{\varphi}}
      & {}
    \end{tikzcd}
  \]
  This shows that there exists a unique~{\klin} map~$f \colon F/U \to P$ with~$\varphi = f \circ \tensor$, namely~$f = \induced{\varphi}$.
\end{proof}


\begin{notationnonum}
  For a right~{\module{$A$}}~$M$ and a left~{\module{$N$}} we denote the (up to unique isomorphism unique) tensor product of~$M$ by~$N$ over~$A$ by~$M \tensor_A N$, and the associated~{\balanced{$A$}} map~$M \times N \to M \tensor_A N$ by~$\tensor$.
  For all~$x \in M$ and~$y \in N$ the resulting element of~$M \tensor_A N$ is hence denoted by~$x \tensor y$.
\end{notationnonum}


\begin{warning*}
  For~{\modules{$A$}}~$\indmodule{M}[A]$ and~$\indmodule[A]{N}$ not every element of their tensor product~$M \tensor_A N$ has to be of the form~$x \tensor y$ for some~$x \in M$ and~$y \in Y$.
  Such elements of~$M \tensor_A N$ are \emph{simple tensors}\index{tensor!simple}\index{simple!tensor}, and every elements of~$M \tensor_A N$ is a~{\klin} combination of simple tensors.
  
  It often sufficies to check things for simple tensors, since for every~{\module{$\kf$}}~$P$ and every~{\klin} map~$f \colon M \tensor_A N \to P$, the action of~$f$ is already uniquely determined by its action on the simple tensors.
  But to construct such a~{\klin} map one \emph{has to} (in some instance) evoke the universal property of the tensor product;
  it does not sufficie to check~\dash{well}{definedness} (or~\dash{$\kf$}{linearity}) on simple tensors.
\end{warning*}


\begin{lemma}
  Let~$A$,~$B$ and~$C$ be~{\kalgs} and let~$\indmodule[A]{M}[B]$ and~$\indmodule[B]{N}[C]$ be bimodules.
  Then~$M \tensor_B N$ is an~{\module{$A$}[$C$]} via
  \[
      a (x \tensor y)
    = (ax) \tensor y
    \quad\text{and}\quad
      (x \tensor y) c
    = x \tensor (yc)
  \]
  for all~$x \in M$,~$y \in N$,~$a \in A$,~$b \in C$.
\end{lemma}


\begin{proof}
  The tensor product~$M \tensor_A N$ is by definition a~{\module{$\kf$}}, so it remains to show that the actions of~$A$ and~$C$ are {\welldef}, that they are module structures and that they are mutually compatible.
  
  To show that the proposed action of~$A$ on~$M \tensor_B N$ is {\welldef} let~$a \in A$.
  Then the map
  \[
            \tau_a
    \colon  M \times N
    \to     M \tensor_A N \,,
    \quad   (x,y)
    \mapsto (ax) \tensor y
  \]
  is~{\balanced{$B$}} because
  \begin{gather*}
    \begin{aligned}
          \tau_a(x_1 + x_2, y)
       =  (a(x_1 + x_2)) \tensor y
       =  (a x_1 + a x_2) \tensor y
      &=  (a x_1) \tensor y + (a x_2) \tensor y \\
      &=  \tau_a(x_1, y) + \tau_a(x_2, y) \,,
    \end{aligned}
  \\
      \tau_a(x, y_1 + y_2)
    = (ax) \tensor (y_1 + y_2)
    = (ax) \tensor y_1 + (ax) \tensor y_2
    = \tau_a(x, y_1) + \tau_a(x, y_2) \,,
  \\
      \tau_a(\lambda x, y)
    = (a \lambda x) \tensor y
    = (\lambda a x) \tensor y
    = \lambda ((ax) \tensor y)
    = \lambda \tau_a(x,y) \,,
  \\
      \tau_a(x, \lambda y)
    = (ax) \tensor (\lambda y)
    = \lambda ((ax) \tensor y)
    = \lambda \tau_a(x,y) \,,
  \\
      \tau_a(xb,y)
    = (axb) \tensor y
    = (ax) \tensor (by)
    = \tau_a(x,by)
  \end{gather*}
  for all~$x, x_1, x_2 \in M$,~$y, y_1, y_2 \in N$,~$\lambda \in \kf$,~$b \in B$.
  It follows that that~$\tau_a$ induces a {\welldef}~{\klin} map
  \[
            M \tensor_B N
    \to     M \tensor_B N \,,
    \quad   x \tensor y
    \mapsto (ax) \tensor y \,.
  \]
  This shows that the proposed action of~$A$ on~$M \tensor_B N$ is {\welldef}.
  It can be shown in the same way that the proposed action of~$C$ on~$M \tensor_B N$ is {\welldef}.
  
  It can be checked that the action of~$A$ on~$M \tensor_B N$ satisfies the various left module axioms, and is therefore a left~{\module{$A$}} structure on~$M \tensor_B N$.
  The action of~$C$ on~$M \tensor_B N$ is similarly a right~{\module{$C$}} structure.
  
  The left~{\module{$A$}} structure of~$M \tensor_B N$ is compatible with the right~{\module{$C$}} structure because
  \[
      ( a(x \tensor y) )c
    = ((ax) \tensor y)c
    = (ax) \tensor (yc)
    = a( x \tensor (yc) )
    = a( (x \tensor y)c )
  \]
  for all~$x \in M$,~$y \in N$,~$a \in A$,~$c \in C$.
\end{proof}


\begin{remark}
  Let~$\varphi \colon B \to A$ be a {\kalg} homomorphism and let~$\indmodule[B]{N}$ be a left~{\module{$B$}}.
  Then~$A$ carries the structure of an~{\module{$A$}[$B$]} via
  \[
      a x b
    = a x \varphi(b)
  \]
  for all~$x \in A$ and all~$a \in A$,~$b \in B$.
  We can regard~$N$ as a~{\module{$B$}[$\kf$]}.
  It then follows that~$A \tensor_B N$ carries the structure of an~{\module{$A$}[$\kf$]}, and hence the structure of a left~{\module{$A$}} via
  \[
      a' \cdot (a \tensor y)
    = (a' a) \tensor y
  \]
  for all~$a',a \in A$,~$y \in N$.
  This construction is the \emph{extension of scalars}\index{extension!of scalars} or \emph{induction}\index{induction} from~$B$ to~$A$.
\end{remark}


\begin{lemma}
  Let~$A$,~$B$,~$C$ and~$D$ be {\kalgs}.
  \begin{enumerate}
    \item
      Let~$( \indmodule[A]{(M_i)}[B] )_{i \in I}$ be a family of bimodules and let~$\indmodule[B]{N}[C]$ be a bimodule.
      Then there exists a unique homomorphism of~{\modules{$A$}[$C$]}
      \begin{gather*}
                \Phi_1
        \colon  \left( \bigdsum_{i \in I} M_i \right) \tensor_B N
        \to     \bigdsum_{i \in I} (M_i \tensor_B N)
      \shortintertext{with}
          \Phi_1\left( (x_i)_{i \in I} \tensor y \right)
        = ( x_i \tensor y )_{i \in I}
      \end{gather*}
      for all~$(x_i)_{i \in I} \in \bigoplus_{i \in I} M_i$ and~$y \in N$, and this homomorphism~$\Phi_1$ is an isomorphism.
    \item
      Let~$\indmodule[A]{M}[B]$ be a bimodule and let~$( \indmodule[B]{(N_j)}[C] )_{j \in J}$ be a family of bimodules.
      Then there exists a unique homomorphism of~{\modules{$A$}[$C$]}
      \begin{gather*}
                \Phi_2
        \colon  M \tensor_B \left( \bigdsum_{j \in J} N_j \right)
        \to     \bigdsum_{j \in J} (M \tensor_B N_j)
      \shortintertext{with}
          \Phi_2\left( x \tensor (y_j)_{j \in J} \right)
        = ( x \tensor y_j )_{j \in J}
      \end{gather*}
      for all~$x \in M$ and~$(y_j)_{j \in J} \in \bigoplus_{j \in J} N_j$, and this homomorphism~$\Phi_2$ is an isomorphism.
    \item
      Let~$\indmodule[A]{M}[B]$,~$\indmodule[B]{N}[C]$ and $\indmodule[C]{P}[D]$ be bimodules.
      There exists a unique homomorphism of~{\modules{$A$}[$D$]}
      \begin{gather*}
                \Phi_3
        \colon  (M \tensor_B N) \tensor_C P
        \to     M \tensor_B (N \tensor_C P)
      \shortintertext{with}
          \Phi_3( (x \tensor y) \tensor z )
        = x \tensor (y \tensor z)
      \end{gather*}
      for all~$x \in M$,~$y \in N$,~$z \in P$, and this homomorphism~$\Phi_3$ is an isomorphism.
    \item
      For every bimodule~$\indmodule[A]{M}[B]$ there exist unique homomorphisms of~{\modules{$A$}[$B$]}
      \begin{align*}
                \Phi_4
        \colon  A \tensor_A M
        \to     M 
        \quad&\text{and}\quad
                \Phi_5
        \colon  M \tensor_B B
        \to     M
      \shortintertext{with}
        \Phi_4(a \tensor m) = m
        \quad&\text{and}\quad
        \Phi_5(m \tensor b) = m
      \end{align*}
      for all~$x \in M$,~$a \in A$,~$b \in B$, and the homomorphisms~$\Phi_4$ and~$\Phi_5$ are isomorphisms.
  \end{enumerate}
\end{lemma}


\begin{proof}
  The proofs are the same as in the commutative case, see Proposition~2.27 in Algebra~I.
  One only has to additionally check that the maps are already homomorphisms of bimodules.
\end{proof}


\begin{proposition}
  \label{hom tensor adjunction}
  Let~$A$ and~$B$ be {\kalgs} and let~$\indmodule{M}[A]$,~$\indmodule[A]{N}[B]$ and~$\indmodule{P}[B]$ be (bi)modules.
  Then the map
  \begin{gather*}
            \Phi
    \colon  \Hom_B( M \tensor_A N, P)
    \to     \Hom_A( M, \Hom_B(N, P) )
  \intertext{given by}
      \Phi(f)(x)(y)
    = f(x \tensor y)
  \end{gather*}
  is a {\welldef} isomorphism of~{\modules{$\kf$}}, that is natural in~$M$,~$N$ and~$P$.
\end{proposition}


\begin{proof}
  For every~$f \in \Hom_B(M \tensor_A N, P)$ and every~$x \in M$ the map~$\Phi(f)(x) \colon N \to P$ is a homomorphism of right~{\modules{$B$}} because
  \begin{gather*}
    \begin{aligned}
          \Phi(f)(x)(y_1 + y_2)
      &=  f(x \tensor (y_1 + y_2))
       =  f(x \tensor y_1 + x \tensor y_2)  \\
      &=  f(x \tensor y_1) + f(x \tensor y_2)
       =  \Phi(f)(x)(y_1) + \Phi(f)(x)(y_2)
    \end{aligned}
    \intertext{and}
      \Phi(f)(x)(yb)
    = f(x \tensor (yb))
    = f((x \tensor y) b)
    = f(x \tensor y) b
    = \Phi(f)(x)(y) b
  \end{gather*}
  for all~$y, y_1, y_2 \in N$,~$b \in B$.
  This shows that the map~$\Phi(f)(x)$ is a {\welldef} element of~$\Hom_B(N,P)$, which menas     that for every~$f \in \Hom_B(M \tensor_A N, P)$ the map
  \[
            \Phi(f)
    \colon  M
    \to     \Hom_B(N,P)
  \]
  is {\welldef}.
  The map~$\Phi(f)$ is then a homomorphism of right~{\modules{$A$}} because
  \begin{gather*}
    \begin{aligned}
          \Phi(f)(x_1 + x_2)(y)
      &=  f((x_1 + x_2) \tensor y)
       =  f(x_1 \tensor y + x_2 \tensor y)
       =  f(x_1 \tensor y) + f(x_2 \tensor y) \\
      &=  \Phi(f)(x_1)(y) + \Phi(f)(x_2)(y)
       =  ( \Phi(f)(x_1) + \Phi(f)(x_2) )(y)
    \end{aligned}
  \shortintertext{and}
      \Phi(f)(xa)(y)
    = f((xa) \tensor y)
    = f(x \tensor (ay))
    = \Phi(f)(x)(ay)
    = (\Phi(f)(x)a)(y) \,.
  \end{gather*}
  for all~$x, x_1, x_2 \in M$,~$a \in A$,~$y \in N$.
  This shows that the proposed map
  \[
            \Phi
    \colon  \Hom_B(M \tensor_A N, P)
    \to     \Hom_A(M, \Hom_B(N,P))
  \]
  is {\welldef}.
  The map~$\Phi$ is a homomorphism of~{\modules{$\kf$}} because
  \begin{gather*}
    \begin{aligned}
          \Phi(f_1 + f_2)(x)(y)
      &=  (f_1 + f_2)(x \tensor y)
       =  f_1(x \tensor y) + f_2(x \tensor y) \\
      &=  \Phi(f_1)(x)(y) + \Phi_2(f_2)(x)(y)
       =  ( \Phi(f_1) + \Phi(f_2) )(x)(y)
    \end{aligned}
  \shortintertext{and}
      \Phi(\lambda f)(x)(y)
    = (\lambda f)(x \tensor y)
    = \lambda f(x \tensor y)
    = \lambda \Phi(f)(x)(y)
    = (\lambda \Phi)(f)(x)(y)
  \end{gather*}
  for all~$f, f_1, f_2 \in \Hom_B(M \tensor_A N, P)$,~$\lambda \in \kf$,~$x \in M$,~$y \in N$.
  
  To show that~$\Phi$ is already an isomorphism of~{\modules{$\kf$}} we construct its inverse map:
  Let~$g \in \Hom_A(M, \Hom_B(N,P))$.
  Then the map
  \[
            \psi(g)
    \colon  M \times N
    \to     P \,,
    \quad   (x,y)
    \mapsto g(x)(y)
  \]
  is~{\balanced{$A$}} because
  \begin{gather*}
    \begin{aligned}
          \psi(g)(x_1 + x_2, y)
      &=  g(x_1 + x_2)(y)
       =  (g(x_1) + g(x_2))(y)  \\
      &=  g(x_1)(y) + g(x_2)(y)
       =  \psi(g)(x_1,y) + \psi(g)(x_2,y) \,,
    \end{aligned}
  \\
      \psi(g)(x, y_1 + y_2)
    = g(x)(y_1 + y_2)
    = g(x)(y_1) + g(x)(y_2)
    = \psi(g)(x, y_1) + \psi(g)(x, y_2) \,,
  \\
      \psi(g)(xa, y)
    = g(xa)(y)
    = (g(x)a)(y)
    = g(x)(ay)
    = \psi(g)(x,ay) \,.
  \end{gather*}
  for all~$x, x_1, x_2 \in M$,~$y, y_1, y_2 \in M$,~$a \in A$.
  It follows from the universal property of the tensor product~$M \tensor_A N$ that the~{\balanced{$A$}} map~$\psi(g)$ induces a {\welldef}~{\klin} map~$\Psi(g) \colon M \tensor_A N \to P$ with
  \[
      \Psi(g)(x \tensor y)
    = \psi(g)(x, y)
    = g(x)(y)
  \]
  for all~$x \in M$,~$y \in N$, i.e.\ such that the following triangle commutes:
  \[
    \begin{tikzcd}[sep = large]
        M \times N
        \arrow{r}[above]{\psi(g)}
        \arrow{d}[left]{\tensor}
      & P
      \\
        M \tensor_A N
        \arrow[dashed]{ur}[below right]{\Psi(g)}
      & {}
    \end{tikzcd}
  \]
  The maps~$\Psi(g)$ are already homomorphismso of right~{\modules{$B$}} because
  \[
      \Psi(g)((x \tensor y) b)
    = \Psi{g}(x \tensor (yb))
    = g(x)(yb)
    = g(x)(y) b
    = \Psi(g)(x \tensor y) b
  \]
  for all~$x \in M$,~$y \in N$,~$b \in B$.
  This shows that~$\Psi(g)$ is a {\welldef} element of the~{\module{$\kf$}}~$\Hom_B(M \tensor_A N, P)$, and hence that the map
  \begin{align*}
              \Psi
     \colon   \Hom_A(M, \Hom_B(N,P))
    &\to      \Hom_B(M \tensor_A N, P) \,, \\
              g
    &\mapsto  \left[
                        \Phi(g)
                \colon  x \tensor y
                \mapsto g(x)(y)
              \right]
  \end{align*}
  is {\welldef}.
  
  The maps~$\Phi$ and~$\Psi$ are mutually inverse:
  It holds for every~$f \in \Hom_B(M \tensor_A N, P)$ that
  \[
      \Psi(\Phi(f))(x \tensor y)
    = \Phi(f)(x)(y)
    = f(x \tensor y)
  \]
  for all~$x \in M$ and~$y \i N$, which shows that~$\Psi \circ \Phi = \id$.
  It similarly holds for every~$g \in \Hom_A(\Hom_B(N, P))$ that
  \[
      \Phi(\Psi(g))(x,y)
    = \Psi(g)(x \tensor y)
    = g(x)(y) \,,
  \]
  for all~$x \in X$ and~$y \in Y$, which shows that~$\Phi \circ \Psi = \id$.
  
  The naturality in~$M$,~$N$ and~$P$ follows from a direct calculation.
\end{proof}
% TODO: Check the naturality.


% TODO: Add the other tensor hom adjunction.


