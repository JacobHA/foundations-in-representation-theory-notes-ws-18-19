\section{Natural Transformations}


\begin{definition}
  Let~$\Ccat$,~$\Dcat$ be two categories and let~$F, G \colon \Ccat \to \Dcat$ be two parallel functors between them.
  A \emph{natural transformation}\index{natural!transformation}~$\eta \colon F \to G$ is a family~$\eta = (\eta_X)_{X \in \Ob(\Ccat)}$ of morphisms~$\eta_X \colon F(X) \to G(X)$ such that for every morphism~$f \colon X \to Y$ in~$\Ccat$ the following square (in~$\Dcat$) commutes:
  \[
    \begin{tikzcd}
        F(X)
        \arrow{r}[above]{\eta_X}
        \arrow{d}[left]{F(f)}
      & G(X)
        \arrow{d}[right]{G(f)}
      \\
        F(Y)
        \arrow{r}[above]{\eta_Y}
      & G(Y)
    \end{tikzcd}
  \]
\end{definition}


% TODO: Natural transformations via their actions on morphisms.
%       Consequence: Natural transformations with the object free definition of categories.


\begin{remark}
  Let~$\Ccat$ and~$\Dcat$ be categories.
  \begin{enumerate}
    \item
      Let~$F, G, H \colon \Ccat \to \Dcat$ be functors and let~$\eta \colon F \to G$ and~$\zeta \colon G \to H$ be natural transformations.
      Their composition~$\zeta \circ \eta$ is the natural transformation~$F \to H$ that is given by
      \[
        (\zeta \circ \eta)_X
        \defined
        ( \zeta_X \circ \eta_X \colon F(X) \to H(X) )
      \]
      at every object~$X \in \Ob(\Ccat)$.
      To see that this is indeed a natural transfomation we consider the following diagram:
      \[
        \begin{tikzcd}
            F(X)
            \arrow[dashed, bend left = 45]{rr}[above]{(\zeta \circ \eta)_X}
            \arrow{r}[above]{\eta_X}
            \arrow{d}[left]{F(f)}
          & G(X)
            \arrow{r}[above]{\zeta_X}
            \arrow{d}[left]{G(f)}
          & H(X)
            \arrow{d}[right]{H(f)}
          \\
            F(Y)
            \arrow{r}[above]{\eta_Y}
            \arrow[dashed, bend right = 45]{rr}[below]{(\zeta \circ \eta)_Y}
          & G(Y)
            \arrow{r}[above]{\zeta_Y}
          & H(Y)
        \end{tikzcd}
      \]
      In this diagram the left and right squares commute because~$\eta$ and~$\zeta$ are natural trasformations, and the upper and lower triangles commute by definition of~$\zeta \circ \eta$.
      It follows that the outer square commutes, which shows that~$\zeta \circ \eta$ is again a natural transformation.
    \item
      For any functor~$F \colon \Ccat \to \Dcat$ its \emph{identical natural transformation}\index{identical natural transformation}\index{natural!transformation!identical}~$\id_F \colon F \to F$ is given by~$(\id_F)_X = \id_{F(X)}$ for every~$X \in \Ob(\Ccat)$.
      This is indeed a natural transformation because the square
      \[
        \begin{tikzcd}
            F(X)
            \arrow{r}[above]{F(f)}
            \arrow{d}[left]{\id_X}
          & F(Y)
            \arrow{d}[right]{\id_Y}
          \\
            F(X)
            \arrow{r}[below]{F(f)}
          & F(Y)
        \end{tikzcd}
      \]
      commutes for every morphism~$f \colon X \to Y$ in~$\Ccat$.
      It holds for every other functor~$G \colon \Ccat \to \Dcat$ that~$\eta \circ {\id_F} = \eta$ for every natural transformation~$\eta \colon F \to G$, and also~${\id_F} \circ \zeta = \zeta$ for every natural transformation~$\zeta \colon G \to F$.
  \end{enumerate}
\end{remark}


\begin{example}
  Consider the two functors
  \[
              G
    \defined  (-)^\times
    \colon    \kAlg
    \to       \Group
    \quad\text{and}\quad
              F
    \defined  k[-]
    \colon    \Group
    \to       \kAlg \,.
  \]
  We examine how the compositions~$F \circ G \colon \Group \to \Group$ and~$G \circ F \colon \kAlg \to \kAlg$ relate to the identity functors~$\Id_{\Group}$ and~$\Id_{\kAlg}$.
  
  Let us first examine the composition~$G \circ F$:
  This functor is on objects given by
  \[
      (G \circ F)(\Gamma)
    = \kf[\Gamma]^\times
  \]
  for every groups~$\Gamma$.
  If~$\varphi \colon \Gamma \to \Delta$ is a homomorphism of groups, then the group homomorphism
  \[
            (G \circ F)(\varphi)
    =       \kf[\varphi]^\times
    \colon  \kf[\Gamma]^\times
    \to     \kf[\Delta]^\times
  \]
  is the restriction of the induced homomorphism of~{\kalgs}~$\kf[\varphi] \colon \kf[\Gamma] \to \kf[\Delta]$ to the groups of units on both sides.
  We observe that we have every group~$\Gamma$ a group homomorphism
  \[
            \eta_\Gamma
    \colon  \Gamma
    \to     \kf[\Gamma]^\times \,,
    \quad   \gamma
    \mapsto [\gamma] \,.
  \]
  The resulting family~$\eta \defined (\eta_X)_{X \in \Ob(\Group)}$ is a natural transformation~$\eta \colon \Id_{\Group} \to G \circ F$, i.e.\ the square
  \[
    \begin{tikzcd}[column sep = large]
        \Gamma
        \arrow{r}[above]{\varphi}
        \arrow{d}[left]{\eta_\Gamma}
      & \Delta
        \arrow{d}[right]{\eta_\Delta}
      \\
        \kf[\Gamma]^\times
        \arrow{r}[above]{\kf[\varphi]^\times}
      & \kf[\Delta]^\times
    \end{tikzcd}
  \]
  commutes for every group homomorphism~$f \colon \Gamma \to \Delta$.
  This holds because
  \[
      \kf[\varphi]^\times( \eta_\Gamma( \gamma ) )
    = \kf[\varphi]^\times( [\gamma] )
    = [\varphi(\gamma)]
    = \eta_\Delta( \varphi( \gamma ) )
  \]
  for every~$\gamma \in \Gamma$.
  
  The functor~$F \circ G$ is given on objects by
  \[
      (F \circ G)(A)
    = \kf[A^\times]
  \]
  for every~{\kalg}~$A$.
  If~$f \colon A \to B$ is a homomorphism of~{\kalgs} then the induced~{\kalg} homomorphism~$(F \circ G)(f)$ is given by
  \[
            (F \circ G)(f)
    =       \kf[f^\times]
    \colon  \kf[A^\times]
    \to     \kf[B^\times] \,,
    \quad   \sum_{a \in A^\times} \lambda_a [a]
    \mapsto \sum_{a \in A^\times} \lambda_a [f(a)] \,.
  \]
  For every~{\kalg}~$A$ the identity~$A^\times \to A^\times$ corresponds (as seen on Exercise~sheet~1) to a homomorphism of~{\kalg}~$\varepsilon_A \colon \kf[A^\times] \to A$, that is given by
  \[
            \varepsilon_A
    \colon  \kf[A^\times]
    \to     A \,,
    \quad   \sum_{a \in A^\times} \lambda_a [a]
    \mapsto \sum_{a \in A^\times} \lambda_a a \,.
  \]
  The resulting family~$\varepsilon \defined (\varepsilon_A)_{A \in \Ob(\kAlg)}$ is a natural transformation~$F \circ G \to \Id_{\kAlg}$, i.e.\ the square
  \[
    \begin{tikzcd}[column sep = large]
        \kf[A^\times]
        \arrow{r}[above]{\kf[f^\times]}
        \arrow{d}[left]{\varepsilon_A}
      & \kf[B^\times]
        \arrow{d}[right]{\varepsilon_B}
      \\
        A
        \arrow{r}[above]{f}
      & B
    \end{tikzcd}
  \]
  commutes for every homomorphism of~{\kalgs}~$f \colon A \to B$.
  This holds because
  \begin{align*}
        \varepsilon_B\left( \kf[f^\times]\left( \sum_{a \in A^\times} \lambda_a [a] \right) \right)
    &=  \varepsilon_B\left( \sum_{a \in A^\times} \lambda_a [f(a)] \right)
     =  \sum_{a \in A^\times} \lambda_a f(a)  \\
    &=  f\left( \sum_{a \in A^\times} \lambda_a a \right)
     =  f\left( \varepsilon_A\left( \sum_{a \in A^\times} \lambda_a [a] \right) \right)
  \end{align*}
  for every~$\sum_{a \in A^\times} \lambda_a [a] \in \kf[A^\times]$.
  
  We will see later on that the functors~$F \colon \Group \to \kAlg$ and~$G \colon \kAlg \to \Group$ are \emph{adjoint} (with~$F$ left adjoint to~$G$), and how this can be expressed via the natural transformations~$\eta \colon \Id_{\Group} \to G \circ F$ and~$\varepsilon \colon F \circ G \to \Id_{\kAlg}$.
\end{example}





\lecturend{5}





\begin{remarkdefinition}
  Let~$F, G \colon \Ccat \to \Dcat$ be two functors.
  A natural transformation~$\eta \colon F \to G$ is a \emph{natural isomorphism}\index{natural!isomorphism}\index{isomorphism!natural} if at every object~$X \in \Ob(\Ccat)$ the morphism~$\eta_X \colon F(X) \to G(X)$ is an isomorphism.
  
  The natural transformation~$\eta$ is a natural isomorphism if and only if there exists a natural transformation~$\zeta \colon G \to F$ with~$\zeta \circ \eta = \id_F$ and~$\eta \circ \zeta = \id_G$:
  
  If such a natural transformation exists then it holds for exery~$X \in \Ob(\Ccat)$ that
  \[
      \zeta_X \circ \eta_X
    = (\zeta \circ \eta)_X
    = (\id_F)_X
    = \id_{F(X)}
  \]
  and similarly~$\eta_X \circ \zeta_X = \id_{G(X)}$.
  This then shows that~$\eta_X$ is for every~$X \in \Ob(\Ccat)$ an isomorphism, with inverse given by~$\eta_X^{-1} = \zeta_X$.
  
  If on the other hand~$\eta_X \colon F(X) \to G(X)$ is an isomorphism for every~$X \in \Ob(\Ccat)$, then it follows for every morphism~$f \colon X \to Y$ in~$\Ccat$ from the commutativity of the square
  \[
    \begin{tikzcd}
        F(X)
        \arrow{r}[above]{F(f)}
        \arrow{d}[left]{\eta_X}
      & F(Y)
        \arrow{d}[right]{\eta_Y}
      \\
        G(X)
        \arrow{r}[above]{G(f)}
      & G(Y)
    \end{tikzcd}
  \]
  that the square
  \[
    \begin{tikzcd}
        F(X)
        \arrow{r}[above]{F(f)}
      & F(Y)
      \\
        G(X)
        \arrow{u}[left]{\eta_X^{-1}}
        \arrow{r}[above]{G(f)}
      & G(Y)
        \arrow{u}[right]{\eta_Y^{-1}}
    \end{tikzcd}
  \]
  also commutes.
  This shows that the family~$\zeta \defined (\eta_X^{-1})_{X \in \Ob(\Ccat)}$ is a natural transformation~$\zeta \colon G \to F$.
  It then holds by construction of~$\zeta$ that~$\zeta \circ \eta = \id_F$ and~$\eta \circ \zeta = \id_G$.
  
  That~$\eta$ is a natural isomorphism is denoted by
  \[
                \eta
    \colon      F
    \xto{\sim}  G  \,.
  \]
  The two functors~$F$ and~$G$ are \emph{isomorphic}\index{isomorphic functors}\index{functor!isomorphic} if there exist a natural isomorphism~$F \xto{\sim} G$.
  That the functors~$F$ and~$G$ are isomorphic is denoted by~$F \cong G$.
\end{remarkdefinition}


\begin{definition}
  Let~$\Ccat$ and~$\Dcat$ be two categories.
  A functor~$F \colon \Ccat \to \Dcat$ is an \emph{equivalence of categories}\index{equivalence of categories}\index{functor!equivalence} if there exists a functor~$G \colon \Dcat \to \Ccat$ with~$G \circ F \cong \Id_{\Ccat}$ and~$F \circ G \cong \Id_{\Dcat}$.
  The categories~$\Ccat$ and~$\Dcat$ are \emph{equivalent}\index{category!equivalent} if there exists an equivalence of categories between them.
  That~$\Ccat$ and~$\Dcat$ are equivalent is denoted by~$\Ccat \simeq \Dcat$.
\end{definition}


\begin{example}
  \label{examples for equivalences}
  \leavevmode
  \begin{enumerate}
    \item
      \label{quiver reps are modules over path algebra}
      Let~$Q$ be a finite quiver.
      \Cref{quiver rep are modules} shows that~$\Rep{\kf}{Q} \simeq \Modl{\kf Q}$, where~$\Rep{\kf}{Q}$ denotes the category of representations of~$Q$ over~$\kf$;
      the only missing ingredient is that the constructed isomorphisms
      \[
              FG(M)
        \cong M
        \quad\text{and}\quad
              GF(X)
        \cong X
      \]
      for~$M \in \Ob(\Modl{\kf Q})$ and~$X \in \Ob(\Rep{\kf}{Q})$ are natural, i.e.\ that for every homomorphism of left~{\modules{$\kf Q$}}~$f \colon M \to N$ the squrae
      \[
        \begin{tikzcd}[column sep = large]
            FG(M)
            \arrow{r}[above]{FG(f)}
            \arrow{d}[left]{\sim}
          & FG(N)
            \arrow{d}[right]{\sim}
          \\
            M
            \arrow{r}[above]{f}
          & N
        \end{tikzcd}
      \]
      commutes, and that for every homomorphism of representations~$f \colon X \to Y$ the square
      \[
        \begin{tikzcd}[column sep = large]
            GF(X)
            \arrow{r}[above]{GF(f)}
            \arrow{d}[left]{\sim}
          & GF(Y)
            \arrow{d}[right]{\sim}
          \\
            X
            \arrow{r}[above]{f}
          & Y
        \end{tikzcd}
      \]
      commutes.
    \item
      Let~$G$ be a group.
      A \emph{representation}\index{representation!of a group}\index{group representation} of~$G$ over~$\kf$ is a pair~$(V,\rho)$ consisting of a~{\module{$\kf$}}~$V$ and a group homomorphism~$\rho \colon G \to \GL(V)$.
      A \emph{homomorphism of representations}\index{homomorphism!of group representations}~$f \colon (V, \rho) \to (W,\sigma)$ is a~{\klin} map~$f \colon V \to W$ such that the square
      \[
        \begin{tikzcd}
            V
            \arrow{r}[above]{\rho(g)}
            \arrow{d}[left]{f}
          & V
            \arrow{d}[right]{f}
          \\
            W
            \arrow{r}[above]{\sigma(g)}
          & W
        \end{tikzcd}
      \]
      commutes for every~$g \in G$.
      It holds for the category~$\Rep{\kf}{G}$ of representations of~$G$ over~$\kf$ that~$\Rep{\kf}{G} \simeq \Modl{\kf[G]}$.
    \item
      Let~$\Ccat$ be the category whose objects are pairs~$(A, \varphi)$ consisting of a ring~$A$ and a ring homomorphism~$\varphi \colon \kf \to \ringcenter(A)$, and where a morphism~$f \colon (A, \varphi) \to (B, \psi)$ is a ring homomorphism~$f \colon A \to B$ that makes the diagram
      \[
        \begin{tikzcd}[column sep = small]
            A
            \arrow{rr}[above]{f}
          & {}
          & B
          \\
            \ringcenter(A)
            \arrow[phantom]{u}[description, rotate=90]{\subseteq}
          & {}
          & \ringcenter(B)
            \arrow[phantom]{u}[description, rotate=90]{\subseteq}
          \\
            {}
          & \kf
            \arrow{ul}
            \arrow{ur}
          & {}
        \end{tikzcd}
      \]
      commute.
      Then \cref{characterization of algebras} shows that the categories~$\kAlg$ and~$\Ccat$ are equivalent.
    \item
      Let~$A$ be a~{\kalg}.
      Let~$\Dcat$ be the category whose objects are pairs~$(V, \varphi)$ consisting of a~{\module{$\kf$}}~$V$ and a homomorphism of~{\kalgs}~$\varphi \colon A \to \End_\kf(V)$, and where a morphism~$f \colon (V,\varphi) \to (W,\psi)$ is a~{\klin} map~$f \colon V \to W$ that makes the squares
      \[
        \begin{tikzcd}
            V
            \arrow{r}[above]{\varphi(a)}
            \arrow{d}[left]{f}
          & V
            \arrow{d}[right]{f}
          \\
            W
            \arrow{r}[above]{\psi(a)}
          & W
        \end{tikzcd}
      \]
      commute for every~$a \in A$.
      \Cref{modules as homomorphisms into endomorphisms} shows that the categories~$\Modl{A}$ and~$\Dcat$ are equivalent.
  \end{enumerate}
\end{example}


% TODO: Equivalence of categories is an equivalence relation.
%       Introduce whiskering earlier to prove this.




