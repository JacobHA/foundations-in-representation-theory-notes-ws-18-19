\section{Representable Functors}


\begin{lemma}[Yoneda’s lemma]\index{Yoneda’s lemma}
  \label{yoneda lemma}
  Let~$\Ccat$ be a category and let~$X \in \Ob(\Ccat) = \Ob(\Ccat^\op)$.
  \begin{enumerate}
    \item
      \label{covariant yoneda}
      Let~$F \colon \Ccat \to \Set$ be a (covariant) functor.
      Then the map
      \[
                Y^{F,X}
        \colon  \Fun(\Ccat, \Set)(h^X, F)
        \to     F(X) \,,
        \quad   ( \eta \colon h^X \to F)
        \mapsto \eta_X(\id_X)
      \]
      is a bijection.
    \item
      \label{contravariant yoneda}
      Let~$G \colon \Ccat^\op \to \Set$ be a (contravariant) functor.
      Then the map
      \[
                Y_{G,X}
        \colon  \Fun(\Ccat, \Set)(h_X, G)
        \to     G(X) \,,
        \quad   ( \eta \colon h_X \to G)
        \mapsto \eta_X(\id_X)
      \]
      is a bijection.
  \end{enumerate}
\end{lemma}


\begin{proof}
  Part~\ref*{covariant yoneda} follows from part~\ref*{contravariant yoneda} because~$\Ccat = (\Ccat^\op)^\op$ and
  \[
      h^X_{(\Ccat)}
    = \Ccat(X,-)
    = \Ccat^\op(-,X)
    = h_X^{(\Ccat^\op)}
  \]
  for every~$X \in \Ob(\Ccat) = \Ob(\Ccat^\op)$.
  We therefore only prove part~\ref*{contravariant yoneda}.
  
  To show that the map~$Y_{G,X}$ is injective we need to show that a natural transformation~$\eta \colon h_X \to G$ is uniquely determined by its value~$\eta_X(\id_X)$.
  This holds because it follows for every~$Y \in \Ob(\Ccat)$ and every~$f \in h_X(Y) = \Ccat(Y,X)$ from the commutativity of the diagram
  \[
    \begin{tikzcd}
        h_X(X)
        \arrow{r}[above]{f^*}
        \arrow{d}[left]{\eta_X}
      & h_X(Y)
        \arrow{d}[right]{\eta_Y}
      \\
        G(X)
        \arrow{r}[above]{G(f)}
      & G(Y)
    \end{tikzcd}
  \]
  that
  \[
      \eta_Y(f)
    = \eta_Y(\id_X \circ f)
    = \eta_Y(f^*(\id_X))
    = G(f)(\eta_X(\id_X)) \,.
  \]
  
  To show the surjectivity of~$Y_{G,X}$ let~$z \in G(X)$.
  For every~$Y \in \Ob(\Ccat)$ let
  \[
            \zeta_Y
    \colon  h_X(Y)
    =       \Ccat(Y,X)
    \to     G(Y) \,,
    \quad   f
    \mapsto G(f)(z) \,.
  \]
  It then holds for every morphism~$g \colon Y \to Y'$ in~$\Ccat$ that the diagram
  \[
    \begin{tikzcd}
        h_X(Y')
        \arrow{r}[above]{g^*}
        \arrow{d}[left]{\zeta_{Y'}}
      & h_X(Y)
        \arrow{d}[right]{\zeta_Y}
      \\
        G(Y')
        \arrow{r}[above]{G(g)}
      & G(Y)
    \end{tikzcd}
  \]
  commutes, because
  \[
      G(g)( \zeta_{Y'}( f ) )
    = G(g)( G(f)(z) )
    = G(f \circ g)(z)
    = \zeta_Y(f \circ g)
    = \zeta_Y( g^*(f) )
  \]
  for every~$f \in h_X(Y')$.
  This shows that~$\zeta \defined (\zeta_X)_{X \in \Ob(\Ccat)}$ is a natural transformation~$\zeta \colon h_X \to F$.
  It holds that
  \[
      Y_{G,X}(\zeta)
    = \zeta_X(\id_X)
    = G(\id_X)(z)
    = \id_{G(X)}(z)
    = z \,,
  \]
  which altogether shows that~$Y_{G,X}$ is surjective.
\end{proof}


\begin{remark}
  The above proof shows that the inverse of the map~$Y_{G,X}$ is given by
  \begin{align*}
              Y_{G,X}^{-1}
     \colon   G(X)
    &\to      \Fun(\Ccat, \Set) \,,
    \\
              z
    &\mapsto  \bigl( \eta_Y \colon h_X(Y) \to G(Y), f \mapsto G(f)(z) \bigr)_{Y \in \Ob(\Ccat)} \,.
  \end{align*}
  The inverse of the map~$Y^{G,X}$ is similarly given by
  \begin{align*}
              (Y^{G,X})^{-1}
     \colon   G(Y)
    &\to      \Fun(\Ccat^\op, \Set) \,,
    \\
              z
    &\mapsto  \bigl( \eta_Y \colon h^X(Y) \to G(Y), f \mapsto G(f)(z) \bigr)_{Y \in \Ob(\Ccat^\op)} \,.
  \end{align*}
\end{remark}





\lecturend{6}


\begin{theorem}[Yoneda embedding]
  \index{Yoneda embedding}
  Let~$\Ccat$ be a category.
  \begin{enumerate}
    \item
      The functor~$h^{(-)} \colon \Ccat^\op \to \Fun(\Ccat, \Set)$ is fully faithful.
    \item
      \label{contravariant yoneda embedding}
      The functor~$h_{(-)} \colon \Ccat \to \Fun(\Ccat^\op, \Set)$ is fully faithful.
  \end{enumerate}
\end{theorem}


\begin{proof}
  It again sufficies to show part~\ref*{contravariant yoneda embedding}.
  
  We need to show that for all objects~$X, Y \in \Ob(\Ccat)$ the map
  \[
            \Phi
    \colon  \Ccat(X,Y)
    \to     \Fun(\Ccat^\op, \Set)(h_X, h_Y) \,,
    \quad   f
    \mapsto f_*
  \]
  is a bijection.
  We do so by exhibiting an inverse for~$\Phi$.
  For this we apply Yoneda’s~lemma to the functors~$h_X, h_Y \colon \Ccat^\op \to \Set$ to find that the map
  \begin{align*}
              \Psi
     \colon   \Fun(\Ccat^\op, \Set)(h_X, h_Y)
    &\longto  h_Y(X)
     =        \Ccat(X,Y)  \,,
    \\
                  \zeta
    &\longmapsto  \zeta_X(\id_X)
  \end{align*}
  is a bijection.
  We claim that this is the required inverse to~$\Phi$.
  Indeed, we have that
  \[
      \Psi(\Phi(f))
    = (\Phi(f))_X(\id_X)
    = f_*(\id_X)
    = f \circ \id_X
    = f
  \]
  for all~$f \in \Ccat(X,Y)$, and hence~$\Psi \circ \Phi = \id$.
  This shows that~$\Phi$ is a right inverse to~$\Psi$, and hence the ({\twosided}) inverse of~$\Psi$ (because~$\Psi$ is a bijection).
%   \begin{align}
%         \Phi(\Psi(\zeta))_Z(f)  \notag
%     &=  \Psi(\zeta)_*(f)  \notag  \\
%     &=  ( \zeta_X(\id_X) )_*(f) \notag  \\
%     &=  \zeta_X(\id_X) \circ f  \notag  \\
%     &=  f^*(\zeta_X(\id_X)) \notag  \\
%     &=  \zeta_Z(f^*(\id_X)) \label{naturality of zeta}  \\
%     &=  \zeta_Z(\id_X \circ f)  \notag  \\
%     &=  \zeta_Z(f)  \notag
%   \end{align}
%   for every~$Z \in \Ob(\Ccat)$ and every~$f \in h_X(Z) = \Ccat(Z,X)$.
%   Here we have used for step~\eqref{naturality of zeta} the commutativity of the following diagram:
%   \[
%     \begin{tikzcd}[row sep = large]
%           h_X(X)
%           \arrow{rrr}[above]{f^*}
%           \arrow{ddd}[left]{\zeta_X}
%           \arrow[equal]{dr}
%         & {}
%         & {}
%         & h_X(Z)
%           \arrow{ddd}[right]{\zeta_Z}
%           \arrow[equal]{dl}
%       \\
%           {}
%         & \Ccat(X,X)
%           \arrow{r}[above]{f^*}
%           \arrow{d}[left]{\zeta_X}
%         & \Ccat(Z,X)
%           \arrow{d}[right]{\zeta_Z}
%         & {}
%       \\
%           {}
%         & \Ccat(X,Y)
%           \arrow{r}[below]{f^*}
%         & \Ccat(Z,Y)
%         & {}
%       \\
%           h_Y(X)
%           \arrow{rrr}[below]{f^*}
%           \arrow[equal]{ur}
%         & {}
%         & {}
%         & h_Y(Z)
%           \arrow[equal]{ul}
%     \end{tikzcd}
%   \]
%   This shows that also~$\Phi \circ \Psi = \id$.
\end{proof}


\begin{definition}
  Let~$\Ccat$ be a category.
  \begin{enumerate}
    \item
      A (covariant) functor~$F \colon \Ccat \to \Set$ is \emph{representable}\index{representable functor}\index{functor!representable} if it is naturally isomorphic to a functor~$h^X \colon \Ccat \to \Set$ for some object~$X \in \Ob(\Ccat)$.
    \item
      A (contravariant) functor~$F \colon \Ccat^\op \to \Set$ is \emph{representable}\index{representable functor}\index{functor!representable} if it is naturally isomorphic to a functor~$h_X \colon \Ccat^\op \to \Set$ for some object~$X \in \Ob(\Ccat)$.
  \end{enumerate}
  The object~$X$ is then a \emph{representing object}\index{representing object}\index{object!representing} for~$F$.
\end{definition}


\begin{remark*}
  If a functor~$F \colon \Ccat \to \Set$ or~$F \colon \Ccat^\op \to \Set$ admits a representing object~$X \in \Ob(\Ccat)$, then~$X$ is unique up to unique isomorphism.
% TODO: Add an explanation to what this means.
\end{remark*}




