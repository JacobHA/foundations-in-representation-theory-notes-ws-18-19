\section{Isomorphisms}


\begin{definition}
  A morphism~$f \colon X \to Y$ in a category~$\Ccat$ is an \emph{isomorphism}\index{isomorphism!in a category} if there exist a morphism~$g \colon Y \to X$ with~$g \circ f = \id_X$ and~$f \circ g = \id_Y$.
\end{definition}


\begin{definition}[continues=properties of functors]
  \leavevmode
  \begin{enumerate}[start=4]
    \item
      The functor~$F$ is \emph{dense}\index{dense}\index{functor!dense} or \emph{essentially surjective}\index{essentially surjective}\index{functor!essentially surjective} if there exist for every object~$Y \in \Ob(\Dcat)$ an object~$X \in \Ob(\Ccat)$ with~$Y \cong F(X)$.
  \end{enumerate}
\end{definition}


\begin{remark}
  Let~$\Ccat$ and~$\Dcat$ be two categories.
  \begin{enumerate}
    \item
      If~$f \colon X \to Y$ is an isomorphism in~$\Ccat$ then the morphism~$g \colon Y \to X$ with~$g \circ f = \id_X$ and~$f \circ g = \id_Y$ is uniquely determined.
      Indeed, if~$g'$ is another such morphism then
      \[
        g
        =
        g \circ {\id_Y}
        =
        g \circ f \circ g'
        =
        {\id_X} \circ g'
        =
        g'  \,.
      \]
      The morphisms~$g$ is the \emph{inverse}\index{inverse}\index{morphism!inverse} of~$f$, and is denoted by~$f^{-1}$.
    \item
      For every~$X \in \Ob(\Ccat)$ its identity~$\id_X \colon X \to X$ is an isomorphism.
    \item
      If~$F \colon \Ccat \to \Dcat$ is a functor and~$f$ is an isomorphism in~$\Ccat$ then~$F(f)$ is an isomorphism in~$\Dcat$, and it holds that~$F(f)^{-1} = F(f^{-1})$.
  \end{enumerate}
\end{remark}


\begin{remark*}
  Let~$f \colon X \to Y$ and~$g \colon Y \to Z$ be two composable morphisms in a category~$\Ccat$.
  If both~$f$ and~$g$ are isomorphisms then their composition~$g \circ f$ is again an isomorphism.
  It then holds that
  \[
    (g \circ f)^{-1}
    =
    f^{-1} \circ g^{-1} \,.
  \]
\end{remark*}


\begin{example}
  \leavevmode
  \begin{enumerate}
    \item
      In the categories~$\Set$,~$\Group$,~$\Modl{A}$, \dots, a morphism~$f$ is an isomorphism if and only if it is bijective (as a \dash{set}{theoretic} map).
    \item
      In the category~$\Top$ the isomorphisms are precisely the homeomorphisms, i.e.\ the continuous maps that are both bijective and open.
    \item
      In the path category~$\Path(Q)$ of a quiver~$Q$ the isomorphisms are precisely the lazy paths~$\varepsilon_i = \id_i$ for~$i \in Q_0 = \Ob(\Path(Q))$.
  \end{enumerate}
\end{example}




