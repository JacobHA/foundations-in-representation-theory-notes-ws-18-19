\chapter{Categories and Functors}





% need to draw the cat myself
% \begin{tikzpicture}[remember picture,overlay]
%   \node[anchor=east,inner sep=0pt]
%   at (current page text area.east|-0,2cm)
%   {\includegraphics[scale=0.15]{cat.pdf}};
% \end{tikzpicture}
% \vspace{-0.7cm}





\section{Categories}

\begin{definition}
  A \emph{category}~$\Ccat$\index{category} consists of the following data:
  \begin{itemize}
    \item
      A class~$\Ob(\Ccat)$.
      The elements are the \emph{objects}\index{objects of a category} of~$\Ccat$.
    \item
      For any two objects~$X, Y \in \Ob(\Ccat)$ a set~$\Ccat(X,Y)$.% dont let footnote see this line break
      \footnote{Other common notations are~$\Hom_{\Ccat}(X,Y)$ or~$\operatorname{Mor}_{\Ccat}(X,Y)$.}
      The elements of~$\Ccat(X, Y)$ are the \emph{morphisms}\index{morphism!in a category} from~$X$ to~$Y$.
      That~$f \in \Ccat(X,Y)$ is denoted by~$f \colon X \to Y$ or~$X \xto{f} Y$.
    \item
      For any three objects~$X, Y, Z \in \Ob(\Ccat)$ a map
      \[
                \Ccat(Y,Z) \times \Ccat(X,Y)
        \to     \Ccat(X,Z) \,,
        \quad   (g, f)
        \mapsto g \circ f \,.
      \]
      For any two morphisms~$f \colon X \to Y$ and~$g \colon Y \to Z$ the morphism~$g \circ f \colon X \to Z$ is the \emph{composition} of~$g$ and~$f$.
  \end{itemize}
  These data are subject to the following conditions:
  \begin{enumerate}[label=(C\arabic*)]
    \item
      The composition of morphisms is associative\index{associativity}:
      For all objects~$X, Y, Z, W \in \Ob(\Ccat)$ and morphisms~$f \colon X \to Y$,~$g \colon Y \to Z$ and~$h \colon Z \to W$ it holds that
      \[
          (h \circ g) \circ f
        = h \circ (g \circ f) \,.
      \]
    \item
      There exists for every object~$X \in \Ob(\Ccat)$ an \emph{identity morphism}\index{identity!morphism}~$\id_X \colon X \to X$ such that
      \[
        f \circ \id_X = f
        \quad\text{and}\quad
        \id_X \circ g = g
      \]
      for all morphisms~$f \colon X \to Y$ and~$g \colon Y \to X$ in~$\Ccat$.
  \end{enumerate}
\end{definition}


\begin{remark}
  Let~$\Ccat$ be a category.
  \begin{enumerate}
    \item
      It could happen for objects~$X, Y \in \Ccat$ that~$\Ccat(X,Y) = \emptyset$, i.e.\ that there exists no morphism from~$X$ to~$Y$ in~$\Ccat$.
    \item
      For every object~$X \in \Ccat$ the identity morphism~$\id_X$ is unique.
      If~$\id'_X$ is another identity morphism of~$X$ then
      \[
          \id_X
        = \id_X \id'_X
        = \id'_X \,.
      \]
  \end{enumerate}
\end{remark}


\begin{remark}
  We sometimes want to consider categories whose objects are all sets (which satisfy certain conditions).
  This can lead to set theoretic problems (also known as \emph{set theoretic difficulties}\index{set theoretic difficulties}).
  One way out of this predicament are \emph{universes}\index{universe}.
  (See \cite[I.6]{Working} and \cite[3.2]{Schubert} for more details on this.)
  We will always fix a universe~$U$ and say that
  \begin{itemize}
    \item
      $X$ is a set if~$X \in U$, and that
    \item
      $X$ is a class if~$X \subseteq U$.
  \end{itemize}
\end{remark}





\lecturend{4}




\begin{example}
  \leavevmode
  \begin{enumerate}
    \item
      The category~$\Set$\index{category!of sets} of sets:
      The objects of~$\Set$ are given by
      \[
          \Ob(\Set)
        = \{
            \text{sets (which are elements of the fixed universe)}
          \} \,,
      \]
      and for any two sets~$X$ and~$Y$ the morphism set~$\Set(X,Y)$ is given by
      \[
          \Set(X,Y)
        = \{
            \text{maps~$f \colon X \to Y$}
          \} \,.
      \]
      The composition of morphisms in~$\Set$ is the usual composition of maps.
    \item
      The category~$\Group$\index{category!of groups} of groups:
      The objects of~$\Group$ are given by
      \[
          \Ob(\Group)
        = \{
            \text{groups}
          \} \,,
      \]
      and for any two groups~$G$ and~$H$ the morphism set~$\Group(G,H)$ is given by
      \[
          \Group(G,H)
        = \{
            \text{group homomorphisms~$f \colon G \to H$}
          \} \,.
      \]
      The composition of morphisms in~$\Group$ is the usual composition of group homomorphisms.
    \item
      The category~$\kAlg$\index{category!of $k$-algebras} of~{\kalgs}:%
      \footnote{This example was not given in the lecture, but will be used \cref{examples for functors}.}
      The objects of~$\kAlg$ are given by
      \[
          \Ob(\kAlg)
        = \{
            \text{{\kalgs}}
          \} \,,
      \]
      and for any two {\kalgs}~$A$ and~$B$ the morphism set~$(\kAlg)(A,B)$ is given by
      \[
          \kAlg(A,B)
        = \{
            \text{{\kalg} homomorphisms~$f \colon A \to B$}
          \} \,.
      \]
      The composition of morphisms in~$\kAlg$ is the usual composition of~{\kalg} homomorphisms.
    \item
      For a~{\kalg}~$A$ the category~$\Modl{A}$\index{category!of $A$-modules} of left~{\modules{$A$}}:
      The objects of~$\Modl{A}$ are given by
      \[
          \Ob(\Modl{A})
        = \{
            \text{left~{\modules{$A$}}}
          \} \,,
      \]
      and for any two left~{\modules{$A$}}~$M$ and~$N$ the morphism set~$(\Modl{A})(M,N)$ is given by
      \[
          (\Modl{A})(M,N)
        = \left\{
            \text{homomorphisms of left~{\modules{$A$}}~$f \colon M \to N$}
          \right\} \,.
      \]
      The composition of morphisms in~$\Modl{A}$ is the usual composition of module homomorphisms.
      
      The category~$\Modr{A}$ of right~{\modules{$A$}} is defined analogous.
    \item
      The category~$\Top$\index{category!of topological spaces} of topological spaces:
      The objects of~$\Top$ are given by
      \[
          \Ob(\Top)
        = \{
            \text{topological spaces}
          \} \,,
      \]
      and for any two topological spaces~$X$ and~$Y$ the morphism set~$\Top(X,Y)$ is given by
      \[
          \Top(X,Y)
        = \{
            \text{continuous maps~$f \colon X \to Y$}
          \} \,.
      \]
      The composition of morphisms in~$\Top$ is the usual composition of continuous maps.
    \item
      Let~$G$ be a group.
      We can then define a category~$\Gcat$ which consists of a single object~$\Ob(\Gcat) = \{ \ast \}$, the single morphism set~$\Gcat(\ast,\ast) = G$, and for which the composition of morphisms is the multiplication of~$G$, i.e.\ the composition is given as
      \[
                  g \circ h
        \defined  gh
      \]
      for all~$g, h \in \Gcat(\ast,\ast) = G$.
      One may pictorially represent the category~$\Gcat$ as follows:
      \[
        \begin{tikzcd}
          \ast
          \arrow[out=30,in=110,loop]{}[above right, near start]{G}
          \arrow[out=150,in=230,loop,looseness=6.5]
          \arrow[out=270,in=350,loop]
        \end{tikzcd}
      \]
    \item
      Let~$Q$ be a quiver.
      Its \emph{category of paths}\index{category!of paths}~$\Path(Q)$, is defined as follows:%
      \footnote{In the lecture, the notation~$Q_*$ is used for the category of paths in~$Q$.}
      The objects of~$\Path(Q)$ are the vertices of~$Q$, hence
      \[
          \Ob(\Path(Q))
        = Q_0 \,,
      \]
      and for any two vertices~$i$ and~$j$ in~$Q$ the morphism set~$\Path(Q)(i,j)$ is given by
      \[
          \Path(Q)(i,j)
        = \{
            \text{paths from~$i$ to~$j$ in~$Q$}
          \} \,.
      \]
      The composition of morphisms in~$\Path(Q)$ is given by concatenation of paths.
      This gives a category because concatenation of paths is associative, and for every~$i \in Q_0$ the lazy path~$\varepsilon_i \in \Path(Q)(i,i)$ acts as the identity of the object~$i$.
  \end{enumerate}
\end{example}


\begin{definition}
  Let~$\Ccat$ be a category.
  The \emph{opposite category}\index{opposite!category}~$\Ccat^\op$ results from~$\Ccat$ by formally reversing the direction of all morphisms:
  The objects of~$\Ccat^\op$ are given by
  \[
      \Ob(\Ccat^\op)
    = \Ob(\Ccat) \,,
  \]
  for any two objects~$X, Y \in \Ob(\Ccat^\op)$ the morphisms set~$\Ccat^\op(X,Y)$ is given by
  \[
      \Ccat^\op(X,Y)
    = \Ccat(Y,X) \,.
  \]
  The composition of any two morphisms~$f \colon X \to Y$ and~$g \colon Y \to Z$ in~$\Ccat^\op$ is given by
  \[
      g \circ_{\Ccat^\op} f
    = f \circ_{\Ccat} g \,.
  \]
\end{definition}





\section{Functors}


\begin{definition}
  Let~$\Ccat$ and~$\Dcat$ be categories.
  A \emph{functor}\index{functor}~$F \colon \Ccat \to \Dcat$ consists of the following data:
  \begin{itemize}
    \item
      A map (of classes)~$F \colon \Ob(\Ccat) \to \Ob(\Dcat)$, $X \mapsto F(X)$.
    \item
      For any two objects~$X$ and~$Y$ of $\Ccat$ a map
      \[
                \Ccat(X, Y)
        \to     \Dcat(F(X), F(Y)) \,,
        \quad   f
        \mapsto F(f) \,.
      \]
  \end{itemize}
  These data are subject to the following conditions:
  \begin{enumerate}[label=(F\arabic*)]
    \item
      It holds for every object~$X$ of~$\Ccat$ that~$F(\id_X) = \id_{F(X)}$.
    \item
      It holds for any two composable morphisms~$f \colon X \to Y$ and~$g \colon Y \to Z$ in~$\Ccat$ that
      \[
          F(g \circ f)
        = F(g) \circ F(f) \,.
      \]
  \end{enumerate}
\end{definition}


\begin{remark}
  What we call a \enquote{functor} is sometimes called a \enquote{covariant functor}\index{functor!covariant}\index{covariant functor|see {functor}}.
  A \emph{contravariant functor}~$G \colon \Ccat \to \Dcat$\index{functor!contravariant}\index{contravariant functor|see {functor}} is a (covariant) functor~$G \colon \Ccat^\op \to \Dcat$.
  This means in terms of the category~$\Ccat$ that~$G$ assigns to every object~$X \in \Ob(\Ccat)$ an object~$G(X) \in \Dcat$ and to every morphism~$f \colon X \to Y$ in~$\Ccat$ a morphism~$G(f) \colon G(Y) \to G(X)$ in~$\Dcat$, in such a way that
  \begin{itemize}
    \item
      $G(\id_X) = \id_{G(X)}$ for every~$X \in \Ccat$, and
    \item
      $G(g \circ f) = G(f) \circ G(g)$ for every pair of composable morphisms~$f \colon X \to Y$ and~$g \colon Y \to Z$ in~$\Ccat$.
  \end{itemize}
\end{remark}


\begin{remark}
  \leavevmode
  \begin{enumerate}
    \item
      Let~$\Ccat$ be a category.
      The \emph{identity functor}~$\Id_{\Ccat} \colon \Ccat \to \Ccat$\index{identity!functor} is given by
      \[
                \Id_\Ccat
        \colon  \left\{
                  \begin{aligned}
                    X &\mapsto  X \,, \\
                    f &\mapsto  f \,.
                  \end{aligned}
                \right.
      \]
    \item
      If~$\Ccat$,~$\Dcat$ and~$\Ecat$ are categories and~$F \colon \Ccat \to \Dcat$ and~$G \colon \Dcat \to \Ecat$ are functors, then they can be composed to a functor~$G \circ F \colon \Ccat \to \Ecat$ given by
      \[
                (F \circ G)
        \colon  \left\{
                  \begin{aligned}
                    X &\mapsto  G(F(X)) \,, \\
                    f &\mapsto  G(F(f)) \,.
                  \end{aligned}
                \right.
      \]
  \end{enumerate}
\end{remark}


\begin{example}
  \label{examples for functors}
  \leavevmode
  \begin{enumerate}
    \item
      We can define two functors which assign to each set~$X$ its power set:
      
      We define a functor~$P_* \colon \Set \to \Set$ which assigns to each set~$X$ its power set~$\power(X)$, and to each map~$f \colon X \to Y$ the induced map
      \[
                P_*(f)
        \colon  \power(X)
        \to     \power(Y) \,,
        \quad   A
        \mapsto f(A) \,.
      \]
      We can also define a (contravariant) functor~$P^* \colon \Set^\op \to \Set$ which again assigns to each set~$X$ its power set~$\power(X)$, but to each map~$f \colon X \to Y$ the induced map
      \[
                P^*(f)
        \colon  \power(Y)
        \to     \power(X) \,,
        \quad   B
        \mapsto f^{-1}(B) \,.
      \]
    \item
      We have two functors between the categories~$\kAlg$ and~$\Group$:
      
      In the one direction there exists a functor~$(-)^\times \colon \kAlg \to \Group$ which assigns to each~{\kalg}~$A$ its group of units~$A^\times$ and to every homomorphism of~{\kalgs}~$f \colon A \to B$ its induces group homomorphism~$f^\times \colon A^\times \to B^\times$.
      
      In the other direction there exists a functor~$\kf[-] \colon \Group \to \kAlg$ which assigns to each group~$G$ its group algebra~$\kf[G]$, and to each group homomorphism~$\varphi \colon G \to H$ the induced homorphism of~{\kalgs}~$\kf[\varphi] \colon \kf[G] \to \kf[H]$, i.e.\ the unique homomorphism of~{\kalgs}~$\kf[G] \to \kf[H]$ which makes the diagram
      \[
        \begin{tikzcd}
            G
            \arrow{r}[above]{\varphi}
            \arrow[hook]{d}
          & H
            \arrow[hook]{d}
          \\
            \kf[G]
            \arrow{r}[above]{\kf[\varphi]}
          & \kf[H]
        \end{tikzcd}
      \]
      commute.
      (Recall from the first exercise sheet that for every~{\kalg}~$A$, every group homomorphism~$G \to A^\times$ extends uniquely to a homomorphism of~{\kalgs}~$\kf[G] \to A$.
      By applying this to the composition~$G \to H \inclusion k[H]^\times$ it follows that there exists a unique homomorphism~$\kf[G] \to \kf[H]$ which makes the above diagram commute.)
    \item
      The functor
      \[
                V
        \colon  \Group
        \to     \Set \,,
        \quad   \left\{
                  \begin{aligned}
                              G
                    &\mapsto  (\text{$G$ as a set}) \,,
                    \\
                              f
                    &\mapsto  f
                  \end{aligned}
                \right.
      \]
      is the \emph{forgetful functor}\index{forgetful functor}\index{functor!forgetful}.
      More generally, we call every functor \emph{forgetful} if it forgets part of the structure.
      We have for example forgetful functors~$\Top \to \Set$ and~$\Modl{A} \to \Modl{\kf}$, where~$A$ is a~{\kalg}.
  \end{enumerate}
\end{example}


\begin{example}
  Let~$\Ccat$ be a category.
  \begin{enumerate}
    \item
      Every object~$X \in \Ob(\Ccat)$ gives rise to a functor
      \[
                h^X
        \colon  \Ccat
        \to     \Set \,,
        \quad   \left\{
                  \begin{aligned}
                              Y
                    &\mapsto  \Ccat(X,Y) \,,
                    \\
                              \left( Y \xlongto{f} Y' \right)
                    &\mapsto  \left( \Ccat(X,Y) \xlongto{f_*} \Ccat(X,Y') \right) \,,
                  \end{aligned}
                \right.
      \]
      where the induced map~$f_* \colon \Ccat(X,Y) \to \Ccat(X,Y')$ is given by~$f_*(g) = f \circ g$ for every~$g \in \Ccat(X,Y)$.
      The functor~$h^X$ is also denoted by~$\Ccat(X,-)$.
    \item
      Every object~$X \in \Ob(\Ccat)$ gives rise to a (contravariant) functor
      \[
                h_X
        \colon  \Ccat^\op
        \to     \Set \,,
        \quad   \left\{
                  \begin{aligned}
                              Y
                    &\mapsto  \Ccat(Y,X) \,,
                    \\
                              \left( Y \xlongto{f} Y' \right)
                    &\mapsto  \left( \Ccat(Y',X) \xlongto{f^*} \Ccat(Y,X) \right) \,,
                  \end{aligned}
                \right.
      \]
      where the induced map~$f^* \colon \Ccat(Y',X) \to \Ccat(Y,X)$ is given by~$f^*(g) = g \circ f$ for every~$g \in \Ccat(Y',X)$.
      The functor~$h_X$ is also denoted by~$\Ccat(-,X)$.
  \end{enumerate}
\end{example}


\begin{definition}[label=properties of functors]
  Let~$F \colon \Ccat \to \Dcat$ be a functor.
  \begin{enumerate}
    \item
      The functor~$F$ is \emph{faithful}\index{faithful}\index{functor!faithful} if the induced map~$\Ccat(X,Y) \to \Dcat(F(X),F(Y))$,~$f \mapsto F(f)$ is injective for all~$X, Y \in \Ccat$.
    \item
      The functor~$F$ is \emph{full}\index{full}\index{functor!full} if the induced map~$\Ccat(X,Y) \to \Dcat(F(X),F(Y))$,~$f \mapsto F(f)$ is surjective for all~$X, Y \in \Ccat$.
    \item
      The functor~$F$ is \emph{fully faithful}\index{fully faithful}\index{functor!fully faithful} if it is both full and faithful, i.e.\ if the induced map~$\Ccat(X,Y) \to \Dcat(F(X),F(Y))$,~$f \mapsto F(f)$ is injective for all~$X, Y \in \Ccat$.
  \end{enumerate}
\end{definition}





\section{Isomorphisms}


\begin{definition}
  A morphism~$f \colon X \to Y$ in a category~$\Ccat$ is an \emph{isomorphism}\index{isomorphism!in a category} if there exist a morphism~$g \colon Y \to X$ with~$g \circ f = \id_X$ and~$f \circ g = \id_Y$.
\end{definition}


\begin{definition}[continues=properties of functors]
  \leavevmode
  \begin{enumerate}[start=4]
    \item
      The functor~$F$ is \emph{dense}\index{dense}\index{functor!dense} or \emph{essentially surjective}\index{essentially surjective}\index{functor!essentially surjective} if there exist for every object~$Y \in \Ob(\Dcat)$ some object~$X \in \Ob(\Ccat)$ with~$Y \cong F(X)$.
  \end{enumerate}
\end{definition}


\begin{remark}
  Let~$\Ccat$ and~$\Dcat$ be categories.
  \begin{enumerate}
    \item
      If~$f \colon X \to Y$ is an isomorphism in~$\Ccat$ the the morphism~$g \colon Y \to X$ with~$g \circ f = \id_X$ and~$f \circ g = \id_Y$ is uniquely determined.
      The morphisms~$g$ is the \emph{inverse}\index{inverse}\index{morphism!inverse} of~$f$, and is denoted by~$f^{-1}$.
    \item
      For every~$X \in \Ob(\Ccat)$ its identity~$\id_X \colon X \to X$ is an isomorphism.
    \item
      If~$F \colon \Ccat \to \Dcat$ is a functor and~$f$ is an isomorphism in~$\Ccat$ then~$F(f)$ is an isomorphism in~$\Dcat$, and it holds that~$F(f)^{-1} = F(f^{-1})$.
  \end{enumerate}
\end{remark}


\begin{example}
  \leavevmode
  \begin{enumerate}
    \item
      In the categories~$\Set$,~$\Group$,~$\Modl{A}$, \dots, a morphism~$f$ is an isomorphism if and only if it is bijective (as a \dash{set}{theoretic} map).
    \item
      In the category~$\Top$, the isomorphisms are precisely the homeomorphisms, i.e.\ the continuous maps which are both bijective and open.
    \item
      In the path category~$\Path(Q)$ of a quiver~$Q$ the isomorphisms are precisely the lazy paths~$\varepsilon_i = \id_i$ for~$i \in Q_0 = \Ob(\Path(Q))$.
  \end{enumerate}
\end{example}





\section{Natural Transformations}


\begin{definition}
  Let~$\Ccat$,~$\Dcat$ be two categories and let~$F, G \colon \Ccat \to \Dcat$ be two functors between them.
  A \emph{natural transformation}\index{natural!transformation}~$\eta \colon F \to G$ is a family~$\eta = (\eta_X)_{X \in \Ob(\Ccat)}$ of morphisms~$\eta_X \colon F(X) \to G(X)$ such that for every morphism~$f \colon X \to Y$ in~$\Ccat$ the following diagram (in~$\Dcat$) commutes:
  \[
    \begin{tikzcd}
        F(X)
        \arrow{r}[above]{\eta_X}
        \arrow{d}[left]{F(f)}
      & G(X)
        \arrow{d}[right]{G(f)}
      \\
        F(Y)
        \arrow{r}[above]{\eta_Y}
      & G(Y)
    \end{tikzcd}
  \]
\end{definition}


\begin{remark}
  Let~$\Ccat$ and~$\Dcat$ be categories.
  \begin{enumerate}
    \item
      Let~$F, G, H \colon \Ccat \to \Dcat$ be functors and let~$\eta \colon F \to G$ and~$\zeta \colon G \to H$ be natural transformations.
      Their composition~$\zeta \circ \eta$ is the natural transformation~$F \to H$ which is given by~$(\zeta \circ \eta)_X \defined \zeta_X \circ \eta_X \colon F(X) \to H(X)$ for every~$X \in \Ob(\Ccat)$.
      To see that this is indeed a natural transfomation we consider the following diagram:
      \[
        \begin{tikzcd}
            F(X)
            \arrow[bend left]{rr}[above]{(\zeta \circ \eta)_X}
            \arrow{r}[above]{\eta_X}
            \arrow{d}[left]{F(f)}
          & G(X)
            \arrow{r}[above]{\zeta_X}
            \arrow{d}[left]{G(f)}
          & H(X)
            \arrow{d}[right]{H(f)}
          \\
            F(Y)
            \arrow{r}[above]{\eta_Y}
            \arrow[bend right]{rr}[below]{(\zeta \circ \eta)_Y}
          & G(Y)
            \arrow{r}[above]{\zeta_Y}
          & H(Y)
        \end{tikzcd}
      \]
      In this diagram the left and right squares commute because~$\eta$ and~$\zeta$ are natural trasformations, and the upper and lower triangles commute by definition of~$\zeta \circ \eta$.
      It follows that the outer square commutes, which shows that~$\zeta \circ \eta$ is again a natural transformation.
    \item
      For any functor~$F \colon \Ccat \to \Dcat$ its \emph{identical natural transformation}\index{identical natural transformation}\index{natural transformation!identical}~$\id_F \colon F \to F$ is given by~$(\id_F)_X = \id_{F(X)}$ for every~$X \in \Ob(\Ccat)$.
      This is indeed a natural transformation, and it holds for every other functor~$G \colon \Ccat \to \Dcat$ that~$\eta \circ \id_F = \eta$ for every natural transformation~$\eta \colon F \to G$, and also~$\id_F \circ \zeta = \zeta$ for every natural transformation~$\zeta \colon G \to F$.
  \end{enumerate}
\end{remark}


\begin{example}
  Consider the two functors
  \[
              G
    \defined  (-)^\times
    \colon    \kAlg
    \to       \Group
    \quad\text{and}\quad
              F
    \defined  k[-]
    \colon    \Group
    \to       \kAlg \,.
  \]
  We examine how the compositions~$F \circ G \colon \Group \to \Group$ and~$G \circ F \colon \kAlg \to \kAlg$ relate to the identity functors~$\id_{\Group}$ and~$\id_{\kAlg}$.
  
  Let us first examine the composition~$G \circ F$:
  This functor is on objects given by
  \[
      (G \circ F)(\Gamma)
    = \kf[\Gamma]^\times
  \]
  for every groups~$\Gamma$.
  If~$\varphi \colon \Gamma \to \Delta$ is a homomorphism of groups, then the group homomorphism
  \[
            (G \circ F)(\varphi)
    =       \kf[\varphi]^\times
    \colon  \kf[\Gamma]^\times
    \to     \kf[\Delta]^\times
  \]
  is the restriction of the induced homomorphism of~{\kalgs}~$\kf[\varphi] \colon \kf[\Gamma] \to \kf[\Delta]$ to the unit groups.
  For every group~$\Gamma$ we have a group homomorphism
  \[
            \eta_\Gamma
    \colon  \Gamma
    \to     \kf[\Gamma]^\times \,,
    \quad   \gamma
    \mapsto [\gamma] \,.
  \]
  The resulting family~$\eta \defined (\eta_X)_{X \in \Ob(\Group)}$ is a natural transformation~$\eta \colon \Id_{\Group} \to G \circ F$, i.e.\ the diagram
  \[
    \begin{tikzcd}[column sep = large]
        \Gamma
        \arrow{r}[above]{\varphi}
        \arrow{d}[left]{\eta_\Gamma}
      & \Delta
        \arrow{d}[right]{\eta_\Delta}
      \\
        \kf[\Gamma]^\times
        \arrow{r}[above]{\kf[\varphi]^\times}
      & \kf[\Delta]^\times
    \end{tikzcd}
  \]
  commutes for every group homomorphism~$f \colon \Gamma \to \Delta$.
  This holds because
  \[
      \kf[\varphi]^\times( \eta_\Gamma( \gamma ) )
    = \kf[\varphi]^\times( [\gamma] )
    = [\varphi(\gamma)]
    = \eta_\Delta( \varphi( \gamma ) )
  \]
  for every~$\gamma \in \Gamma$.
  
  The functor~$F \circ G$ is given on objects by
  \[
      (F \circ G)(A)
    = \kf[A^\times]
  \]
  for every~{\kalg}~$A$.
  If~$f \colon A \to B$ is a homomorphism of~{\kalgs} then the induced~{\kalg} homomorphism~$(F \circ G)(f)$ is given by
  \[
            (F \circ G)(f)
    =       \kf[f^\times]
    \colon  \kf[A^\times]
    \to     \kf[B^\times] \,,
    \quad   \sum_{a \in A^\times} \lambda_a [a]
    \mapsto \sum_{a \in A^\times} \lambda_a [f(a)] \,.
  \]
  For every~{\kalg}~$A$ the identity~$A^\times \to A^\times$ corresponds (as seen on the first exercise sheet) to a homomorphism of~{\kalg}~$\varepsilon_A \colon \kf[A^\times] \to A$, which is given by
  \[
            \varepsilon_A
    \colon  \kf[A^\times]
    \to     A \,,
    \quad   \sum_{a \in A^\times} \lambda_a [a]
    \mapsto \sum_{a \in A^\times} \lambda_a a \,.
  \]
  The resulting family~$\varepsilon \defined (\varepsilon_A)_{A \in \Ob(\kAlg)}$ is a natural transformation~$F \circ G \to \Id_{\kAlg}$, i.e.\ the diagram
  \[
    \begin{tikzcd}[column sep = large]
        \kf[A^\times]
        \arrow{r}[above]{\kf[f^\times]}
        \arrow{d}[left]{\varepsilon_A}
      & \kf[B^\times]
        \arrow{d}[right]{\varepsilon_B}
      \\
        A
        \arrow{r}[above]{f}
      & B
    \end{tikzcd}
  \]
  commutes for every homomorphism of~{\kalgs}~$f \colon A \to B$.
  This holds because
  \begin{align*}
        \varepsilon_B\left( \kf[f^\times]\left( \sum_{a \in A^\times} \lambda_a [a] \right) \right)
    &=  \varepsilon_B\left( \sum_{a \in A^\times} \lambda_a [f(a)] \right)
     =  \sum_{a \in A^\times} \lambda_a f(a)  \\
    &=  f\left( \sum_{a \in A^\times} \lambda_a a \right)
     =  f\left( \varepsilon_A\left( \sum_{a \in A^\times} \lambda_a [a] \right) \right)
  \end{align*}
  for every~$\sum_{a \in A^\times} \lambda_a [a] \in \kf[A^\times]$.
  
  We will later see that the functors~$F \colon \Group \to \kAlg$ and~$G \colon \kAlg \to \Group$ form an \emph{adjunction}, and how this can be expressed via the natural transformations~$\eta \colon \Id_{\Group} \to G \circ F$ and~$\varepsilon \colon F \circ G \to \Id_{\kAlg}$.
\end{example}





\lecturend{5}





\begin{remarkdefinition}
  Let~$F, G \colon \Ccat \to \Dcat$ be two functors.
  A natural transformation~$\eta \colon F \to G$ is a \emph{natural isomorphism}\index{natural!isomorphism}\index{isomorphism!natural} if for every~$X \in \Ob(\Ccat)$ the morphism~$\eta_X \colon F(X) \to G(X)$ is an isomorphism.
  
  The natural transformation~$\eta$ is a natural isomorphism if and only if there exists a natural transformation~$\zeta \colon G \to F$ with~$\zeta \circ \eta = \id_F$ and~$\eta \circ \zeta = \id_G$:
  
  If such a natural transformation exists then it holds for exery~$X \in \Ob(\Ccat)$ that
  \[
      \zeta_X \circ \eta_X
    = (\zeta \circ \eta)_X
    = (\id_F)_X
    = \id_{F(X)}
  \]
  and similarly~$\eta_X \circ \zeta_X = \id_{G(X)}$.
  This then shows that~$\eta_X$ is for every~$X \in \Ob(\Ccat)$ an isomorphism, with inverse given by~$\eta_X^{-1} = \zeta_X$.
  
  If on the other hand~$\eta_X \colon F(X) \to G(X)$ is an isomorphism for every~$X \in \Ob(\Ccat)$, then it follows for every morphism~$f \colon X \to Y$ in~$\Ccat$ from the commutativity of the diagram
  \[
    \begin{tikzcd}
        F(X)
        \arrow{r}[above]{F(f)}
        \arrow{d}[left]{\eta_X}
      & F(Y)
        \arrow{d}[right]{\eta_Y}
      \\
        G(X)
        \arrow{r}[above]{G(f)}
      & G(Y)
    \end{tikzcd}
  \]
  that the diagram
  \[
    \begin{tikzcd}
        F(X)
        \arrow{r}[above]{F(f)}
      & F(Y)
      \\
        G(X)
        \arrow{u}[left]{\eta_X^{-1}}
        \arrow{r}[above]{G(f)}
      & G(Y)
        \arrow{u}[right]{\eta_Y^{-1}}
    \end{tikzcd}
  \]
  also commutes.
  This shows that the family~$\zeta \defined (\eta_X^{-1})_{X \in \Ob(\Ccat)}$ is a natural transformation~$\zeta \colon G \to F$.
  It holds by construction of~$\zeta$ that~$\zeta \circ \eta = \id_F$ and~$\eta \circ \zeta = \id_G$.
  
  That~$\eta$ is a natural isomorphism is denoted by~$\eta \colon F \xto{\sim} G$.
  The two functors~$F$ and~$G$ are \emph{isomorphic}\index{isomorphic functors}\index{functor!isomorphic} if there exist a natural isomorphism~$F \xto{\sim} G$.
  That~$F$ and~$G$ are isomorphic is denoted by~$F \cong G$.
\end{remarkdefinition}


\begin{definition}
  Let~$\Ccat$ and~$\Dcat$ be categories.
  A functor~$F \colon \Ccat \to \Dcat$ is an \emph{equivalence of categories}\index{equivalence of categories}\index{functor!equivalence} if there exists a functor~$G \colon \Dcat \to \Ccat$ with~$G \circ F \cong \Id_{\Ccat}$ and~$F \circ G \cong \Id_{\Dcat}$.
  The categories~$\Ccat$ and~$\Dcat$ are \emph{equivalent}\index{category!equivalent} if there exists an equivalence of categories between them.
  That~$\Ccat$ and~$\Dcat$ are equivalent is denoted by~$\Ccat \simeq \Dcat$.
\end{definition}


\begin{example}
  \leavevmode
  \begin{enumerate}
    \item
      Let~$Q$ be a finite quiver.
      \Cref{quiver rep are modules} shows that~$\Rep{\kf}{Q} \simeq \Modl{\kf Q}$, where~$\Rep{\kf}{Q}$ denotes the category of representations of~$Q$ over~$\kf$;
      the only missing ingredient is that the constructed isomorphisms
      \[
              FG(M)
        \cong M
        \quad\text{and}\quad
              GF(X)
        \cong X
      \]
      for~$M \in \Ob(\Modl{\kf Q})$ and~$X \in \Ob(\Rep{\kf}{Q})$ are natural, i.e.\ that for every homomorphism of left~{\modules{$\kf Q$}}~$f \colon M \to N$ the diagram
      \[
        \begin{tikzcd}[column sep = large]
            FG(M)
            \arrow{r}[above]{FG(f)}
            \arrow{d}[left]{\sim}
          & FG(N)
            \arrow{d}[right]{\sim}
          \\
            M
            \arrow{r}[above]{f}
          & N
        \end{tikzcd}
      \]
      commutes, and that for every homomorphism of representations~$f \colon X \to Y$ the diagram
      \[
        \begin{tikzcd}[column sep = large]
            GF(X)
            \arrow{r}[above]{GF(f)}
            \arrow{d}[left]{\sim}
          & GF(Y)
            \arrow{d}[right]{\sim}
          \\
            X
            \arrow{r}[above]{f}
          & Y
        \end{tikzcd}
      \]
      commutes.
    \item
      Let~$G$ be a group.
      A \emph{representation}\index{representation!of a group}\index{group representation} of~$G$ over~$\kf$ is a pair~$(V,\rho)$ consisting of a~{\module{$\kf$}}~$V$ and a group homomorphism~$\rho \colon G \to \GL(V)$.
      A \emph{homomorphism of representations}\index{homomorphism!of group representations}~$f \colon (V, \rho) \to (W,\sigma)$ is a~{\klin} map~$f \colon V \to W$ such that the diagram
      \[
        \begin{tikzcd}
            V
            \arrow{r}[above]{\rho(g)}
            \arrow{d}[left]{f}
          & V
            \arrow{d}[right]{f}
          \\
            W
            \arrow{r}[above]{\sigma(g)}
          & W
        \end{tikzcd}
      \]
      commutes for every~$g \in G$.
      It holds for the category~$\Rep{\kf}{G}$ of representations of~$G$ over~$\kf$ that~$\Rep{\kf}{G} \simeq \Modl{\kf[G]}$.
    \item
      Let~$\Ccat$ be the category whose objects are pairs~$(A, \varphi)$ consisting of a ring~$A$ and a ring homomorphism~$\varphi \colon \kf \to \ringcenter(A)$, and where a morphism~$f \colon (A, \varphi) \to (B, \psi)$ is a ring homomorphism~$f \colon A \to B$ which makes the diagram
      \[
        \begin{tikzcd}[column sep = small]
            A
            \arrow{rr}[above]{f}
          & {}
          & B
          \\
            \ringcenter(A)
            \arrow[hook]{u}
          & {}
          & \ringcenter(B)
            \arrow[hook]{u}
          \\
            {}
          & \kf
            \arrow{ul}
            \arrow{ur}
          & {}
        \end{tikzcd}
      \]
      commute.
      Then \cref{characterization of algebras} shows that the categories~$\kAlg$ and~$\Ccat$ are equivalent.
    \item
      Let~$A$ be a~{\kalg}.
      Let~$\Dcat$ be the category whose objects are pairs~$(V, \varphi)$ consisting of a~{\module{$\kf$}}~$V$ and a homomorphism of~{\kalgs}~$\varphi \colon A \to \End_\kf(V)$, and where a morphism~$f \colon (V,\varphi) \to (W,\psi)$ is a~{\klin} map~$f \colon V \to W$ which makes the diagram
      \[
        \begin{tikzcd}
            V
            \arrow{r}[above]{\varphi(a)}
            \arrow{d}[left]{f}
          & V
            \arrow{d}[right]{f}
          \\
            W
            \arrow{r}[above]{\psi(a)}
          & W
        \end{tikzcd}
      \]
      commute for every~$a \in A$.
      \Cref{modules as homomorphisms into endomorphisms} shows that the categories~$\Modl{A}$ and~$\Dcat$ are equivalent.
  \end{enumerate}
\end{example}





\section{Functor Categories}


\begin{definition}
  Let~$\Ccat$ and~$\Dcat$ be two categories.
  The \emph{functor category}\index{functor!category}\index{category!functor}~$\Fun(\Ccat, \Dcat)$ has as objects
  \[
              \Ob(\Fun(\Ccat, \Dcat))
    \defined  \{
                \text{functors~$F \colon \Ccat \to \Dcat$}
              \} \,,
  \]
  and for any two functors~$F, G \colon \Ccat \to \Dcat$ their morphism set~$\Fun(\Ccat, \Dcat)(F,G)$ is given by
  \[
              \Fun(\Ccat, \Dcat)(F,G)
    \defined  \{
                \text{natural transformations~$\eta \colon F \to G$}
              \} \,.
  \]
  The composition of morphisms in~$\Fun(\Ccat, \Dcat)$ is the composition of natural transformations.
\end{definition}


\begin{remark}
  We’re running into \dash{set}{theoretic} issues again:
  If~$\Ccat$ and~$\Dcat$ are two categories in our fixed universe~$U$ (i.e.\~$\Ob(\Ccat), \Ob(\Dcat) \subseteq U$) then~$\Ob(\Fun(\Ccat, \Dcat))$ might not be a subset of~$U$.
  The solution to this problem is to choose another universe~$V$ with~$U \in V$.
  Then~$\Ob(\Fun(\Ccat, \Dcat)) \subseteq V$, so~$\Fun(\Ccat, \Dcat)$ becomes a category with respect to the universe~$V$.
\end{remark}


\begin{example}
  Let~$Q$ be a quiver and consider the category~$\Fun(\Path(Q),\Modl{\kf})$.
  
  Every functor~$V \in \Ob(\Fun(\Path(Q),\Modl{\kf}))$ gives rise to a representation~$F(V)$ of~$Q$ over~$\kf$ with
  \[
              F(V)_i
    \defined  V(i)
  \]
  for every~$i \in Q_0 = \Ob(\Path(Q))$ and
  \[
              F(V)_\alpha
    \defined  V(\alpha)
    \colon    F(V)_i
    \to       F(V)_j
  \]
  for every arrow~$\alpha$ from~$i$ to~$j$ in~$Q_0$ (and thus morphisms from~$i$ to~$j$ in~$\Path(Q)$).
  In this way we obtain a functor
  \[
            F
    \colon  \Fun(\Path(Q), \Modl{\kf})
    \to     \Rep{\kf}{Q}  \,.
  \]
  
  Conversely, let~$X$ be a representation of~$Q$ over~$\kf$.
  We can use~$X$ to define a functor~$G(X) \colon \Path(Q) \to \Modl{\kf}$ via
  \[
            G(X)
    \colon  \left\{
              \begin{aligned}
                          i
                &\mapsto  X_i \,, \\
                          (p = \alpha_\ell \dotsm \alpha_1)
                &\mapsto  X_{\alpha_\ell} \circ \dotsb \circ X_{\alpha_1} \,.
              \end{aligned}
            \right.
  \]
  This construction yields a functor
  \[
            G
    \colon  \Rep{\kf}{Q}
    \to     \Fun(\Path(Q), \Modl{\kf}) \,.
  \]
  It can now be checked that~$G \circ F = \Id_{\Fun(\Path(Q), \Modl{\kf})}$ and~$F \circ G = \Id_{\Rep{\kf}{Q}}$, which shows that~$\Rep{\kf}{Q} \simeq \Fun(\Path(Q), \Modl{\kf})$.
\end{example}


\begin{definition}
  Let~$\Ccat$ and~$\Dcat$ betwo categories.
  For every object~$X \in \Ob(\Ccat)$ the \emph{evaluation}\index{evaluation}\index{functor!evaluation} at~$X$ is the functor~$\ev_X \colon \Fun(\Ccat, \Dcat) \to \Dcat$ given by
  \[
            \ev_X
    \colon  \left\{
              \begin{aligned}
                          F
                &\mapsto  F(X) \,,
                \\
                          (F \xto{\eta} G)
                &\mapsto  (\eta_X \colon F(X) \to G(X)) \,.
              \end{aligned}
            \right.
  \]
\end{definition}


\begin{remark}
  Let~$\Ccat$ and~$\Dcat$ be two categories.
  If~$f \colon X \to Y$ is a morphism in~$\Ccat$ then we get an induced natural transformation~$\ev_f \colon \ev_X \to \ev_Y$ given by
  \[
              (\ev_f)_F
    \defined  F(f)
  \]
  for every functor~$F \in \Ob(\Fun(\Ccat, \Dcat))$.
  This is indeed a natural transformation:
  Let~$F, G \in \Ob(\Fun(\Ccat, \Dcat))$ be functors and let~$\eta \colon F \to G$ be a natural transformation between them.
  We then have the following diagram:
  \[
    \begin{tikzcd}
        \ev_X(F)
        \arrow{rrr}[above]{\ev_X(\eta)}
        \arrow{ddd}[left]{(\ev_f)_F}
        \arrow[equal]{dr}
      & {}
      & {}
      & \ev_X(G)
        \arrow{ddd}[right]{(\ev_f)_G}
        \arrow[equal]{dl}
      \\
        {}
      & F(X)
        \arrow{r}[above]{\eta_X}
        \arrow{d}[left]{F(f)}
      & G(X)
        \arrow{d}[right]{G(f)}
      & {}
      \\
        {}
      & F(Y)
        \arrow{r}[below]{\eta_Y}
      & G(Y)
      & {}
      \\
        \ev_Y(F)
        \arrow{rrr}[below]{\ev_Y(\eta)}
        \arrow[equal]{ur}
      & {}
      & {}
      & \ev_Y(G)
        \arrow[equal]{ul}
    \end{tikzcd}
  \]
  The inner square commutes because~$\eta$ is a natural transformation, and so it follows that the outer square commutes.
  
  It also holds that~$\ev_{\id_X} = \id_{\ev_X}$ and that~$\ev_{f \circ g} = \ev_f \circ \ev_g$ for every two composable morphisms~$f \colon X \to Y$ and~$g \colon Y \to Z$ in~$\Ccat$.
  We hence obtain a functor
  \[
            \ev
    \colon  \Ccat
    \to     \Fun( \Fun(\Ccat, \Dcat), \Dcat ) \,.
  \]

\end{remark}





\section{Representable Functors}


\begin{lemma}[Yoneda’s lemma]\index{Yoneda’s lemma}
  \label{yoneda lemma}
  Let~$\Ccat$ be a category and let~$X \in \Ob(\Ccat) = \Ob(\Ccat^\op)$.
  \begin{enumerate}
    \item
      \label{covariant yoneda}
      Let~$F \colon \Ccat \to \Set$ be a (covariant) functor.
      Then the map
      \[
                Y^{F,X}
        \colon  \Fun(\Ccat, \Set)(h^X, F)
        \to     F(X) \,,
        \quad   ( \eta \colon h^X \to F)
        \mapsto \eta_X(\id_X)
      \]
      is a bijection.
    \item
      \label{contravariant yoneda}
      Let~$G \colon \Ccat^\op \to \Set$ be a (contravariant) functor.
      Then the map
      \[
                Y_{G,X}
        \colon  \Fun(\Ccat, \Set)(h_X, G)
        \to     G(X) \,,
        \quad   ( \eta \colon h_X \to G)
        \mapsto \eta_X(\id_X)
      \]
      is a bijection.
  \end{enumerate}
\end{lemma}


\begin{proof}
  Part~\ref*{covariant yoneda} follows from part~\ref*{contravariant yoneda} because~$\Ccat = (\Ccat^\op)^\op$ and
  \[
      h^X_{(\Ccat)}
    = \Ccat(X,-)
    = \Ccat^\op(-,X)
    = h_X^{(\Ccat^\op)}
  \]
  for every~$X \in \Ob(\Ccat) = \Ob(\Ccat^\op)$.
  We therefore only prove part~\ref*{contravariant yoneda}.
  
  To show that the map~$Y_{G,X}$ is injective we need to show that a natural transformation~$\eta \colon h_X \to G$ is uniquely determined by its value~$\eta_X(\id_X)$.
  This holds because it follows for every~$Y \in \Ob(\Ccat)$ and every~$f \in h_X(Y) = \Ccat(Y,X)$ from the commutativity of the diagram
  \[
    \begin{tikzcd}
        h_X(X)
        \arrow{r}[above]{f^*}
        \arrow{d}[left]{\eta_X}
      & h_X(Y)
        \arrow{d}[right]{\eta_Y}
      \\
        G(X)
        \arrow{r}[above]{G(f)}
      & G(Y)
    \end{tikzcd}
  \]
  that
  \[
      \eta_Y(f)
    = \eta_Y(\id_X \circ f)
    = \eta_Y(f^*(\id_X))
    = G(f)(\eta_X(\id_X)) \,.
  \]
  
  To show the surjectivity of~$Y_{G,X}$ let~$z \in G(X)$.
  For every~$Y \in \Ob(\Ccat)$ let
  \[
            \zeta_Y
    \colon  h_X(Y)
    =       \Ccat(Y,X)
    \to     G(Y) \,,
    \quad   f
    \mapsto G(f)(z) \,.
  \]
  It then holds for every morphism~$g \colon Y \to Y'$ in~$\Ccat$ that the diagram
  \[
    \begin{tikzcd}
        h_X(Y')
        \arrow{r}[above]{g^*}
        \arrow{d}[left]{\zeta_{Y'}}
      & h_X(Y)
        \arrow{d}[right]{\zeta_Y}
      \\
        G(Y')
        \arrow{r}[above]{G(g)}
      & G(Y)
    \end{tikzcd}
  \]
  commutes, because
  \[
      G(g)( \zeta_{Y'}( f ) )
    = G(g)( G(f)(z) )
    = G(f \circ g)(z)
    = \zeta_Y(f \circ g)
    = \zeta_Y( g^*(f) )
  \]
  for every~$f \in h_X(Y')$.
  This shows that~$\zeta \defined (\zeta_X)_{X \in \Ob(\Ccat)}$ is a natural transformation~$\zeta \colon h_X \to F$.
  It holds that
  \[
      Y_{G,X}(\zeta)
    = \zeta_X(\id_X)
    = G(\id_X)(z)
    = \id_{G(X)}(z)
    = z \,,
  \]
  which altogether shows that~$Y_{G,X}$ is surjective.
\end{proof}





\lecturend{6}




