\section{Chain and Cochain Complexes}


\begin{definition}
  \leavevmode
  \begin{enumerate}
    \item
      A \emph{chain complex}\index{chain!complex}\index{complex!chain} in~$\Acat$ is a pair~$\Ccc = ( (C_n)_{n \in \Integer}, (d_n)_{n \in \Integer} )$ consisting of
      \begin{itemize}
        \item
          a family of objects~$C_n \in \Ob(\Acat)$, where~$n \in \Integer$, and
        \item
          a family of morphisms~$d_n \colon C_n \to C_{n-1}$, where again~$n \in \Integer$,
      \end{itemize}
      subject to the condition~$d_{n-1} d_n = 0$ for every~$n \in \Integer$.
      The family~$d = (d_n)_{n \in \Integer}$ is the \emph{differential}\index{differential} of~$\Ccc$.
    \item
      Dually, a \emph{cochain complex}\index{cochain complex}\index{complex!cochain} in~$\Acat$ is a par~$\Cccc = ( (C^n)_{n \in \Integer}, (d^n)_{n \in \Integer} )$ consisting of
      \begin{itemize}
        \item
          a family of objects~$C^n \in \Ob(\Acat)$, where~$n \in \Integer$, and
        \item
          a family of morphisms~$d^n \colon C^n \to C^{n+1}$, where again~$n \in \Integer$,
      \end{itemize}
      subject to the condition~$d^{n+1} d^n = 0$ for every~$n \in \Integer$.
      The family~$d \defined (d^n)_{n \in \Integer}$ is the \emph{differential}\index{differential} of~$\Cccc$.%
      \footnote{The author thinks that~$d$ ought to be called \emph{codifferential} instead.}
    \item
      Let~$\Ccc$ and~$\Dcc$ be two chain complexes in~$\Acat$.
      A \emph{morphism}\index{morphism!of!chain complexes} of chain complexes~$f \colon \Ccc \to \Dcc$ is a family~$f = (f_n)_{n \in \Integer}$ of morphisms~$f_n \colon C_n \to D_n$ such that~\enquote{$df = fd$}, i.e.\ such that for every~$n \in \Integer$ the following diagram commutes:
      \[
        \begin{tikzcd}
            C_n
            \arrow{r}[above]{d_n}
            \arrow{d}[left]{f_n}
          & C_{n-1}
            \arrow{d}[right]{f_{n-1}}
          \\
            D_n
            \arrow{r}[above]{d_n}
          & D_{n-1}
        \end{tikzcd}
      \]
      We obtain with this notion of morphism a category~$\Ch(\Acat)$ of chain complexes in~$\Acat$.
      
      The notion of a morphism of cochain complexes is defined similarly, and we obtain a category~$\CCh(\Acat)$ of cochain complexes in~$\Acat$.
  \end{enumerate}
\end{definition}


\begin{remark*}
  If~$f = (f_n)_n \colon \Ccc \to \Dcc$ is a morphism of chain complexes such that~$f_n$ is for every~$n \in \Integer$ an isomorphism then~$(f_n^{-1})_{n \in \Integer}$ is a morphism of chain complexes~$\Dcc \to \Ccc$.
  The morphism~$f$ is therefore an isomorphism if and only if~$f_n$ is for every~$n \in \Integer$ an isomorphism.
\end{remark*}


\begin{remark}
  \leavevmode
  \begin{enumerate}
    \item
      A cochain complexes in~$\Acat$ is the same as a chain complex in~$\Acat^\op$ shifted by~$1$:
      If~$\Cccc$ is a cochain complex in~$\Acat$ then~$\cc{\widetilde{C}} = ( (\widetilde{C}_n)_{n \in \Integer}, (\tilde{d}_n)_{n \in \Integer} )$ given by~$\widetilde{C}_n \defined C^n$ and~$\tilde{d}_n \defined d^{n-1}$ for every~$n \in \Integer$ is a chain complex in~$\Acat^\op$.
      
      To every chain complex~$\Ccc$ in~$\Acat$ we can also associate a cochain complex~$\ccc{\widetilde{C}}$ in~$\Acat$ by setting~$\widetilde{C}^n \defined C_{-n}$ for every~$n \in \Integer$, as well as~$\tilde{d}^n \defined d_{-n}$ for every~$n \in \Integer$.
      
      These constructions can be extended to equivalences of categories
      \[
                \CCh(\Acat)
        \simeq  \Ch(\Acat^\op)
        \quad\text{and}\quad
                \Ch(\Acat)
        \simeq  \CCh(\Acat)
      \]
    \item
      One can also describe chain complexes as \emph{differential graded objects}\index{differential graded object}\index{object!differential graded}:
      
      Suppose that the category~$\Acat$ has countable coproducts, which will be denoted by~$\bigoplus$.
      A~\dash{$\Integer$}{graded}\index{graded object}\index{object!graded} object of~$\Acat$ is an object~$C$ together with a decomposition~$C = \bigoplus_{n \in \Integer} C_n$ for some countable family~$(C_n)_{n \in \Integer}$ of objects~$C_n \in \Ob(\Acat)$.%
      \footnote{The author thinks that it is more appropiate to define the described~\dash{$\Integer$}{graded} object as simply the family~$(C_n)_{n \in \Integer}$ itself.}
      
      If~$C = \bigoplus_{n \in \Integer}$ and~$D = \bigoplus_{n \in \Integer} D_n$ are~\dash{$\Integer$}{graded} objects in~$\Acat$, then a morphism~$C \to D$ (in~$\Acat$) is \emph{homogeneous}\index{morphism!homogeneous}\index{homogeneous morphism}\index{homogeneous morphism}\index{morphism!homogeneous} of degree~$d \in \Integer$ if for every~$n \in \Integer$ the morphism~$f$ \enquote{restricts to a morphism~$C_n \to D_{n+d}$}, in the sense that there exist a morphism~$f_n \colon C_n \to D_{n+d}$ that makes the following square commute:
      \[
        \begin{tikzcd}
            C
            \arrow{r}[above]{f}
          & D
          \\
            C_n
            \arrow{u}
            \arrow{r}[above]{f_n}
          & D_{n+d}
            \arrow{u}
        \end{tikzcd}
      \]
      A \emph{morphism of~\dash{$\Integer$}{graded} objects}\index{morphism!of!graded objects}\index{graded object!morphism}~$f \colon C \to D$ is a homogeneous morphism of degree~$0$.
      
      If~$\Ccc$ is a chain complex in~$\Acat$ then~$C \defined \bigoplus_{n \in \Integer} C_n$ is (together with this decomposition) a~\dash{$\Integer$}{graded} object in~$\Acat$, and the differential~$(d_n)_{n \in \Natural}$ induces a morphism~$d \colon C \to C$ of degree~$-1$ with~$d^2 = 0$;
      this morphism is the unique one that makes the following square commute:
      \[
        \begin{tikzcd}
            C
            \arrow{r}[above]{d}
          & C
          \\
            C_n
            \arrow{u}
            \arrow{r}[above]{d_{n-1}}
          & C_{n-1}
            \arrow{u}
        \end{tikzcd}
      \]
      Every morphism of chain complexes~$(f_n)_{n \in \Integer} \colon \Ccc \to \Dcc$ results in a morphism of~\dash{$\Integer$}{graded} objects~$f \colon C \to D$ in~$\Acat$, namely the unique morphism~$C \to D$ such that the following square commutes for every~$n \in \Integer$:
      \[
        \begin{tikzcd}
            C
            \arrow[dashed]{r}[above]{f}
          & D
          \\
            C_n
            \arrow{u}
            \arrow{r}[above]{f_n}
          & D_n
            \arrow{u}
        \end{tikzcd}
      \]
      (See \cref{functoriality of (co)product} for more details on this induced morphism and its functoriality.)
      This morphism then satisfies~$df = fd$.
      Indeed, we have for every~$n \in \Integer$ the following diagram:
      \[
        \begin{tikzcd}[column sep = large]
            C_n
            \arrow{rrr}[above]{d_n}
            \arrow{dr}[above right]{i_n}
            \arrow{ddd}[left]{f_n}
          & {}
          & {}
          & C_{n-1}
            \arrow{dl}[above left]{i_{n-1}}
            \arrow{ddd}[right]{f_{n-1}}
        \\
            {}
          & C
            \arrow{r}[above]{d}
            \arrow{d}[left]{f}
          & C
            \arrow{d}[right]{f}
          & {}
          \\
            {}
          & D
            \arrow{r}[above]{d}
          & D
          & {}
          \\
            D_n
            \arrow{ur}[below right]{j_n}
            \arrow{rrr}[above]{d_n}
          & {}
          & {}
          & D_{n-1}
            \arrow{ul}[below left]{j_{n-1}}
        \end{tikzcd}
      \]
      The four trapezoids commute by the definitions of~$d$,$d$ and~$f$, and the outer sqare commutes because~$(f_n)_{n \in \Integer}$ is a morphism of chain complexes.
      It follows that the inner square also commutes;
      indeed, it follows that for every~$n \in \Integer$ that
      \[
          d f i_n
        = d j_n f_n
        = j_{n-1} d_n f_n
        = j_{n-1} f_{n-1} d_n
        = f i_{n-1} d_n
        = f d i_n \,,
      \]
      and hence overall that~$df = fd$ by the universal property of the coproduct~$C$
      
      This leads us to consider the category~$\Ccat$ where
      \begin{itemize}
        \item
          an object of~$\Ccat$ is a pair~$(C,d)$ consisting of a~\dash{$\Integer$}{graded} object~$C$ and a homogeneous morphism~$d \colon C \to C$ of degree~$-1$ with~$d^2 = 0$, and
        \item
          a morphism~$f \colon (C,d) \to (D,d)$ is a morphism~$f \colon C \to D$ of~\dash{$\Integer$}{graded} objects such that~$d f = f d$.
      \end{itemize}
      We have above constructed a functor~$F \colon \Ch(\Acat) \to \Ccat$, and this functor is an equivalence of categories.
      The objects of~$\Ccat$ are called \emph{differential graded objects} in~$\Acat$.
  \end{enumerate}
\end{remark}


% TODO: Remark regarding leaving out indices.
% informal explanation & also via morphisms of graded modules

\begin{definition}
  \leavevmode
  \begin{enumerate}
    \item
      Let~$\Ccc$ be a chain complex.
      For every~$n \in \Integer$, the~\emph{\dash{$n$}{th} cycle object}\index{cycle} of~$\Ccc$ is
      \[
                  \Zl_n(\Ccc)
        \defined  \ker(d_n) \,,
      \]
      and the~\emph{\dash{$n$}{th} boundary object}\index{boundary} of~$\Ccc$ is
      \[
                  \Bl_n(\Ccc)
        \defined  \im(d_{n+1})  \,.
      \]
      It follows for every~$n \in \Integer$ from~$d_n d_{n+1} = 0$ that there exist a unique morphism~$\Bl_n(\Ccc) \to \Zl_n(\Ccc)$ that makes the diagram
      \[
        \begin{tikzcd}[column sep = tiny]
            \dotsb
            \arrow{rr}
          & {}
          & C_{n+1}
            \arrow{rr}[above]{d_{n+1}}
            \arrow{dr}
          & {}
          & C_n
            \arrow{rr}[above]{d_n}
          & {}
          & C_{n-1}
            \arrow{rr}
          & {}
          & \dotsb
          \\
            {}
          & {}
          & {}
          & \Bl_n(\Ccc)
            \arrow{ur}
            \arrow[dashed]{rr}
          & {}
          & \Zl_n(\Ccc)
            \arrow{ul}
          & {}
          & {}
          & {}
        \end{tikzcd}
      \]
      commute, and this morphism is a monomorphism.
      The~\emph{\dash{$n$}{th} homology object}\index{homology} of the chain complex~$\Ccc$ is
      \[
                    \Hl_n(\Ccc)
        \defined    \coker(\Bl_n(\Ccc) \to \Zl_n(\Ccc)) \,.
      \]
      (One may think about~$\Hl_n(\Ccc)$ as a quotient~$\Hl_n(\Ccc) = {\Zl_n(\Ccc)}/{\Bl_n(\Ccc)}$.)
    \item
      Dually, let~$\Cccc$ be a cochain complex in~$\Acat$.
      For every~$n \in \Integer$, the~\emph{\dash{$n$}{th} cocycle object}\index{cocycle} of~$\Cccc$ is
      \[
        \Zl^n(\Cccc) \defined \ker(d^n) \,,
      \]
      and the~\emph{\dash{$n$}{th} coboundary object}\index{coboundary} of~$\Cccc$ is
      \[
                  \Bl^n(\Cccc)
        \defined  \im(d^{n-1})  \,.
      \]
      The~\emph{\dash{$n$}{th} cohomology object}\index{cohomology} of~$\Cccc$ is
      \[
                  \Hl^n(\Cccc)
        \defined  \coker( \Bl^n(\Cccc)\to \Zl^n(\Cccc) )
      \]
      where~$\Bl^n(\Cccc) \to \Zl^n(\Cccc)$ is the unique morphism that makes the following diagram commute:
      \[
        \begin{tikzcd}[column sep = tiny]
            \dotsb
            \arrow{rr}
          & {}
          & C^{n-1}
            \arrow{rr}[above]{d^{n-1}}
            \arrow{dr}
          & {}
          & C^n
            \arrow{rr}[above]{d^n}
          & {}
          & C^{n+1}
            \arrow{rr}
          & {}
          & \dotsb
          \\
            {}
          & {}
          & {}
          & \Bl^n(\Cccc)
            \arrow{ur}
            \arrow[dashed]{rr}
          & {}
          & \Zl^n(\Cccc)
            \arrow{ul}
          & {}
          & {}
          & {}
        \end{tikzcd}
      \]
  \end{enumerate}
\end{definition}


\begin{example*}
  Let~$A$ be a~{\kalg} and let~$\Ccc$ be a chain complex of~{\modules{$A$}}, i.e.\ a chain complex in the abelian category~$\Modl{A}$.
  Then~$\Bl_n(\Ccc) \subseteq \Zl_n(\Ccc) \subseteq C_n$ are submodules for every~$n \in \Integer$, whose quotient~$\Hl_n(\Ccc) = {\Zl_n(\Ccc)}/{\Bl_n(\Ccc)}$ is the~\dash{$n$}{th} homology of~$\Ccc$.
  The elements of~$\Zl_n(\Ccc)$ are the~\emph{\dash{$n$}{cycles}} of~$\Ccc$, and the elements of~$\Bl_n(\Ccc)$ are the~\emph{\dash{$n$}{boundaries}} of~$\Ccc$.
\end{example*}


\begin{remark*}
  \label{Hn as cokernel with Cn}
  Let~$\Ccc$ be a chain complex in~$\Acat$ and let~$n \in \Integer$.
  Then the canonical morphism~$C_{n+1} \to \Bl_n(\Ccc)$ is an epimorphism, and hence
  \[
      \Hl_n(\Ccc)
    = \coker( \Bl_n(\Ccc) \to \Zl_n(\Ccc) )
    = \coker( C_{n+1} \to \Bl_n(\Ccc) \to \Zl_n(\Ccc) ) \,.
  \]
  The morphism~$C_{n+1} \to \Zl_n(\Ccc)$ is induced by the differential~$d_{n+1} \colon C_{n+1} \to C_n$ via the universal property of the kernel, which can be used because~$d_{n+1} d_n = 0$.
\end{remark*}


\begin{remark}[Functoriality of homology]
  \label{functoriality of homology}
  \leavevmode
  \begin{enumerate}
    \item
      Let~$f = (f_n)_n \colon \Ccc \to \Dcc$ be a morphism of chain complexes.
      It follows for every~$n \in \Integer$ from the commutativity of the square
      \[
        \begin{tikzcd}
            C_n
            \arrow{r}[above]{d_n}
            \arrow{d}[left]{f_n}
          & C_{n-1}
            \arrow{d}[right]{f_{n-1}}
          \\
            D_n
            \arrow{r}[above]{d_n}
          & D_{n-1}
        \end{tikzcd}
      \]
      that there exists a unique induced morphism~$\Zl_n(f) \colon \Zl_n(\Ccc) \to \Zl_n(\Dcc)$ that makes the following diagram commute:
      \[
        \begin{tikzcd}
            \Zl_n(\Ccc)
            \arrow{r}
            \arrow[dashed]{d}[left]{\Zl_n(f)}
          & C_n
            \arrow{r}[above]{d_n}
            \arrow{d}[left]{f_n}
          & C_{n-1}
            \arrow{d}[right]{f_{n-1}}
          \\
            \Zl_n(\Dcc)
            \arrow{r}
          & D_n
            \arrow{r}[above]{d_n}
          & D_{n-1}
        \end{tikzcd}
      \]
      This induced morphism is functorial in the following sense:
      \begin{itemize}
        \item
          If~$\Dcc = \Ccc$ and~$f = \id_{\Ccc}$ then~$\Zl_n(\id_{\Ccc}) = \id_{\Zl_n(\Ccc)}$ for every~$n \in \Integer$.
        \item
          If~$\Ecc$ is another chain complex and~$g \colon \Dcc \to \Ecc$ is another morphism of chain complexes, then
          \[
              \Zl_n(g \circ f)
            = \Zl_n(g) \circ \Zl_n(f)
          \]
          for every~$n \in \Integer$.
        \item
          If~$g \colon \Ccc \to \Dcc$ is another morphisms of chain complexes that is parallel to the morphism~$f$, then
          \[
              \Zl_n(f + g)
            = \Zl_n(f) + \Zl_n(g) \,.
          \]
          for every~$n \in \Integer$.
      \end{itemize}
      (See \cref{functoriality of (co)kernel} for more details on this induced morphism and its functoriality and additivity.)
    \item
      It also follows for every~$n \in \Integer$ from the commutativity of the square
      \[
        \begin{tikzcd}[column sep = large]
            C_{n+1}
            \arrow{r}[above]{d_{n+1}}
            \arrow{d}[left]{f_{n+1}}
          & C_n
            \arrow{d}[right]{f_n}
          \\
            D_{n+1}
            \arrow{r}[above]{d_{n+1}}
          & D_n
        \end{tikzcd}
      \]
      that there exists a unique induced morphism~$\Bl_n(f) \colon \Bl_n(\Ccc) \to \Bl_n(\Dcc)$ that makes the following diagram commute:
      \[
        \begin{tikzcd}[column sep = large]
            C_{n+1}
            \arrow{r}[above]{d_{n+1}}
            \arrow{d}[left]{f_{n+1}}
          & C_n
            \arrow{r}
            \arrow{d}[right]{f_n}
          & \Bl_n(\Ccc)
            \arrow[dashed]{d}[right]{\Bl_n(f)}
          \\
            D_{n+1}
            \arrow{r}[above]{d_{n+1}}
          & D_n
            \arrow{r}
          & \Bl_n(\Dcc)
        \end{tikzcd}
      \]
      This induced morphism is functorial in the following sense:
      \begin{itemize}
        \item
          If~$\Dcc = \Ccc$ and~$f = \id_{\Ccc}$ then~$\Bl_n(\id_{\Ccc}) = \id_{\Bl_n(\Ccc)}$ for every~$n \in \Integer$.
        \item
          If~$\Ecc$ is another chain complex and~$g \colon \Dcc \to \Ecc$ is another morphism of chain complexes, then
          \[
              \Bl_n(g \circ f)
            = \Bl_n(g) \circ \Bl_n(f)
          \]
          for every~$n \in \Integer$.
        \item
          If~$g \colon \Ccc \to \Dcc$ is another morphism of chain complexes that is parallel to the morphism~$g$ then
          \[
              \Bl_n(f + g)
            = \Bl_n(f) + \Bl_n(g)
          \]
          for every~$n \in \Integer$.
      \end{itemize}
      (See \cref{functoriality of (co)image} for more details on this induced morphism and its functoriality and additivity.)
    \item
      We get from the above for every~$n \in \Integer$ the following diagram:
      \[
        \begin{tikzcd}[row sep = large, column sep = 2.3em]
            \dotsb
            \arrow{r}
          & C_{n+1}
            \arrow{rr}[above]{d_{n+1}}
            \arrow{dd}[left, near start]{f_{n+1}}
            \arrow{dr}[below left]{p}
          & {}
          & C_n
            \arrow{rr}[above]{d_n}
            \arrow{dd}[right, near start]{f_n}
          & {}
          & C_{n-1}
            \arrow{r}
            \arrow{dd}[right, near start]{f_{n-1}}
          & \dotsb
          \\
            {}
          & {}
          & \Bl_n(\Ccc)
            \arrow{ur}[above left]{i}
          & {}
          & \Zl_n(\Ccc)
            \arrow{ul}[above right]{j}
            \arrow[from=ll, dashed, crossing over, "\lambda", near start]
          & {}
          & {}
          \\
            \dotsb
            \arrow{r}
          & D_{n+1}
            \arrow{rr}[above, near start]{d_{n+1}}
            \arrow{dr}[below left]{p'}
          & {}
          & D_n
            \arrow{rr}[above, near end]{d_n}
          & {}
          & D_{n-1}
            \arrow{r}
          & \dotsb
          \\
            {}
          & {}
          & \Bl_n(\Dcc)
            \arrow{ur}[above left]{i'}
            \arrow[dashed]{rr}[above]{\lambda'}
            \arrow[from=uu, dashed, crossing over, "\Bl_n(f)", near start]
          & {}
          & \Zl_n(\Dcc)
            \arrow{ul}[above right]{j'}
            \arrow[from=uu, dashed, crossing over, "\Zl_n(f)", swap, near start]
          & {}
          & {}
        \end{tikzcd}
      \]
      This diagram commutes:
      It remains to show that the dashed square in the front commutes.
      This holds because
      \begin{align*}
            j' \Zl_n(f) \lambda p
        =  f_n j \lambda p
        =  f_n i p
        =  f_n d_{n+1}
        =  d_{n+1} f_{n+1}
        =  i' p' f_{n+1}
        &=  j' \lambda' p' f_{n+1}  \\
        &=  j' \lambda' \Bl_n(f) p \,,
      \end{align*}
      and hence~$\Zl_n(f) \lambda = \lambda' \Bl_n(f)$ because~$p$ is an epimorphism and~$j'$ is a monomorphism.
      
      It follows from this commutativity of the frontal square
      \[
        \begin{tikzcd}[row sep = large]
            \Bl_n(\Ccc)
            \arrow{r}[above]{\lambda}
            \arrow{d}[left]{\Bl_n(f)}
          & \Zl_n(\Ccc)
            \arrow{d}[right]{\Zl_n(f)}
          \\
            \Bl_n(\Dcc)
            \arrow{r}[above]{\lambda'}
          & \Zl_n(\Dcc)
        \end{tikzcd}
      \]
      that there exists a unique induced morphism~$\Hl_n(f) \colon \Hl_n(\Ccc) \to \Hl_n(\Dcc)$ that makes the following diagram commute:
      \[
        \begin{tikzcd}[row sep = large]
            \Bl_n(\Ccc)
            \arrow{r}[above]{\lambda}
            \arrow{d}[left]{\Bl_n(f)}
          & \Zl_n(\Ccc)
            \arrow{r}
            \arrow{d}[right]{\Zl_n(f)}
          & \Hl_n(\Ccc)
            \arrow[dashed]{d}[right]{\Hl_n(f)}
          \\
            \Bl_n(\Dcc)
            \arrow{r}[above]{\lambda'}
          & \Zl_n(\Dcc)
            \arrow{r}
          & \Hl_n(\Dcc)
        \end{tikzcd}
      \]
    \end{enumerate}

    It follows from the functorialty of both~$\Bl_n$ and~$\Zl_n$, together with the functoriality of the cokernel (as explained in \cref{functoriality of (co)kernel}), that this induced morphism is both functorial and additive in the following sense:
    \begin{itemize}
      \item
        If~$\Dcc = \Ccc$ and~$f = \id_{\Ccc}$ then~$\Hl_n(\id_{\Ccc}) = \id_{\Hl_n(\Ccc)}$ for every~$n \in \Integer$.
      \item
        If~$\Ecc$ is another chain complex and~$g \colon \Dcc \to \Ecc$ is another morphism of chain complexes, then
        \[
            \Hl_n(g \circ f)
          = \Hl_n(g) \circ \Hl_n(f)
        \]
        for every~$n \in \Integer$.
      \item
        If~$g \colon \Ccc \to \Dcc$ is another morphism of chain complexes that is parallel to the morphism~$f$, then
        \[
            \Hl_n(f + g)
          = \Hl_n(f) + \Hl_n(g)
        \]
        for every~$n \in \Integer$.
    \end{itemize}
\end{remark}


\begin{remark*}
  We have seen in~\cref{Hn as cokernel with Cn} that the~\dash{$n$}{th} homology~~$\Hl_n(\Ccc)$ of a chain complex~$\Ccc$ can alse be described as the cokernel of the morphism~$C_{n+1} \to \Zl_n(\Ccc)$ that is induced by the morphism~$d_{n+1} \colon C_{n+1} \to C_n$ via the universal propery of the kernel~$\Zl_n(\Ccc) = \ker(d_n)$ .
  This description of the~\dash{$n$}{th} homology leads to the same induced morphism between homology objects:
  
  Let~$f \colon \Ccc \to \Dcc$ be a morphism of chain complexes.
  We have the following commutative diagram in~$\Acat$:
  \[
    \begin{tikzcd}[row sep = large]
        C_{n+1}
        \arrow{rr}[above]{d_{n+1}}
        \arrow[dashed]{dd}[left, near start]{f_{n+1}}
        \arrow{dr}
      & {}
      & C_n
        \arrow{rr}[above]{d_n}
        \arrow{dd}[left, very near start]{f_n}
      & {}
      & C_{n-1}
        \arrow{dd}[right, near start]{f_{n-1}}
      & {}
      \\
        {}
      & \Bl_n(\Ccc)
      & {}
      & \Zl_n(\Ccc)
        \arrow{ul}
        \arrow[dashed, from=ulll, crossing over]
        \arrow[from=ll, crossing over]
      & {}
      & \Hl_n(\Ccc)
        \arrow[dashed,from=ll, crossing over]
        \arrow[dashed]{dd}[right]{\Hl_n(f)}
      \\
        D_{n+1}
        \arrow{rr}[above, near start]{d_{n+1}}
        \arrow[dashed]{drrr}
        \arrow{dr}
      & {}
      & D_n
        \arrow{rr}[above, near start]{d_n}
      & {}
      & D_{n-1}
      & {}
      \\
        {}
      & \Bl_n(\Dcc)
        \arrow{rr}
        \arrow[from=uu, crossing over, "\Bl_n(f)", near start]
      & {}
      & \Zl_n(\Dcc)
        \arrow{ul}
        \arrow[dashed]{rr}
        \arrow[dashed,from=uu, crossing over, "\Zl_n(f)", near start]
      & {}
      & \Hl_n(\Dcc)
    \end{tikzcd}
  \]
  The commutativity of the dashed subdiagram
  \[
    \begin{tikzcd}
        C_{n+1}
        \arrow{r}
        \arrow{d}[left]{f_{n+1}}
      & \Zl_n(\Ccc)
        \arrow{r}
        \arrow{d}[right]{\Zl_n(f)}
      & \Hl_n(\Ccc)
        \arrow[dashed]{d}[right]{\Hl_n(f)}
      \\
        D_{n+1}
        \arrow{r}
      & \Zl_n(\Dcc)
        \arrow{r}
      & \Hl_n(\Dcc)
    \end{tikzcd}
  \]
  shows the claim.
\end{remark*}


\begin{example*}
  Let~$A$ be a~{\kalg}.
  Let~$\Ccc$ and~$\Dcc$ be chain complexes of~{\modules{$A$}}, i.e.\ chain complexes in the abelian category~$\Modl{A}$, and let~$f \colon \Ccc \to \Dcc$ be a morphism of chain complexes.
  Then~$f(\Zl_n(\Ccc)) \subseteq \Zl_n(\Dcc)$ and~$f(\Bl_n(\Ccc)) \subseteq \Bl_n(\Dcc)$ for every~$n \in \Integer$, and the homomorphisms~$\Zl_n(f)$ and~$\Bl_n(f)$ are the resulting restrictions of~$f$.
  The induced morphism~$\Hl_n(f) \colon \Hl_n(\Ccc) \to \Hl_n(\Dcc)$ is on elements given by
  \[
        \Hl_n(f)( \class{x} )
    =   \class{f(x)}
    \in \Hl_n(\Dcc)
  \]
  for every~$\class{x} \in \Hl_n(\Ccc)$.
\end{example*}


\begin{definition}
  A morphism~$f \colon \Ccc \to \Dcc$ of chain complexes is a \emph{{\qim}}\index{quasi-isomorism}\index{chain complex!quasi-isomorphic} if for every~$n \in \Integer$ the morphism~$\Hl_n(f) \colon \Hl_n(\Ccc) \to \Hl_n(\Dcc)$ is an isomorphism.
  Dually, a morphism~$f \colon \Cccc \to \Dccc$ of cochain complexes is a \emph{{\qim}} if for every~$n \in \Integer$ the morphism~$\Hl^n(f) \colon \Hl^n(\Cccc) \to \Hl^n(\Dccc)$ is an isomorphism.
\end{definition}


\begin{remarkdefinition}
  For a chain complex~$\Ccc$ in~$\Acat$ the following conditions are equivalent:
  \begin{enumerate}
    \item
      The sequence
      \[
        \dotsb
        \to
        C_{n+1}
        \xlongto{d_{n+1}}
        C_n
        \xlongto{d_n}
        C_{n-1}
        \to
        \dotsb
      \]
      is exact.
    \item
      It holds that~$\Hl_n(\Ccc) = 0$ for every~$n \in \Integer$.
    \item
      The morphism~$0 \to \Ccc$ is a {\qim}.
    \item[iii')]
%     TODO: Makes this less hacky.
      The morphism~$\Ccc \to 0$ is a {\qim}.
    \item
      The chain complex~$\Ccc$ is {\qic} to the zero complex.
  \end{enumerate}
  If the chain complex~$\Ccc$ satisfies these equivalent conditions then~$\Ccc$ is \emph{acyclic}\index{acyclic}\index{chain complex!acyclic}.
\end{remarkdefinition}


\begin{example}
  Consider the chain complex~$\Ccc$ of abelian groups, i.e.\ in the abelian category~$\Ab$, given by
  \[
              C_n
    \defined  \begin{cases}
                \Integer/8  & \text{if~$n \geq 0$}  \,, \\
                0           & \text{if~$n < 0$} \,,
              \end{cases}
  \]
  together with the differential morphisms~$d_n \colon C_n \to C_{n-1}$ given by
  \[
              d_n
    \defined  \begin{cases}
                \text{multiplication by~$4$}  & \text{if~$n > 0$} \,, \\
                0                             & \text{if~$n \leq 0$}  \,.
              \end{cases}
  \]
  This is indeed a chain complex, and its homology is given by
  \[
          \Hl_n(\Ccc)
    \cong \begin{cases}
            0           & \text{if~$n < 0$}     \,, \\
            \Integer/4  & \text{if~$n = 0$}     \,, \\
            \Integer/2  & \text{if~$n \geq 1$}  \,. \\
          \end{cases}
  \]
  Indeed, we have for every~$n < 0$ that~$C_n = 0$, hence~$\Zl_n(\Ccc) = 0$ and therefore also~$\Hl_n(\Ccc) = 0$.
  For~$n = 0$ we have that~$\Zl_0(\Ccc) = \Integer/8$ and~$\Bl_0(\Ccc) = 4\Integer/8$, and hence that
  \[
          \Hl_0(\Ccc)
    =     \Zl_0(\Ccc) / \Bl_0(\Ccc)
    =     (\Integer/8) / (4\Integer/8)
    \cong \Integer/4 \,.
  \]
  For~$n \geq 1$ we have that~$\Zl_n(\Ccc) = 2\Integer/8$ and~$\Bl_n(\Ccc) = 4\Integer/8$, and hence that
  \[
          \Hl_n(\Ccc)
    =     \Zl_n(\Ccc) / \Bl_n(\Ccc)
    =     (2 \Integer/8) / (4\Integer/8)
    \cong 2\Integer/4
    \cong \Integer/2 \,.
  \]
\end{example}


\begin{example}
  For every~$n \geq 0$ let
  \[
              \Delta^n
    \defined  \conv(e_0, \dotsc, e_n)
    =         \left\{
                (t_0, \dotsc, t_n) \in \Real^{n+1}
              \suchthat*
                t_0, \dotsc, t_n \geq 0,
                \sum_{i=0}^n t_i = 1
              \right\}
  \]
  be the \emph{standard~\simplex{$n$}}, together with the usual topology.
  For every~$k = 0, \dotsc, n$ let
  \[
            f^{(n)}_k
    \colon  \Delta^{n-1}
    \to     \Delta^n \,,
    \quad   (t_0, \dotsc, t_{n-1})
    \mapsto (t_0, \dotsc, t_{i-1}, 0, t_i, \dotsc, t_{n-1})
  \]
  be the inclusion of~$\Delta^{n-1}$ into~$\Delta^n$ as the~\dash{$k$}{th} face.
  
  Let now~$X$ be a topological space.
  A~\simplex{$n$} in~$X$ is a continuous map~$\sigma \colon \Delta^n \to X$.
  For every~$n \geq 0$ let
  \[
              C^\sing_n(X)
    \defined  \text{free abelian group on the set~$\{ \text{\simplices{$n$} $\sigma \colon \Delta^n \to X$} \}$} \,,
  \]
  and for every~$n < 0$ let~$C^\sing_n(X) \defined 0$.
  We define for every~$n \in \Integer$ a differential
  \[
    d^\sing_n
    \colon
    C^\sing_n(X)
    \to
    C^\sing_{n-1}(X)
  \]
  by~$d^\sing_n = 0$ for~$n \leq 0$, and for~$n > 0$ on basis elements by
  \[
      d^\sing_n(\sigma)
    = \sum_{i=0}^n (-1)^i \sigma \circ f^{(n)}_i
  \]
  for every~{\simplex{$n$}}~$\sigma$ in~$X$.
  This resulting chain complex
  \[
              \Ccc^\sing
    \defined  ( (C^\sing_n(X))_{n \in \Integer}, (d_n^\sing)_{n \in \Integer} )
  \]
  is the \emph{singular chain complex}\index{singular!chain complex}\index{chain complex!singular} of~$X$.
  (That this really defines a chain complex of abelian groups follows from Exercise~4 of Exercise~sheet~8.)
  Its~\dash{$n$}{th} homology
  \[
              \Hl_n^\sing(X)
    \defined  \Hl_n( \Ccc^\sing(X) )
  \]
  is the~\dash{$n$}{th} \emph{singular homology}\index{singular!homology}\index{homology!singular} of~$X$.
  
  The \emph{singular cochain complex}\index{singular!cochain complex}\index{chochain complex!singular} of~$X$ is denoted by~$\Cccc_\sing(X)$;
  it is given by
  \[
              C^n_\sing(X)
    \defined  \Hom_\Integer( C^\sing_n(X), \Integer )
  \]
  for every~$n \in \Integer$, and the differential~$d^n_\sing \colon C^n_\sing(X) \to C^{n+1}_\sing(X)$ is for every~$n \in \Integer$ the dual map to~$d^\sing_{n+1} \colon C_{n+1}^\sing(X) \to C_n^\sing(X)$.
  The~\dash{$n$}{th} cohomology of this cochain complex~$\Cccc_\sing(X)$,
  \[
              \Hl^n_\sing(X)
    \defined  \Hl^n( \Cccc_\sing(X) ) \,,
  \]
  is the~\dash{$n$}{th} \emph{singular cohomology}\index{singular!cohomology}\index{cohomology!singular} of~$X$.
\end{example}


\begin{lemma}
  \label{chain complexes are additive}
  The categories~$\Ch(\Acat)$ and~$\CCh(\Acat)$ are additive.
  Moreover, both sums of morphisms and biproducts can be computed componentwise.
\end{lemma}


\begin{proof}
  It sufficies to consider the category~$\Ch(\Acat)$.
  For any two parallel morphisms of chain complexes~$f, g \colon \Ccc \to \Dcc$ their sum is given by
  \[
              (f + g)_n
    \defined  f_n + g_n
  \]
  for every~$n \in \Integer$.
  This makes~$\Ch(\Acat)(\Ccc, \Dcc)$ into an abelian group.
  
  The category~$\Ch(\Acat)$ has biproducts:
  Let~$\Ccc^{(1)}, \dotsc, \Ccc^{(k)}$ be chain complexes in~$\Acat$.
  Their biproduct~$\Ccc$ is given by
  \[
              C_n
    \defined  C^{(1)}_n \oplus \dotsb \oplus C^{(k)}_n
  \]
  for every~$n \in \Integer$, together with the differentials~$d_n \colon C_n \to C_{n-1}$ given by
  \[
              d_n
    \defined  \begin{bmatrix}
                d^{(1)}_n &         &           \\
                          & \ddots  &           \\
                          &         & d^{(k)}_n
              \end{bmatrix}
    \colon    C_n
    \to       C_{n-1}
  \]
  for every~$n \in \Integer$.
  The canonical morphisms~$c_i \colon \Ccc^{(i)} \to \Ccc$ and~$p_i \colon \Ccc \to \Ccc^{(i)}$ are given in components by
  \[
              c_i
    \defined  (c_{i,n})_{n \in \Integer}
    \quad\text{and}\quad
              p_i
    \defined  (p_{i,n})_{n \in \Integer} \,,
  \]
  where
  \[
            c_{i,n}
    \colon  C^{(i)}_n
    \to     C^{(1)}_n \oplus \dotsb \oplus C^{(k)}_n
    \quad\text{and}\quad
            p_{i,n}
    \colon  C^{(1)}_n \oplus \dotsb \oplus C^{(k)}_n
    \to     C^{(i)}_n
  \]
  are for every~$n \in \Integer$ the canonical morphisms belonging to the biproduct in~$\Acat$.
  These are indeed morphisms of chain complexes.
% TODO: Check that these are indeed morphisms of chain complexes.
  It remains check that~$p_j c_i = 0$ for all~$i \neq j$, and that~$\sum_{i=1}^k c_i p_i = \id_{\Ccc}$.
  This holds true because it holds componentwise.
\end{proof}


\begin{lemma}
  \label{chain complexes have (co)kernels}
  Let~$f \colon \Ccc \to \Dcc$ be a morphism of chain complexes.
  It follows for every~$n \in \Integer$ from the commutativity of the square
  \[
    \begin{tikzcd}
        C_n
        \arrow{r}[above]{d_n}
        \arrow{d}[left]{f_n}
      & C_{n-1}
        \arrow{d}[right]{f_{n-1}}
      \\
        D_n
        \arrow{r}[above]{d_n}
      & D_{n-1}
    \end{tikzcd}
  \]
  that there exist unique morphisms
  \begin{gather*}
    d^{\,\ker}_n \colon \ker(f_n) \to \ker(f_{n-1})
  \shortintertext{and}
    d^{\,\coker}_n \colon \coker(f_n) \to \coker(f_{n-1})
  \end{gather*}
  that make the following two squares commute:
  \[
    \begin{tikzcd}[column sep = large]
        \ker(f_n)
        \arrow[dashed]{r}[above]{d^{\,\ker}_n}
        \arrow{d}[left]{k_n}
      & \ker(f_{n-1})
        \arrow{d}[right]{k_{n-1}}
      \\
        C_n
        \arrow{r}[above]{d_n}
      & C_{n-1}
    \end{tikzcd}
    \qquad\qquad
    \begin{tikzcd}[column sep = large]
        D_n
        \arrow{r}[above]{d_n}
        \arrow{d}[left]{c_n}
      & D_{n-1}
        \arrow{d}[right]{c_{n-1}}
      \\
        \coker(f_n)
        \arrow[dashed]{r}[above]{d^{\,\coker}_n}
      & \coker(f_{n-1})
    \end{tikzcd}
  \]
  Then~$\ker(f) \defined ( (\ker(f_n))_{n \in \Integer}, (d^{\,\ker}_n)_{n \in \Integer} )$ is again a chain complex,~$k \defined (k_n)_{n \in \Integer}$ is a morphism of chain complexes~$k \colon \ker(f) \to \Ccc$, and~$k$ is a kernel of~$f$.
  Dually,~$\coker(f) \defined ( (\coker(f_n))_{n \in \Integer}, (d^{\,\coker}_n)_{n \in \Integer} )$ is again a chain complex,~$c \defined (c_n)_{n \in \Integer}$ is a morphism of chain complexes~$c \colon \Dcc \to \coker(f)$, and~$c$ is a cokernel of~$f$.
\end{lemma}


\begin{proof}
  This is part of Exercise~1 of Exercise~sheet~8.
\end{proof}





\lecturend{15}


\begin{theorem}
  The categories~$\Ch(\Acat)$ and~$\CCh(\Acat)$ are again abelian.
\end{theorem}


\begin{proof}
  It sufficies to show that~$\Ch(\Acat)$ is abelian.
  It follows from \cref{chain complexes are additive} and \cref{chain complexes have (co)kernels} that the category~$\Ch(\Acat)$ is additive and has kernels and cokernels.
  It remains to show that for a morphism~$f = (f_n)_{n \in \Integer} \colon \Ccc \to \Dcc$ of chain complexes the induced morphism~$\tilde{f} \colon \coim(f) \to \im(f)$, i.e.\ the unique morphism that makes the square
  \[
    \begin{tikzcd}
        \Ccc
        \arrow{r}[above]{f}
        \arrow{d}
      & \Dcc
        \arrow{d}
      \\
        \coim(f)
        \arrow[dashed]{r}[above]{\tilde{f}}
      & \im(f)
    \end{tikzcd}
  \]
  commute, is an isomorphism.
  Let~$\Ecc \defined \coim(f)$ and~$\Fcc \defined \im(f)$.
  It follows from \cref{chain complexes have (co)kernels} that~$C_n \to E_n$ is for every~$n \in \Integer$ a coimage of~$f_n$ and that~$F_n \to D_n$ is for every~$n \in \Integer$ an image of~$f_n$.
  It holds for~$\tilde{f} = (\tilde{f}_n)_{n \in \Integer}$ that~$\tilde{f}_n \colon E_n \to F_n$ is for every~$n \in \Integer$ the canonical morphism from the \hyperref[canonical factorization]{canonical factorization lemma} induced by~$f_n$, because the square
  \[
    \begin{tikzcd}
        C_n
        \arrow{r}[above]{f_n}
        \arrow{d}
      & D_n
        \arrow{d}
      \\
        \coim(f_n)
        \arrow[dashed]{r}[above]{\tilde{f}_n}
      & \im(f_n)
    \end{tikzcd}
  \]
  commutes.
  It follows that~$\tilde{f}_n$ is an isomorphism for every~$n \in \Integer$ because~$\Acat$ is abelian.
  Hence~$\tilde{f}$ is an isomorphism.
\end{proof}


\begin{remark*}
  \label{exactness for chain complexes is computed degreewise}
  Kernels and cokernels in the abelian categories~$\Ch(\Acat)$ and~$\CCh(\Acat)$ are computed degreewise, so the same goes for images and coimages.
  It follows that a sequence~$\Ccc' \to \Ccc \to \Ccc''$ of chain complexes in~$\Acat$ (resp.\ a sequence~$\bp\Cccc \to \Cccc \to \bpp\Cccc$ of cochain complexes in~$\Acat$) is exact if and only if it is exact in each degree, i.e.\ if and only if the sequence~$C'_n \to C_n \to C''_n$ (resp.\ the sequence~$\bp C^n \to C^n \to \bpp C^n$) is exact for every~$n \in \Integer$.
\end{remark*}




