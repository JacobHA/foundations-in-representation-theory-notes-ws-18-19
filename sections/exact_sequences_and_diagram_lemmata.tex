\section{Exact Sequences and Diagram Lemmata}


\begin{remarknonum}(Orally)
  We will (at least for now) neither use nor state the Freyd--Mitchell embedding theorem.
  We will instead explain how one can \enquote{diagram chase} in abelian categories.
\end{remarknonum}


\begin{remarkdefinition}
  \leavevmode
  \begin{enumerate}
    \item
      Let~$f \colon X \to Y$ and~$g \colon Y \to Z$ be two composable morphisms in~$\Acat$.
      If~$g f = 0$ then there exists a unique morphism~$\lambda \colon \im(f) \to \ker(g)$ that makes the diagram
      \[
        \begin{tikzcd}[column sep = small, row sep = large]
            X
            \arrow{rr}[above]{f}
            \arrow{dr}[below left]{f'}
          & {}
          & Y
            \arrow{rr}[above]{g}
          & {}
          & Z
          \\
            {}
          & \im(f)
            \arrow{ur}[above left]{i}
            \arrow[dashed]{rr}[above]{\lambda}
          & {}
          & \ker(g)
            \arrow{ul}[above right]{k}
          & {}
        \end{tikzcd}
      \]
      commute, and this morphism is a monomorphism.
      Indeed, it holds that
      \[
          g i f'
        = g f
        = 0
      \]
      and hence~$g i = 0$ because~$f'$ is an epimorphism.
      The existence and uniqueness of~$\lambda$ thus follow from the universal property of the kernel~$k \colon \ker(g) \to Y$.
      
      The sequence~$X \xto{f} Y \xto{g} Z$ is \emph{exact}\index{exact!sequence}\index{sequence!exact} if
      \begin{enumerate}[label=(E\arabic*)]
        \item
          $gf = 0$, and
        \item
          the canonical morphism~$\lambda \colon \im(f) \to \ker(g)$ is an ismorphism.
      \end{enumerate}
    \item
      A (possibly infinite) sequence
      \[
          \dotsb
        \longto
          X_{i-1}
        \xlongto{f_{i-1}}
          X_i
        \xlongto{f_i}
          X_{i+1}
        \longto
          \dotsb
      \]
      of composable morphisms in~$\Acat$ is \emph{exact}\index{exact sequence} if at every position~$i$ for which both an incoming morphism~$f_{i-1} \colon X_{i-1} \to X_i$ and an outgoing morphism~$f_i \colon X_i \to X_{i+1}$ exist, the resulting sequence
      \[
          X_{i-1}
        \xlongto{f_{i-1}}
          X_i
        \xlongto{f_i}
          X_{i+1}
      \]
      is exact.
    \item
      We have the following special cases of exact sequences:
      \begin{itemize}
        \item
          A sequence of the form
          \[
            0
            \to
            X
            \xlongto{f}
            Y
          \]
          is exact if and only if the morphism~$f$ is a monomorphism.
        \item
          A sequence of the form
          \[
            Y
            \xlongto{g}
            Z
            \to
            0
          \]
          is exact if and only if the morphism~$g$ is an epimorphism.
        \item
          A sequence of the form
          \[
            0
              \to
            X
              \xlongto{f}
            Y
              \xlongto{g}
            Z
              \to
            0
          \]
          is exact if and only if the morphism~$f$ is a monomorphism, the morphism~$g$ is an epimorphism, and the canonical morphism~$\im(f) \to \ker(g)$ is an isomorphism.
          This is furthermore equivalent to~$f$ being a kernel of~$g$ and~$g$ being a cokernel of~$f$ (at the same time).
      \end{itemize}
  \end{enumerate}
\end{remarkdefinition}


\begin{definition*}
  Let~$\Acat$ be an abelian category.
  \begin{enumerate}
    \item
      A \emph{short exact sequence}\index{short exact sequence}\index{exact sequence!short}\index{sequence!short exact} in~$\Acat$ is an exact sequence of the form
      \[
            0
        \to X'
        \to X
        \to X''
        \to 0 \,.
      \]
    \item
      A \emph{left exact sequence}\index{left exact!sequence}\index{exact!sequence!left}\index{sequence!left exact} in~$\Acat$ is an exact sequence of the form
      \[
            0
        \to X'
        \to X
        \to X'' \,.
      \]
    \item
      A \emph{right exact sequence}\index{right exact!sequence}\index{exact!sequence!right}\index{sequence!right exact} in~$\Acat$ is an exact sequence of the form
      \[
            X'
        \to X
        \to X''
        \to 0 \,.
      \]
  \end{enumerate}
\end{definition*}


\begin{remark*}
  \label{language of left and right exact}
  Let~$\Acat$ be an abelian category.
  \begin{enumerate}
    \item
      Sometimes a sequence~$X' \to X \to X''$ in~$\Acat$ is called \emph{short exact}\index{short exact sequence}\index{exact sequence!short}\index{sequence!short exact} if the corresponding sequence~$0 \to X' \to X \to X'' \to 0$ is short exact.
      
      Similarly, a sequence~$X' \to X \to X''$ is sometimes called \emph{left exact}\index{left exact sequence}\index{exact!sequence!left}\index{sequence!left exact} (resp.\ \emph{right exact}\index{right exact sequence}\index{exact!sequence!right}\index{sequence!right exact}) if the sequence~$0 \to X' \to X \to X''$ is (left) exact (resp.\ if the sequence~$X' \to X \to X'' \to 0$ is (right) exact).
    \item
      A sequence~$0 \to X' \to X \to X''$ in~$\Acat$ is (left) exact if and only if the morphism~$X' \to X$ is a kernel of the morphism~$X \to X''$.
      
      Dually, a sequence~$X' \to X \to X'' \to 0$ is (right) exact if and only if the morphism~$X \to X''$ is a cokernel of the morphism~$X' \to X$.
  \end{enumerate}
\end{remark*}


\begin{proposition}
  \label{kernels of pullbacks}
  Let~$\Acat$ be an abelian category and let
  \[
    \begin{tikzcd}[sep = large]
        {}
      & {}
      & Y'
        \arrow{r}[above]{g'}
        \arrow{d}[right]{h'}
        \arrow[phantom]{dr}[description]{\pb}
      & Z'
        \arrow{d}[right]{h}
      & {}
      \\
        0
        \arrow{r}
      & X
        \arrow{r}[above]{f}
      & Y
        \arrow{r}[above]{g}
      & Z
        \arrow{r}
      & 0
    \end{tikzcd}
  \]
  be a commutative diagram in~$\Acat$ where the bottom row is exact and the square is a pullback square.
  Then there exists a unique morphism~$f' \colon X \to Y'$ such that the diagram
  \begin{equation}
    \label{resulting pb of ses}
    \begin{tikzcd}[sep = large]
        0
        \arrow{r}
      & X
        \arrow[dashed]{r}[above]{f'}
        \arrow[equal]{d}
      & Y'
        \arrow{r}[above]{g'}
        \arrow{d}[right]{h'}
        \arrow[phantom]{dr}[description]{\pb}
      & Z'
        \arrow{d}[right]{h}
        \arrow{r}
      & 0
      \\
        0
        \arrow{r}
      & X
        \arrow{r}[above]{f}
      & Y
        \arrow{r}[above]{g}
      & Z
        \arrow{r}
      & 0
    \end{tikzcd}
  \end{equation}
  commutes and such that the upper row
  \begin{equation}
    \label{upper row}
    0
    \to
    X'
    \xlongto{f'}
    Y'
    \xlongto{g'}
    Z'
    \to
    0
  \end{equation}
  is exact.
\end{proposition}


\begin{proof}
  It follows from \cref{mono epi under pull push} that~$g'$ is an epimorphism because~$g$ is one.
  
  To construct the desired morphism~$f' \colon X \to Y$ we use that the given square is a pullback square:
  The diagram
  \[
    \begin{tikzcd}[sep = large]
        X
        \arrow[dashed, bend left]{drr}[above right]{0}
        \arrow[bend right]{ddr}[below left]{f}
      & {}
      & {}
      \\
        {}
      & Y'
        \arrow{r}[above]{g'}
        \arrow{d}[right]{h'}
        \arrow[phantom]{dr}[description]{\pb}
      & Z'
        \arrow{d}[right]{h}
      \\
        {}
      & Y
        \arrow{r}[above]{g}
      & Z
    \end{tikzcd}
  \]
  commutes because~$gf = 0$.
  So it follows that there exists a unique morphism~$f' \colon X \to Y'$ that makes the diagram
  \[
    \begin{tikzcd}[sep = large]
        X
        \arrow[bend left]{drr}[above right]{0}
        \arrow[dashed]{dr}[above right]{f'}
        \arrow[bend right]{ddr}[below left]{f}
      & {}
      & {}
      \\
        {}
      & Y'
        \arrow{r}[above]{g'}
        \arrow{d}[right]{h'}
        \arrow[phantom]{dr}[description]{\pb}
      & Z'
        \arrow{d}[right]{h}
      \\
        {}
      & Y
        \arrow{r}[above]{g}
      & Z
    \end{tikzcd}
  \]
  commute.
  This means precisely that the morphism~$f'$ makes the diagram~\eqref{resulting pb of ses} commute and satisfies~$g' f' = 0$.
  The morphism~$f$ is a monomorphism because the composition~$h' f' = f$ is a monomorphism.
  
  To show the exactness of the row~\eqref{upper row} it remains to show that~$f'$ is already a kernel of~$g'$.
  So let~$u \colon W \to Y'$ be a morphism with~$g' u = 0$.
  \[
    \begin{tikzcd}[sep = large]
        {}
      & {}
      & W
        \arrow[dashed]{d}[right]{u}
        \arrow[dashed, bend left]{dr}[above right]{0}
      & {}
      & {}
      \\
        0
        \arrow{r}
      & X
        \arrow{r}[above]{f'}
        \arrow[equal]{d}
      & Y'
        \arrow{r}[above]{g'}
        \arrow{d}[right]{h'}
        \arrow[phantom]{dr}[description]{\pb}
      & Z'
        \arrow{d}[right]{h}
        \arrow{r}
      & 0
      \\
        0
        \arrow{r}
      & X
        \arrow{r}[above]{f}
      & Y
        \arrow{r}[above]{g}
      & Z
        \arrow{r}
      & 0
    \end{tikzcd}
  \]
  Then
  \[
      g h' u
    = h g' u
    = h \circ 0
    = 0
  \]
  and hence~$h' u$ factors uniquely over the kernel of~$g$, which is~$f$.
  So there exists a unique morphism~$\lambda \colon W \to X$ with~$f \lambda = h' u$.
  \[
    \begin{tikzcd}[sep = large]
        {}
      & {}
      & W
        \arrow{d}[right]{u}
        \arrow[bend left]{dr}[above right]{0}
        \arrow[dashed,bend right]{dl}[above left]{\lambda}
      & {}
      & {}
      \\
        0
        \arrow{r}
      & X
        \arrow{r}[above, near end]{f'}
        \arrow[equal]{d}
      & Y'
        \arrow{r}[above, near start]{g'}
        \arrow{d}[right]{h'}
        \arrow[phantom]{dr}[description]{\pb}
      & Z'
        \arrow{r}
      & 0
      \\
        0
        \arrow{r}
      & X
        \arrow[from=uur, dashed, bend right = 15, near start, swap, "\lambda", crossing over]
        \arrow{r}[above]{f}
      & Y
        \arrow{r}[above]{g}
      & Z
        \arrow{r}
        \arrow[from=uul, dashed, bend left = 15, near start, "0", crossing over]
        \arrow[from=u, right, "h"]
      & 0
    \end{tikzcd}
  \]
  To show that~$u = f' \lambda$, i.e.\ that the above diagram commutes, we again use that the \dash{right}{hand} square is a pullback square:
  The two parallel morphisms~$u, f' \lambda \colon W \to Y'$ coincide if and only if they coincide after composition with both~$g'$ and~$h'$.
  We have that
  \[
      h' u
    = f \lambda
    = f \id_X \lambda
    = h' f' \lambda
  \]
  and also
  \[
      g' u
    = 0
    = g' f' \lambda
  \]
  because~$g' f' = 0$.
  Hence~$u = f' \lambda$, which shows the existence of the desired morphism~$\lambda$.
  The morphism~$\lambda$ is uniquely determined by the composition~$f' \lambda = u$ because~$f'$ is a monomorphism.
  This shows the uniqueness of~$\lambda$.
\end{proof}


\begin{remark*}
  One can reformulate \cref{kernels of pullbacks}:
  If
  \[
    \begin{tikzcd}
        0
        \arrow{r}
      & X'
        \arrow{r}[above]{f'}
        \arrow{d}[right]{h'}
      & Y'
        \arrow{r}[above]{g'}
        \arrow{d}[right]{h}
        \arrow[phantom]{dr}[description]{\pb}
      & Z'
        \arrow{r}
        \arrow{d}[right]{h''}
      & 0
      \\
        0
        \arrow{r}
      & X
        \arrow{r}[above]{f}
      & Y
        \arrow{r}[above]{g}
      & Z
        \arrow{r}
      & 0
    \end{tikzcd}
  \]
  is a commutative diagram with (short) exact rows such that the \dash{right}{hand} square is a pullback square, then~$h' \colon X' \to X$ is an isomorphism.
  Hence kernels of epimorphisms stay the same under pullbacks.
% TODO: Explain this in more detail.
\end{remark*}


\begin{remarkdefinition}[label=abstract points]
  Let~$\Acat$ be an abelian category and let~$X$ be an object in~$\Acat$.
  \begin{enumerate}
    \item
      For every object~$A \in \Ob(\Acat)$ let
      \[
                  X(A)
        \defined  \Hom_\Acat(A,X) \,.
      \]
      The elements of~$X(A)$ are the~\emph{\dash{$A$}{valued} points}\index{points of an object}\index{object!points of} of~$X$.
    \item
      We denote by~$x  \inA X$ that~$x$ is an~\dash{$A$}{valued} points of~$X$ for some~$A \in \Ob(\Acat)$.
  \end{enumerate}
\end{remarkdefinition}


\begin{remark*}
  \label{motivation for abstract elements}
  Let~$\Acat$ be an abelian category and let~$f \colon X \to Y$ be a morphism in~$\Acat$.
  \begin{enumerate}
    \item
      \label{elements of kernel}
      We can reformulate the universal property of the kernel in terms of points of objects:
      A morphism~$k \colon K \to X$ is a kernel of~$f$ if and only if for every point~$x \in X$ with~$fx = 0$ there exist a unique point~$\tilde{x} \inA K$ with~$k \tilde{x} = x$.
      (If~$x$ is an~\dash{$A$}{valued} point then it automatically follows that~$\tilde{x}$ is again~\dash{$A$}{valued}, because~$f \tilde{x}$ is~\dash{$A$}{valued}.)
    \item
      However, another similar rule from diagram chasing does not hold (or at least not up to equality, as we will see below):
      If the morphism~$f$ is an epimorphism and~$y \inA Y$ is a point, say~$y \in Y(A)$, then there does not have to exist a point~$x \in X$ with~$fx = y$.
      In other words, there does not have to exist a lift~$x \colon A \to X$ of the morphism~$y \colon A \to Y$ along~$f$; here we mean by \enquote{lift} a morphism that makes the triangle
      \[
        \begin{tikzcd}
            {}
          & X
            \arrow{d}[right]{f}
          \\
            A
            \arrow[dashed]{ur}[above left]{x}
            \arrow{r}[below]{y}
          & Y
        \end{tikzcd}
      \]
      commute.
      Indeed, we may consider as a (counter)example the abelian category~$\Acat = \Ab$ and choose~$f \colon \Integer \to \Integer/2$ to be the canonical projection.
      Then for the morphism~$\id_\Integer \colon \Integer/2 \to \Integer/2$ no such lift~$\Integer/2 \to \Integer$ exists.
      
      We will in the following circumvent this problem by relaxing under what conditions we consider two points of~$X$ to be \enquote{the same}:
      Instead of equality of points we will work with equivalence of points, as we will now explain.
  \end{enumerate}
\end{remark*}


\begin{remarkdefinition}[continues=abstract points]
  \leavevmode
  \begin{enumerate}[start=3]
    \item
      Two points~$x, y \inA X$, say~$x \in X(A)$ and~$y \in X(B)$, are \emph{equivalent}\index{equivalence!of points of an object}\index{points of an object!equivalence} if there exists for some object~$C \in \Ob(\Acat)$ epimorphisms~$u \colon C \to A$ and~$v \colon C \to B$ that make the square
      \[
        \begin{tikzcd}
            C
            \arrow[dashed]{r}[above]{u}
            \arrow[dashed]{d}[left]{v}
          & A
            \arrow{d}[right]{x}
          \\
            B
            \arrow{r}[above]{y}
          & X
        \end{tikzcd}
      \]
      commute.
      That the points~$x$ and~$y$ are equivalent is denoted by~$x \equiv y$.
      
      This concept of equivalence~$\equiv$ defines an equivalence relation on the class of points of~$X$:
      Every point~$x \inA X$, say~$x \in X(A)$, is equivalent to itself because square
      \[
        \begin{tikzcd}
            A
            \arrow[dashed]{r}[above]{\id_A}
            \arrow[dashed]{d}[left]{\id_A}
          & A
            \arrow{d}[right]{x}
          \\
            A
            \arrow{r}[above]{x}
          & X
        \end{tikzcd}
      \]
      commutes.
      The relation~$\equiv$ is symmetric because the definition of~$x \equiv y$ is symmetric in~$x$ and~$y$.
      Let~$x, y, z \inA X$, say
      \[
        x \in X(A) \,,
        \quad
        y \in X(B) \,,
        \quad
        z \in X(C) \,,
      \]
      with~$x \equiv y$ and~$y \equiv z$.
      Let
      \[
        u \colon D \to A \,,
        \quad
        v \colon D \to B \,,
        \quad
        r \colon E \to B \,,
        \quad
        s \colon E \to C \,,
      \]
      be epimorphism that make the squares
      \[
        \begin{tikzcd}
            D
            \arrow[dashed]{r}[above]{u}
            \arrow[dashed]{d}[left]{v}
          & A
            \arrow{d}[left]{x}
          \\
            B
            \arrow{r}[above]{y}
          & X
        \end{tikzcd}
        \qquad\text{and}\qquad
        \begin{tikzcd}
            E
            \arrow[dashed]{r}[above]{r}
            \arrow[dashed]{d}[left]{s}
          & B
            \arrow{d}[left]{y}
          \\
            C
            \arrow{r}[above]{z}
          & X
        \end{tikzcd}
      \]
      commute.
      Together with the pullback square
      \[
        \begin{tikzcd}
            F
            \arrow[dashed]{r}[above]{r'}
            \arrow[dashed]{d}[left]{v'}
            \arrow[phantom]{dr}[description]{\pb}
          & D
            \arrow{d}[left]{v}
          \\
            E
            \arrow{r}[above]{r}
          & B
        \end{tikzcd}
      \]
      we get the following commutative diagram:
      \[
        \begin{tikzcd}[sep = large]
            F
            \arrow[dashed]{r}[above]{r'}
            \arrow[dashed]{d}[left]{v'}
            \arrow[phantom]{dr}[description]{\pb}
          & D
            \arrow[dashed]{r}[above]{u}
            \arrow[dashed]{d}[left]{v}
          & A
            \arrow{d}[left]{x}
          \\
            E
            \arrow[dashed]{r}[above]{r}
            \arrow[dashed]{d}[left]{s}
          & B
            \arrow{r}[above]{y}
            \arrow{d}[left]{y}
          & X
            \arrow[dashed]{d}[left]{\id_X}
          \\
            C
            \arrow{r}[above]{z}
          & X
            \arrow[dashed]{r}[above]{\id_X}
          & X
        \end{tikzcd}
      \]
      The morphisms~$v'$ and~$r'$ are epimorphisms by \cref{mono epi under pull push} because the morphisms~$v$ and~$r$ are epimorphisms.
      The compositions~$u r' \colon F \to A$ and~$s v' \colon F \to C$ are therefore epimorphisms, and they make the outer square
      \[
        \begin{tikzcd}[sep = large]
            F
            \arrow[dashed]{r}[above]{u r'}
            \arrow[dashed]{d}[left]{s v'}
          & A
            \arrow{d}[left]{x}
          \\
            C
            \arrow{r}[above]{z}
          & X
        \end{tikzcd}
      \]
      commute.
      This shows that also~$x \equiv z$.
    \item
      Let~$0$ be the zero morphism~$0_\Acat \to X$.
      It holds for every point~$x \inA X$ that~$x \equiv 0$ if and only if there exists an epimorphism~$u \colon B \to A$ that makes the square
      \[
        \begin{tikzcd}
            B
            \arrow[dashed]{r}[above]{u}
            \arrow[dashed]{d}
          & A
            \arrow{d}[right]{x}
          \\
            0_\Acat
            \arrow{r}
          & X
        \end{tikzcd}
      \]
      commute (because the zero morphism~$B \to 0$ is automatically an epimorphism), i.e.\ such that~$xu = 0_{B,X}$.
      It then follows that~$x = 0_{A,X}$ because~$u$ is an epimorphism;
      and if on the other hand~$x = 0_{A,X}$ then we can choose~$B = A$ and~$u = \id_A$ for the above commutative square.
      This shows that~$x \equiv 0$ if and only if~$x$ is the zero morphism~$A \to X$.
      
      This shows in particular that the various zero morphisms~$A \to X$ with~$A \in \Ob(\Acat)$ all give the same point of~$X$, namely~$0 \inA X$.
    \item
      Let~$Y \in \Ob(\Acat)$ be another object and let~$f \colon X \to Y$ be a morphism.
      Then~$f$ induces for every object~$A \in \Ob(\Acat)$ a map
      \[
                X(A)
        \to     Y(A) \,,
        \quad   x
        \mapsto fx \,.
      \]
      If~$x, y \inA X$ with~$x \equiv y$ then also~$fx \equiv fy$:
      If~$x \in X(A)$ and~$y \in X(B)$ then there exist epimorphisms~$u \colon C \colon A$ and~$v \colon C \to B$ that make the square
      \[
        \begin{tikzcd}
            C
            \arrow[dashed]{r}[above]{u}
            \arrow[dashed]{d}[left]{v}
          & A
            \arrow{d}[right]{x}
          \\
            B
            \arrow{r}[above]{y}
          & X
        \end{tikzcd}
      \]
      commute.
      We then get the following commutative diagram:
      \[
        \begin{tikzcd}
            C
            \arrow{r}[above]{u}
            \arrow{d}[left]{v}
          & A
            \arrow[dashed, bend left]{ddr}[above right]{fx}
            \arrow{d}[left]{x}
          & {}
          \\
            B
            \arrow{r}[above]{y}
            \arrow[dashed, bend right]{drr}[below left]{fy}
          & X
            \arrow{dr}[above right]{f}
          & {}
          \\
            {}
          & {}
          & Y
        \end{tikzcd}
      \]
      The commutativity of the outer square
      \[
        \begin{tikzcd}
            C
            \arrow[dashed]{r}[above]{u}
            \arrow[dashed]{d}[left]{v}
          & A
            \arrow{d}[right]{fx}
          \\
            B
            \arrow{r}[above]{fy}
          & Y
        \end{tikzcd}
      \]
      shows that~$fx \equiv fy$.
  \end{enumerate}
\end{remarkdefinition}


\begin{remark*}
  Let~$\Acat$ be an abelian category and let~$X$ be an object in~$\Acat$.
  
  If~$x \inA X$, say~$x \in X(A)$, and~$u \colon A' \to A$ is an epimorphism, then~$x \equiv xu$.
  This follows from the commutativity of the following square:
  \[
    \begin{tikzcd}
        A'
        \arrow[dashed]{r}[above]{u}
        \arrow[dashed]{d}[left]{\id_{A'}}
      & A
        \arrow{d}[right]{x}
      \\
        A'
        \arrow{r}[above]{xu}
      & X
    \end{tikzcd}
  \]
  
  The equivalence relation~$\equiv$ is already determined by this property:
  Let~$\equiv'$ be the equivalence relation on the class of points of~$X$ that is generated by~$x \equiv' y$ for all~$x, y \inA X$ for which there exists an epimorphism~$u$ with~$y = xu$.
  Then the equivalence relations~$\equiv$ and $\equiv'$ coincide.
  
  Indeed, we have seen above that the equivalence relation~$\equiv$ is finer than the equivalence relation~$\equiv'$.
  Suppose on the other hand  that $x, y \inA X$ are points with~$x \equiv y$, say~$x \in X(A)$ and~$y \in X(B)$.
  Then let~$u \colon C \to A$ and~$v \colon C \to B$ be epimorphisms that make the square
  \[
    \begin{tikzcd}
        C
        \arrow[dashed]{r}[above]{u}
        \arrow[dashed]{d}[left]{v}
      & A
        \arrow{d}[right]{x}
      \\
        B
        \arrow{r}[above]{y}
      & X
    \end{tikzcd}
  \]
  commute.
  Then
  \[
            x
    \equiv' x u 
    =       y v
    \equiv' y \,.
  \]
  This shows that the equivalence relation~$\equiv'$ is finer than the equivalence relation~$\equiv$.
\end{remark*}


\begin{theorem}[Rules for diagram chase]
  \label{rules for diagram chase}
  Let~$\Acat$ be an abelian category.
  \begin{enumerate}
    \item
      \label{abstract injective}
      For a morphism~$f \colon X \to Y$ in~$\Acat$ the following three conditions are equivalent:
      \begin{enumerate}
        \item
          The morphism~$f$ is a monomorphism.
        \item
          It follows for all~$x, x' \inA X$ from~$fx \equiv fx'$ that~$x \equiv x'$.
        \item
          It follows for every~$z \inA X$ from~$fz \equiv 0$ that~$z \equiv 0$.
      \end{enumerate}
    \item
    \label{abstract surjective}
      For a morphism~$g \colon Y \to Z$ in~$\Acat$ the following two conditions are equivalent:
      \begin{enumerate}
        \item
          The morphism~$g$ is an epimorphism.
        \item
          There exists for every~$z \inA Z$ some~$y \inA Y$ with~$gy \equiv z$.
      \end{enumerate}
    \item
      For a sequence~$X \xto{f} Y \xto{g} Z$ in~$\Acat$ the following two conditions are equivalent:
      \begin{enumerate}
        \item
          The sequence~$X \xto{f} Y \xto{g} Z$ is exact.
        \item
          It holds that~$gf = 0$ and there exists for every~$y \inA Y$ with~$gy \equiv 0$ some~$x \inA X$ with~$fx \equiv y$.
      \end{enumerate}
    \item
      \label{difference element}
      Let~$f \colon X \to Y$ be a morphism in~$\Acat$ and let~$x, x' \inA X$ be two points with~$fx \equiv fx'$.
      Then there exists a point~$\tilde{x} \inA X$ such that
      \begin{enumerate}
        \item
          $f \tilde{x} \equiv 0$, and
        \item
          it holds for every morphism~$h \colon X \to W$ that
          \begin{itemize}
            \item
              if~$hx' \equiv 0$ then~$h \tilde{x} \equiv hx$, and
            \item
              if~$hx \equiv 0$ then~$h \tilde{x} \equiv -hx'$.
          \end{itemize}
      \end{enumerate}
  \end{enumerate}
\end{theorem}


\begin{notationnonum}
  The morphism~$\tilde{x}$ from part~\ref*{difference element} of \cref{rules for diagram chase} is denoted by~$x - x'$.
\end{notationnonum}


\begin{proof}
  \leavevmode
  \begin{enumerate}
    \item
      Suppose that~$f$ is a monomorphism and let~$x, x' \inA X$ be two points with~$fx \equiv fx'$, say~$x \in X(A)$ and~$x' \in X(A')$.
      Then there exist epimorphisms~$u \colon B \to A$ and~$v \colon B \to A'$ that make the square
      \[
        \begin{tikzcd}
            B
            \arrow{rr}[above]{u}
            \arrow{dd}[left]{v}
          & {}
          & A
            \arrow{d}[right]{x}
          \\
            {}
          & {}
          & X
            \arrow{d}[right]{f}
          \\
            A'
            \arrow{r}[above]{x'}
          & X
            \arrow{r}[above]{f}
          & Y
        \end{tikzcd}
      \]
      commute, i.e.\ such that~$f x u = f x' v$.
      It follows from~$f$ being a monomorphism that already~$x u = x' v$, i.e.\ that the square
      \[
        \begin{tikzcd}
            B
            \arrow{r}[above]{u}
            \arrow{d}[left]{v}
          & A
            \arrow{d}[right]{x}
          \\
            A'
            \arrow{r}[above]{x'}
          & X
        \end{tikzcd}
      \]
      commutes.
      This shows that~$x \equiv x'$.
      
      Suppose that it follows for all points~$x, x' \inA X$ from~$f x \equiv f x'$ that~$x \equiv x'$, and let~$z \inA X$ with~$f z \equiv 0$.
      Then~$fz \equiv 0 = f \circ 0$ and hence~$z \equiv 0$.
      
      Suppose lastly that~$z \equiv 0$ for every~$z \inA X$ with~$fz \equiv 0$.
      It then follows for every~$x \inA X$ that
      \[
                  fx = 0
        \implies  fx \equiv 0
        \implies  x \equiv 0
        \implies  x = 0 \,.
      \]
      This shows that~$\ker(f) = 0$ and hence that~$f$ is a monomorphism.
    \item
      Suppose that~$g$ is an epimorphism and let~$z \inA Z$.
      We consider the following pullback square:
      \[
        \begin{tikzcd}
            A'
            \arrow{r}[above]{g'}
            \arrow{d}[left]{y}
            \arrow[phantom]{dr}[description]{\pb}
          & A
            \arrow{d}[right]{z}
          \\
            Y
            \arrow{r}[above]{g}
          & Z
        \end{tikzcd}
      \]
      The morphism~$g'$ is by \cref{mono epi under pull push} again an epimorphism because~$g$ is an epimorphism.
      It therefore holds that
      \[
                z
        \equiv  z g'
        =       g y \,.
      \]
      
      Suppose on the other hand that for every point~$z \inA Z$ there exists some~$y \inA Y$ with~$gy \equiv z$.
      By choosing~$z = \id_Z$ we find that there exists some~$y \inA Y$ with ~$gy \equiv \id_Z$, say~$y \in X(A)$.
      There then exist epimorphisms~$u \colon B \to Z$ and~$v \colon B \to A$ that make the square
      \[
        \begin{tikzcd}
            B
            \arrow{r}[above]{u}
            \arrow{d}[left]{v}
          & Z
            \arrow{d}[right]{\id_Z}
          \\
            A
            \arrow{r}[above]{gy}
          & Z
        \end{tikzcd}
      \]
      commute.
      The composition~$gyv = u$ is therefore an epimorphism, hence the morphism~$g$ is an epimorphism.
  \end{enumerate}
  
  
  
  
  
  \lecturend{13}
  
  
  
  
  
  \begin{enumerate}[resume]
    \item 
      If~$gf = 0$ then let~$\lambda \colon \im(f) \to \ker(g)$ be the canonical morphism, i.e.\ the unique morphism~$\lambda \colon \im(f) \to \ker(g)$ that makes the diagram
      \[
        \begin{tikzcd}[column sep = small, row sep = large]
            X
            \arrow{rr}[above]{f}
            \arrow{dr}[below left]{f'}
          & {}
          & Y
            \arrow{rr}[above]{g}
          & {}
          & Z
          \\
            {}
          & \im(f)
            \arrow{ur}[above left]{i}
            \arrow[dashed]{rr}[above]{\lambda}
          & {}
          & \ker(g)
            \arrow{ul}[above right]{k}
          & {}
        \end{tikzcd}
      \]
      commute.
      If~$y \inA Y$ with~$gy \equiv 0$ then~$gy = 0$ and it follows from the universal property of the kernel~$k \colon \ker(g) \to Y$ that there exist a unique morphism~$\tilde{y} \colon A \to \ker(g)$ that makes the following diagram commute:
      \[
        \begin{tikzcd}[column sep = small, row sep = large]
            {}
          & {}
          & A
            \arrow{d}[left]{y}
          & {}
          & {}
          \\
            X
            \arrow{rr}[above]{f}
            \arrow{dr}[below left]{f'}
          & {}
          & Y
            \arrow{rr}[above, near end]{g}
          & {}
          & Z
          \\
            {}
          & \im(f)
            \arrow{ur}[above left]{i}
            \arrow{rr}[above]{\lambda}
          & {}
          & \ker(g)
            \arrow[from = uul, dashed, bend left, crossing over, near start, "\tilde{y}"]
            \arrow{ul}[above right]{k}
          & {}
        \end{tikzcd}
      \]
      
      Suppose now that the sequence
      \[
        X
        \xlongto{f}
        Y
        \xlongto{g}
        Z
      \]
      is exact.
      Then the canonical morphism~$\lambda \colon \im(f) \to \ker(g)$ is an isomorphism.
      Let~$y \inA Y$ such that~$gy \equiv 0$, hence~$gy = 0$.
      Then
      \[
        \lambda^{-1} \tilde{y} \in_A \im(f) \,.
      \]
      The morphism~$f' \colon X \to \im(f)$ is an epimorphism, hence it follows from part~\ref*{abstract surjective} that there exists some point~$x \inA X$ with~$f' x \equiv \lambda^{-1} \tilde{y}$.
      It holds for this point that
      \[
                f x
        =       i f' x
        \equiv  i \lambda^{-1} \tilde{y}
        =       k \tilde{y}
        =       y \,.
      \]
      
      Suppose on the other hand that~$gf = 0$ and that for every~$y \inA Y$ with~$gy \equiv 0$ there exist some~$x \inA X$ with~$fx = y$.
      We need to show that the morphism~$\lambda \colon \im(f) \to \ker(g)$ is an isomorphism.
      We already know that~$\lambda$ is a monomorphism (as this is always the case), so it remains to show that~$\lambda$ is an epimorphism.
      For this we use part~\ref*{abstract surjective}:
      Let~$\tilde{y} \inA \ker(g)$ and set~$y \defined k \tilde{y} \inA Y$.
      Then
      \[
          g y
        = g k \tilde{y}
        = 0 \circ \tilde{y}
        = 0 \,.
      \]
      It follows by assumption that there exist some point~$x \inA X$ with~$f x \equiv y$.
      For the point
      \[
                  \tilde{x}
        \defined  f' x \inA \im(f)
      \]
      we have that~$\lambda \tilde{x} = \tilde{y}$;
      indeed, we have that
      \[
                k \lambda \tilde{x}
        =       i \tilde{x}
        =       i f' x
        =       f x
        \equiv  y
        =       k \tilde{y} \,,
      \]
      and hence~$\lambda \tilde{x} \equiv \tilde{y}$ because~$k$ is a monomorphism (by part~\ref*{abstract injective}).
      This shows by part~\ref*{abstract surjective} that~$\lambda$ is an epimorphism.
    \item
      Let~$x, x' \inA X$ with~$f x \equiv f x'$, say~$x \in X(A)$ and~$x' \in X(A')$.
      It follows from~$f x \equiv f x'$ that there exist epimorphisms~$u \colon B \to A$ and~$v \colon B \to A'$ that make the diagram
      \[
        \begin{tikzcd}
            B
            \arrow{rr}[above]{u}
            \arrow{dd}[left]{v}
          & {}
          & A
            \arrow{d}[right]{x}
          \\
            {}
          & {}
          & X
            \arrow{d}[right]{f}
          \\
            A'
            \arrow{r}[above]{x'}
          & X
            \arrow{r}[above]{f}
          & Y
        \end{tikzcd}
      \]
      commute.
      It follows for the element
      \begin{gather*}
                  \tilde{x}
        \defined  x v - x' u
        \in       X(B)
      \shortintertext{that}
          f \tilde{x}
        = f (x v - x' u)
        = f x v - f x' u
        = 0
      \end{gather*}
      and hence~$f \tilde{x} \equiv 0$.
      If~$h \colon X \to W$ is a morphism with~$h x' \equiv 0$ then~$h x' = 0$, hence
      \[
                h \tilde{x}
        =       h (x v - x' u)
        =       h x v - h x' u
        =       h x v
        \equiv  h x
      \]
      because~$v$ is an epimorphism.
      We similarly find that if~$h x \equiv 0$ then~$h \tilde{x} \equiv - h x'$.
    \qedhere
  \end{enumerate}
\end{proof}





\textline{Below this line proofs are currently missing.}





\begin{lemma}[Snake lemma]
  \index{snake lemma}\index{lemma!snake}
  Let~$\Acat$ be an abelian category.
  Let
  \begin{equation}
    \label{presnake}
    \begin{tikzcd}
        {}
      & X'
        \arrow{r}[above]{i}
        \arrow{d}[right]{f'}
      & X
        \arrow{r}[above]{p}
        \arrow{d}[right]{f}
      & X''
        \arrow{d}[right]{f''}
        \arrow{r}
      & 0
      \\
        0
        \arrow{r}
      & Y'
        \arrow{r}[below]{j}
      & Y
        \arrow{r}[below]{q}
      & Y''
      & {}
    \end{tikzcd}
  \end{equation}
  be a commutative diagram in~$\Acat$ with exact rows.
  Let
  \[
    k' \colon \ker(f') \to X' \,,
    \quad
    k \colon \ker(f) \to X \,,
    \quad
    k'' \colon \ker(f'') \to X''
  \]
  be kernels of~$f'$,~$f$ and~$f''$, and let
  \[
    \tilde{i} \colon \ker(f') \to \ker(f)
    \quad\text{and}\quad
    \tilde{p} \colon \ker(f) \to \ker(f'')
  \]
  be the unique morphisms which make the following diagram commute:
  \[
    \begin{tikzcd}
        \ker(f')
        \arrow[dashed]{r}[above]{\tilde{i}}
        \arrow{d}[right]{k'}
      & \ker(f)
        \arrow[dashed]{r}[above]{\tilde{p}}
        \arrow{d}[right]{k}
      & \ker(f'')
        \arrow{d}[right]{k''}
      \\
        X'
        \arrow{r}[above]{i}
      & X
        \arrow{r}[above]{p}
      & X''
    \end{tikzcd}
  \]
  Dually, let
  \[
    c' \colon Y' \to \coker(f') \,,
    \quad
    c \colon Y \to \coker(f) \,,
    \quad
    c'' \colon Y'' \to \coker(f'')
  \]
  be cokernels of~$f'$,~$f$ and~$f''$, and let
  \[
    \bar{j} \colon \coker(f') \to \coker(f)
    \quad\text{and}\quad
    \bar{q} \colon \coker(f) \to \coker(f'')
  \]
  be the unique morphisms that make the following diagram commute:
  \[
    \begin{tikzcd}
        Y'
        \arrow{r}[above]{j}
        \arrow{d}[right]{c'}
      & Y
        \arrow{r}[above]{q}
        \arrow{d}[right]{c}
      & Y''
        \arrow{d}[right]{c''}
      \\
        \coker(f')
        \arrow[dashed]{r}[above]{\bar{j}}
      & \coker(f)
        \arrow[dashed]{r}[above]{\bar{q}}
      & \coker(f'')
    \end{tikzcd}
  \]
  (See \cref{functoriality of (co)kernel} for a more thorough explanation on these induced morphisms.)

  \begin{enumerate}
    \item
      There exists a morphism
      \[
                \delta
        \colon  \ker(f'')
        \to     \coker(f')
      \]
      such that the sequence
      \begin{equation}
        \label{induced snake}
        \begin{tikzcd}
            \ker(f')
            \arrow{r}[above]{\tilde{i}}
          & \ker(f)
            \arrow{r}[above]{\tilde{p}}
            \arrow[d, phantom, ""{coordinate, name=Z}]
          & \ker(f'')
            \arrow[ dll,
                    "\delta",
                    rounded corners,
                    to path={ -- ([xshift=2ex]\tikztostart.east)
                              |- (Z) \tikztonodes
                              -| ([xshift=-2ex]\tikztotarget.west)
                              -- (\tikztotarget)}
                  ]
          \\
            \coker(f')
            \arrow{r}[above]{\bar{j}}
          & \coker(f)
            \arrow{r}[above]{\bar{q}}
          & \coker(f'')
        \end{tikzcd}
      \end{equation}
      is exact.
    \item
      \label{snake inherits mono epi}
      If~$i$ is a monomorphism then~$\tilde{i} \colon \ker(f') \to \ker(f)$ is again a monomorphism, and if~$q$ is an epimorphism then~$\bar{q} \colon \coker(f) \to \coker(f'')$ is again an epimorphism.
  \end{enumerate}
\end{lemma}


\begin{remark*}
  \leavevmode
  \begin{enumerate}
    \item
      One can also draw the exact sequence~\eqref{induced snake} into the diagram~\eqref{presnake}.
      This then results in the following diagram:
      \[
        \begin{tikzcd}[row sep = huge]
            {}
          & \ker(f')
            \arrow[dashed]{r}
            \arrow{d}
          & \ker(f)
            \arrow{d}
            \arrow[dashed]{r}
            \arrow[ddd, phantom, ""{coordinate, name=Z}]
          & \ker(f'')
            \arrow{d}
          & {}
          \\
            {}
          & X'
            \arrow{r}[above]{i}
            \arrow{d}[left, near start]{f'}
          & X
            \arrow{r}[above]{p}
            \arrow{d}[left, near start]{f}
          & X''
            \arrow{r}
            \arrow{d}[left, near start]{f''}
          & 0
          \\
            0
            \arrow{r}
          & Y'
            \arrow{r}[above]{j}
            \arrow{d}
          & Y
            \arrow{r}[above]{q}
            \arrow{d}
          & Y''
            \arrow{d}
          & {}
          \\
            {}
          & \coker(f')
            \arrow[ from=uuurr,
                    dashed,
                    rounded corners,
                    crossing over,
                    "\delta",
                    to path={ -- ([xshift=2ex]\tikztostart.east)
                              |- ([yshift=-1.5ex]Z) \tikztonodes
                              -| ([xshift=-2ex]\tikztotarget.west)
                              -- (\tikztotarget)}
                  ]
            \arrow[dashed]{r}
          & \coker(f)
            \arrow[dashed]{r}
          & \coker(f'')
          & {}
        \end{tikzcd}
      \]
      The name \enquote{snake lemma} stems from the form of this dashed sequence.
    \item
      Part~\ref*{snake inherits mono epi} of the snake lemma states that the sequence
      \[
        \begin{tikzcd}
            0
            \arrow[dashed]{r}
          & \ker(f')
            \arrow{r}
          & \ker(f)
            \arrow{r}
            \arrow[d, phantom, ""{coordinate, name=Z}]
          & \ker(f'')
            \arrow[ dll,
                    "\delta",
                    rounded corners,
                    to path={ -- ([xshift=2ex]\tikztostart.east)
                              |- (Z) \tikztonodes
                              -| ([xshift=-2ex]\tikztotarget.west)
                              -- (\tikztotarget)}
                  ]
          & {}
          \\
            {}
          & \coker(f')
            \arrow{r}
          & \coker(f)
            \arrow{r}
          & \coker(f'')
            \arrow[dotted]{r}
          & 0
        \end{tikzcd}
      \]
      inherits on the left and right the same additional exactness conditions as the original diagram:
      \[
        \begin{tikzcd}
            0
            \arrow[dashed]{r}
          & X'
            \arrow{r}[above]{i}
            \arrow{d}[right]{f'}
          & X
            \arrow{r}[above]{p}
            \arrow{d}[right]{f}
          & X''
            \arrow{d}[right]{f''}
            \arrow{r}
          & 0
          \\
            0
            \arrow{r}
          & Y'
            \arrow{r}[below]{j}
          & Y
            \arrow{r}[below]{q}
          & Y''
            \arrow[dotted]{r}
          & 0
        \end{tikzcd}
      \]
    \item
      The morphism~$\delta$ from the snake lemma is known as the \enquote{connecting morphism}.
    \item
      The connecting morphism in natural in the following sense:
      Suppose that we are given a commutative diagram of the following form, where the four horizontal rows are exact.
      \[
        \begin{tikzcd}[column sep = 1.88em, row sep = 2.5em]
            {}
          & {}
          & {}
          & X'
            \arrow{rr}[above]{i}
            \arrow{dd}[right, near start]{f'}
            \arrow[dashed]{dl}[above left]{\alpha'}
          & {}
          & X
            \arrow{rr}[above]{p}
            \arrow{dd}[right, near start]{f}
            \arrow[dashed]{dl}[above left]{\alpha}
          & {}
          & X''
            \arrow{dd}[right, near start]{f''}
            \arrow{rr}
            \arrow[dashed]{dl}[above left]{\alpha''}
          & {}
          & 0
          \\
            {}
          & {}
          & W'
          & {}
          & W
            \arrow[from=ll, crossing over, above, "i'", near end]
          & {}
          & W''
            \arrow[from=ll, crossing over, above, "p'", near end]
          & {}
          & 0
            \arrow[from=ll, crossing over]
          & {}
          \\
            {}
          & 0
            \arrow{rr}
          & {}
          & Y'
            \arrow{rr}[below, near start]{j}
            \arrow[dashed]{dl}[below right]{\beta'}
          & {}
          & Y
            \arrow{rr}[below, near start]{q}
            \arrow[dashed]{dl}[below right]{\beta}
          & {}
          & Y''
            \arrow[dashed]{dl}[below right]{\beta''}
          & {}
          & {}
          \\
            0
            \arrow{rr}
          & {}
          & Z'
            \arrow[from=uu, crossing over, "g'", swap, near start]
            \arrow{rr}[below]{j'}
          & {}
          & Z
            \arrow[from=uu, crossing over, "g", swap, near start]
            \arrow{rr}[below]{q'}
          & {}
          & Z''
            \arrow[from=uu, crossing over, "g''", swap, near start]
          & {}
          & {}
        \end{tikzcd}
      \]
      We can then apply the snake lemma to both the back and the front of this diagram.
      It then follows from the commutativity of the square
      \[
        \begin{tikzcd}
            X''
            \arrow[dashed]{r}[above]{\alpha''}
            \arrow{d}[left]{f''}
          & W''
            \arrow{d}[right]{g''}
          \\
            Y''
            \arrow[dashed]{r}[above]{\beta''}
          & Z''
        \end{tikzcd}
      \]
      that~$\alpha''$ induces a unique morphism~$\ker(f'') \to \ker(g'')$ that make the square
      \[
        \begin{tikzcd}
            \ker(f'')
            \arrow[dashed]{r}
            \arrow{d}
          & \ker(g'')
            \arrow{d}
          \\
            X''
            \arrow[dashed]{r}[above]{\alpha''}
          & W''
        \end{tikzcd}
      \]
      commute.
      Similarly, the commutativity of the square
      \[
        \begin{tikzcd}
            X'
            \arrow[dashed]{r}[above]{\alpha'}
            \arrow{d}[left]{f'}
          & W'
            \arrow{d}[right]{g'}
          \\
            Y'
            \arrow[dashed]{r}[above]{\beta'}
          & Z'
        \end{tikzcd}
      \]
      shows that~$\beta'$ induces a unique morphism~$\coker(f') \to \coker(g')$ that makes the square
      \[
        \begin{tikzcd}
            Y'
            \arrow[dashed]{r}[above]{\beta'}
            \arrow{d}
          & Z'
            \arrow{d}
          \\
            \coker(f')
            \arrow[dashed]{r}
          & \coker(g')
        \end{tikzcd}
      \]
      commute.
      (See \cref{functoriality of (co)kernel} for a more detailed explanation on this.)
      The connecting morphisms~$\delta \colon \ker(f'') \to \coker(f')$ and~$\delta' \ker(g'') \to \coker(g')$ then make the square
      \[
        \begin{tikzcd}
            \ker(f'')
            \arrow{r}[above]{\delta}
            \arrow[dashed]{d}
          & \coker(f')
            \arrow[dashed]{d}
          \\
            \ker(g'')
            \arrow{r}[above]{\delta'}
          & \coker(g')
        \end{tikzcd}
      \]
      commute.
      
      It then follows more more generally that the diagram
      \[
        \begin{tikzcd}[column sep = 1.8em]
            \ker(f')
            \arrow{r}
            \arrow[dashed]{d}
          & \ker(f)
            \arrow{r}
            \arrow[dashed]{d}
          & \ker(f'')
            \arrow{r}[above]{\delta}
            \arrow[dashed]{d}
          & \coker(f')
            \arrow{r}
            \arrow[dashed]{d}
          & \coker(f)
            \arrow{r}
            \arrow[dashed]{d}
          & \coker(f'')
            \arrow[dashed]{d}
          \\
            \ker(g')
            \arrow{r}
          & \ker(g)
            \arrow{r}
          & \ker(g'')
            \arrow{r}[above]{\delta'}
          & \coker(g')
            \arrow{r}
          & \coker(g)
            \arrow{r}
          & \coker(g'')
        \end{tikzcd}
      \]
      commutes, where the rows are the exact sequences from the snake lemma.
      The vertical morphisms in this diagram are induced in the same way as~$\ker(f'') \to \ker(g'')$ and~$\coker(f') \to \coker(g')$, as explained above.
      
      One may also think about this naturality of the connecting morphism as the functoriality of the exact squences from the snake lemma.
  \end{enumerate}
\end{remark*}


\begin{definition}
  Let~$F \colon \Acat \to \Bcat$ be an additive functor between abelian categories~$\Acat$ and~$\Bcat$.
  The functor~$F$ is
  \begin{enumerate}
    \item
      exact, resp.
    \item
      left exact, resp.
    \item
      right exact,
  \end{enumerate}
  if for every short exact sequence~$0 \to X' \to X \to X'' \to 0$ in~$\Acat$
  \begin{enumerate}
    \item
      the sequence~$0 \to F(X') \to F(X) \to F(X'') \to 0$ is again (short) exact, resp.
    \item
     the sequence~$0 \to F(X') \to F(X) \to F(X'')$ is again (left) exact, resp.
    \item
      the sequence~$F(X') \to F(X) \to F(X'') \to 0$ is again (right) exact.
  \end{enumerate}
\end{definition}


\begin{remark}
  Let~$F \colon \Acat \to \Bcat$ be a functor between abelian categories~$\Acat$ and~$\Bcat$.
  \begin{enumerate}
    \item
      Left exactness can be detected via left exact sequences:
      The functor~$F$ is left exact if and only if for every (left) exact sequence~$0 \to X' \to X \to X''$ in~$\Acat$ the sequence~$0 \to F(X') \to F(X) \to F(X'')$ in~$\Bcat$ is again (left) exact.
    \item
      Right exactness can similarly be detected via right exact sequences:
      The functor~$F$ is right exact if and only if for every (right) exact sequence~$X' \to X \to X'' \to 0$ in~$\Acat$ the sequence~$F(X') \to F(X) \to F(X'') \to 0$ in~$\Bcat$ is again (right) exact.
  \end{enumerate}
\end{remark}


% TODO: Add proofs for this.


\begin{remark*}
  \leavevmode
  \begin{enumerate}
    \item
      In other words, a functor~$F \colon \Acat \to \Bcat$ between abelian categories~$\Acat$ and~$\Bcat$ is left exact (resp.\ right exact) if and only if it respects kernels (resp.\ cokernels);
      this follows from the characterization of left exact sequences via kernels and right exact sequences via cokernels, as explained in \cref{language of left and right exact}.
    \item
      If a functor~$F \colon \Acat \to \Bcat$ between abelian categories~$\Acat$ and~$\Bcat$ is exact then for every exact sequence
      \[
        \dotsb
        \to
        X_{i-1}
        \to
        X_i
        \to
        X_{i+1}
        \to
        \dotsb
      \]
      in~$\Acat$ the sequence
      \[
        \dotsb
        \to
        F(X_{i-1})
        \to
        F(X_i)
        \to
        F(X_{i+1})
        \to
        \dotsb
      \]
      in~$\Bcat$ is again exact.
  \end{enumerate}
\end{remark*}


\begin{example}
  Let~$\Acat$ be an abelian category.
  \begin{enumerate}
    \item
      For every object~$X \in \Ob(\Acat)$ the covariant~\dash{$\Hom$}{functor}
      \[
        \Hom_\Acat(X,-)
        \colon
        \Acat
        \to
        \Ab
      \]
      is left exact.
    \item
      For every object~$X \in \Ob(\Acat)$ the contravariant~\dash{$\Hom$}{functor}
      \[
        \Hom_\Acat(-,X)
        \colon
        \Acat^\op
        \to
        \Ab
      \]
      is left exact.
    \item
      Let~$A$ be a~{\kalg}.
      Then for every left~{\module{$A$}}~$\indmodule[A]{N}$ the functor
      \[
        (-) \tensor_A N
        \colon
        \Modr{A}
        \to
        \Modl{\kf}
      \]
      is right exact.
      Similarly, for every right~{\module{$A$}}~$\indmodule{M}[A]$ the functor
      \[
        M \tensor_A (-)
        \colon
        \Modl{A}
        \to
         \Modl{\kf}
      \]
      is right exact.
    \item
      Let~$X$ be a topological space.
      The inclusion functor~$I \colon \Sheaf(X) \to \Presheaf(X)$ is left exact.
      But it is not right exact because it does not respect cokernels.
      The sheafification functor~$S \colon \Presheaf(X) \to \Sheaf(X)$ is exact.
%     TODO: Explain this via the given facts.
    \item
      Let again~$X$ be a topological space.
      For every presheaf~$\Fs$ on~$X$ let
      \[
                  \Gamma(X,\Fs)
        \defined  \Fs(X).%
        \footnote{A more general notation is~$\Gamma(U,\Fs) \defined \Fs(U)$ for every open subseteq~$U \subseteq X$ and every presheaf~$\Fs$ on~$X$.}
      \]
      Then~$\Gamma(X,-) \colon \Presheaf(X) \to \Ab$ is the \emph{global section functor}, and this functor is exact.
%     TODO: Explain this properly.
      The restriction~$\Gamma(X,-) \colon \Sheaf(X) \to \Ab$ is again left exact, but not right exact.
%     TODO: Properly explain this.
  \end{enumerate}
\end{example}





\lecturend{14}








