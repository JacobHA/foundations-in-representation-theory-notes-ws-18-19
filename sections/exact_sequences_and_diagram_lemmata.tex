\section{Exact Sequences and Diagram Lemmata}


\begin{remarknonum}(Orally)
  We will (at least for now) neither use nor state the Freyd--Mitchell embedding theorem.
  We will instead explain how one can \enquote{diagram chase} in abelian categories.
\end{remarknonum}


\begin{remarkdefinition}
  \leavevmode
  \begin{enumerate}
    \item
      Let~$f \colon X \to Y$ and~$g \colon Y \to Z$ be two composable morphisms in~$\Acat$.
      
      If~$g f = 0$ then there exists a unique morphism~$\lambda \colon \im(f) \to \ker(g)$ that makes the diagram
      \[
        \begin{tikzcd}[column sep = large]
            X
            \arrow{rr}[above]{f}
            \arrow{dr}[below left]{f'}
          & {}
          & Y
            \arrow{rr}[above]{g}
          & {}
          & Z
          \\
            {}
          & \im(f)
            \arrow{ur}[above left]{i}
            \arrow[dashed]{rr}[above]{\lambda}
          & {}
          & \ker(g)
            \arrow{ul}[above right]{k}
          & {}
        \end{tikzcd}
      \]
      commute, and~$\lambda$ is a monomorphism.
      Indeed, it holds that
      \[
          g i f'
        = g f
        = 0
      \]
      and hence~$g i = 0$ because~$f'$ is an epimorphism.
      The existence and uniqueness of~$\lambda$ thus follow from the universal property of the kernel~$\ker(g) \to Y$.
      
      The sequence~$X \xto{f} Y \xto{g} Z$ is \emph{exact}\index{exact sequence} if
      \begin{enumerate}[label=(E\arabic*)]
        \item
          $gf = 0$, and
        \item
          the canonical morphism~$\lambda \colon \im(f) \to \ker(g)$ is an ismorphism.
      \end{enumerate}
    \item
      A (possibly infinite) sequence
      \[
          \dotsb
        \longto
          X_{i-1}
        \xlongto{f_{i-1}}
          X_i
        \xlongto{f_i}
          X_{i+1}
        \longto
          \dotsb
      \]
      of composable morphisms in~$\Acat$ is \emph{exact}\index{exact sequence} if at every position~$i$ for which both an incoming arrow~$f_{i-1} \colon X_{i-1} \to X_i$ and outgoing~$f_i \colon X_i \to X_{i+1}$ exist, the resulting sequence
      \[
          X_{i-1}
        \xlongto{f_{i-1}}
          X_i
        \xlongto{f_i}
          X_{i+1}
      \]
      is exact.
    \item
      We have the following special cases of exact sequences:
      \begin{itemize}
        \item
          A sequence of the form
          \[
            0
            \to
            X
            \xlongto{f}
            Y
          \]
          is exact if and only if~$f$ is a monomorphism.
        \item
          A sequence of the form
          \[
            Y
            \xlongto{g}
            Z
            \to
            0
          \]
          is exact if and only if~$g$ is an epimorphism.
        \item
          A sequence of the form
          \[
            0
              \to
            X
              \xlongto{f}
            Y
              \xlongto{g}
            Z
              \to
            0
          \]
          is exact if and only if~$f$ is a monomorphism,~$g$ is an epimorphism and the canonical morphism~$\im(f) \to \ker(g)$ is an isomorphism.
          This is furthermore equivalent to~$f$ being a kernel of~$g$ and~$g$ being a cokernel of~$f$ (at the same time).
      \end{itemize}
  \end{enumerate}
\end{remarkdefinition}


\begin{proposition}
  Let~$\Acat$ be an abelian category and let
  \[
    \begin{tikzcd}[sep = large]
        {}
      & {}
      & Y'
        \arrow{r}[above]{g'}
        \arrow{d}[right]{h'}
        \arrow[phantom]{dr}[description]{\pb}
      & Z'
        \arrow{d}[right]{h}
      & {}
      \\
        0
        \arrow{r}
      & X
        \arrow{r}[above]{f}
      & Y
        \arrow{r}[above]{g}
      & Z
        \arrow{r}
      & 0
    \end{tikzcd}
  \]
  be a commutative diagram in~$\Acat$ where the bottom row is exact and the square is a pullback square.
  Then there exist a unique morphism~$f' \colon X \to Y'$ such that the diagram
  \begin{equation}
    \label{resulting pb of ses}
    \begin{tikzcd}[sep = large]
        0
        \arrow{r}
      & X
        \arrow[dashed]{r}[above]{f'}
        \arrow[equal]{d}
      & Y'
        \arrow{r}[above]{g'}
        \arrow{d}[right]{h'}
        \arrow[phantom]{dr}[description]{\pb}
      & Z'
        \arrow{d}[right]{h}
        \arrow{r}
      & 0
      \\
        0
        \arrow{r}
      & X
        \arrow{r}[above]{f}
      & Y
        \arrow{r}[above]{g}
      & Z
        \arrow{r}
      & 0
    \end{tikzcd}
  \end{equation}
  commutes and the upper row
  \begin{equation}
    \label{upper row}
    0
    \to
    X
    \xlongto{f}
    Y
    \xlongto{g}
    Z
    \to
    0
  \end{equation}
  is exact.
\end{proposition}


\begin{proof}
  It follows from \cref{mono epi under pull push} that~$g'$ is an epimorphism because~$g$ is one.
  
  To construct the desired morphism~$f' \colon X \to Y$ we use that the given square is a pullback square:
  The diagram
  \[
    \begin{tikzcd}[sep = large]
        X
        \arrow[bend left]{drr}[above right]{0}
        \arrow[bend right]{ddr}[below left]{f}
      & {}
      & {}
      \\
        {}
      & Y'
        \arrow{r}[above]{g'}
        \arrow{d}[right]{h'}
        \arrow[phantom]{dr}[description]{\pb}
      & Z'
        \arrow{d}[right]{h}
      \\
        {}
      & Y
        \arrow{r}[above]{g}
      & Z
    \end{tikzcd}
  \]
  commutes because~$gf = 0$.
  So it follows that there exists a unique morphism~$f' \colon X \to Y'$ that makes the diagram
  \[
    \begin{tikzcd}[sep = large]
        X
        \arrow[bend left]{drr}[above right]{0}
        \arrow[dashed]{dr}[above right]{f'}
        \arrow[bend right]{ddr}[below left]{f}
      & {}
      & {}
      \\
        {}
      & Y'
        \arrow{r}[above]{g'}
        \arrow{d}[right]{h'}
        \arrow[phantom]{dr}[description]{\pb}
      & Z'
        \arrow{d}[right]{h}
      \\
        {}
      & Y
        \arrow{r}[above]{g}
      & Z
    \end{tikzcd}
  \]
  commute.
  This means precisely that the morphism~$f'$ makes the diagram~\eqref{resulting pb of ses} commute and satisfies~$g' f' = 0$.
  
  The morphism~$f$ is a monomorphism because the composition~$h' f' = f$ is a monomorphism.
  
  To show the exactness of the row~\eqref{upper row} it remains to show that~$f'$ is already a kernel of~$g'$.
  So let~$u \colon W \to Y'$ be a morphism with~$g' u = 0$.
  \[
    \begin{tikzcd}[sep = large]
        {}
      & {}
      & W
        \arrow[dashed]{d}[right]{u}
        \arrow[dashed, bend left]{dr}[above right]{0}
      & {}
      & {}
      \\
        0
        \arrow{r}
      & X
        \arrow{r}[above]{f'}
        \arrow[equal]{d}
      & Y'
        \arrow{r}[above]{g'}
        \arrow{d}[right]{h'}
        \arrow[phantom]{dr}[description]{\pb}
      & Z'
        \arrow{d}[right]{h}
        \arrow{r}
      & 0
      \\
        0
        \arrow{r}
      & X
        \arrow{r}[above]{f}
      & Y
        \arrow{r}[above]{g}
      & Z
        \arrow{r}
      & 0
    \end{tikzcd}
  \]
  Then
  \[
      g h' u
    = h g' u
    = h \circ 0
    = 0
  \]
  and hence~$h' u$ factors uniquely over the kernel of~$g$, which is~$f$.
  So there exists some morphism~$\lambda \colon W \to X$ with~$f \lambda = h' u$.
  \[
    \begin{tikzcd}[sep = large]
        {}
      & {}
      & W
        \arrow{d}[right]{u}
        \arrow[bend left]{dr}[above right]{0}
        \arrow[dashed,bend right]{dl}[above left]{\lambda}
      & {}
      & {}
      \\
        0
        \arrow{r}
      & X
        \arrow{r}[above, near end]{f'}
        \arrow[equal]{d}
      & Y'
        \arrow{r}[above, near start]{g'}
        \arrow{d}[right]{h'}
        \arrow[phantom]{dr}[description]{\pb}
      & Z'
        \arrow{r}
      & 0
      \\
        0
        \arrow{r}
      & X
        \arrow[from=uur, dashed, bend right = 15, near start, swap, "\lambda", crossing over]
        \arrow{r}[above]{f}
      & Y
        \arrow{r}[above]{g}
      & Z
        \arrow{r}
        \arrow[from=uul, dashed, bend left = 15, near start, "0", crossing over]
        \arrow[from=u, right, "h"]
      & 0
    \end{tikzcd}
  \]
  To show that~$u = f' \lambda$, i.e.\ that the above diagram commutes, we again use that the \dash{right}{hand} square is a pullback square:
  The two parallel morphisms~$u, f' \lambda \colon W \to Y$ coincide if and only if they coincide after composition with both~$g'$ and~$h'$.
  We have that
  \[
      h' u
    = f \lambda
    = f \id_X \lambda
    = h' f' \lambda
  \]
  and also
  \[
      g' u
    = 0
    = g' f' \lambda
  \]
  because~$g' f' = 0$.
  Hence~$u = f' \lambda$, which shows the existence of the desired morphism~$\lambda$.
  The morphism~$\lambda$ is uniquely determined by the composition~$f' \lambda = u$ because~$f'$ is a monomorphism.
  This shows the uniqueness of~$\lambda$.
\end{proof}


\begin{remarkdefinition}
  Let~$\Acat$ be an abelian category and let~$X$ be an object in~$\Acat$.
  \begin{enumerate}
    \item
      For every object~$A \in \Ob(\Acat)$ let
      \[
                  X(A)
        \defined  \Hom_\Acat(A,X) \,.
      \]
      The elements of~$X(A)$ are the~\emph{\dash{$A$}{valued} points}\index{points of an object}\index{object!points of} of~$X$.
    \item
      We denote by~$x  \inA X$ that~$x$ is an~\dash{$A$}{valued} points of~$X$ for some~$A \in \Ob(\Acat)$.
    \item
      Two point~$x, y \inA X$, say~$x \in X(A)$ and~$y \in X(B)$, are \emph{equivalent}\index{equivalence!of points of an object}\index{points of an object!equivalence} if there exists for some object~$C \in \Ob(\Acat)$ epimorphisms~$u \colon C \to A$ and~$v \colon C \to B$ that make the square
      \[
        \begin{tikzcd}
            C
            \arrow{r}[above]{u}
            \arrow{d}[left]{v}
          & A
            \arrow{d}[right]{x}
          \\
            B
            \arrow{r}[above]{y}
          & X
        \end{tikzcd}
      \]
      commute.
      That the points~$x$ and~$y$ are equivalent is denoted by~$x \equiv y$.
      
      This concept of equivalence~$\equiv$ defines an equivalence relation on the class of points of~$X$:
      Every point~$x \inA X$, say~$x \in X(A)$, is equivalent to itself because square
      \[
        \begin{tikzcd}
            A
            \arrow{r}[above]{\id_A}
            \arrow{d}[left]{\id_A}
          & A
            \arrow{d}[right]{x}
          \\
            A
            \arrow{r}[above]{x}
          & X
        \end{tikzcd}
      \]
      commutes.
      The relation~$\equiv$ is symmetric because the definition of~$x \equiv y$ is symmetric in~$x$ and~$y$.
      Let~$x, y, z \inA X$, say
      \[
        x \in X(A) \,,
        \quad
        y \in X(B) \,,
        \quad
        z \in X(C) \,,
      \]
      with~$x \equiv y$ and~$y \equiv z$.
      Let
      \[
        u \colon D \to A \,,
        \quad
        v \colon D \to B \,,
        \quad
        r \colon E \to B \,,
        \quad
        s \colon E \to C \,,
      \]
      be epimorphism that make the squares
      \[
        \begin{tikzcd}
            D
            \arrow{r}[above]{u}
            \arrow{d}[left]{v}
          & A
            \arrow{d}[left]{x}
          \\
            B
            \arrow{r}[above]{y}
          & X
        \end{tikzcd}
        \qquad\text{and}\qquad
        \begin{tikzcd}
            E
            \arrow{r}[above]{r}
            \arrow{d}[left]{s}
          & B
            \arrow{d}[left]{y}
          \\
            C
            \arrow{r}[above]{z}
          & X
        \end{tikzcd}
      \]
      commute.
      Together with the pullback square
      \[
        \begin{tikzcd}
            F
            \arrow[dashed]{r}[above]{r'}
            \arrow[dashed]{d}[left]{v'}
            \arrow[phantom]{dr}[description]{\pb}
          & D
            \arrow{d}[left]{v}
          \\
            E
            \arrow{r}[above]{r}
          & B
        \end{tikzcd}
      \]
      we get the following commutative diagram:
      \[
        \begin{tikzcd}[sep = large]
            F
            \arrow[dashed]{r}[above]{r'}
            \arrow[dashed]{d}[left]{v'}
            \arrow[phantom]{dr}[description]{\pb}
          & D
            \arrow{r}[above]{u}
            \arrow{d}[left]{v}
          & A
            \arrow{d}[left]{x}
          \\
            E
            \arrow{r}[above]{r}
            \arrow{d}[left]{s}
          & B
            \arrow{r}[above]{y}
            \arrow{d}[left]{y}
          & X
            \arrow[dashed]{d}[left]{\id_X}
          \\
            C
            \arrow{r}[above]{z}
          & X
            \arrow[dashed]{r}[above]{\id_X}
          & X
        \end{tikzcd}
      \]
      The morphisms~$v'$ and~$r'$ are epimorphisms by \cref{mono epi under pull push} because the morphisms~$v$ and~$r$ are epimorphisms.
      The compositions~$u r' \colon F \to A$ and~$s v' \colon F \to C$ are therefore epimorphisms, and they make the square
      \[
        \begin{tikzcd}[sep = large]
            F
            \arrow{r}[above]{u r'}
            \arrow{d}[left]{s v'}
          & A
            \arrow{d}[left]{x}
          \\
            C
            \arrow{r}[above]{z}
          & X
        \end{tikzcd}
      \]
      commute.
      This shows that also~$x \equiv z$.
    \item
      Let~$0$ be the zero morphism~$0_\Acat \to X$.
      It holds for every point~$x \inA X$ that~$x \equiv 0$ if and only if there exists an epimorphism~$u \colon B \to A$ that makes the square
      \[
        \begin{tikzcd}
            B
            \arrow{r}[above]{v}
            \arrow{d}
          & A
            \arrow{d}[right]{x}
          \\
            0
            \arrow{r}
          & X
        \end{tikzcd}
      \]
      commute (because the zero morphism~$B \to 0$ is an epimorphism), i.e.\ such that~$xv = 0$.
      It then follows that~$x = 0$ because~$v$ is an epimorphism.;
      and if on the other hand~$x = 0$ then we can choose~$B = A$ and~$v = \id_A$.
      This shows that~$x \equiv 0$ if and only if~$x$ is the zero morphism~$A \to X$.
      
      This shows that the zero morphisms~$A \to X$ with~$A \in \Ob(\Acat)$ all give the same point of~$X$, which we will denote by~$0 \inA X$.
    \item
      Let~$Y \in \Ob(\Acat)$ be another object and let~$f \colon X \to Y$ be a morphism.
      Then~$f$ induces for every~$A \in \Ob(\Acat)$ a map
      \[
                X(A)
        \to     Y(A) \,,
        \quad   x
        \mapsto fx \,.
      \]
      If~$x, y \inA X$ with~$x \equiv y$ then also~$fx \equiv fy$:
      If~$x \in X(A)$ and~$y \in X(B)$ then there exist epimorphisms~$u \colon C \colon A$ and~$v \colon C \to B$ which make the square
      \[
        \begin{tikzcd}
            C
            \arrow[dashed]{r}[above]{u}
            \arrow[dashed]{d}[left]{v}
          & A
            \arrow{d}[right]{x}
          \\
            B
            \arrow{r}[above]{y}
          & X
        \end{tikzcd}
      \]
      commute.
      We then get the following commutative diagram:
      \[
        \begin{tikzcd}
            C
            \arrow{r}[above]{u}
            \arrow{d}[left]{v}
          & A
            \arrow[dashed, bend left]{ddr}[above right]{fx}
            \arrow{d}[left]{x}
          & {}
          \\
            B
            \arrow{r}[above]{y}
            \arrow[dashed, bend right]{drr}[below left]{fy}
          & X
            \arrow{dr}[above right]{f}
          & {}
          \\
            {}
          & {}
          & Y
        \end{tikzcd}
      \]
      The commutativity of the outer square
      \[
        \begin{tikzcd}
            C
            \arrow[dashed]{r}[above]{u}
            \arrow[dashed]{d}[left]{v}
          & A
            \arrow{d}[right]{fx}
          \\
            B
            \arrow{r}[above]{fy}
          & Y
        \end{tikzcd}
      \]
      shows that~$fx \equiv fy$.
  \end{enumerate}
\end{remarkdefinition}


\begin{remark*}
  Let~$\Acat$ be an abelian category and let~$X$ be an object in~$\Acat$.
  
  If~$x \inA X$, say~$x \in X(A)$, and~$u \colon A' \to A$ is an epimorphism, then~$x \equiv xu$.
  This follows from the commutativity of the following square:
  \[
    \begin{tikzcd}
        A'
        \arrow{r}[above]{u}
        \arrow{d}[left]{\id_{A'}}
      & A
        \arrow{d}[right]{x}
      \\
        A'
        \arrow{r}[above]{xu}
      & X
    \end{tikzcd}
  \]
  
  The equivalence relation~$\equiv$ is already determined by this property:
  Let~$\equiv'$ be the equivalence relation on the class of points of~$X$ that is generated by~$x \equiv' y$ for~$x, y \inA X$ whenever there exist an epimorphism~$u$ with~$y = xu$.
  Then the equivalence relations~$\equiv$ and $\equiv'$ coincide.
  
  Indeed, we have seen above that the equivalence relation~$\equiv$ is finer than the equivalence relation~$\equiv'$.
  Suppose on the other hand  that $x, y \inA X$ are points with~$x \equiv y$, say~$x \in X(A)$ and~$y \in X(B)$.
  Then let~$u \colon C \to A$ and~$v \colon C \to B$ be epimorphisms that make the square
  \[
    \begin{tikzcd}
        C
        \arrow{r}[above]{u}
        \arrow{d}[left]{v}
      & A
        \arrow{d}[right]{x}
      \\
        B
        \arrow{r}[above]{y}
      & X
    \end{tikzcd}
  \]
  commute.
  Then
  \[
            x
    \equiv' x u 
    =       y v
    \equiv' y \,.
  \]
  This shows that the equivalence relation~$\equiv'$ is finer than the equivalence relation~$\equiv$.
\end{remark*}


\begin{theorem}[Rules for diagram chase]
  \label{rules for diagram chase}
  Let~$\Acat$ be an abelian category.
  \begin{enumerate}
    \item
      For a morphism~$f \colon X \to Y$ in~$\Acat$ the following three conditions are equivalent:
      \begin{enumerate}
        \item
          The morphism~$f$ is a monomorphism.
        \item
          It follows for all~$x, x' \inA X$ from~$fx \equiv fx'$ that~$x \equiv x'$.
        \item
          It follows for every~$z \inA X$ from~$fz \equiv 0$ that~$z \equiv 0$.
      \end{enumerate}
    \item
      For a morphism~$g \colon Y \to Z$ in~$\Acat$ the following two conditions are equivalent:
      \begin{enumerate}
        \item
          The morphism~$g$ is an epimorphism.
        \item
          There exists for every~$z \inA Z$ some~$y \inA Y$ with~$gy \equiv z$.
      \end{enumerate}
    \item
      For a sequence~$X \xto{f} Y \xto{g} Z$ in~$\Acat$ the following two conditions are equivalent:
      \begin{enumerate}
        \item
          The sequence~$X \xto{f} Y \xto{g} Z$ is exact.
        \item
          It holds that~$gf = 0$ and there exists for every~$y \inA Y$ with~$gy \equiv 0$ some~$x \inA X$ with~$fx \equiv y$.
      \end{enumerate}
    \item
      \label{difference element}
      Let~$f \colon X \to Y$ be a morphism in~$\Acat$ and let~$x, x' \inA X$ be two points with~$fx \equiv fx'$.
      Then there exists a point~$\tilde{x} \inA X$ such that
      \begin{enumerate}
        \item
          $f \tilde{x} \equiv 0$, and
        \item
          it holds for every morphism~$h \colon X \to W$ that
          \begin{itemize}
            \item
              if~$hx' \equiv 0$ then~$h \tilde{x} \equiv hx$, and
            \item
              if~$hx \equiv 0$ then~$h \tilde{x} \equiv -hx'$.
          \end{itemize}
      \end{enumerate}
  \end{enumerate}
\end{theorem}


\begin{notationnonum}
  The morphism~$\tilde{x}$ from part~\ref*{difference element} of \cref{rules for diagram chase} is denoted by~$x - x'$.
\end{notationnonum}


\begin{proof}
  \renewcommand{\qedsymbol}{}
  \leavevmode
  \begin{enumerate}
    \item
      Suppose that~$f$ is a monomorphism and let~$x, x' \inA$ be two points with~$fx \equiv fx'$, say~$x \in X(A)$ and~$x' \in X(A')$.
      Then there exist epimorphisms~$u \colon B \to A$ and~$v \colon B \to A'$ that make the square
      \[
        \begin{tikzcd}
            B
            \arrow{rr}[above]{u}
            \arrow{dd}[left]{v}
          & {}
          & A
            \arrow{d}[right]{x}
          \\
            {}
          & {}
          & X
            \arrow{d}[right]{f}
          \\
            A'
            \arrow{r}[above]{x'}
          & X
            \arrow{r}[above]{f}
          & Y
        \end{tikzcd}
      \]
      commute, i.e.\ such that~$f x u = f x' v$.
      It follows from~$f$ being a monomorphism that~$x u = x' v$, i.e.\ that the square
      \[
        \begin{tikzcd}
            B
            \arrow{r}[above]{u}
            \arrow{d}[left]{v}
          & A
            \arrow{d}[right]{x}
          \\
            A'
            \arrow{r}[above]{x'}
          & X
        \end{tikzcd}
      \]
      commutes.
      This shows that~$x \equiv x'$.
      
      Suppose that it follows for all points~$x, x' \inA X$ from~$f x \equiv f x'$ that~$x \equiv x'$, and let~$z \inA X$ with~$f z \equiv 0$.
      Then~$fz \equiv 0 = f \circ 0$ and hence~$z \equiv 0$.
      
      Suppose lastly that~$z \equiv 0$ for every~$z \inA X$ with~$fz \equiv 0$.
      It then follows for every~$x \inA X$ that
      \[
                  fx = 0
        \implies  fx \equiv 0
        \implies  x \equiv 0
        \implies  x = 0 \,.
      \]
      This shows that~$\ker(f) = 0$ and hence that~$f$ is a monomorphism.
    \item
      Suppose that~$g$ is an epimorphism and let~$z \inA Z$.
      We consider the following pullback square:
      \[
        \begin{tikzcd}
            A'
            \arrow{r}[above]{g'}
            \arrow{d}[left]{y}
            \arrow[phantom]{dr}[description]{\pb}
          & A
            \arrow{d}[right]{z}
          \\
            Y
            \arrow{r}[above]{g}
          & Z
        \end{tikzcd}
      \]
      The morphism~$g'$ is by \cref{mono epi under pull push} again an epimorphism because~$g$ is an epimorphism.
      It therefore holds that
      \[
                z
        \equiv  z g'
        =       g y \,.
      \]
      
      Suppose on the other hand that for every point~$z \inA Z$ there exists some~$y \inA Y$ with~$gy \equiv z$.
      By choosing~$z = \id_Z$ we find that there exists some~$y \inA Y$ with ~$gy \equiv \id_Z$, say~$y \in X(A)$.
      Then exist epimorphisms~$u \colon B \to Z$ and~$v \colon B \to A$ that make the square
      \[
        \begin{tikzcd}
            B
            \arrow{r}[above]{u}
            \arrow{d}[left]{v}
          & Z
            \arrow{d}[right]{\id_Z}
          \\
            A
            \arrow{r}[above]{gy}
          & Z
        \end{tikzcd}
      \]
      commute.
      The composition~$gyv = u$ is therefore an epimorphism, hence the morphism~$g$ is an epimorphism.
  \end{enumerate}
  \lecturend{13}
  
\end{proof}














