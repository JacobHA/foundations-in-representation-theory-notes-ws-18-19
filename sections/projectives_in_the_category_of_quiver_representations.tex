\section{Projectives in the Category of Quiver Representations}


\begin{conventionnonum}
  Let~$\kf$ be a field,~$Q$ a finite quiver (i.e.~$Q_0$ and~$Q_1$ are finite), and abbreviate~$A \defined \kf Q$.
\end{conventionnonum}


\begin{definition}
  \leavevmode
  \begin{enumerate}
    \item
      For all vertices~$i, j \in Q_0$ let
      \[
        Q_*(i,j)
        =
        \{
          p \in Q_*
        \suchthat
          s(p) = i,
          t(p) = j
        \}
      \]
      be the set of paths from~$i$ to~$j$ in~$Q$.
    \item
      For every vertex~$i \in Q_0$ let~$P(i)$ be the representation of~$Q$ over~$\kf$ that is given by the following data:
      \begin{itemize}
        \item
          For every vertex~$j \in Q_0$ the~{\kvs}~$P(i)_j$ is the free~{\kvs} with basis~$Q_*(i,j)$.
        \item
          For every arrow~$\alpha \in Q_1$ with~$\alpha \colon j \to k$ the~{\klin} map~$P(i)_\alpha \colon P(i)_j \to P(i)_k$ is given on the basis~$Q(i,j)$ of~$P(i)_j$ by the concatenation of paths.
          We hence have for every~$p \in Q(i,j)$ that
          \[
            P(i)_\alpha(p)
            =
            \alpha \circ p  \,.
          \]
      \end{itemize}
  \end{enumerate}
\end{definition}


\begin{remark}
  Recall the equivalence of categories~$F \colon \Rep{\kf}{Q} \to \Modl{A}$ from part~\ref*{quiver reps are modules over path algebra} of \cref{examples for equivalences}:
  For every representation~$X$ of~$Q$ over~$\kf$ the~{\module{$A$}}~$M \defined F(X)$ has the underlying~{\kvs}
  \[
    M
    =
    \bigoplus_{j \in Q_0} X_j  \,,
  \]
  and the action of an arrow~$\alpha \colon j \to k$ (i.e.\ an element of the basis~$Q_*$ of~$A$) on~$M$ is given by the composition
  \[
    M
    \xlongto{\text{projection}}
    X_j
    \xlongto{X_\alpha}
    X_k
    \xlongto{\text{inclusion}}
    M \,.
  \]
  Under this equivalence of categories the representation~$P(i)$ corresponds to
  \[
    F(P(i))
    \cong
    A \varepsilon_i \,.
  \]
  
  Indeed, the~{\kvs}~$F(P(i)) = \bigoplus_{j \in Q_0} P(i)_j$ has as a basis the set of paths
  \begin{equation}
  \label{paths starting in i}
    \{
      p \in Q_*
    \suchthat
      s(p) = i
    \}  \,.
  \end{equation}
  This is also a basis of~$A \varepsilon_i$:
  We have for every~$p \in Q_*$ that
  \[
    p \varepsilon_i
    =
    \begin{cases}
      p & \text{if~$s(p) = i$}  \,, \\
      0 & \text{otherwise}  \,,
    \end{cases}
  \]
  and hence find that the linearly independent set~\eqref{paths starting in i} is a generating set, and therefore basis, for~$A \varepsilon_i$.
  The action of an arrow~$\alpha \in Q_1$ on a basis element~$p$ from~\eqref{paths starting in i} is for~$\alpha \colon j \to k$ given by
  \[
    \alpha p
    =
    \begin{cases}
      \alpha \circ p  & \text{if~$t(\alpha) = s(p)$}  \,, \\
      0               & \text{otherwise}  \,,
    \end{cases}
    =
    \begin{cases}
      \alpha \circ p  & \text{if~$s(p) = j$}  \,, \\
      0               & \text{otherwise}  \,,
    \end{cases}
    =
    P(i)_\alpha(p)  \,.
  \]
  This shows that the actions of~$A$ on~$M$ and~$A \varepsilon_i$ coincide.
\end{remark}


\begin{corollary}
  The representations~$P(i)$ of~$Q$ are projective objects of~$\Rep{\kf}{Q}$.
\end{corollary}


\begin{proof}
  It sufficies to show that the~{\modules{$A$}}~$A \varepsilon_i$ are projective.
  We observe that
  \begin{align*}
    A
    &=
    \bigoplus_{i \in Q_0}
    A \varepsilon_i
  \shortintertext{because on the level of bases}
    Q_*
    &=
    \coprod_{i \in Q_0}
    \{
      p \in Q_*
    \suchthat
      s(p) = i
    \}  \,.
  \end{align*}
  The~$A \varepsilon_i$ are therefore direct summands of the free~{\module{$A$}}~$A$, whence projective.
\end{proof}


\begin{lemma*}
  \label{homomorphisms out of P(i)}
  For every representation~$X$ of~$Q$ over~$\kf$ the map
  \[
    \Hom(P(i), X)
    \to
    X_i \,,
    \quad
    f
    \mapsto
    f_i(\varepsilon_i)
  \]
  is an isomorphism of~{\kvs}.
\end{lemma*}


\begin{proof}
  This is part~(i) of Exercise~4 of Exercise sheet~11.
\end{proof}


\begin{remark*}
  \label{P(i) projective via Hom}
  That the representations~$P(i)$ are projective can also be seen with the help of~\cref{homomorphisms out of P(i)}:
  For every representation~$X$ of~$Q$ over~$\kf$ let
  \[
    \varphi_X
    \colon
    \Hom(P(i), X)
    \to
    X_i
  \]
  be the above isomorphism.
  Let~$X$ and~$Y$ be two such representations and let~$f \colon X \to Y$ be an epimorphism of representations.
  We note the commutativity of the following square:
  \[
    \begin{tikzcd}
        \Hom(P(i), X)
        \arrow{r}[above]{f_*}
        \arrow{d}[left]{\varphi_X}
      & \Hom(P(i), Y)
        \arrow{d}[right]{\varphi_Y}
      \\
        X_i
        \arrow{r}[below]{f_i}
      & Y_i
    \end{tikzcd}
  \]
  The~{\klin} map~$f_i$ is surjective because~$f$ is an epimorphism.
  It follows that~$f_*$ is also surjective.
  This shows that the functor~$\Hom(P(i), -)$ maps epimorphisms to surjections, which means that~$P(i)$ is projective.
\end{remark*}


\begin{definition*}
  A quiver~$Q$ is \emph{acyclic}\index{acyclic!quiver}\index{quiver!acyclic} if it contains no oriented circles (of length~$\geq 1$).
\end{definition*}


\begin{corollary*}
  If the quiver~$Q$ is acyclic then every~$P(i)$ is~{\fd} with~$\End_k(P(i)) = k$.
\end{corollary*}


\begin{proof}
  This is part~(iii) of Exercise~4 of Exercise sheet~11.
\end{proof}


\begin{remark}
  If the quiver~$Q$ is acyclic then~$A$ is~{\fd} and every~$P(i)$ is indecomposable.
  (This is part~(iv) of Exercise~4 of Exercise sheet~11.)
\end{remark}


\begin{theoremnonum}[Krull--Remak--Schmidt]
  Let~$X$ be a~{\fd} representation of an acyclic quiver~$Q$.
  \begin{enumerate}
    \item
      There exists a decomposition~$X \cong X_1^{\oplus a_1} \oplus \dotsb \oplus X_r^{\oplus a_r}$ with~$X_i$ indecomposable,~$X_i \ncong X_j$ for~$i \neq j$ and~$a_i > 0$ for every~$i$.
    \item
      If~$X \cong Y_1^{\oplus b_1} \oplus \dotsb \oplus Y_s^{\oplus b_s}$ is another such decomposition then~$r = s$ and, up to reordering,~$X_i \cong Y_i$ and~$a_i = b_i$ for every~$i$.
  \end{enumerate}
\end{theoremnonum}


\begin{remarknonum}
  We will not prove the Krull--Remak--Schmidt theorem.
  A proof can be found in \cite{Elements}.
% TODO: Make this reference more specific.
\end{remarknonum}


\begin{remark*}
  The Krull--Remak--Schmidt theorem holds for every~{\kalg}~$A$ and every {\fd}~{\module{$A$}}~$M$;
  it holds even more generally for every ring~$R$ and every~{\module{$R$}}~$M$ of finite length.
  We will use the Krull--Remak--Schmidt theorem to show that every~{\fd} projective representation of~$Q$ over~$\kf$ is isomorphic to a direct sum of copies of~$P(i)$.
% TODO: When do we show this?
\end{remark*}


\begin{theorem}[The standard projective resolution]
  Let~$M$ be an~{\module{$A$}}.%
  \footnote{We do not require~$M$ to be~{\fd}, nor~$Q$ to be acyclic.}
  Then the sequence
  \[
    0
    \to
    \bigoplus_{\alpha \in Q_1}
    A \varepsilon_{t(\alpha)} \tensor_\kf \varepsilon_{s(\alpha)} M
    \xlongto{f}
    \bigoplus_{i \in Q_0}
    A \varepsilon_i \tensor_\kf \varepsilon_i M
    \xlongto{g}
    M
    \to
    0
  \]
  given by the~{\klin}~maps
  \begin{align*}
    g( (a_i \tensor x_i)_i )
    &\defined
    \sum_{i \in Q_0} a_i x_i
  \shortintertext{and}
    f( (a_\alpha \tensor x_\alpha)_\alpha )
    &\defined
    \sum_{\alpha \in Q_1}
    \Bigl(
      \iota_{s(\alpha)}( a_\alpha \alpha \tensor x_\alpha )
    - \iota_{t(\alpha)}( a_\alpha \tensor \alpha x_\alpha )
    \Bigr)
  \end{align*}
  is exact, and the appearing~{\modules{$A$}}
  \[
    P_0
    \defined
    \bigoplus_{i \in Q_0} A \varepsilon_i \tensor_\kf \varepsilon_i M
    \qquad\text{and}\qquad
    P_1
    \defined
    \bigoplus_{\alpha \in Q_1} A \varepsilon_{t(\alpha)} \tensor_\kf \varepsilon_{s(\alpha)} M
  \]
  are projective.
\end{theorem}


\begin{proof}
  The~{\klin} maps~$f$ and~$g$ are {\welldef}, and~$f \circ g = 0$.
  We have for every~$m \in M$ that
  \[
    m
    =
    1 \cdot m
    =
    \sum_{i \in Q_0} \varepsilon_i m
    =
    \sum_{i \in Q_0} \varepsilon_i^2 m
    =
    g\left( \sum_{i \in Q_0} \varepsilon_i \otimes \varepsilon_i m \right)
  \]
  which shows that~$g$ is surjective.
  The~{\modules{$A$}}~$P_0$ and~$P_1$ are projective because
  \[
    P_0
    \cong
    \bigoplus_{i \in Q_0}
    (A \varepsilon_i)^{\oplus \dim \varepsilon_i M}
    \quad\text{and}\quad
    P_1
    \cong
    \bigoplus_{\alpha \in Q_1}
    (A \varepsilon_{t(\alpha)})^{\oplus \dim \varepsilon_{s(\alpha)} M}
  \]
  are direct sums of the~{\modules{$A$}}~$A \varepsilon_i$.
  
  To show the injectivity of~$f$ and the exactness at~$P_0$ we observe that every element~$\xi \in P_0$ can be uniquely written as
  \[
    \xi
    =
    \left(
      \sum_{\substack{p \in Q_* \\ s(p) = i}}
      p \tensor \xi_p
    \right)_{i \in Q_0}
  \]
  with~$\xi_p \in \varepsilon_i A$ for~$i = s(p)$, because the set~$\{ p \in Q_* \suchthat s(p) = i \}$ is a basis for~$A \varepsilon_i$.
  The \emph{length} of~$\xi \neq 0$ is given by
  \[
      \ell(\xi)
    = \max
      \{
        \ell(p)
      \suchthat
        \xi_p \neq 0
      \}  \,.
  \]
  
  \begin{claimnonum}
    For every~$\xi \in P_0$ the residue class~$\xi + \im(f)$ contains either~$0$ or an element of length~$0$.
  \end{claimnonum}
  
  \begin{proof}
    Suppose that~$\xi_p \neq 0$ for a path~$p \in Q_*$ with starting vertex~$i \defined s(p)$ such that~$\ell(p) \geq 1$.
    If~$\alpha$ denotes the starting arrow of~$p$ then there exists a unique path~$p' \in Q_*$ with~$p = p' \alpha$.
    We find for~$\zeta \in P_1$ with components~$\zeta_\alpha = p' \tensor \xi_p$ and~$\zeta_{\alpha'} = 0$ otherwise, i.e.
    \[
      ( \delta_{\alpha, \alpha'} p' \tensor \xi_p )_{\alpha' \in Q_1}
    \]
    that
    \[
      f( \zeta )
      =
        \iota_{s(\alpha)}(p' \alpha \tensor \zeta_p)
      - \iota_{t(\alpha)}(p' \tensor \alpha \zeta_p)
      =
        \iota_{s(\alpha)}(p \tensor \zeta_p)
      - \iota_{t(\alpha)}(p' \tensor \alpha \zeta_p)
    \]
    We have that~$s(\alpha) = s(p) = i$ because the path~$p$ starts with~$\alpha$.
    We hence find that the element
    \[
      \xi'
      \defined
      \xi - f(\zeta)
    \]
    differs from~$\xi$ in (at most) two coordinates:
    In the~\dash{$i$}{th} coordinate we’re losing the summand~$p \tensor \zeta_p$, while gaining the summand~$p' \tensor \alpha \zeta_p$ in the~\dash{$s(p')$}{th} coordinate.
    We note that both~$\xi'$ and~$\xi$ have the same residue class modulo~$\im(f)$, and that~$\ell(p') < \ell(p)$.
    
    We have thus shows that by changing the representative of the residue class~$\xi + \im(f)$ from~$\xi$ to~$\xi'$ we can replace a summand of length~$d$ in one coordinate by a summand of smaller length in another coordinate.
    By repeating this process finitely many times we arrive at a representative of the residue class~$\xi + \im(f)$ that is either of length~$0$ or just~$0$ itself.
%     We find for the element~$\zeta \in P_1$ given by
%     \[
%       \zeta
%       \defined
%       \left(
%         \sum_{\subalign{p &\in Q_* \\ \ell(p) &= d \\ p &= q \alpha}}
%         q \tensor \xi_p
%       \right)_{\alpha \in Q_1}
%     \]
%     that
%     \[
%       f(\zeta)
%       =
%       \sum_{\alpha \in Q_1}
%       \left(
%         \iota_{s(\alpha)}
%         \left(
%           \sum_{\subalign{p &\in Q_* \\ \ell(p) &= d}}
%             p \tensor \xi_p
%         \right)
%         -
%         \iota_{t(\alpha)}
%         \left(
%           \sum_{\subalign{p &\in Q_* \\ \ell(p) &= d \\ p &= q \alpha}}
%             q \tensor \alpha \xi_p
%         \right)
%       \right)
%     \]
%     It follows for the element~$\xi' \in P_0$ that
%     \[
%       \xi'
%       \defined
%       \xi- f(\zeta) \,.
%     \]
%     The elements~$\xi$ and~$\xi'$ have the same residue class module~$\im(f)$.
%     All paths of length~$d$ for~$\xi$ are canceled by~$f(\zeta)$, whence~$\deg(\xi') < d$.
%     The claim follows by induction hypothesis.
  \end{proof}
  
  We now show that~$\ker(g) \subseteq \im(f)$:
  Suppose that there exists some~$\xi \in \ker(g)$ with~$\xi \notin \im(f)$.
  Then the residue class~$\xi + \im(f)$ does not contain~$0$, and so we may assume by the above claim that~$\ell(\xi) = 0$.
  Then
  \[
    \xi
    =
    (
      \varepsilon_i \tensor \xi_{\varepsilon_i}
    )_{i \in Q_0}
  \]
  with~$\xi_{\varepsilon_i} \in \varepsilon_i M$ for every~$i \in Q_0$.
  We have that
  \[
    0
    =
    g(\xi)
    =
    \sum_{i \in Q_0}
    \varepsilon_i \xi_{\varepsilon_i}
    =
    \sum_{i \in Q_0}
    \xi_{\varepsilon_i}
  \]
  because it follows from~$\xi_{\varepsilon_i} \in \varepsilon_i M$ that~$\varepsilon_i \xi_{\varepsilon_i} = \xi_{\varepsilon_i}$.
  We find from the directness of the sum~$M = \bigoplus_{i \in Q_0} \varepsilon_i M$ that~$\xi_{\varepsilon_i} = 0$ for every~$i \in Q_0$, and hence~$\xi = 0$.
  But this contradicts~$\xi \notin \im(f)$.
  We find that indeed~$\ker(g) \subseteq \im(f)$.
  
  All that is left to show is the injectivity of~$f$:
  We may write~$\eta \in P_1$ uniquely as
  \[
    \eta
    =
    \left(
      \sum_{\substack{p \in Q_* \\ s(p) = t(\alpha)}}
      p \tensor \eta_{\alpha,p}
    \right)_{\alpha \in Q_1}
  \]
  with~$\eta_{\alpha, p} \in \varepsilon_{s(\alpha)} A$ for every~$\alpha \in Q_1$ and path~$p \in Q_*$ with~$s(p) = t(\alpha)$.
  Suppose that~$\eta \neq 0$.
  Let~$p_0 \in Q_*$ for which there exists an arrow~$\alpha \in Q_1$ with~$\eta_{\alpha, p_0} \neq 0$, and suppose that~$p_0$ is of maximal length with this property.
 We may write the element
  \[
    \xi
    \defined
    f(\eta)
    =
    \sum_{\alpha \in Q_1}
    \left(
      \iota_{s(\alpha)}
      \left(
        \sum_{\substack{p \in Q_* \\ s(p) = t(\alpha)}}
        p \alpha \tensor \eta_{\alpha,p}
      \right)
      -
      \iota_{t(\alpha)}
      \left(
        \sum_{\substack{p \in Q_* \\ s(p) = t(\alpha)}}
        p \tensor \alpha \eta_{\alpha,p}
      \right)
    \right)
  \]
  as
  \[
    \xi
    =
    \left(
      \sum_{\substack{p \in Q_* \\ s(p) = i}}
      p \tensor \xi_p
    \right)_{i \in Q_0}
  \]
  in the same way as before.
  We then find that
  \[
    \xi_{p_0 \alpha}
    =
    \eta_{\alpha, p_0}
    \neq
    0 \,.
  \]
  and hence that~$f(\eta) \neq 0$.
  This shows that~$\ker(f) = 0$.
\end{proof}


\begin{remark}
  Under the equivalence of categories~$F \colon \Rep{\kf}{Q} \to \Modl{\kf Q}$ the standard projective resolution of a representation~$X$ is given by
  \[
    0
    \to
    \bigoplus_{\alpha \in Q_1}
    P(t(\alpha)) \tensor_\kf X_{s(\alpha)}
    \to
    \bigoplus_{i \in Q_0}
    P(i) \tensor_\kf X_i
    \to
    X
    \to
    0 \,.
  \]
\end{remark}


\begin{corollary}
  Let~$X$ be a representation~$Q$ over~$\kf$.
  Then $\Right^n \Hom(-,X) = 0$ for every~$n \geq 2$.
\end{corollary}


\begin{definition}
  The~\dash{$\Integer$}{bilinear} form
  \[
    \bil{-,-}
    \colon
    \Integer^{Q_0} \times \Integer^{Q_0}
    \to
    \Integer
  \]
  given by
  \[
    \bil{d,e}
    =
    \sum_{i \in Q_0}
    d_i e_i
    -
    \sum_{\alpha \in Q_1}
    d_{s(\alpha)} e_{t(\alpha)}
  \]
  is the \emph{Euler form}\index{Euler form} of~$Q$.
\end{definition}


\begin{definition*}
  The \emph{dimension vector}\index{dimension vector} of a {\fd} representation~$X$ of~$Q$ over~$\kf$ is the tupel
  \[
    \dimvect X
    \defined
    (\dim X_i)_{i \in Q_0}
  \]
\end{definition*}


\begin{corollary}
  Let~$X$ and~$Y$ be~{\fd} representations of~$Q$ over~$\kf$.
  Then
  \[
    \dim \Hom(X,Y)
    -
    \dim (\Right^1 \Hom(-,Y))(X)
    =
    \bil{\dimvect X, \dimvect Y}  \,.
  \]
\end{corollary}


\begin{proof}
  Let
  \[
    \underbrace{
    \dotsb
    \to
    0
    \to
    0
    \to
    P_1
    \to
    P_0
    }_{\Pcc}
    \to
    X
    \to
    0
  \]
  be the standard projective resolution of~$X$.
  By applying the functor~$\Hom(-,Y)$ to this resolution we arrive at the cochain complex
  \[
    \Hom(\Pcc, Y)
    =
    \bigl(
      \dotsb
      \to
      0
      \to
      0
      \to
      \Hom(P_0, Y)
      \to
      \Hom(P_1, Y))
      \to
      0
      \to
      0
      \to
      \dotsb
    \bigr)  \,.
  \]
  We find with the universal property of the coproduct and \cref{homomorphisms out of P(i)} that
  \begin{align*}
    \Hom(P_0, Y)
    &=
    \Hom
    \left(
      \bigoplus_{i \in Q_0}
      P(i) \tensor_\kf X_i,
      Y
    \right)
    \\
    &\cong
    \left(
      \bigoplus_{i \in Q_0}
      P(i)^{\oplus \dim X_i},
      Y
    \right)
    \\
    &\cong
    \bigoplus_{i \in Q_0}
    \Hom( P(i), Y )^{\oplus \dim X_i}
    \\
    &\cong
    \bigoplus_{i \in Q_0}
    Y_i^{\oplus \dim X_i}
  \end{align*}
  and hence
  \[
    \dim \Hom(P_0, Y)
    =
    \sum_{i \in Q_0} \dim X_i \dim Y_i  \,.
  \]
  We similarly find that
  \[
    \dim \Hom(P_1, Y)
    =
    \sum_{\alpha \in Q_1}
    \dim X_{s(\alpha)} \dim Y_{t(\alpha)} \,.
  \]
  We get from the short exact sequence
  \[
    0
    \to
    P_1
    \to
    P_0
    \to
    X
    \to
    0
  \]
  the induced (long) exact equence
  \[
    0
    \to
    \Hom(X,Y)
    \to
    \Hom(P_0,Y)
    \to
    \Hom(P_1,Y)
    \to
    (\Right^1 \Hom(-,Y))(X)
    \to
    0
    \to
    \dotsb
  \]
  where we use that~$(\Right^1 \Hom(-,Y))(P_0) = 0$ because~$P_0$ is projective.
  It follows that
  \begin{align*}
    {}&
      \dim \Hom(X,Y)
    - \dim (\Right^1 \Hom(-,Y))(X)
    \\
    ={}&
      \dim \Hom(P_0, Y)
    - \Hom \Hom(P_1, Y)
    \\
    ={}&
      \sum_{i \in Q_0} \dim X_i \dim Y_i
    - \sum_{\alpha \in Q_1} \dim X_{s(\alpha)} \dim Y_{t(\alpha)}
    \\
    ={}&
      \bil{ \dimvect X, \dimvect Y }
  \end{align*}
  as claimed.
\end{proof}





\lecturend{24}




