\chapter{Abelian Categories}





\section{Monomorphisms and Epimorphisms}


\begin{definition}
  Let~$f \colon X \to Y$ be a morphism in a category~$\Ccat$.
  \begin{enumerate}
    \item
      The morphism~$f$ is a \emph{monomorphism}\index{monomorphism}\index{morphism!mono-} if it follows for every pair of parallel morphisms~$u, v \colon Z \to X$ from~$f \circ u = f \circ v$ that already~$u = v$.
    \item
      The morphism~$f$ is an \emph{epimorphism}\index{epimorphism}\index{morphism!epi-} if it follows for every pair of parallel morphisms~$u, v \colon Y \to Z$ from~$u \circ f = v \circ f$ that already~$u = v$.
  \end{enumerate}
\end{definition}


\begin{remark}
  Let~$f \colon X \to Y$ and~$g \colon Y \to Z$ be composable morphisms in a category~$\Ccat$.
  \begin{enumerate}
    \item
      If~$f$ is an isomorphism then it is both a monomorphism and an epimorphism.
    \item
      If both~$f$ and~$g$ are monomorphisms then their composition~$g \circ f$ is again a monomorphism.
      If both~$f$ and~$g$ are epimorphisms then their composition~$g \circ f$ is again an epimorphism.
    \item
      If the composition~$g \circ f$ is a monomorphism then~$f$ is a monomorphism.
      If the composition~$g \circ f$ is an epimorphism then~$g$ is an epimorphism.
    \item
      The morphism~$f$ is a monomorphism (in~$\Ccat$) if and only if it is an epimorphism in~$\Ccat^\op$.
  \end{enumerate}
\end{remark}


\begin{example}
  We give examples of monomorphisms.
  \begin{enumerate}
    \item
      In the category~$\Set$ the monomorphisms are precisely the injective maps.
      The same holds for the categories~$\Modl{A}$,~$\Group$,~$\Ring$,~$\CommRing$,~$\kAlg$,~$\kCommAlg$,~$\Top$.
    \item
      If~$Q$ is a quiver then in its path category~$\Path(Q)$ every morphism is a monomorphism:
      Let~$p = \alpha_\ell \dotsm \alpha_1$ be a morphism in~$Q$, i.e.\ a path in~$Q$.
      If~$u = u_r \dotsm u_1$ and~$v = v_s \dotsm v_1$ are morphisms in~$\Path(Q)$, i.e.\ paths in~$Q$, with~$s(u) = s(v)$ and~$t(u) = t(v) = s(p)$ then the equality~$p \circ u = p \circ v$ means that
      \[
          \alpha_\ell \dotsm \alpha_1 u_r \dotsm u_1
        = \alpha_\ell \dotsm \alpha_1 v_s \dotsm v_1 \,.
      \]
      It then follows that~$r = s$ and~$u_i = v_i$ for all~$i = 1, \dotsc, r$.
    \item
      Let~$\Conn_*$ be the category of pointed, connected topological spaces:
      The objects of~$\Conn_*$ are pairs~$(X, x_0)$ consisting of a connected topological space~$X$ and a base point~$x_0 \in X$.
      A morphism~$f \colon (X, x_0) \to (Y, y_0)$ is a continuous map~$f \colon X \to Y$ with~$f(x_0) = y_0$.
      The morphism~$f \colon (\Real, 0) \to (S^1, 1)$ with~$f(x) = e^{2 \pi i x}$ is then a monomorphism.
  \end{enumerate}
\end{example}


\begin{example}
  We give examples for epimorphisms.
  \begin{enumerate}
    \item
      In the category~$\Set$ a morphism is an epimorphism if and only if it surjective.
      The same holds for the category~$\Group$ of groups.%
      \footnote{This is as evident as one may suspect at first glance.}
    \item
      If~$Q$ is a quiver then in its path category~$\Path(Q)$ every morphism in an epimorphism.
    \item
      Let~$\Haus$ be the category of Hausdorff topological spaces (where morphisms are just continuous maps).
      A morphism~$f \colon X \to Y$ in~$\Haus$ is an epimorphism if and only if it has dense image.
    \item
      Let~$A$ be a commutative ring and let~$S \subseteq A$ be a multiplicative set.
      Then the canonical map~$f \colon A \to S^{-1} A$,~$a \mapsto a/1$ is an epimorphism:
      If~$u,v \colon S^{-1} A \to B$ are two ring homomorphisms with~$u \circ f = v \circ f$ then~$u(a/1) = v(a/1)$ for every~$a \in A$.
      It then follows for every fraction~$a/s \in S^{-1} A$ that
      \[
          u\left( \frac{a}{s} \right)
        = u\left( \frac{a}{1} \right) u\left( \frac{s}{1} \right)^{-1}
        = v\left( \frac{a}{1} \right) v\left( \frac{s}{1} \right)^{-1}
        = v\left( \frac{a}{s} \right) \,,
      \]
      which then shows that~$u = v$.
  \end{enumerate}
\end{example}


% TODO: Add a proof that the epimorphisms in Grp are precisely the surjective homomorphisms.




\section{Special Objects}


\begin{definition}
  Let~$X$ be an object in a category~$\Ccat$.
  \begin{enumerate}
    \item
      The object~$X$ is \emph{initial}\index{initial object}\index{object!initial} if there exists  for every object~$Y \in \Ob(\Ccat)$ a unique morphism~$X \to Y$ in~$\Ccat$.
    \item
      The object~$X$ is \emph{terminal}\index{terminal object}\index{object!terminal} or \emph{final}\index{final object}\index{object!final} if there exists for every object~$Y \in \Ob(\Ccat)$ a unique morphism~$Y \to X$ in~$\Ccat$.
    \item
      The object~$X$ is a \emph{zero object}\index{zero!object}\index{object!zero} if it is both initial and terminal.
  \end{enumerate}
\end{definition}


\begin{remark}
  Let~$\Ccat$ be a category.
  \begin{enumerate}
    \item
      An object~$X$ of~$\Ccat$ is initial (in~$\Ccat$) if and only if it is terminal in~$\Ccat^\op$.
    \item
      Initial and terminal objects are unique up to unique isomorphisms (if they exist).
    \item
      If~$\Ccat$ admits a zero object then it is denoted by~$0 = 0_\Ccat$.
  \end{enumerate}
\end{remark}


\begin{example}
  \leavevmode
  \begin{enumerate}
    \item
      In the category~$\Set$, the empty set~$\emptyset$ is the unique initial object, and every \dash{one}{point} set~$\{\ast\}$ is a final object.
      The analogous statements hold for the category~$\Top$.
    \item
      In the category~$\Modl{A}$ the zero module~$0$ is the zero object.
    \item
      In the category~$\Group$ the trivial groups~$1$ is the zero object.
    \item
      In the category~$\kAlg$ the~{\kalg}~$\kf$ is initial, while the zero algebra~$0$ is final.
  \end{enumerate}
\end{example}


\begin{remarkdefinition}
  Let~$\Ccat$ be a category which admits a zero object~$0$.
  Then there exists for any two objects~$X, Y \in \Ob(\Ccat)$ a unique morphism~$0_{XY} \colon X \to Y$ which factors trough the zero object, i.e.\ which makes the triangle
  \[
    \begin{tikzcd}
        X
        \arrow{dr}
        \arrow{rr}[above]{0_{XY}}
      & {}
      & Y
      \\
        {}
      & 0
        \arrow{ur}
      & {}
    \end{tikzcd}
  \]
  commute.
  The morphism~$0_{XY}$ is the \emph{zero morphism}\index{zero!morphism}\index{morphism!zero} from~$X$ to~$Y$.
\end{remarkdefinition}


\begin{remark*}
  Let~$\Ccat$ be a category which admits a zero object~$0$.
  Then it holds for every morphism~$f \colon X \to Y$ that
  \[
      f \circ 0_{W,X}
    = 0_{W,Y}
    \qquad\text{and}\qquad
      0_{Y,Z} \circ f
    = 0_{X,Z}
  \]
  for all~$W, Z \in \Ob(\Ccat)$.
  Indeed, we have the following commutative diagrams:
  \[
    \begin{tikzcd}
        W
        \arrow[dashed, bend left = 40]{rrr}[above]{f \circ 0_{W,X}}
        \arrow{rr}[above]{0_{W,X}}
        \arrow{dr}
      & {}
      & X
        \arrow{r}[above]{f}
      & Y
      \\
        {}
      & 0
        \arrow{ur}
        \arrow[dashed, bend right]{urr}
      & {}
      & {}
    \end{tikzcd}
    \qquad
    \begin{tikzcd}
        X
        \arrow[dashed, bend left = 40]{rrr}[above]{0_{Y,Z} \circ f}
        \arrow{r}[above]{f}
        \arrow[dashed, bend right]{drr}
      & Y
        \arrow{rr}[above]{0_{Y,Z}}
        \arrow{dr}
      & {}
      & Z
      \\
        {}
      & {}
      & 0
        \arrow{ur}
      & {}
    \end{tikzcd}
  \]
  The commutativity of the first diagram shows that~$f \circ 0_{W,X}$ factors through the zero object~$0$, and is hence the zero morphism~$W \to Y$.
  The commutativity of the second diagram shows similarly that~$0_{Y,Z} \circ f$ factors through the zero objects~$0$ and is therefore the zero morphism~$X \to Z$.
\end{remark*}





\lecturend{8}





\section{Products and Coproducts}

\begin{definition}
  Let~$(X_i)_{i \in I}$ be a family of objects in a category~$\Ccat$.
  \begin{enumerate}
    \item
      A \emph{product}\index{product} of the family of objects~$(X_i)_{i \in I}$ is a pair~$(P, (p_i)_{i \in I})$ consisting of an object~$P \in \Ob(\Ccat)$ and morphisms~$p_i \colon P \to X_i$, such that for every other pair~$(Q, (q_i)_{i \in I})$ consisting of an object~$Q \in \Ob(\Ccat)$ and morphisms~$q_i \colon Q \to X_i$ there exists a unique morphism~$\lambda \colon Q \to P$ which makes the triangle
      \[
        \begin{tikzcd}[sep = large]
            P
            \arrow{d}[left]{p_i}
          & Q
            \arrow{dl}[below right]{q_i}
            \arrow[dashed]{l}[above]{\lambda}
          \\
            X_i
          & {}
        \end{tikzcd}
%         \begin{tikzcd}[sep = large]
%             Q
%             \arrow{dr}[above right]{q_i}
%             \arrow[dashed]{d}[left]{\lambda}
%           & {}
%           \\
%             P
%             \arrow{r}[above]{p_i}
%           & X_i
%         \end{tikzcd}
      \]
      commute for every~$i \in I$.
    \item
      A \emph{coproduct}\index{coproduct} of the family of objects~$(X_i)_{i \in I}$ is a pair~$(C, (c_i)_{i \in I})$ consisting of an object~$C \in \Ob(\Ccat)$ and morphisms~$c_i \colon X_i \to C$, such that for every pair~$(D, (d_i)_{i \in I})$ consisting of an object~$D \in \Ob(\Ccat)$ and morphisms~$d_i \colon X_i \to C$ there exists a unique morphism~$\mu \colon C \to D$ which makes the triangle
      \[
        \begin{tikzcd}[sep = large]
            C
            \arrow[dashed]{r}[above]{\mu}
          & D
          \\
            X_i
            \arrow{u}[left]{c_i}
            \arrow{ur}[below right]{d_i}
          & {}
        \end{tikzcd}
      \]
      commute for every~$i \in I$.
  \end{enumerate}
\end{definition}


\begin{remark}
  Let~$(X_i)_{i \in I}$ be a family of objects~$X_i$ in a category~$\Ccat$.
  \begin{enumerate}
    \item
      A pair~$(P,(p_i)_{i \in I})$ is a product of the family~$(X_i)_{i \in I}$ in~$\Ccat$ if and only if it is a coproduct of this family in~$\Ccat^\op$.
    \item
      Products are unique up to unique isomorphism, i.e.\ if~$(P, (p_i)_i)$ and~$(P', (p'_i)_{i \in I})$ are two products of the family~$(X_i)_{i \in I}$ in~$\Ccat$ then there exist a unique morphism~$\lambda \colon P \to P'$ which makes the triangle
      \[
        \begin{tikzcd}
            P
            \arrow[dashed]{rr}[above]{\lambda}
            \arrow{dr}[below left]{p_i}
          & {}
          & P'
            \arrow{dl}[below right]{p'_i}
          \\
            {}
          & X_i
          & {}
        \end{tikzcd}
      \]
      commute for every~$i \in I$, and the morphism~$\lambda$ is already an isomorphism.
      Similarly, coproducts are unique up to unique isomorphism.
    \item
      The product of the family~$(X_i)_{i \in I}$ is denoted by~$\prod_{i \in I} X_i$, or by~$X_1 \times \dotsb \times X_n$ if~$I = \{1, \dotsc, n\}$.
      The coproduct of the family~$(X_i)_{i \in I}$ is denoted by~$\coprod_{i \in I} X_i$, or by~$X_1 \dcup \dotsb \dcup X_n$ if~$I = \{1, \dotsc, n\}$.%
      \footnote{In the lecture the notations~$\bigsqcap_{i \in I} X_i$, resp.~$\bigsqcup_{i \in I} X_i$ and~$X_1 \sqcup \dotsb \sqcup X_n$ are used instead.}
    \item
      If every family of objects in~$\Ccat$ has a product (resp.\ coproduct) in~$\Ccat$ then we say that that~$\Ccat$ \emph{has products} (resp.\ \emph{has coproducts}).
      If every finite family of objects in~$\Ccat$ has a product (resp.\ coproducts) in~$\Ccat$ then we say that~$\Ccat$ \emph{has finite coproducts} (resp.\ \emph{has finite coproducts}).
  \end{enumerate}
\end{remark}


\begin{example}
  In the following let~$(X_i)_{i \in I}$ be a family objects in the given category~$\Ccat$.
  \begin{enumerate}
    \item
      Let~$\Ccat = \Set$.
      Then the (categorical) product~$\prod_{ \in I} X_i$ is the cartesian product, and the map~$p_i \colon \prod_{j \in I} X_j \to X_i$ is for every~$i \in I$ the usual projections onto the~\dash{$i$}{th} factor.
      The coproduct~$\coprod_{i \in I} X_i$ is the (formal) disjoint union of the sets~$X_i$, and the map~$c_i \colon X_i \to \coprod_{j \in I} X_j$ is for every~$i \in I$ the inclusion into the~\dash{$j$}{th} set.
    \item
      Let~$\Ccat = \Modl{A}$.
      Then the (categorical) product~$\prod_{i \in I} X_i$ is the cartesian product, and the morphism~$p_i \colon \prod_{j \in I} X_j \to X_i$ is for every~$i \in I$ the usual projection onto the~\dash{$i$}{th} factor.
      The coproduct of the family~$(X_i)_{i \in I}$ is the direct sum~$\bigoplus_{i \in I} X_i$, and the morphisms~$c_i \colon X_i \to \bigoplus_{j \in I} X_j$ is for every~$i \in I$ the inclusion into the~\dash{$i$}{th} summand.
    \item
      Let~$\Ccat = \kCommAlg$.
      Then the (categorical) product~$\prod_{i \in I} X_i$ is the cartesian product, and the morphism~$p_i \colon \prod_{j \in I} X_j \to X_i$ is for every~$i \in I$ the usual projection onto the~\dash{$i$}{th} factor.
      The coproduct of finitely many commutative~{\kalg}~$A_1, \dotsc, A_n$ in the category~$\kCommAlg$ is their tensor product~$A_1 \tensor \dotsb \tensor A_n$, and the morphism~$c_i \colon A_i \to A_1 \tensor \dotsb \tensor A_n$ is for every~$i \in I$ the inclusion into the~\dash{$i$}{th} factor, i.e.\ the algebra homomorphism
      \[
                A_i
        \to     A_1 \tensor \dotsb \tensor A_n \,,
        \quad   x
        \mapsto 1 \tensor \dotsb \tensor 1 \tensor x \tensor 1 \tensor \dotsb \tensor 1 \,.
      \]
      (The coproduct of an arbitrary family~$(A_i)_{i \in I}$ of commutative~{\kalgs} in the category~$\kCommAlg$ can be described similarly.)
  \end{enumerate}
\end{example}


\begin{remark*}
  Let~$\Ccat$ be a category.
  A product of an empty family of objects in~$\Ccat$ is the same a terminal object of~$\Ccat$.
  A coproduct of an empty family of objects in~$\Ccat$ is the same an initial object of~$\Ccat$.
\end{remark*}


\begin{lemma}
  \label{existence of coproducts}
  Let~$(X_i)_{i \in I}$ be a family of objects in a category~$\Ccat$.
  \begin{enumerate}
    \item
      The following are equivalent:
      \begin{enumerate}
        \item
          The product~$\prod_{i \in I} X_i$ exists in~$\Ccat$.
        \item
          The functor~$\Ccat^\op \to \Set$ given by~$Y \mapsto \prod_{i \in I} \Ccat(Y,X_i)$ is representable.
      \end{enumerate}
    \item
      \label{for coproducts}
      The following are equivalent:
      \begin{enumerate}
        \item
          The coproduct~$\coprod_{i \in I} X_i$ exists in~$\Ccat$.
        \item
          The functor~$\Ccat \to \Set$ given by~$Y \mapsto \prod_{i \in I} \Ccat(X_i,Y)$ is representable.
      \end{enumerate}
  \end{enumerate}
\end{lemma}


\begin{proof}
  It sufficies to show part~\ref*{for coproducts}.
  We denote the given functor~$\Ccat \to \Set$ by~$F$.
  
  Suppose first that the coproduct~$\coprod_{i \in I} X_i$ exists, and denote the associated morphisms by~$c_i \colon X_i \to \coprod_{j \in I} X_j$.
  We claim that the functor~$F$ is represented by the coproduct~$\coprod_{i \in I} X_i$;
  we thus need to construct a natural isomorphism~$\eta \colon h^{(\coprod_{i \in I} X_i)} \to F$.
  We define the components of~$\eta$ as
  \begin{align*}
              \eta_Y
     \colon   h^{(\coprod_{i \in I} X_i)}(Y)
     =        \Ccat\left( \coprod_{i \in I} X_i, Y \right)
    &\to      \prod_{i \in I} \Ccat(X_i, Y)
     =        F(Y) \,,
     \\
              f
    &\mapsto  (f \circ c_i)_{i \in I} \,.
  \end{align*}
  Then the family~$\eta \defined (\eta_Y)_{Y \in \Ob(\Ccat)}$ defines a natural transformation~$\eta \colon h^{(\coprod_{i \in I} X_i)} \to F$.
  Indeed, for every morphism~$g \colon Y \to Y'$ in~$\Ccat$ the diagram
  \[
    \begin{tikzcd}
        h^{(\coprod_{i \in I} X_i)}(Y)
        \arrow{rrr}[above]{h^{(\coprod_{i \in I} X_i)}(g)}
        \arrow{ddd}[left]{\eta_Y}
        \arrow[equal]{dr}
      & {}
      & {}
      & h^{(\coprod_{i \in I} X_i)}(Y')
        \arrow{ddd}[right]{\eta_{Y'}}
        \arrow[equal]{dl}
      \\
        {}
      & \Ccat\left( \coprod_{i \in I} X_i, Y \right)
        \arrow{r}[above]{g_*}
        \arrow{d}[left]{\eta_Y}
      & \Ccat\left( \coprod_{i \in I} X_i, Y' \right)
        \arrow{d}[right]{\eta_{Y'}}
      & {}
      \\
        {}
      & \prod_{i \in I} \Ccat(X_i, Y)
        \arrow{r}[below]{\prod_{i \in I} g_*}
      & \prod_{i \in I} \Ccat(X_i, Y')
      & {}
      \\
        F(Y)
        \arrow[equal]{ur}
        \arrow{rrr}[below]{F(g)}
      & {}
      & {}
      & F(Y')
        \arrow[equal]{ul}
    \end{tikzcd}
  \]
  commutes, because the inner square is given on elements by
  \[
    \begin{tikzcd}[column sep = huge, row sep = large]
        f
        \arrow[maps to]{r}[above]{g_*}
        \arrow[maps to]{d}[left]{\eta_Y}
      & g \circ f
        \arrow[maps to]{d}[right]{\eta_{Y'}}
      \\
        (f \circ c_i)_{i \in I}
        \arrow[maps to]{r}[above]{\prod_{i \in I} g_*}
      & (g \circ f \circ c_i)_{i \in I}
    \end{tikzcd}
  \]
  and thus commutes.
  
  The natural transformation~$\eta$ is already a natural isomorphism:
  There exist at every objects~$Y \in \Ob(\Ccat)$ for every family of morphisms~$(h_i)_{i \in I} \in \prod_{i \in I} \Ccat(X_i, Y)$ by the definition of the coproduct~$\coprod_{i \in I} X_i$ a unique morphism~$g \in \Ccat(\coprod_{i \in I} X_i, Y)$ with~$g \circ c_i = h_i$ for every~$i \in I$, i.e.\ with~$\eta_Y(g) = (h_i)_{i \in I}$.
  This means that~$\eta_Y$ is bijective at every~$Y \in \Ob(\Ccat)$.
  
  Suppose now that the functor~$F$ is representable.
  Let~$C$ be a representing object for~$F$ and let~$\eta \colon h^C \to F$ be a natural isomorphism.
  Then~$\eta_C$ is a map
  \[
      \eta_C
    \colon
      h^C(C)
    =
      \Ccat(C,C)
    \to
      F(C)
    =
      \prod_{i \in I} \Ccat(X_i, C) \,.
  \]
  By setting $(c_i)_{i \in I} \defined \eta_C(\id_C)$ we therefore get for every~$i \in I$ a morphism~$c_i \colon X_i \to C$.
  We show that the pair~$(C, (c_i)_{i \in I})$ is a coproduct of the family~$(X_i)_{i \in I}$.
  So let~$(D, (d_i)_{i \in I})$ be another pair consisting of an object~$D \in \Ob(\Dcat)$ and a family~$(d_i)_{i \in I}$ of morphisms~$d_i \colon X_i \to D$.
  Then
  \[
        (d_i)_{i \in I}
    \in \prod_{i \in I} \Ccat(X_i, D)
    =   F(D)
  \]
  and it follows from~$\eta_D \colon h^C(D) \to F(D)$ being a bijection that there exist a unique element~$\lambda \in h^C(D) = \Ccat(C,D)$, i.e.\ morphism~$\lambda \colon C \to D$, with~$(d_i)_{i \in I} = \eta_D(\lambda)$.
  It follows from the naturality of~$\eta$ that the diagram
  \[
    \begin{tikzcd}[row sep = large]
        \Ccat(C,C)
        \arrow[equal]{r}
        \arrow{d}[left]{\lambda_*}
      & h^C(C)
        \arrow{r}[above]{\eta_C}
        \arrow{d}[left]{h^C(\lambda)}
      & F(C)
        \arrow{d}[right]{F(\lambda)}
        \arrow[equal]{r}
      & \prod_{i \in I} \Ccat(X_i, C)
        \arrow{d}[right]{\prod_{i \in I} \lambda_*}
      \\
        \Ccat(C,D)
        \arrow[equal]{r}
      & h^C(D)
        \arrow{r}[below]{\eta_D}
      & F(D)
        \arrow[equal]{r}
      & \prod_{i \in I} \Ccat(X_i, D)
    \end{tikzcd}
  \]
  commutes.
  It therefore follows for the element~$\id_C \in \Ccat(\Ccat, \Ccat) = h^C(C)$ that
  \begin{align*}
        (d_i)_{i \in I}
     =  \eta_D( \lambda )
     =  \eta_D( \lambda_*( \id_C) )
    &=  \eta_D( h^C(\lambda)( \id_C ) )  \\
    &=  F(\lambda)( \eta_C( \id_C) ) )
     =  F(\lambda)( (c_i)_{i \in I} )
     =  (\lambda \circ c_i)_{i \in I} \,,
  \end{align*}
  and hence that~$\lambda \circ c_i = d_i$ for every~$i \in I$.
  This shows the existence of the required morphism~$\lambda \colon C \to D$.
  By reading the above argumentation from the bottom to the top we also find that the morphism~$\lambda$ is unique.
  
  This shows that the object~$C$ together with the morphisms~$c_i \colon X_i \to C$ is a coproduct of the family~$(X_i)_{i \in I}$.
\end{proof}


\begin{remark*}
  The above proof actually shows that an object~$C \in \Ob(\Ccat)$ is a coproduct of the family~$(X_i)_{i \in I}$, with respect to some suitable morphisms~$c_i \colon X_i \to C$, if and only if it represents the functor~$\prod_{i \in I} \Ccat(X_i, -) \colon \Ccat \to \Set$.
  This statement is stronger than the formulation in \cref{existence of coproducts}.
  
  One can also show a slightly stronger version of this:
  It follows for every object~$C \in \Ob(\Ccat)$ from Yoneda’s~lemma that the map
  \begin{align*}
              \{ \text{natural transformations~$\eta \colon h^C \to F$} \}
    &\to      F(C) \,,
    \\
              \eta
    &\mapsto  \eta_C(\id_C)
  \end{align*}
  is a bijection.
  An element of the right hand side is an element of~$F(C)$, i.e.\ a family~$(c_i)_{i \in I}$ of morphisms~$c_i \colon X_i \to C$.
  It then holds that a natural transformation~$\eta \colon h^C \to F$ is an isomorphism if and only if the corresponding family~$(c_i)_{i \in I}$ makes the pair~$(C, (c_i)_{i \in I})$ into a coproduct of the family~$(X_i)_{i \in I}$.
  
  Indeed, that the natural transformation~$\eta$ is a natural isomorphism means that at every object~$D \in \Ob(\Ccat)$ the map
  \[
            \eta_D
    \colon  \Ccat(C,D)
    =       h^C(D)
    \to     F(D)
  \]
  is a bijection.
  The component~$\eta_D$ can be expressed via the element~$(c_i)_{i \in I} \in F(C)$ as
  \[
      \eta_D(\lambda)
    = F(\lambda)( (c_i)_{i \in I} )
    = (\lambda \circ c_i)_{i \in I} \,.
  \]
  The bijectivity of~$\eta_D$ therefore means that for every~$(d_i)_{i \in I} \in F(D) = \prod_{i \in I} \Ccat(X_i, D)$ there exists a unique element~$\lambda \in \Ccat(C, D)$ with~$(\lambda \circ c_i)_{i \in I} = (d_i)_{i \in I}$.
  In other words, there exists for every object~$D \in \Ob(\Ccat)$ and every family~$(d_i)_{i \in I}$ of morphisms~$d_i \colon X_i \to D$ a unique morphism~$\lambda \colon C \to D$ with~$d_i = \lambda \circ c_i$ for every~$i \in I$.
  But this is precisely what it means for the pair~$(C, (c_i)_{i \in I})$ to be a coproduct of the family~$(X_i)_{i \in I}$.
  
  This shows that for every object~$C \in \Ob(\Ccat)$, a family~$(c_i)_{i \in I}$ of morphisms~$c_i \colon X_i \to C$ that makes~$(C, (c_i)_{i \in I})$ into a coproduct of the family~$(X_i)_{i \in I}$ is \enquote{the same} as a natural isomorphism~$\eta \colon h^C \to F$ (via Yoneda’s~lemma).
  It follows in particular that~$C$ is a coproduct of the family~$(X_i)_{i \in I}$, with respect to a suitable choice of morphisms~$c_i \colon X_i \to C$, if and only if there exist a natural isomorphism~$h^C \to F$.
  (This is what was shown in the above proof.)
\end{remark*}





\section{Additive Categories}


\begin{definition}
  A \emph{preadditive~category}\index{pre-additive category}\index{category!pre-additive} is a category~$\Acat$ together with the structure of an abelian group on~$\Acat(X,Y)$ for all~$X, Y \in \Ob(\Acat)$ such that the composition in~$\Ccat$ is~{\Zbilin}, i.e.\ such that
  \[
    k \circ (g + h) = k \circ g + k \circ h
    \quad\text{and}\quad
    (g + h) \circ f = g \circ f + h \circ f
  \]
  for all morphisms~$f \colon W \to X$,~$g, h \colon X \to Y$ and~$k \colon Y \to Z$ in~$\Acat$.
\end{definition}


\begin{remark}
  \leavevmode
  \begin{enumerate}
    \item
      Preddative categories are also known as~\emph{\dash{$\Ab$}{categories}}\index{Ab-category@$\Ab$-category}\index{category!Ab-@$\Ab$-} (where~$\Ab$ denotes the category of abelian groups).
      (In the language of enriched category theory, an~\dash{$\Ab$}{category} is precisely a category that is enriched over the monoidal category~$(\Ab,{\tensor})$.)
    \item
      If~$\kf$ is a commutative ring then one can similarly define the notation of a \emph{{\preklin} category}\index{pre-k-linear category@{\preklin} category}\index{category!pre-k-linear@{\preklin}} (also known as~\emph{\dash{$\Modl{k}$}{category}}\index{k-Mod-category@$\Modl{\kf}$-category}\index{category!k-Mod-@$\Modl{\kf}$-})~$\Ccat$.
      Every~$\Ccat(X,Y)$ is then endowed with the structure of a~{\module{$\kf$}} and the composition is~{\kbilin}.
  \end{enumerate}
\end{remark}


\begin{example}
  \leavevmode
  \begin{enumerate}
    \item
      The category~$\Ab = \Modl{\Integer}$ is preadditive.
    \item
      If~$A$ is a~{\kalg} then the categories~$\Modl{A}$ and~$\Modr{A}$ are {\preklin}.
    \item
      If~$R$ is a ring then we may think about~$R$ as a preadditive category~$\Rcat$ consisting of a single object~$\Ob(\Rcat) = \{ \ast \}$ with~$\Rcat(\ast,\ast) = R$.
      The composition in~$\Rcat$ is given by the multiplication of~$R$, i.e.\ by~$g \circ f = gf$ for all~$f, g \in R$, and the addition of morphisms is the addition in~$R$.
      
      One can similarly regard every~{\kalg}~$A$ as a {\preklin} category~$\Acat$ which consists of a single object~$\Ob(\Acat) = \{\ast\}$ with~$\Acat(\ast,\ast) = A$.
    \item
      Let~$\Ccat$ be any category and let~$\Acat$ be a preadditive category.
      Then the functor category~$\Fun(\Ccat, \Acat)$ is again preadditive:
      For any two natural transformations~$\eta, \zeta \colon F \to G$ between functors~$F, G \in \Ob(\Fun(\Ccat, \Acat))$, their sum~$\eta + \zeta$ is at an object~$X \in \Ob(\Ccat)$ given by
      \[
          (\eta + \zeta)_X
        = \eta_X + \zeta_X \,.
      \]
      This defines again a natural transformation~$\eta + \zeta \colon F \to G$.
      Indeed, for every morphism~$f \colon X \to X'$ in~$\Ccat$ the square
      \[
        \begin{tikzcd}[sep = large]
            F(X)
            \arrow{r}[above]{F(f)}
            \arrow{d}[left]{\eta_X + \zeta_X}
          & F(X')
            \arrow{d}[right]{\eta_{X'} + \zeta_{X'}}
          \\
            G(X)
            \arrow{r}[above]{G(f)}
          & G(X')
        \end{tikzcd}
      \]
      commutes because
      \begin{align*}
            (\eta_{X'} \circ \zeta_{X'}) \circ F(f)
        &=  \eta_{X'} \circ F(f) + \zeta_{X'} \circ F(f)  \\
        &=  G(f) \circ \eta_X + G(f) \circ \zeta_X
         =  G(f) \circ (\eta_X + \zeta_X) \,.
      \end{align*}
      
      We find similarly that for every category~$\Ccat$ and every {\preklin} category~$\Acat$ the functor category~$\Fun(\Ccat, \Acat)$ is again~{\preklin}. 
  \end{enumerate}
\end{example}


\begin{remark*}
  \leavevmode
  \begin{enumerate}
    \item
      A preadditive category is the same as a pre\nobreakdash-$\Integer$\nobreakdash-linear category.
    \item
      If~$\Acat$ is a preadditive (resp.\ {\preklin}) category, then the opposite category~$\Acat^\op$ is again preadditive (resp.\ {\preklin}) with the same addition (resp.~{\module{$\kf$}} structure) of morphisms.
  \end{enumerate}
\end{remark*}


\begin{definition}
  Let~$F \colon \Acat \to \Bcat$ be a functor between categories~$\Acat$ and~$\Bcat$.
  \begin{enumerate}
    \item
      If~$\Acat$ and~$\Bcat$ are preaddive categories then the functor~$F$ is \emph{additive}\index{additive!functor}\index{functor!additive} if
      \[
          F(f + g)
        = F(f) + F(g)
      \]
      for all morphisms~$f, g \colon X \to X'$ in~$\Acat$, i.e.\ if the map
      \[
                    \Acat(X, Y)
        \xlongto{F} \Bcat(F(X), F(Y))
      \]
      is a group homorphism for all~$X, Y \in \Ob(\Acat)$.
    \item
      If~$\Acat$ and~$\Bcat$ are {\preklin} categories then the functor~$F$ is~\emph{{\klin}}\index{k-linear@$\kf$-linear!functor}\index{functor!k-linear@$\kf$-linear} if
      \[
        F(f + g) = F(f) + F(g)
        \quad\text{and}\quad
        F(\lambda f) = \lambda F(f)
      \]
      for all morphisms~$f, g \colon X \to X'$ in~$\Acat$ and scalars~$\lambda \in \kf$, i.e.\ if the map
      \[
                    \Acat(X, Y)
        \xlongto{F} \Bcat(F(X), F(Y))
      \]
      is~{\klin} for all~$X, Y \in \Ob(\Acat)$.%
      \footnote{The notion of a~{\klin} functor was not introduced in the lecture.}
  \end{enumerate}
\end{definition}


\begin{lemma}
  \label{inital terminal zero}
  Let~$\Acat$ be a preadditive category.
  \begin{enumerate}
    \item
      For any object~$X \in \Ob(\Acat)$ the following conditions are equivalent:
      \begin{enumerate}
        \item
          The object~$X$ is unital in~$\Acat$.
        \item
          The object~$X$ is terminal in~$\Acat$.
        \item
          The object~$X$ is a zero object for~$\Acat$.
        \item
          It holds that~$\id_X = 0_{\Acat(X,X)}$.
        \item
          The abelian group~$\Acat(X,X)$ consists of only a single element.
      \end{enumerate}
    \item
      Suppose that the category~$\Acat~$ has a zero object.
      Then it holds for any two objects~$X, Y \in \Ob(\Acat)$ that~$0_{X,Y} = 0_{\Acat(X,Y)}$.
  \end{enumerate}
\end{lemma}


\begin{proof}
  This is Exercise~3 of the fifth exercise sheet.
\end{proof}





\lecturend{9}


\begin{definition}
  Let~$\Acat$ be a preadditive category and let~$X_1, \dotsc, X_n \in \Ob(\Acat)$ be objects, where~$n \in \Integer_{\geq 0}$.
  A \emph{biproduct}\index{biproduct} of~$X_1, \dotsc, X_n$ is a triple~$(X, (p_1, \dotsc, p_n), (c_1, \dotsc, c_n))$ consisting of an object~$X \in \Ob(\Acat)$ together with morphisms~$p_i \colon X \to X_i$ and morphisms~$c_i \colon X_i \to X$ in~$\Acat$,
  \begin{equation}
    \label{no abuse of notation}
      p_j c_i
    = \begin{cases}
        \id_{X_i}     & \text{if~$i = j$}     \,, \\
        0_{X_i, X_j}  & \text{if~$i \neq j$}  \,,
      \end{cases}
  \end{equation}
  for all~$i,j = 1, \dots, n$, and
  \[
      \sum_{i=1}^n c_i p_i
    = {\id_X} \,.
  \]
% Add notation.
\end{definition}


\begin{remark*}
  In the lecture, the formula~\ref{no abuse of notation} was instead written as
  \[
      p_j c_i
    = \delta_{ij} \id_{X_i} \,.
  \]
  This is an abuse of notation:
  For~$j \neq i$ the composition~$p_j c_i$ is a morphism~$X_i \to X_j$, whereas~$\delta_{ij} \id_{X_i} = 0 \cdot \id_{X_i} = 0_{X_i, X_i}$ is the zero morphism~$X_i \to X_i$.
  The author tries to avoid this abuse of notation, but will still sometimes write~$p_j c_i = \delta_{ij}$ as an abbreviation for~\eqref{no abuse of notation}.
\end{remark*}


\begin{remarknonum}
  For a preadditive category~$\Acat$, a biproduct of an empty family of objects in~$\Acat$ is the same as a zero object of~$\Acat$.
\end{remarknonum}


\begin{lemma}
  \label{product coproduct biproduct}
  Let~$\Acat$ be a preadditive category, let~$X_1, \dotsc, X_n \in \Acat$ where~$n \in \Integer_{\geq 0}$.
  \begin{enumerate}
    \item
      If~$(X, (p_1, \dotsc, p_n), (c_1, \dotsc, c_n))$ is a biproduct of~$X_1, \dotsc, X_n$ then~$(X, (p_1, \dotsc, p_n))$ is a product of~$X_1, \dotsc, X_n$ and~$(X, (c_1, \dotsc, c_n))$ is a coproduct of~$X_1, \dotsc, X_n$.
    \item
      \label{products into biproducts}
      Suppose that~$(X, (p_1, \dotsc, p_n))$ is a product of~$X_1, \dotsc, X_n$.
      Then there exist for every~$i = 1, \dotsc, n$ a unique morphism~$c_i \colon X_i \to X$ with~$p_j c_i = \delta_{ij} \id_{X_i}$ for every~$j = 1, \dotsc, n$.
      The triple~$(X, (p_1, \dotsc, p_n), (c_1, \dotsc, c_n))$ is then a biproduct of~$X_1, \dotsc, X_n$.
    \item
      Dually, suppose that~$(X, (c_1, \dotsc, c_n))$ is a coproduct of~$X_1, \dotsc, X_n$.
      Then there exist for every~$i = 1, \dotsc, n$ a unique morphism~$p_i \colon X \to X_i$ with~$p_i c_j = \delta_{ij} \id_{X_i}$ for every~$j = 1, \dotsc, n$.
      The triple~$(X, (p_1, \dotsc, p_n), (c_1, \dotsc, c_n))$ is then a biproduct of~$X_1, \dotsc, X_n$.
  \end{enumerate}
\end{lemma}


\begin{proof}
  For security reasons we consider the case~$n = 0$ separately:
  The product over the empty family is a final object of~$\Acat$, the coproduct over the empty family is an initial object of~$\Acat$, and the biproduct over the empty family is a zero object of~$\Acat$.
  The statements therefore follow for~$n = 0$ from \cref{inital terminal zero}.
  In the following we consider the case~$n \geq 1$.
  \begin{enumerate}
    \item
      It sufficies by duality to show that~$(X, (c_i)_i)$ is a coproduct for~$X_1, \dotsc, X_n$.
      Let~$(D, (d_i)_i)$ be another pair consisting of an object~$D \in \Ob(\Acat)$ and morphisms~$d_i \colon X_i \to D$.
      We need to show that there exists a unique morphism~$\mu \colon X \to D$ which makes the triangle
      \[
        \begin{tikzcd}[sep = large]
            X_i
            \arrow{r}[above]{c_i}
            \arrow{dr}[below left]{d_i}
          & X
            \arrow[dashed]{d}[right]{\mu}
          \\
            {}
          & D
        \end{tikzcd}
      \]
      commute for every~$i = 1, \dotsc, n$. 
      If such a morphism~$\mu$ exists then
      \[
          \mu
        = \mu \id_X
        = \mu \sum_{i=1}^n c_i p_i
        = \sum_{i=1}^n \mu c_i p_i
        = \sum_{i=1}^n d_i p_i \,,
      \]
      which shows that~$\mu$ is unique.
      If we define on the other hand~$\mu \defined \sum_{i=1}^n d_i p_i$ then
      \[
          \mu c_i
        = \sum_{j=1}^n d_j \underbrace{p_j c_i}_{= \delta_{ij}}
        = d_i \,,
      \]
      which shows the existence of~$\mu$.
    \item
      By the universal property of the product there exists for every~$i = 1, \dotsc, n$ a unique morphism~$c_i \colon X_i \to X$ which makes for all~$j \neq i$ the triangles
      \[
        \begin{tikzcd}[sep = large]
            X_i
            \arrow{dr}[above right]{\id_{X_i}}
            \arrow[dashed]{d}[left]{c_i}
          & {}
          \\
            X
            \arrow{r}[below]{p_i}
          & X_i
        \end{tikzcd}
        \qquad\text{and}\qquad
        \begin{tikzcd}[sep = large]
            X_i
            \arrow{dr}[above right]{0}
            \arrow[dashed]{d}[left]{c_i}
          & {}
          \\
            X
            \arrow{r}[below]{p_j}
          & X_j
        \end{tikzcd}
      \]
      commute.
      This means that~$p_j c_i = \delta_{ij}$ for all~$i, j = 1, \dotsc, n$.
      
      We now show that~$\sum_{i=1}^n c_i p_i = \id_X$.
      Indeed, we find for every~$j = 1, \dotsc, n$ that
      \[
          p_j \circ \sum_{i=1}^n c_i p_i
        = \sum_{i=1}^n \underbrace{ p_j c_i }_{= \delta_{ij}} p_i
        = p_j \,.
      \]
      That shows that for every~$j = 1, \dotsc, n$ the triangle
      \[
        \begin{tikzcd}
            X
            \arrow{dr}[below left]{p_j}
            \arrow[dashed]{rr}[above]{\sum_{i=1}^n c_i p_i}
          & {}
          & X
            \arrow{dl}[below right]{p_j}
          \\
            {}
          & X_j
          &
        \end{tikzcd}
      \]
      commutes.
      But it follows from the uniqueness of products up to unique isomorphism that there exist a \emph{unique} morphism~$X \to X$ which makes this triangle commute.
      The identity~$\id_X \colon X \to X$ also makes the above triangle commute, and so it follows that~$\sum_{i=1}^n c_i p_i = \id_X$.
    \item
      This can be shown dually to part~\ref*{products into biproducts}.
    \qedhere
  \end{enumerate}
\end{proof}


\begin{remark}
  It follows from \cref{product coproduct biproduct} that for a preadditive category~$\Acat$ the following are equivalent:
  \begin{enumerate}
    \item
      $\Acat$ has finite products.
    \item
      $\Acat$ has finite coproducts.
    \item
      $\Acat$ has finite biproducts.
  \end{enumerate}
\end{remark}


\begin{definition}
  A preadditve (or {\preklin}) category~$\Acat$ is \emph{additive}\index{additive!category}\index{category!additive} (resp.~{\klin}\index{k-linear@$\kf$-linear!category}\index{category!k-linear@$\kf$-linear}) if it has finite biproducts (and thus equivalently finite products, and equivalently finite coproducts).
\end{definition}


\begin{remarknonum}
  Additive (and~{\klin}) categories have zero objects, as these are the biproducts of empty family of objects.
\end{remarknonum}


\begin{remark*}
  A category~$\Acat$ is additive (resp.~{\klin}) if and only if its dual category~$\Acat^\op$ is additive (resp.\ {\klin}).
\end{remark*}


\begin{remark*}
  In a preadditive catgory~$\Acat$ one can express morphisms between biproducts as matrices:
  Let~$X_1, \dotsc, X_n$ and~$Y_1, \dotsc, Y_m$ be two families of objects in~$\Acat$ whose biproducts~$X_1 \oplus \dotsb \oplus X_n$ and~$Y_1 \oplus \dotsb \oplus Y_m$ exist, and denote the associated morphisms by
  \begin{align*}
    c_i \colon X_i \to X_1 \oplus \dotsb \oplus X_n
    \quad&\text{and}\quad
    p_i \colon X_1 \oplus \dotsb \oplus X_n \to X_i \,,
  \shortintertext{and}
    d_i \colon Y_i \to Y_1 \oplus \dotsb \oplus Y_m
    \quad&\text{and}\quad
    q_i \colon Y_1 \oplus \dotsb \oplus Y_m \to Y_i \,.
  \end{align*}
  
  Suppose first that we are given a morphism
  \[
            f
    \colon  X_1 \oplus \dotsb \oplus X_n
    \to     Y_1 \oplus \dotsb \oplus Y_m
  \]
  in~$\Acat$.
  It then follows from the calculation
  \begin{align*}
        f
    &=  \id_{Y_1 \oplus \dotsb \oplus Y_m} \circ f \circ \id_{X_1 \oplus \dotsb \oplus X_n} \\
    &=  \left( \sum_{i=1}^n d_i q_i \right) \circ f \circ \left( \sum_{j=1}^m c_j p_j \right)
     =  \sum_{i=1}^n \sum_{j=1}^m d_i (q_i \circ f \circ c_j) p_j \,.
  \end{align*}
  that the morphism~$f$ is unique determined by the compositions~$q_i \circ f \circ c_j$.
  We will refer to the composition
  \[
    [f]_{ij} \defined q_i \circ f \circ c_j
  \]
  as the~\dash{$(i,j)$}{th} component of~$f$.
  The above calculation shows that the morphism~$f$ can be retrieved from its components via the formula
  \[
    f = \sum_{i=1}^n \sum_{j=1}^m d_i [f]_{ij} p_j \,.
  \]
  To better visualize the relation between~$f$ and its components, we may arrange these components in the form of an~\dash{$(m \times n)$}{matrix}
  \[
    \begin{bmatrix}
      f_{11}  & \cdots  & f_{1n}  \\
      \vdots  & \ddots  & \vdots  \\
      f_{m1}  & \cdots  & f_{mn}
    \end{bmatrix} \,.
  \]
  We will refer to this matrix as~$[f]$.
  (Note that~$[f]_{ij}$ is hence the~\dash{$(i,j)$}{th} entry of the matrix~$[f]$.)
  
  Suppose on the other hand that we are given a collection of morphisms~$g_{ij} \colon X_j \to Y_i$ where~$i = 1, \dotsc, m$ and~$j = 1, \dotsc, n$.
  We can then define a morphism~$g \colon X \to Y$ via
  \[
              g
    \defined  \sum_{i=1}^m \sum_{j=1}^n d_i g_{ij} p_j \,.
  \]
  The components~$[g]_{ij}$ of the morphism~$g$ are then for all~$i = 1, \dotsc, m$ and~$j = 1, \dotsc, n$ given by
  \begin{align*}
        [g]_{ij}
     =  q_i \circ g \circ c_j
    &=  q_i \circ \left( \sum_{i'=1}^m \sum_{j'=1}^n d_{i'} g_{i'j'} p_{j'} \right) \circ c_j \\
    &=  \sum_{i=1}^m \sum_{j=1}^n
        \underbrace{q_i d_{i'}}_{= \delta_{i,i'}} g_{i'j'} \underbrace{p_{j'} c_j}_{= \delta_{j',j}}
     =  g_{ij} \,.
  \end{align*}
  The components~$[g]_{ij}$ of~$g$ are hence the morphisms~$g_{ij}$ that we started with.
  
  This shows overall that we have constructed a bijection
  \begin{align*}
      \Acat(X_1 \oplus \dotsb \oplus X_n, Y_1 \oplus \dotsb \oplus Y_m)
    &\longleftrightarrow
      \left\{
        \begin{bsmallmatrix}
          g_{11}  & \cdots  & g_{1n}  \\
          \vdots  & \ddots  & \vdots  \\
          g_{m1}  & \cdots  & g_{mn}
        \end{bsmallmatrix}
      \suchthat*
        g_{ij} \in \Acat(X_j, Y_i)
      \right\}  \,,
    \\
      f
    &\longmapsto
      [f] \,,
    \\
      \sum_{i=1}^m \sum_{j=1}^n q_i g_{ij} c_j
    =
      g
    &\longmapsfrom
      \begin{bsmallmatrix}
          g_{11}  & \cdots  & g_{1n}  \\
          \vdots  & \ddots  & \vdots  \\
          g_{1n}  & \cdots  & g_{mn}
        \end{bsmallmatrix}  \,.
  \end{align*}
  This way of representing morphisms between biproducts as matrices is compatible with both sums and composition of morphisms, and if~$\Acat$ is~{\preklin} then also with scalar multiplication of morphisms.
  \begin{itemize}
    \item
      Let~$f_1, f_2 \colon X_1 \oplus \dotsb \oplus X_n \to Y_1 \oplus \dotsb \oplus Y_m$ be two parallel morphisms in~$\Acat$.
      Then the morphism~$f_1 + f_2$ has for all~$i = 1, \dotsc, m$ and~$j = 1, \dotsc, n$ the components
      \[
          [f_1 + f_2]_{ij}
        = q_i \circ (f_1 + f_2) \circ c_j
        = q_i \circ f_1 \circ c_j + q_i \circ f_2 \circ c_j
        = [f_1]_{ij} + [f_2]_{ij} \,.
      \]
      This shows that indeed
      \[
          [f_1 + f_2]
        = [f_1] + [f_2] \,.
      \]
    \item
      Let~$Z_1, \dotsc, Z_l$ be objects in~$\Acat$ whose biproduct~$Z_1 \oplus \dotsb \oplus Z_l$ exists in~$\Acat$, and let
      \[
        e_i \colon Z_i \to Z_1 \oplus \dotsb \oplus Z_l
      \quad\text{and}\quad
        r_i \colon Z_1 \oplus \dotsb \oplus Z_l \to Z_i \,,
      \]
      be the associated morphisms.
      It then holds for any two composable morphisms
      \[
          X_1 \oplus \dotsb \oplus X_n
        \xlongto{f}
          Y_1 \oplus \dotsb \oplus Y_m
        \xlongto{g}
          Z_1 \oplus \dotsb \oplus Z_l
      \]
      in~$\Acat$ that
      \[
          [g \circ f]
        = [g] \cdot [f]
      \]
      where the product on the right hand side is taken in the naive way.
      Indeed the composition~$g \circ f$ has for all~$i = 1, \dotsc, l$ and~$k = 1, \dotsc, n$ the components
      \begin{align*}
            [g \circ f]_{ik}
        &=  r_i \circ (g \circ f) \circ c_k
         =  r_i \circ g \circ \id_Y \circ f \circ c_k \\
        &=  r_i \circ g \circ \left( \sum_{j=1}^m d_j q_j \right) \circ f \circ c_k \\
        &=  \sum_{j=1}^m (r_i \circ g \circ d_j) \circ (q_j \circ f \circ c_k)
         =  \sum_{j=1}^m [g]_{ij} [f]_{jk} \,.
      \end{align*}
      The resulting term~$\sum_{j=1}^m [g]_{ij} [f]_{jk}$ is precisely the~\dash{$(i,k)$}{th} entry of the matrix product~$[g] \cdot [f]$.
    \item
      If~$\Acat$ also~{\preklin} then let~$f \colon X_1 \oplus \dotsb \oplus X_n \to Y_1 \oplus \dotsb \oplus Y_m$ be a morphism in~$\Acat$ and let~$\lambda \in \kf$ be a scalar.
      Then the morphism~$\lambda f$ has for all~$i = 1, \dotsc, m$ and~$j = 1, \dotsc, n$ the components
      \[
          [\lambda f]_{ij}
        = q_i \circ (\lambda f) \circ c_j
        = \lambda (q_i \circ f \circ c_j)
        = \lambda [f]_{ij} \,.
      \]
      This shows that indeed
      \[
          [\lambda f]
        = \lambda [f] \,.
      \]
  \end{itemize}
  
  In the following we will notationally often not distinguish between the morphism~$f \colon X_1 \oplus \dotsb \oplus X_n \to Y_1 \oplus \dotsb \oplus Y_m$ and its matrix representation.
  So instead of
  \[
      [f]
    = \begin{bmatrix}
        f_{11}  & \cdots  & f_{1n}  \\
        \vdots  & \ddots  & \vdots  \\
        f_{m1}  & \cdots  & f_{mn}
      \end{bmatrix}
  \]
  (where~$f_{ij} = [f]_{ij}$ is the~\dash{$(i,j)$}{th} component of~$f$) we will just write
  \[
      f
    = \begin{bmatrix}
        f_{11}  & \cdots  & f_{1n}  \\
        \vdots  & \ddots  & \vdots  \\
        f_{m1}  & \cdots  & f_{mn}
      \end{bmatrix} \,.
  \]
  If one of the morphisms~$f_{ij}$ is the identity~$\id_Z$ of some object~$Z$ (which is then necessarily given by~$Z = X_j = Y_i$) then we will often just write the corresponding matrix entry as~$1$ instead of~$\id_Z$.
  
  We finish this remark by pointing out that the morphisms~$c_i \colon X_i \to X_1 \oplus \dotsb \oplus X_n$ and~$p_i \colon X_1 \oplus \dotsb \oplus X_n \to X_i$ are by these notional conventions given by the matrices
  \[
      c_i
    = \begin{bsmallmatrix}
        {} \\0 \\ \vdots \\ 0 \\ 1 \\ 0 \\ \vdots \\ 0 \\ {}
      \end{bsmallmatrix}
    \qquad\text{and}\qquad
      p_i
    = \begin{bsmallmatrix}
        0 & \cdots & 0 & 1 & 0 & \cdots & 0
      \end{bsmallmatrix} \,.
  \]
  
  (This whole remark was not present in the lecture.
  The idea of representing a morphism between biproducts as a matrix was explained only for the special case~$n = m = 2$, since is used in the upcoming \cref{sum via category structure}.
  But the author found this ad\nobreakdash-hoc explanation a bit insufficient, and thus decided to add a more detailed explanation for these notes.)
\end{remark*}


\begin{remark}
  \label{sum via category structure}
  Let$~\Acat$ be an additive category.
  For any objects~$X \in \Ob(\Acat)$ we can define the \emph{diagonal \textup(morphism\textup)}\index{diagonal morphism}
  \[
            \diag_X
    \colon  X
    \to     X \oplus X
  \]
  by using the universal property of the product for~$X \oplus X$, as the unique morphism~$X \to X \oplus X$ which make the diagram
  \[
    \begin{tikzcd}[sep = large]
        {}
      & X
        \arrow[dashed]{d}[right]{\diag_X}
        \arrow{dl}[above left]{\id_X}
        \arrow{dr}[above right]{\id_X}
      & {}
      \\
        X
      & X \oplus X
        \arrow{l}[below]{p_1}
        \arrow{r}[below]{p_2}
      & X
    \end{tikzcd}
  \]
  commute.
  This means that
  \[
    p_1 \circ \diag_X = \id_X
    \quad\text{and}\quad
    p_2 \circ \diag_X = \id_X \,,
  \]
  so the morphism~$\diag_X$ can be written is matrix form as
  \[
      \diag_X
    = \begin{bmatrix}
        1 \\ 1
      \end{bmatrix} \,.
  \]
  We can dually define the \emph{codiagonal \textup(morphism\textup)}\index{codiagonal morphism}
  \[
            \codiag_X
    \colon  X \oplus X
    \to     X
  \]
  by using the universal property of the coproduct for~$X \oplus X$, as the unique morphism~$X \oplus X \to X$ which makes the diagram
  \[
    \begin{tikzcd}[sep = large]
        X
        \arrow{dr}[below left]{\id_X}
      & X \oplus X
        \arrow[dashed]{d}[right]{\codiag_X}
        \arrow{l}[above]{c_1}
        \arrow{r}[above]{c_2}
      & X
        \arrow{dl}[below right]{\id_X}
      \\
        {}
      & X
      & {}
    \end{tikzcd}
  \]
  commute.
  This means that
  \[
    \codiag_X \circ c_1 = \id_X
    \quad\text{and}\quad
    \codiag_X \circ c_2 = \id_X \,,
  \]
  so the morphism~$\codiag_X$ can be written in matrix form as
  \[
      \codiag_X
    = \begin{bmatrix}
        1 & 1
      \end{bmatrix} \,.
  \]
  
  (Note that for~$\Acat = \Modl{A}$, where~$A$ is a~{\kalg}, the diagonal~$\diag_X$ is the usual diagonal map~$\diag_X(x) = (x,x)$, and the codiagonal~$\codiag_X$ is the addition~$\codiag_X(x_1, x_2) = x_1 + x_2$.)
  
  We can now describe the sum~$f + g$ of two parallel morphisms~$f, g \colon X \to Y$ in~$\Acat$ as the compositions
  \begin{equation}
    \label{composition for sum}
      X
    \xlongto{\diag_X}
      X \oplus X
    \xlongto{\begin{bsmallmatrix} f & 0 \\ 0 & g \end{bsmallmatrix}}
      Y \oplus Y
    \xlongto{\codiag_Y}
      Y \,.
  \end{equation}
  Indeed, we find by matrix multiplication that
  \[
      \codiag_Y
      \circ
      \begin{bmatrix}
        f & 0 \\
        0 & g
      \end{bmatrix}
      \circ
      \diag_X
    = \begin{bmatrix}
        1 & 1
      \end{bmatrix}
      \begin{bmatrix}
        f & 0 \\
        0 & g
      \end{bmatrix}
      \begin{bmatrix}
        1 \\
        1
      \end{bmatrix}
    = f + g \,.
  \]
  By using that
  \[
    \begin{bmatrix}
      1 & 1
    \end{bmatrix}
    \begin{bmatrix}
      f & 0 \\
      0 & g
    \end{bmatrix}
    =
    \begin{bmatrix}
      f & g
    \end{bmatrix}
    \quad\text{and}\quad
    \begin{bmatrix}
      f & 0 \\
      0 & g
    \end{bmatrix}
    \begin{bmatrix}
      1 \\
      1
    \end{bmatrix}
    =
    \begin{bmatrix}
      f \\
      g
    \end{bmatrix}
  \]
  we can also rewrite the composition~\eqref{composition for sum} as
  \[
      X
    \xlongto{\diag_X}
      X \oplus X
    \xlongto{\begin{bsmallmatrix} f & g \end{bsmallmatrix}}
      Y
    \qquad\text{or}\qquad
      X
    \xlongto{\begin{bsmallmatrix} f \\ g \end{bsmallmatrix}}
      Y \oplus Y
    \xlongto{\codiag_Y}
      Y \,.
  \]
  
  This shows that the addition of~$\Acat$ can be retrieved from the categorical structure of~$\Acat$.
  It follows that an arbitrary category~$\Acat$ can be made into an additive category in at most one way.
  We can therefore regard \enquote{being additive} as a property of a category.
\end{remark}


\begin{definition}
  Let~$F \colon \Ccat \to \Dcat$ be a functor between arbitrary categories~$\Ccat$ and~$\Dcat$.
  \begin{enumerate}
    \item
      The functor~$F$ \emph{respects \textup(finite\textup) products}\index{functor!respects!products} if it holds for every (finite) family~$(X_i)_{i \in I}$ of objects~$X_i \in \Ob(\Ccat)$ and every product~$(P, (p_i)_{i \in I})$ of this family that the pair~$(F(P), (F(p_i))_{i \in I})$ is a product of the family~$(F(X_i))_{i \in I}$.
    \item
      The functor~$F$ \emph{respects \textup(finite\textup) coproducts}\index{functor!respects!coproducts} if it holds for every (finite) family~$(X_i)_{i \in I}$ of objects~$X_i \in \Ob(\Ccat)$ and every coproduct~$(C, (c_i)_{i \in I})$ of this family that the pair~$(F(C), (F(c_i))_{i \in I})$ is a coproduct of the family~$(F(X_i))_{i \in I}$.
  \end{enumerate}
  Suppose now that~$F \colon \Acat \to \Bcat$ is a functor between preadditive categories~$\Acat$ and~$\Bcat$.
  \begin{enumerate}[resume]
    \item
      The functor~$F$ \emph{respects biproducts}\index{functor!respects!biproducts} if it holds for all objects~$X_1, \dotsc, X_n \in \Ob(\Acat)$ (where~$n \geq 0$) and every biproduct~$(X, (p_1, \dotsc, p_n), (c_1, \dotsc, c_n))$ of these objects that the triple~$(F(X), (F(p_1), \dotsc, F(p_n)), (F(c_1), \dotsc, F(c_n)))$ is a biproduct of the objects~$F(X_1), \dotsc, F(X_n)$.
  \end{enumerate}
\end{definition}


\begin{theorem}
  Let~$F \colon \Acat \to \Bcat$ be a functor between preadditive categories~$\Acat$ and~$\Bcat$.
  \begin{enumerate}
    \item
      \label{additive preserves biproducts}
      If the functor~$F$ is additive then it respects biproducts (and hence also finite products and finite coproducts).
    \item
      If the categories~$\Acat$ and~$\Bcat$ are already additive, then the following conditions on~$F$ are equivalent:
      \begin{enumerate}
        \item
          \label{is additive}
          $F$ is additive.
        \item
          \label{respects biproducts}
          $F$ respects biproducts.
        \item
          \label{respects finite products}
          $F$ respects finite products.
        \item
          \label{respects finite coproducts}
          $F$ respects finite coproducts.
      \end{enumerate}
  \end{enumerate}
\end{theorem}


\begin{proof}
  \leavevmode
  \begin{enumerate}
    \item
      Let~$n \geq 0$, let~$X_1, \dotsc, X_n \in \Ob(\Acat)$ and let~$(X, (p_i)_i, (c_i)_i)$ be a biproduct of~$X_1, \dotsc, X_n$.
      
      Once again we consider the case~$n = 0$ separately:
      The biproduct~$X$ is then a zero object of~$\Acat$.
      It follows that $\id_X = 0_{X,X}$ by \cref{inital terminal zero}, and hence
      \[
          \id_{F(X)}
        = F( \id_X )
        = F( 0_{X,X} )
        = 0_{F(X), F(X)} \,,
      \]
      by the additivity of~$F$.
      This shows that~$F(X)$ is a zero object for~$\Bcat$.
      
      Let now~$n \geq 1$.
      We then calculate that
      \begin{gather*}
          F(p_j)  F(c_i)
        = F(p_j c_i)
        = F
          \left(
              \begin{cases}
                \id_{X_i}     & \text{if~$i = j$} \\
                0_{X_i, X_j}  & \text{if~$i \neq j$}
              \end{cases}
          \right)
        = \begin{cases}
            \id_{F(X_i)}        & \text{if~$i = j$}     \,, \\
            0_{F(X_i), F(X_j)}  & \text{if~$i \neq j$}  \,,
          \end{cases}
      \intertext{and}
          \sum_{i=1}^n F(c_i) F(p_i)
        = \sum_{i=1}^n F(c_i p_i)
        = F\left( \sum_{i=1}^n c_i p_i \right)
        = F( \id_X )
        = \id_{F(X)} \,.
      \end{gather*}
    \item
      \begin{description}
        \item[\ref*{is additive}~$\implies$~\ref*{respects biproducts}]
          This has been shown in part~\ref*{additive preserves biproducts}.
          
        \item[\ref*{respects biproducts}~$\implies$~\ref*{respects finite products}]
          Let~$(X, (p_i)_i)$ be a product of the objects~$X_1, \dotsc, X_n$.
          It follows from \cref{product coproduct biproduct} unique morphism~$c_i \colon X_i \to X$ such that the triple~$(X, (p_i)_i, (c_i)_i)$ is a biproduct for~$X_1, \dotsc, X_n$.
          It follows that the triple~$(F(X), (F(p_i))_i, (F(c_i))_i)$ is a biproduct for the objects~$F(X_1), \dotsc, F(X_n)$ because the functor~$F$ preserves biproducts.
          This entails that the tuple~$(F(X), (F(p_i))_i)$ is a product of these objects.
        
        \item[\ref*{respects finite products}~$\implies$~\ref*{respects biproducts}]
          It follows from~$F$ respecting products that~$F$ respects terminal objects, because a terminal object is the same as an empty product.
          Hence~$F(0) = 0$.
          It follows that~$F(0_{X,Y}) = 0_{F(X), F(Y)}$ for any two objects~$X, Y \in \Ob(\Acat)$.
          Indeed, by applying the fuctor~$F$ to the commutative triangle
          \[
            \begin{tikzcd}[column sep = small]
                X
                \arrow{rr}[above]{0_{X,Y}}
                \arrow{dr}
              & {}
              & Y
              \\
                {}
              & 0
                \arrow{ur}
              & {}
            \end{tikzcd}
          \]
          we get the following commutative triangle:
          \[
            \begin{tikzcd}[column sep = small]
                F(X)
                \arrow{rr}[above]{F(0_{X,Y})}
                \arrow{dr}
              & {}
              & F(Y)
              \\
                {}
              & 0
                \arrow{ur}
              & {}
            \end{tikzcd}
          \]
          The commutativity of this triangle shows that the morphism~$F(0_{X,Y})$ factors through the zero object~$0$, which is precisly what is means for~$F(0_{X,Y})$ to be the zero morphism.
          
          Let now~$(X, (p_i)_i, (c_i)_i)$ be a biproduct of some objects~$X_1, \dotsc, X_n \in \Ob(\Acat)$.
          Then~$(X, (p_i)_i)$ is a product of~$X_1, \dotsc, X_n$, and it follows that~$(F(X), (F(p_i))_i)$ is a product of~$F(X_1), \dotsc, F(X_n)$, because~$F$ respects finite products.
          We find for the morphisms~$F(c_i) \colon F(X_i) \to F(X)$ that
          \[
              F(p_i) \circ F(c_i)
            = F(p_i \circ c_i)
            = F( \id_{X_i} )
            = \id_{F(X_i)}
          \]
          and we also find for~$j \neq i$ that
          \[
              F(p_j) \circ F(c_i)
            = F(p_j \circ c_i)
            = F(0_{X_i, X_j})
            = 0_{F(X_i), F(X_j)} \,.
          \]
          It follows from \cref{product coproduct biproduct} that the triple~$(F(X), (F(p_i))_i, (F(c_i))_i)$ is a biproduct of~$F(X_1), \dotsc, F(X_n)$.
          
        \item[\ref*{respects biproducts}~$\iff$~\ref*{respects finite coproducts}]
          This can be shown dually to the equivalence of~\ref*{respects biproducts} and~\ref*{respects finite products}.
        
        \item[\ref*{respects biproducts}~$\implies$~\ref*{is additive}]
          We find as in the implication \ref*{respects finite products}~$\implies$~\ref*{respects biproducts} that~$F(0) = 0$ and that consequently~$F(0_{X,Y}) =  0_{F(X), F(Y)}$ for all~$X, Y \in \Ob(\Acat)$.
          It also follows from the already proven implications~\ref*{respects biproducts}~$\implies$~\ref*{respects finite products} and \ref*{respects biproducts}~$\implies$~\ref*{respects finite coproducts} that~$F$ respects products and coproducts.
          We hence find that
          \[
              F(\diag_X)
            = \diag_{F(X)}
            \quad\text{and}\quad
              F(\codiag_Y)
            = \codiag_{F(Y)}
          \]
          for all~$X, Y \in \Ob(\Acat)$.
          
          Let~$f, g \colon X \to Y$ be two parallel morphisms in~$\Acat$.
          We can then describe their sum~$f+g$ as the composition
          \begin{equation}
            \label{sum as composition}
              f+g
            \colon
              X
            \xlongto{\diag_X}
              X \oplus X
            \xlongto{\begin{bsmallmatrix} f & 0 \\ 0 & g \end{bsmallmatrix}}
              Y \oplus Y
            \xlongto{\codiag_Y}
              Y \,.
          \end{equation}
          It follows from~$F$ preserving coproducts and products, as well as identities and zero morphisms, that
          \[
            F
            \left(
              \begin{bmatrix}
                f & 0 \\
                0 & g
              \end{bmatrix}
            \right)
            =
            \begin{bmatrix}
              F(f)  & 0     \\
              0     & F(g)
            \end{bmatrix} \,.
          \]
          By using that $F(\diag_X) = \diag_{F(X)}$ and~$F(\codiag_Y) = \codiag_{F(Y)}$ we altogether find that applying the functor~$F$ to the compositon~$\eqref{sum as composition}$ exhibits the morphism~$F(f+g)$ as the composition
          \[
              F(f+g)
            \colon
              F(X)
            \xlongto{\diag_{F(X)}}
              F(X) \oplus F(X)
            \xlongto{\begin{bsmallmatrix} F(f) & 0 \\ 0 & F(g) \end{bsmallmatrix}}
              F(Y) \oplus F(Y)
            \xlongto{\codiag_{F(Y)}}
              Y \,.
          \]
          This composition is precisely~$F(f) + F(g)$, so~$F(f + g) = F(f) + F(g)$.
        \qedhere
      \end{description}
  \end{enumerate}
\end{proof}





\section{Kernels and Cokernels}


\begin{definition}
  Let~$\Ccat$ be a category that has a zero object, or that is preaddive.
  Let~$f \colon X \to Y$ be a morphism in~$\Ccat$.
  \begin{enumerate}
      \item
        A \emph{kernel}\index{kernel} of~$f$ is a pair~$(K,k)$ consisting of an object~$K \in \Ob(\Ccat)$ and a morphism~$k \colon K \to X$ with~$f \circ k = 0$, such that for every morphism~$\ell \colon L \to X$ in~$\Ccat$ with~$f \circ \ell = 0$ there exists a unique morphism~$\lambda \colon L \to K$ which makes the following diagram commute:
        \[
          \begin{tikzcd}[row sep = small, column sep = large]
              K
              \arrow{dr}[above right]{k}
              \arrow[bend left]{drr}[above right]{0}
            & {}
            & {}
            \\
              {}
            & X
              \arrow{r}[above, near start]{f}
            & Y
            \\
              L
              \arrow{ur}[below right]{\ell}
              \arrow[bend right]{urr}[below right]{0}
              \arrow[dashed]{uu}[left]{\lambda}
            & {}
            & {}
          \end{tikzcd}
        \]
      \item
        A \emph{cokernel}\index{cokernel} of~$f$ is a pair~$(C,c)$ consisting of an object~$C \in \Ob(\Ccat)$ and a morphism~$c \colon Y \to C$ with~$c \circ f = 0$, such that for every morphism~$d \colon Y \to D$ in~$\Ccat$ with~$d \circ f = 0$ there exists a unique morphism~$\mu \colon C \to D$ which makes the following diagram commute:
        \[
          \begin{tikzcd}[row sep = small, column sep = large]
              {}
            & {}
            & C
              \arrow[dashed]{dd}[right]{\mu}
            \\
              X
              \arrow{r}[above]{f}
              \arrow[bend left]{urr}[above left]{0}
              \arrow[bend right]{drr}[below left]{0}
            & Y
              \arrow{ur}[above left]{c}
              \arrow{dr}[below left]{d}
            & {}
            \\
              {}
            & {}
            & D
          \end{tikzcd}
        \]
  \end{enumerate}
\end{definition}


\begin{remark}
  Let~$\Ccat$ be a category that has a zero object, or that is preaddive.
  Let~$f \colon X \to Y$ be a morphism in~$\Ccat$.
  \begin{enumerate}
    \item
      A pair~$(K,k)$ is a kernel of~$f$ in~$\Ccat$ if and only if it is a cokernel of~$f$ in~$\Ccat^\op$.
    \item
      Kernels and cokernels are unique up to unique isomorphism.
    \item
      If every morphism in~$\Ccat$ has a kernel (resp.\ a cokernel) then the category~$\Ccat$ \emph{has kernels}\index{category!has!kernels}\index{has!kernels} (resp.\ \emph{has cokernels}\index{category!has!cokernels}\index{has!cokernels}).
    \item
      The kernel of~$f$ is denoted by~$\ker(f) \to X$, and the cokernel of~$f$ is denoted by~$Y \to \coker(f)$.
  \end{enumerate}
\end{remark}





\lecturend{10}





\begin{notation*}
  \leavevmode
  \begin{enumerate}
    \item
      Let~$\Ccat$ be a category that has a zero object and let~$f \colon X \to Y$ be a morphism in~$\Ccat$.
      Then we write~$\ker(f) = 0$ to mean that the zero morphism~$0 \to X$ is a kernel of~$f$.
      We dually write~$\coker(f) = 0$ to mean that the zero morphism~$Y \to 0$ is a cokernel of~$f$.
    \item
      Let~$\Ccat$ be a category that is preadditive, or that has a zero object, and let~$f \colon X \to Y$ be a morphism in~$\Ccat$.
      If~$g \colon Y \to Z$ is another morphism in~$\Ccat$ then we write that~$\ker(f) = \ker(gf)$ to mean that a morphism~$k \colon K \to X$ in~$\Ccat$ is a kernel of~$f$ if and only if it is a kernel of~$gf$.
      (This entails in particular that a kernel for~$f$ exists if and only if a kernel for~$gf$ exist.)
      
      Dually, if~$h \colon W \to X$ is a morphism in~$\Ccat$, then we write that~$\coker(f) = \coker(fh)$ to mean that a morphism~$c \colon Y \to C$ in~$\Ccat$ is a cokernel of~$f$ if and only if it is a cokernel of~$fh$.
      (This entails in particular that a cokernel for~$f$ exists if and only if a cokernel for~$fh$ exists.)
  \end{enumerate}
\end{notation*}


\begin{remark*}
  Let~$\Ccat$ be a category that has a zero object and let~$f \colon X \to Y$ be a morphism in~$\Ccat$.
  Then~$\ker(f) = 0$ if and only if it follows for every morphism~$u \colon W \to X$ in~$\Ccat$ with~$f \circ u = 0$ that already~$u = 0$.
  
  Indeed, the composition~$0 \to X \xto{f} Y$ is the zero morphism.
  That~$\ker(f) = 0$ therefore means that every morphism~$u \colon W \to X$ with~$f \circ u = 0$ factors uniquely trough the zero morphism~$0 \to X$, i.e.\ that there exists a unique morphism~$W \to 0$ that makes the triangle
  \[
    \begin{tikzcd}[sep = large]
        W
        \arrow[dashed]{d}
        \arrow{dr}[above right]{u}
        \arrow[bend left]{drr}[above right]{0}
      & {}
      & {}
      \\
        0
        \arrow{r}
      & X
        \arrow{r}[above]{f}
      & Y
    \end{tikzcd}
  \]
  commute.
  That~$u$ factors trough the zero morphism~$0 \to X$ is equivalent to~$u = 0$, and this factorization is necessarily unique because there exist only one morphism~$W \to 0$.
  
  It holds dually that~$\coker(f) = 0$ if and only if it follows for every morphism~$v \colon Y \to Z$ in~$\Ccat$ with~$v \circ f = 0$ that already~$v = 0$.
\end{remark*}


\begin{lemma}
  Let~$\Ccat$ be a category that is preadditive, or that has a zero object.
  Let~$f \colon X \to Y$ be a morphism in~$\Ccat$.
  \begin{enumerate}
    \item
      If~$k \colon \ker(f) \to X$ is a kernel of~$f$ then the morphism~$k$ is a monomorphism.
      Dually, if~$c \colon Y \to \coker(f)$ is a cokernel of~$f$ then the morphism~$c$ is an epimorphism.
    \item
      Suppose that~$\Ccat$ is both preadditive and has a zero object (e.g.~$\Ccat$ is additive).
      If~$\ker(f) = 0$ then~$f$ is a monomorphism.
      Dually, if~$\coker(f) = 0$ then~$f$ is an epimorphism.
    \item
      Suppose that~$\Ccat$ has a zero object.
      If~$f$ is a monomorphism then~$\ker(f) = 0$, and if~$f$ is an epimorphism then~$\coker(f) = 0$.
    \item
      If~$u \colon Y \to Z$ is a monomorphism in~$\Ccat$ then~$\ker(f) = \ker(uf)$.
      Dually, if~$p \colon W \to X$ is an epimorphism in~$\Ccat$ then~$\coker(f) = \coker(fp)$.
  \end{enumerate}
\end{lemma}


\begin{proof}
  \leavevmode
  \begin{enumerate}
    \item
      Let~$u, v \colon W \to \ker(f)$ be two parallel morphisms with~$k \circ u = k \circ v$, and denote this composition by~$w \colon W \to X$.
      Then
      \[
          f \circ w
        = f \circ k \circ u
        = 0 \circ u
        = 0 \,.
      \]
      It follows from the universal property of the kernel~$k \colon \ker(f) \to X$ that there exist a unique morphism~$W \to \ker(f)$ which makes the triangle
      \[
        \begin{tikzcd}[sep = large]
            W
            \arrow[dashed]{d}
            \arrow{dr}[above right]{w}
          & {}
          \\
            \ker(f)
            \arrow{r}[above, near start]{k}
          & X
        \end{tikzcd}
      \]
      commute.
      Both~$u$ and~$v$ make this triangle commute, and so it follows that~$u = v$.
      
      That the cokernel~$c \colon Y \to \coker(f)$ is an epimorphism can be shown dually.
    \item
      Let~$u,v \colon W \to X$ be two morphisms with~$f \circ u = f \circ v$.
      Then
      \[
          f \circ (u-v)
        = f \circ u - f \circ v
        = 0 \,,
      \]
      and it follows from the universal property of the kernel~$\ker(f)$ that the difference~$u-v$ factors through~$\ker(f) = 0$, which results in the following commutative triangle:
      \[
        \begin{tikzcd}[sep = large]
            W
            \arrow[dashed]{d}
            \arrow{dr}[above right]{u-v}
          & {}
          \\
            0
            \arrow{r}
          & X
        \end{tikzcd}
      \]
      The morphisms~$W \to 0$ and~$0 \to X$ are necessarily the zero morphisms, so it follows that~$u-v = 0 \circ 0 = 0$, and hence~$u = v$.
      
      That~$f$ is an epimorphism if~$\coker(f) = 0$ can be shown dually.
    \item
      If~$u \colon W \to X$ is any morphism with~$f \circ u = 0$ then
      \[
          f \circ u
        = 0
        = f \circ 0
      \]
      and hence~$u = 0$.
      
      That~$\coker(f) = 0$ if~$f$ is an epimorphism can be shown dually.
    \item
      It holds for every morphism~$v \colon W \to X$ that
      \[
              f \circ v = 0
        \iff  uf \circ v = u \circ 0
        \iff  uf \circ v = 0 \,,
      \]
      where the first equivalence holds because~$u$ is a monomorphism.
      We thus find for every morphism~$k \colon K \to X$ in~$\Ccat$ that
      \begin{align*}
            {}& \text{$k$ is a kernel of~$f$} \\
        \iff{}& \text{every morphism~$v \colon W \to X$ with~$f \circ v = 0$ factors uniquely through~$k$}  \\
        \iff{}& \text{every morphism~$v \colon W \to X$ with~$uf \circ v = 0$ factors uniquely through~$k$} \\
        \iff{}& \text{$k$ is a kernel of~$uf$} \,.
      \end{align*}
      
      That~$\coker(f) = \coker(fp)$ can be shown dually.
    \qedhere
  \end{enumerate}
\end{proof}


\begin{notation*}
  Let~$\Ccat$ be a category that is preadditive, or that has a zero object.
  We say that a morphism~$k \colon K \to X$ in~$\Ccat$ \emph{is a kernel}\index{is a!kernel} if it is a kernel for some morphism~$f \colon X \to Y$.
  Dually, we say that a morphism~$c \colon Y \to C$ in~$\Ccat$ \emph{is a cokernel}\index{is a!cokernel} if it is a cokernel for some morphism~$f \colon X \to Y$.
\end{notation*}


\begin{example}
  \leavevmode
  \begin{enumerate}
    \item
      Let~$A$ be a~{\kalg} and consider the module category~$\Modl{A}$.
      Let~$f \colon M \to N$ be a homomorphism of~{\modules{$A$}}.
      Then the submodule~$\ker(f) = f^{-1}(0)$ of~$M$ together with the inclusion~$k \colon \ker(f) \to M$ is a kernel of~$f$.
      The quotient moduel~$\coker(f) = N/f(M)$ together with the canonical projection~$c \colon N \to \coker(f)$ is a cokernel of~$f$.
      
      Note that in the category~$\Modl{A}$, a morphism~$k$ is a monomorphism if and only if~$k$ is a kernel, and similarly that a morphism~$c$ is an epimorphism if and only if~$c$ is a cokernel.
      (Recall that the monomorphisms in~$\Modl{A}$ are preciely the injective module homomorphisms, and that the epimorphisms in~$\Modl{A}$ are precisely the surjective module homomorphisms.)
    \item
      Consider the category~$\Group$ and let~$f \colon G \to H$ be a group homorphism.
      
      Then the subgroup~$\ker(f) = f^{-1}(1)$ together with the inclusion~$k \colon \ker(f) \to G$ is a kernel of~$f$.
      Note that~$\ker(f)$ is always a normal subgroup of~$G$, while for every subgroup~$K' \subseteq G$ the inclusion~$k' \colon K' \to G$ is a monomorphism.
      So in~$\Group$ not every monomorphism is a kernel.
      
      A cokernel of~$f$ is given by the quotient group~$\coker(f) = H/\closure{f(G)}$ together with the canonical projection~$c \colon H \to \coker(f)$, where
      \[
                  \closure{f(G)}
        \defined  \gen{
                    h f h^{-1}
                  \suchthat
                    h \in H, g \in G
                  }
      \]
      is the normal subgroup of~$H$ generated by~$f(G)$, i.e.\ the normal closure of~$f(G)$ in~$H$.
      If~$p \colon H \to C$ is any epimorphism in~$\Group$, then~$p$ is surjective, and hence
      \[
              C
        \cong H/\ker(p)
        =     \coker(\ker(p) \to H)
      \]
      This shows that in~$\Group$ every epimorphism is a cokernel.
    \item
      Let~$\Set_*$ be the category of pointed sets\index{category!of pointed sets}:
      
      The objects of~$\Set_*$ are pairs~$(X,x_0)$ consisting of a set~$X$ and a base point~$x_0 \in X$.
      A morphism~$f \colon (X,x_0) \to (Y,y_0)$ in~$\Set_*$ is a map~$f \colon X \to Y$ with~$f(x_0) = y_0$.
      (Such maps are also known as \emph{pointed maps}\index{pointed map}.)%
      \footnote{One may think about pointed sets as somewhat similar to~\dash{$\Finite_1$}{vector spaces}.}
      The category~$\Set_*$ has the singleton~$(\{\ast\}, \ast)$ as a zero object.
      The monomorphisms in~$\Set_*$ are precisely the injective pointed maps, and the epimorphisms are the surjective pointed maps.
      
      Let~$f \colon (X,x_0) \to (Y,y_0)$ be a morphism in~$\Set_*$.
      A kernel for~$f$ is given by~$\ker(f) = (f^{-1}(y_0), x_0)$ together with the inclusion~$k \colon \ker(f) \to (X,x_0)$.
      A cokernel for~$f$ is given by~$\coker(f) = (Y/{\sim}, \class{y_0})$ together with the canonical projection~$c \colon (Y, y_0) \to \coker(f)$, where~$\sim$ is the equivalence relation on~$Y$ generated by~$y \sim y'$ for all~$y, y' \in f(X)$.
      More explicitely, we have for any two~$y, y' \in Y$ that
      \[
              y \sim y'
        \iff (\text{$y = y'$ or~$y, y' \in f(X)$}) \,.
      \]
      
      Every monomorphism in~$\Set_*$ is a kernel (namely that of its cokernel).
      But not every epimorphism in~$\Set_*$ is a cokernel, because it holds for every cokernel~$c \colon (Y,y_0) \to (C,c_0)$ that every element~$z \in C$ with~$z \neq c_0$ has precisely one preimage under~$c$.
      (This follows from the above explicit description of the cokernel.)
%     TODO: Prove the above claims.
    \item
      The ring~$A \defined \Integer[t_1, t_2, t_3, \dotsc]$ is not noetherian because the ideal~$I \defined \genideal{t_1, t_2, t_3, \dotsc}$ of~$A$ is not finitely generated.
      Consider the category~$\Modlfg{A}$ of finitely generated~{\modules{$A$}}.
      This category is additive.
      
      Every morphism~$f \colon M \to N$ in~$\Modlfg{A}$ has a cokernel in~$\Modlfg{A}$ because the cokernel of~$f$ in~$\Modl{A}$ is already contained in~$\Modlfg{A}$ (and the zero morphisms in~$\Modlfg{A}$ coincides with the one in~$\Modl{A}$).
      
      Let~$f \colon A \to A/I$ be the canonical projection;
      note that~$A/I$ is finitely generated and hence contained in~$\Modlfg{A}$ (even though the~{\modules{$A$}}~$I$ is not contained in~$\Modl{A}$).
      The morphism~$f$ has no kernel in~$\Modl{A}^\fg$:
      
      Morally speaking, the problem is that the kernel of~$f$ in~$\Modl{A}$, which is~$I$, is not contained in~$\Modlfg{A}$.
      But this is not yet a proper proof because a kernel of~$f$ in~$\Modlfg{A}$ does not necessarily have to also be a kernel of~$f$ in~$\Modl{A}$.
      
      So instead, assume that there exists a kernel~$k \colon K \to A$ of~$f$ in~$\Modlfg{A}$.
      Then~$f \circ k = 0$ and hence~$k(K) \subseteq I$.
      Then for every~$i \in I$ the morphism
      \[
                \ell_i
        \colon  A
        \to     A \,,
        \quad   a
        \mapsto t_i a
      \]
      satisfies~$\ell_i(A) \subseteq I$ and hence~$f \circ \ell_i = 0$.
      It follows that the morphism~$\ell_i$ factors uniquely trough the kernel~$k$, i.e.\ that there exists a unique morphism~$\lambda_i \colon A \to K$ which makes the triangle
      \[
        \begin{tikzcd}[sep = large]
            A
            \arrow[dashed]{d}[left]{\lambda_i}
            \arrow{dr}[above right]{\ell_i}
          & {}
          \\
            K
            \arrow{r}[below]{k}
          & A
        \end{tikzcd}
      \]
      commute.
      It follows for every~$i \in I$ that
      \[
            t_i
        =   \ell_i(1)
        =   (k \circ \lambda_i)(1)
        =   k(\lambda_i(1))
        \in k(K)
      \]
      and hence~$t_i \in k(K)$.
      This shows that also~$I \subseteq k(K)$, and hence~$I = k(K)$.
      But it follows from~$K$ being finitely generated that~$k(K)$ is also finitely generated, which contradicts~$I$ not being finitely generated.
  \end{enumerate}
\end{example}


% TODO: Add example: divisible abelian groups.


\begin{definition}
  Let~$\Ccat$ be a category that is preadditive, or that has a zero object, and which has kernels and cokernels.
  Let~$f \colon X \to Y$ be a morphism in~$\Ccat$.
  \begin{enumerate}
    \item
      An \emph{image}\index{image} of~$f$ is a kernel of a cokernel of~$f$, and is denotedy by~$\im(f) \to Y$.
    \item
      A \emph{coimage}\index{coimage} of~$f$ is a cokernel of a kernel of~$f$, and is denoted by~$X \to \coim(f)$.
  \end{enumerate}
\end{definition}


\begin{example*}
  Let~$A$ be a~{\kalg} and let~$f \colon M \to N$ be a morphism in~$\Modl{A}$ (or~$\Modr{A}$).
  Then an image of~$f$ is given by the submodule~$\im(f) = f(M)$ of~$N$ together with the inclusion~$\im(f) \to N$.
  A coimage of~$f$ is given by the quotient module~$\coim(f) = M/\ker(f)$ together with the canonical projection~$M \to \coim(f)$.
\end{example*}


\begin{remarknonum}
  Images and coimages are unique up to unique isomorphisms.
\end{remarknonum}


\begin{lemma}[Canonical factorization]
  \index{canonical factorization}
  \label{canonical factorization}
  Let~$\Ccat$ be a category that is preadditive, or that has a zero object, and that has kernels and cokernels.
  Let~$f \colon X \to Y$ be a morphism in~$\Ccat$.
  \begin{enumerate}
    \item
      \label{restriction to image}
      There exists a unique morphism~$f' \colon X \to \im(f)$ in~$\Ccat$ which makes the following triangle commute:
      \[
        \begin{tikzcd}
            X
            \arrow{r}[above]{f}
            \arrow[dashed]{dr}[below left]{f'}
          & Y
          \\
            {}
          & \im(f)
            \arrow{u}
        \end{tikzcd}
      \]
    \item
      There exists a unique morphism~$\tilde{f} \colon \coim(f) \to Y$ in~$\Ccat$ which makes the following triangle commute:
      \[
        \begin{tikzcd}
            X
            \arrow{r}[above]{f}
            \arrow{d}
          & Y
          \\
            \coim(f)
            \arrow[dashed]{ur}[below right]{\tilde{f}}
          & {}
        \end{tikzcd}
      \]
    \item
      There exists a unique morphism~$\bar{f} \colon \coim(f) \to \im(f)$ in~$\Ccat$ which makes the following square commute:
      \begin{equation}
        \label{canonical morphism from coim to im}
        \begin{tikzcd}
            X
            \arrow{r}[above]{f}
            \arrow{d}
          & Y
          \\
            \coim(f)
            \arrow[dashed]{r}[above]{\bar{f}}
          & \im(f)
            \arrow{u}
        \end{tikzcd}
      \end{equation}
    \item
      The morphism~$\bar{f}$ is compatible with the morphisms~$f'$ and~$\tilde{f}$ in the sense that the diagrams
      \begin{equation}
        \label{compatibility of induced morphisms}
        \begin{tikzcd}[column sep = large, row sep = huge]
            X
            \arrow{r}[above]{f}
            \arrow{dr}[above right]{f'}
            \arrow{d}
          & Y
          \\
            \coim(f)
            \arrow{r}[above]{\bar{f}}
          & \im(f)
            \arrow{u}
        \end{tikzcd}
        \qquad\text{and}\qquad
        \begin{tikzcd}[column sep = large, row sep = huge]
            X
            \arrow{r}[above]{f}
            \arrow{d}
          & Y
          \\
            \coim(f)
            \arrow{r}[above]{\bar{f}}
            \arrow{ur}[above left]{\tilde{f}}
          & \im(f)
            \arrow{u}
        \end{tikzcd}
      \end{equation}
      commute.
  \end{enumerate}
\end{lemma}


\begin{proof}
  We denote the various kernels and cokernels by
  \[
    k \colon \ker(f) \to X  \,,
    \quad
    c \colon Y \to \coker(f)  \,,
    \quad
    i \colon \im(f) \to Y \,,
    \quad
    p \colon X \to \coim(f) \,.
  \]
  \begin{enumerate}
    \item
      The morphism~$i \colon \im(f) \to Y$ is a kernel of the cokernel~$c \colon Y \to \coker(f)$, hence it follows from~$c \circ f = 0$ that there exist a unique morphism~$f' \colon X \to \im(f)$ which makes the following diagram commute:
      \[
        \begin{tikzcd}[row sep = large]
            {}
          & \coker(f)
          \\
            X
            \arrow{ur}[above left]{0}
            \arrow{r}[above]{f}
            \arrow[dashed]{dr}[below left]{f'}
          & Y
            \arrow{u}[right]{c}
          \\
            {}
          & \im(f)
            \arrow{u}[right]{i}
        \end{tikzcd}
%         \begin{tikzcd}
%             X 
%             \arrow{r}[above]{f}
%             \arrow[bend left = 40]{rr}[above left]{0}
%             \arrow[dashed]{dr}[below left]{f'}
%           & Y
%             \arrow{r}[above]{c}
%           & \coker(f)
%           \\
%             {}
%           & \im(f)
%             \arrow{u}[right]{i}
%           & {}
%         \end{tikzcd}
      \]
    \item
      This can be shown dually to part~\ref*{restriction to image}.
    \item
      We construct~$\bar{f}$ by using the already constructed morphism~$f' \colon X \to \im(f)$:
      The morphism~$p \colon X \to \coim(f)$ is a cokernel of the the kernel~$k \colon \ker(f) \to X$.
      It holds that
      \[
          i \circ f' \circ  k
        = f \circ k
        = 0
        = i \circ 0 \,,
      \]
      and hence~$f' \circ k = 0$ because~$i$ is a monomorphism.
      It follows from the universal property of the cokernel~$p \colon X \to \coim(f)$ that there exist a unique morphism~$\bar{f} \colon \coim(f) \to \im(f)$ which makes the following diagram commute:
      \[
        \begin{tikzcd}[column sep = 5em, row sep = huge]
            \ker(f)
            \arrow{d}[left]{k}
            \arrow[bend left]{dr}[above right]{0}
          & {}
          \\
            X
            \arrow{r}[above, near start]{f}
            \arrow{dr}[above right]{f'}
            \arrow{d}[left]{p}
          & Y
          \\
            \coim(f)
            \arrow[dashed]{r}[below]{\bar{f}}
          & \im(f)
            \arrow{u}[right]{i}
            \arrow[dashed, from=uul, bend left = 15, crossing over, near start, "0"]
        \end{tikzcd}
      \]
      This shows the existence of~$\bar{f}$.
      Suppose that~$\bar{f}' \colon \coim(f) \to \im(f)$ is another morphism which makes the diagram~\eqref{canonical morphism from coim to im} commute.
      Then
      \[
          i \circ \bar{f}' \circ p
        = f
        = i \circ \bar{f} \circ p \,.
      \]
      It follows from~$i$ being a monomorphism that
      \[
          \bar{f}' \circ p
        = \bar{f} \circ p \,,
      \]
      and it then further follows from~$p$ being a epimorphism that
      \[
        \bar{f}' = \bar{f} \,.
      \]
      This shows the desired uniqueness of the morphism~$\bar{f}$.
    \item
      It follows from the above construction of~$\bar{f}$ that of the two diagrams in~\eqref{compatibility of induced morphisms} the left one commutes.
      We could have dually constructed~$\bar{f}$ by using the morphism~$\tilde{f}$ instead of~$f'$, which would then give us that the right diagram commutes.
      We can alternatively check the commutativity of the right hand diagram by hand:
      It holds that
      \[
          i \circ \bar{f} \circ p
        = f
        = \tilde{f} \circ p
      \]
      and hence~$i \circ \bar{f} = \tilde{f}$ because~$p$ is an epimorphism.
    \qedhere
  \end{enumerate}
\end{proof}


\begin{example*}
  Let~$A$ be a~{\kalg} and consider the category~$\Modl{A}$.
  Let~$f \colon M \to N$ be a homomorphism of left~{\modules{$A$}}.
  Then the morphism~$f' \colon \im(f) \to N$ is the inclusion~$n \mapsto n$, the morphism~$\tilde{f} \colon M \to \coker(f) = M/\ker(f)$ is the canonical projection~$m \mapsto \class{m}$, and the morphism~$\bar{f} \colon M/\ker(f) \to \im(f)$ is the isomorphism~$\class{m} \mapsto f(m)$.
\end{example*}





\section{Abelian Categories}


\begin{definition}
  An \emph{abelian category}\index{abelian category}\index{category!abelian} is an additive category~$\Acat$ that has kernels and cokernels and in which for every morphism~$f \colon X \to Y$ the induced morphism~$\coim(f) \to \im(f)$ from the \hyperref[canonical factorization]{canonical factorization lemma} is an isomorphism.
\end{definition}


\begin{remark*}
  Let~$\Acat$ be a category.
  We have previously seen that \enquote{being additive} is a property of~$\Acat$.
  It follows that \enquote{being abelian} is also a property of~$\Acat$.
\end{remark*}


\begin{remark}
  \leavevmode
  \begin{enumerate}
    \item
      A category~$\Acat$ is abelian if and only if its dual category~$\Acat^\op$ is abelian.
    \item
      If $\Acat$ is an abelian category then we will for~$X, Y \in \Ob(\Acat)$ often write~$\Hom_\Acat(X,Y)$ instead of~$\Acat(X,Y)$.
    \item
      If~$F \colon \Acat \to \Bcat$ is an equivalence of categories and~$\Acat$ is abelian, then~$\Bcat$ is also abelian.%
      \footnote{In the lecture it was instead stated that one can \enquote{use~$F$ to make~$\Bcat$ into an abelian category}.
      It was then sketched how one can use~$F$ to transfer the group structure on the morphism sets of~$\Acat$ to a group structure of the morphism sets of~$\Bcat$;
      one can  then check that~$F$ respects biproducts, kernels and cokernels (this was not proven in the lecture).
      The author finds this presentation somewhat misleading, as \enquote{being abelian} is really a property of a category, and not an additional structure.
      Hence we can’t \enquote{make} a category abelian.}
  \end{enumerate}
\end{remark}


\begin{example}
  \leavevmode
  \begin{enumerate}
    \item
      The category~$\Ab$ of abelian groups is an abelian category (and the term \enquote{abelian category} comes from this example).
    \item
      If more generally~$A$ is a~{\kalg} then the module categories~$\Modl{A}$ and~$\Modr{A}$ are abelian.
    \item
      If~$G$ is a group then the category~$\Rep{k}{G}$ is equivalent to the the module category~$\Modl{\kf[G]}$ and hence abelian.
    \item
      If~$A$ is a left noetherian ring then the additive category~$\Modlfg{A}$ of finitely generated left~{\modules{$A$}} is again abelian:
      
      Let~$f \colon M \to N$ be a homomorphism of~{\modules{$A$}} where both~$M$ and~$N$ are finitely generated.
      Then the submodule~$\ker(f) \subseteq M$ is again finitely generated because~$M$ is noetherian (since finitely generated modules over noetherian rings are noetherian themselves), and the quotient module~$N/f(M)$ is also again finitely generated.
      We can therefore utilize for~$\Modlfg{A}$ the usual kernels and cokernels.
      
      The analogous statement and argumentation also holds for~$\Modrfg{A}$, the category of finitely generated right~{\modules{$A$}}.
    \item
      Let~$A$ be a {\fd}~{\kalg} over a field~$\kf$.
      Then the category~$\Modlfd{A}$ of {\fd} left~{\modules{$A$}} coincides with the category~$\Modlfg{A}$ of finitely generated left~{\modules{$A$}}, and is hence abelian.
      The analogous observation holds for~$\Modrfd{A}$, the category of {\fd} right~{\modules{$A$}}.
    \item
      If~$\Ccat$ is a category and~$\Acat$ is an abelian category then the functor category~$\Fun(\Ccat, \Acat)$ is again abelian.
      (This will be an exercise on one of the upcoming exercise sheets.)
  \end{enumerate}
\end{example}


\begin{remark*}
  If~$\Acat$ is an abelian category and~$f \colon X \to Y$ is a morphism in~$\Acat$ which is both a monomorphism and an epimorphism, then~$f$ is already an isomorphism:
  
  The zero morphism~$0 \to X$ is a kernel of~$f$ because~$f$ is a monomorphism.
  The identity~$\id_X \colon X \to X$ is a cokernel of the zero morphism~$0 \to X$, and hence a coimage of~$f$.
  We find dually that the identity~$\id_Y \colon Y \to Y$ is an image of~$f$ because~$f$ is an epimorphism.
  The induced morphism~$f \colon \coim(f) \to \im(f)$ of the \hyperref[canonical factorization]{canonical factorization lemma} is just~$f$ itself because the square
  \[
    \begin{tikzcd}
        X
        \arrow{r}[above]{f}
        \arrow{d}[left]{\id_X}
      & Y
      \\
        X 
        \arrow{r}[above]{f}
      & Y
        \arrow{u}[right]{\id_Y}
    \end{tikzcd}
  \]
  commutes.
  It hence follows from the definition of an abelian category that~$f$ is an isomorphism.
\end{remark*}


\begin{example}[Non-example]
  Consider the category~$F^* \Ab$ of filtered abelian groups:
  The objects of~$F^* \Ab$ are pairs~$(A,F^*)$ consisting of an abelian group~$A$ and an increasing \dash{$\Integer$}{filtration}~$F^*$ on~$A$, i.e.\ an increasing sequence
  \[
              \dotsb
    \subseteq F^{i-1}(A)
    \subseteq F^i(A)
    \subseteq F^{i+1}(A)
    \subseteq \dotsb
  \]
  of subgroups~$F^i(A) \subseteq A$ with~$i \in \Integer$.
  A morphism~$f \colon (A, F^*) \to (B,G^*)$ in~$F^* \Ab$ is a group homomorphism~$f \colon A \to B$ with~$f(F^i(A)) \subseteq G^i(B)$ for every~$i \in \Integer$.
  The category~$F^* \Ab$ is additive and has kernels and cokernels:
  If~$f \colon (A,F^*) \to (B,G^*)$ is a morphism in~$F^* \Ab$ then a kernel of~$f$ is given by
  \[
      \ker_{F^* \Ab}(f)
    = ( \ker_{\Ab}(f), ( F^i(A) \cap \ker_{\Ab}(f) )_{i \in \Integer} ) \,,
  \]
  and a cokernel of~$f$ is given by
  \[
      \coker_{F^* \Ab}(f)
    = ( \coker_{\Ab}(f), ( (F^i(B) + f(A))/f(A) )_{i \in \Integer} ) \,.
  \]
  
  Let now~$A$ be an abelian group and suppose that~$F^*$ and~$\tilde{F}^*$ are two filtrations on~$A$ such that~$F_i(A) \subseteq \tilde{F}_i(A)$ for all~$i \in \Integer$, but~$F_j(A) \subsetneq \tilde{F}_j(A)$ for some~$j \in \Integer$.
  (One can for example choose any nontrivial abelian group~$A$ and define such filtrations via
  \[
      F_i(A)
    = \begin{cases}
        0 & \text{if~$i < 1$} \,, \\
        A & \text{if~$i \geq 1$}  \,,
      \end{cases}
    \qquad\text{and}\qquad
      \tilde{F}_i(A)
    = \begin{cases}
        0 & \text{if~$i < 0$} \,, \\
        A & \text{if~$i \geq 0$}  \,.
      \end{cases}
  \]
  Then both filtrations agree except for~$F_0(A) = 0 \subsetneq A = F_1(A)$.)
  We can now regard the identity~$\id_A$ as a morphism~$f \colon (A, F^*) \to (A, \tilde{F}^*)$.
  It then follows from~$\ker_{\Ab}(f) = 0$ and~$\coker_{\Ab}(f) = 0$ that~$\ker_{F^* \Ab}(f) = 0$ and~$\coker_{F^* \Ab}(f) = 0$ by the above explicit construction of kernels and cokernels in~$F^* \Ab$.
  Hence~$f$ is both a monomorphism and an epimorphism.
  But~$f$ is not an isomorphism as the filtrations of~$F^*$ and~$\tilde{F}^*$ differ, so~$F^* \Ab$ cannot be abelian.
\end{example}





\lecturend{11}



\begin{example}
  Let~$X$ be a topological space.
  We will sketch how to show that the category of sheaves of abelian groups over~$X$ is abelian.
  \begin{enumerate}
    \item
      We start with the notion of a presheaf on~$X$:
      
      \begin{definitionnonum}
        A \emph{presheaf}~$\sheaf{F}$\index{presheaf} (of abelian groups over~$X$) consists of
        \begin{itemize}
          \item
            an abelian group~$\sheaf{F}(U)$ for every open subset~$U \subseteq X$, and
          \item
            a group homomorphism~$\rho_{V,U} \colon \sheaf{F}(V) \to \sheaf{F}(U)$ for all open subsets~$U \subseteq V \subseteq X$,
        \end{itemize}
        such that
        \begin{enumerate}[label=(P\arabic*)]
          \item
            $\rho_{U,U} = \id_{\sheaf{F}(U)}$ for every open subset~$U \subseteq X$, and
          \item
            $\rho_{V,U} \circ \rho_{W,V} = \rho_{W,U}$ for all open subsets~$U \subseteq V \subseteq W \subseteq X$.
        \end{enumerate}
      \end{definitionnonum}
      
      For an open subset~$U \subseteq X$ one may think about the abelian group~$\sheaf{F}(U)$ as consisting of certain functions on~$U$ which can be added together.
      One can consequently think about the homomorphism~$\rho_{V,U} \colon \sheaf{F}(V) \to \sheaf{F}(U)$ associated to open subsets~$U \subseteq V \subseteq X$ as restricting the functions on~$V$ to the functions on~$U$.
      
      For an open subset~$U \subseteq X$, the elements of~$\sheaf{F}(U)$ are called \emph{sections}\index{section} of~$\sheaf{F}$ on~$U$.
      For open subsets~$U \subseteq V \subseteq X$, the homomorphism~$\rho_{V,U} \colon \sheaf{F}(V) \to \sheaf{F}(U)$ is the \emph{restriction homomorphism} from~$V$ to~$U$, and for a section~$s \in \sheaf{F}(U)$ one calls~$\rho_{V,U}(s) \in \sheaf{F}(U)$ the \emph{restriction} of~$s$ to~$U$.
      This restriction is also denoted by~$\restrict{s}{U}$ instead of~$\rho_{V,U}(s)$.
      We then have that
      \[
        \restrict{s}{U} = s
      \]
      for every open subset~$U \subseteq X$ and every section~$s \in \sheaf{F}(U)$, and
      \[
        \restrict{ (\restrict{s}{V}) }{U} = \restrict{s}{U}
      \]
      for all open subsets~$U \subseteq V \subseteq W \subseteq X$ and every section~$s \in \sheaf{F}(W)$.
      It also holds that
      \[
          \restrict{(s+t)}{U}
        = \restrict{s}{U} + \restrict{t}{U}
      \]
      for all open subsets~$U \subseteq V \subseteq X$ and all sections~$s, t \in \sheaf{F}(V)$  because the restriction homomorphism~$\rho_{V,U}$ is a group homomorphism.
      
      \begin{examplenonum}
        \leavevmode
        \begin{enumerate}
          \item
            For every open subset~$U \subseteq X$ let
            \[
                        \cont_X(U)
              \defined  \{
                          f \colon U \to \Real
                        \suchthat
                          \text{$f$ is continuous}
                        \} \,,
            \]
            and for all open subsets~$U \subseteq V \subseteq X$ let
            \[
                      \rho_{V,U}
              \colon  \cont_X(V)
              \to     \cont_X(U) \,,
              \quad   f
              \mapsto \restrict{f}{U}
            \]
            be the (literal) restriction homomorphism.
            This defines a presheaf~$\cont_X$ on~$X$, the \emph{presheaf of continuous functions}.
          \item
            For every abelian group~$A$ we can consider the \emph{constant~\dash{$A$}{valued} presheaf}\index{constant!presheaf}\index{presheaf!constant} on~$X$, which is denoted by~$\widetilde{\sheaf{C}}_{X,A}$ and given by~$\widetilde{\sheaf{C}}_{X,A}(U) = A$ for every open subset~$U \subseteq X$, and~$\rho_{V,U} = \id_A$ for all open subsets~$U \subseteq V \subseteq X$.
        \end{enumerate}
      \end{examplenonum}
      
      Let~$\sheaf{F}, \sheaf{G}$ be two presheaves on~$X$.
      A \emph{homomorphism}\index{homomorphism!of presheaves} of presheaves~$f \colon \sheaf{F} \to \sheaf{G}$ is a tuple~$(f_U)_{U \subseteq X}$ of group homomorphisms~$f_U \colon \sheaf{F}(U) \to \sheaf{G}(U)$, where~$U \subseteq X$ ranges through the open subsets of~$X$, such that for all open subsets~$U \subseteq V \subseteq X$ the square
      \[
        \begin{tikzcd}[sep = large]
            \sheaf{F}(V)
            \arrow{r}[above]{f_V}
            \arrow{d}[left]{\rho_{V,U}^\sheaf{F}}
          & \sheaf{G}(V)
            \arrow{d}[right]{\rho_{V,U}^\sheaf{G}}
          \\
            \sheaf{F}(U)
            \arrow{r}[above]{f_U}
          & \sheaf{G}(U)
        \end{tikzcd}
      \]
      commutes.
      Let~$\Pcat \defined \Presheaf_X(\Ab)$ be the category of presheaves on~$X$.
      
    \item
      We can interpret the presheaf category~$\Pcat$ as a functor category:
      Let~$\Xcat$ be the category defined by objects
      \[
                  \Ob(\Xcat)
        \defined  \{
                    U
                  \suchthat
                    \text{$U \subseteq X$ is open}
                  \}
      \]
      and morphism sets
      \[
                  \Xcat(U,V)
        \defined  \begin{cases}
                    \{ i_{U,V} \} & \text{if~$U \subseteq V$} \,, \\
                    \emptyset     & \text{otherwise} \,,
                  \end{cases}
      \]
      where~$i_{U,V} \colon U \to V$ is the inclusion.
      The composition of morphisms in~$\Xcat$ is defined in the only possible way.
      The presheaf category~$\Pcat$ is then equivalent to the functor category~$\Fun(\Xcat^\op, \Ab)$.
      We see from this alternative description of the presheaf category~$\Pcat$ that it is abelian.
      
      \begin{remark*}
        A \emph{preorder}\index{preorder} on a set~$P$ is a relation~$\leq$ which is reflexive and transitive, i.e.\ it holds that~$x \leq x$ for every~$x \in P$, and it holds for all~$x, y, z \in P$ with~$x \leq y$ and~$y \leq z$ that also~$x \leq z$.
        (But in contrast to a partial order, a preorder does not have to be antisymmetric.
        So there may exist~$x, y \in P$ with both~$x \leq y$ and~$y \leq x$ but~$x \neq y$.)
        A \emph{preordered set} is a pair~$(P,\leq)$ consisting of a set~$P$ and a preorder~$\leq$ on~$P$.
        
        If~$(P,\leq)$ is a preordered set then one can define a category~$\Pcat$ whose objects are given by the elements of~$P$, and in which there exists for every two elements~$x, y \in P$ a morphism~$x \to y$ in~$\Pcat$ if and only if~$x \leq y$, and this morphism is then unique.
        More formally speaking, we have that
        \[
            \Ob(\Pcat)
          = P \,,
        \]
        and the morphisms sets of~$\Pcat$ are for any two objects~$x, y \in P$ given by
        \[
            \Pcat(x,y)
          = \begin{cases}
              \{\ast\}  & \text{if~$x = y$} \,, \\
              \emptyset & \text{if~$x \neq y$}  \,.
            \end{cases}
        \]
        The composition of morphisms in defined in the only possible way.
        In the resulting category~$\Pcat$ there exists between any two objects at most one morphism.
        Such categories are called~\emph{thin}\index{thin category}\index{category!thin}.
        
        If on the other hand~$\Tcat$ is any thin category whose class of objects~$T \defined \Ob(\Tcat)$ is a set (and not a proper class), then one can define a preorder~$\leq$ on~$T$ via
        \[
                s \leq t
          \iff  \Tcat(s,t) \neq \emptyset
        \]
        for all~$s, t \in T$.
        This then results in a preordered set~$(T, \leq)$.
        
        These two constructions are mutually inverse, and show that preordered sets~$(P,\leq)$ are \enquote{the same} as thin categories~$\Tcat$ whose class of objects form a set.
        
        In the above example, the preordered set (which is already an ordered set) is given by~$P = \{U \subseteq X \suchthat \text{$U$ is open}\}$ together with the inclusion~$\subseteq$ as a preorder.
        The category~$\Xcat$ then results from the preordered set~$(P, \subseteq)$ via the above construction.
      \end{remark*}
      
    \item
      We are now ready to introduce sheaves:
      
      \begin{definitionnonum}
        A presheaf~$\sheaf{F}$ on~$X$ is a \emph{sheaf}\index{sheaf} if for every open subset~$U \subseteq X$ and every open cover~$\{ U_i \}_{i \in I}$ of~$U$ the following two conditions are satisfied:
        \begin{enumerate}[label=(S\arabic*)]
          \item
            \label{separation axiom}
            If~$s \in \sheaf{F}(U)$ is a section with~$\restrict{s}{U_i} = 0$ for every~$i \in I$ then already~$s = 0$.
          \item
            \label{glueing axiom}
            Suppose that~$s_i \in \sheaf{F}(U_i)$ is a section for every~$i \in I$, so that
            \[
                \restrict{s_i}{U_i \cap U_j}
              = \restrict{s_j}{U_i \cap U_j}
            \]
            for all~$i, j \in I$.
            Then there exists a section~$s \in \sheaf{F}(U)$ with~$\restrict{s}{U_i} = s_i$ for every~$i \in I$.
        \end{enumerate}
        Condition~\ref*{separation axiom} is the \emph{separation~axiom}\index{separation axiom}\index{sheaf!separation axiom} and condition~\ref*{glueing axiom} is the \emph{glueing~axiom}\index{glueing axiom}\index{sheaf!glueing axiom}.
      \end{definitionnonum}
      
      We denote by~$\Scat = \Sheaf_X(\Ab)$ the category of sheaves on~$X$, which is the full subcategory of the presheaf category~$\Pcat = \Presheaf_X(\Ab)$ whose objects are sheaves.
      
      Note that if~$\sheaf{F}$ is any sheaf on~$X$ then~$\sheaf{F}(\emptyset) = 0$:
      We may choose for the open subset~$U = \emptyset$ the empty covering~$U = \bigcup_{i \in \emptyset} U_i$.
      It then follows for any two sections~$s, t \in \sheaf{F}(\emptyset)$ from the separation axiom that~$s = t$, which shows that the abelian group~$\sheaf{F}(\emptyset)$ consists of only a single element.
      
      \begin{examplenonum}
        \leavevmode
        \begin{enumerate}
          \item
            The presheaf of continuous functions~$\cont_X$ is already a sheaf.
          \item
            If~$A$ is a nonzero abelian group then the constant presheaf~$\widetilde{\sheaf{C}}_{X,A}$ is not a sheaf, because~$\widetilde{\sheaf{C}}_{X,A}(\emptyset) = A \neq 0$.
        \end{enumerate}
      \end{examplenonum}
      
    \item
      The sheaf category~$\Scat$ is additive:
      \begin{description}[font=\bfseries]
        \item[Preadditive:]
          The sheaf category~$\Scat$ is a full subcategory of the presheaf category~$\Pcat$, which is a preadditive category.
          It therefore inherits the structure of a preadditive category from~$\Pcat$.
          Note that the inclusion functor~$I \colon \Scat \to \Pcat$ is additive.
        \item[Zero object:]
          The zero presheaf~$0$ is already a sheaf, and hence contained in~$\Scat$.
          It is also a zero object in~$\Scat$ because~$\Scat$ is a full subcategory of~$\Pcat$.
        \item[Biproducts:]
          The biproduct~$\sheaf{F} \oplus \sheaf{G}$ of two presheaves~$\sheaf{F}$ and~$\sheaf{G}$ in the presheaf category~$\Pcat$ is given by
          \[
              (\sheaf{F} \oplus \sheaf{G})(U)
            = \sheaf{F}(U) \oplus \sheaf{G}(U)
          \]
          for every open subset~$U \subseteq X$, together with the restriction homomorphisms
          \[
              \rho^{\sheaf{F} \oplus \sheaf{G}}_{V,U}
            = \rho^\sheaf{F}_{V,U} \oplus \rho^\sheaf{G}_{V,U}
          \]
          for all open subsets~$U \subseteq V \subseteq X$, i.e.\
          \[
              \restrict{(s,t)}{U}
            = ( \restrict{s}{U}, \restrict{t}{U} )
          \]
          for every section ~$(s,t) \in \sheaf{F}(U) \oplus \sheaf{G}(U)$.
          
          If both~$\sheaf{F}$ an~$\sheaf{G}$ are sheaves then their biproduct~$\sheaf{F} \oplus \sheaf{G}$ is again a sheaf:
          Let~$U \subseteq X$ be an open subset and let~$\{ U_i \}_{i \in I}$ be an open cover of~$U$.
          \begin{enumerate}[label=(S\arabic*)]
            \item
              Let~$s \in \sheaf{F}(U) \oplus \sheaf{G}(U)$ be a section with~$\restrict{s}{U_i} = 0$ for every~$i \in I$.
              We may write~$s = (t,u)$ for some sections~$t \in \sheaf{F}(U)$ and~$u \in \sheaf{G}(U)$, and it holds for every~$i \in I$ that
              \[
                  0
                = \restrict{s}{U_i}
                = \restrict{(t,u)}{U_i}
                = ( \restrict{t}{U_i}, \restrict{u}{U_i} ) \,.
              \]
              It therefore holds for every~$i \in I$ that~$\restrict{t}{U_i} = 0$ and~$\restrict{u}{U_i} = 0$.
              It follows that~$t = 0$ and~$u = 0$ because both~$\sheaf{F}$ and~$\sheaf{G}$ are sheaves (and hence satisify the separation axiom).
              This shows that~$s = 0$, and hence that~$\sheaf{F} \oplus \sheaf{G}$ satisfies the separation axiom.
            \item
              For every~$s \in S$ let~$s_i \in U_i$ be a section, such that
              \[
                  \restrict{s_i}{U_i \cap U_j}
                = \restrict{s_j}{U_i \cap U_j}
              \]
              for all~$i,j \in I$.
              Then every~$s_i$ can be written as~$s_i = (t_i, u_i)$ for some sections~$t_i \in \sheaf{F}(U_i)$ and~$u_i \in \sheaf{G}(U_i)$, and it holds for all~$i, j \in I$ that
              \begin{align*}
                    (\restrict{t_i}{U_i \cap U_j}, \restrict{u_i}{U_i \cap U_j})
                &=  \restrict{(t_i, u_i)}{U_i \cap U_j}
                 =  \restrict{s_i}{U_i \cap U_j}  \\
                &=  \restrict{s_j}{U_i \cap U_j}
                 =  \restrict{(t_j, u_j)}{U_i \cap U_j}
                 =  (\restrict{t_j}{U_i \cap U_j}, \restrict{u_j}{U_i \cap U_j}) \,.
              \end{align*}
%               \begin{align*}
%                  {}&  ( \restrict{s^i_1}{U_i \cap U_j}, \dotsc, \restrict{s^i_n}{U_i \cap U_j} )  \\
%                 ={}&  \restrict{(s^i_1, \dotsc, s^i_n)}{U_i \cap U_j} \\
%                 ={}&  \restrict{s^i}{U_i \cap U_j}
%                 =     \restrict{s^j}{U_i \cap U_j}  \\
%                 ={}&  \restrict{(s^j_1, \dotsc, s^j_n)}{U_i \cap U_j} \\
%                 ={}&  ( \restrict{s^j_1}{U_i \cap U_j}, \dotsc, \restrict{s^j_n}{U_i \cap U_j} )
%               \end{align*}
              It therefore holds for all~$i, j \in I$ that
              \[
                  \restrict{t_i}{U_i \cap U_j}
                = \restrict{t_j}{U_i \cap U_j}
                \qquad\text{and}\quad
                  \restrict{u_i}{U_i \cap U_j}
                = \restrict{u_j}{U_i \cap U_j} \,.
              \]
              It follows that there exist sections~$t \in \sheaf{F}(U)$ and~$u \in \sheaf{G}(U)$ with~$\restrict{t}{U_i} = t_i$ and~$\restrict{u}{U_i} = u_i$ for every~$i \in I$ because both~$\sheaf{F}$ and~$\sheaf{G}$ are sheaves (and hence satisfiy the glueing axiom).
              It follows for the section~$s$ that
              \[
                  \restrict{s}{U_i}
                = \restrict{(t,u)}{U_i}
                = ( \restrict{t}{U_i}, \restrict{u}{U_i} )
                = ( t_i, u_i )
                = s_i
              \]
              for every~$i \in I$.
              This shows that~$\sheaf{F} \oplus \sheaf{G}$ satisfies the glueing axiom.
          \end{enumerate}
          
          This shows together that the biproduct~$\sheaf{F} \oplus \sheaf{G}$ is contained in~$\Scat$.
          The sheaf~$\sheaf{F} \oplus \sheaf{G}$ is also a biproduct of~$\sheaf{F}$ and~$\sheaf{G}$ in~$\Scat$ because~$\Scat$ is a full subcategory of~$\Pcat$.
          
          This shows that~$\Scat$ admits binary biproducts;
          it follows inductively that~$\Scat$ admits biproducts~$\sheaf{F}_1 \oplus \dotsb \oplus \sheaf{F}_n$ for any collections of sheaves~$\sheaf{F}_1, \dotsc, \sheaf{F}_n$ with~$n \geq 1$.
          For~$n = 0$ this is also true because~$\Scat$ contains a zero object.
      \end{description}
      
      This shows alltogether that the sheaf category~$\Scat$ is indeed additive.
          
    \item
      The sheaf category~$\Scat$ admits kernels:
      To be more precise, let~$\sheaf{F}$ and~$\sheaf{G}$ be two sheaves and let~$f \colon \sheaf{F} \to \sheaf{G}$ be a homomorphism of sheaves.
      We may forget that~$\sheaf{F}$ and~$\sheaf{G}$ are sheaves and regard them as just presheaves;
      then~$f$ is a homomorphisms of presheaves, and hence a morphism in the abelian category~$\Pcat$.
      We can therefore consider its kernel~$\ker_{\Pcat}(f)$ in the category~$\Pcat$.
      
      This kernel~$\ker_\Pcat(f)$ of~$f$ in the presheaf category~$\Pcat$ is already a sheaf:
      The kernel~$\ker_\Pcat(f)$ is given by
      \[
          (\ker_\Pcat(f))(U)
        = \ker(f_U)
      \]
      for every open subset~$U \subseteq X$, together with the restriction homomorphisms
      \[
        \rho^{\ker_{\Pcat}(f)}_{V,U} \colon \ker(f_V) \to \ker(f_U)
      \]
      for all open subsets~$U \subseteq V \subseteq X$ which are given by restriction of the restriction homomorphisems~$\rho^{\sheaf{F}}_{V,U} \colon \sheaf{F}(V) \to \sheaf{F}(U)$.
      Let~$U \subseteq X$ be open and let~$\{ U_i \}_{i \in I}$ be an open cover of~$U$.
      \begin{enumerate}[label=(S\arabic*)]
        \item
          Let~$s \in (\ker_{\Pcat}(f))(U) = \ker(f_U)$ be a section with~$\restrict{s}{U_i} = 0$ for every~$i \in I$.
          This section~$s$ is an element of~$\sheaf{F}(U)$ (because~$\ker(f_U)$ is a subgroup of~$\sheaf{F}(U)$), and the relation~$\restrict{s}{U_i} = 0$ also holds in the sheaf~$\sheaf{F}$ (i.e.\ it follows from~$\rho^{\ker_{\Pcat}(f)}_{U,U_i}(s) = 0$ that also~$\rho^{\sheaf{F}}_{U,U_i}(s) = 0$) because the square
          \[
            \begin{tikzcd}[sep = large]
                \ker(f_U)
                \arrow[hook]{r}
                \arrow{d}[left]{\rho_{U,U_i}}
              & \sheaf{F}(U)
                \arrow{d}[right]{\rho_{U,U_i}}
              \\
                \ker(f_{U_i})
                \arrow[hook]{r}
              & \sheaf{F}(U_i)
            \end{tikzcd}
          \]
          commutes.
          It follows that~$s = 0$ in~$\sheaf{F}(U)$, and hence also in~$\ker(f_U)$, because~$\sheaf{F}$ is a sheaf and hence satisfies the separation axiom.
          This shows that the presheaf~$\ker_{\Pcat}(f)$ satisfies the separation axiom.
        \item
          For every~$i \in I$ let~$s_i \in (\ker_{\Pcat}(f))(U_i) = \ker(f_{U_i})$ be a section such that~$\restrict{s_i}{U_i \cap U_j} = \restrict{s_j}{U_i \cap U_j}$ for all~$i,j \in I$.
          We can then regard every~$s_i$ as an element of~$\sheaf{F}(U_i)$, and the relation~$\restrict{s_i}{U_i \cap U_j} = \restrict{s_i}{U_i \cap U_j}$ also holds in the sheaf~$\sheaf{F}$ because the diagram
          \[
            \begin{tikzcd}[column sep = huge, row sep = tiny]
                \ker(f_{U_i})
                \arrow{dr}[above right, near start]{\rho_{U_i, U_i \cap U_j}}
                \arrow[hook]{dd}
              & {}
              & \ker(f_{U_j})
                \arrow{dl}[above left, near start]{\rho_{U_j, U_i \cap U_j}}
                \arrow[hook]{dd}
              \\
                {}
              & \ker(f_{U_i \cap U_j})
                \arrow[hook]{dd}
              & {}
              \\
                \sheaf{F}(U_i)
                \arrow{dr}[below left, near end]{\rho_{U_i, U_i \cap U_j}}
              & {}
              & \sheaf{F}(U_j)
                \arrow{dl}[below right, near end]{\rho_{U_j, U_i \cap U_j}}
              \\
                {}
              & \sheaf{F}(U_i \cap U_j)
              & {}
            \end{tikzcd}
          \]
          commutes.
          It follows that there exists a section~$s \in \sheaf{F}(U)$ with~$\restrict{s}{U_i} = s_i$ for every~$i \in I$ because~$\sheaf{F}$is a sheaf and hence satisfies the glueing axiom.
          We need to show that already~$s \in \ker(f_U)$, i.e.\ that~$f_U(s) = 0$.
          We use that~$f$ is a homomorphism of sheaves to calculate
          \[
              \restrict{f_U(s)}{U_i}
            = f_{U_i}( \restrict{s}{U_i} )
            = f_{U_i}(s_i)
            = 0
          \]
          for every~$i \in I$.
          It follows that~$f_U(s) = 0$ because~$\sheaf{G}$ is a sheaf and hence satisfies the separation axiom.
          This shows that~$\ker(f)$ satisfies the glueing axiom.
      \end{enumerate}
      
      We can similarly consider the cokernel~$\coker_{\Pcat}(f)$.
      This presheaf is given by
      \[
        (\coker_{\Pcat}(f))(U) = \coker(f_U)
      \]
      for every open subset~$U \subseteq X$, and the restriction homomorphism
      \[
                \rho^{\coker_{\Pcat}(f)}_{V,U}
        \colon  \sheaf{G}(V)
        \to     \sheaf{G}(U)
      \]
      is for all open subsets~$U \subseteq V \subseteq X$ induced by the restriction homomorphism~$\rho^{\sheaf{G}}_{V,U} \colon \sheaf{G}(V) \to \sheaf{G}(U)$ in the sense that the following square commutes:
      \[
        \begin{tikzcd}[sep = large]
            \sheaf{G}(V)
            \arrow[two heads]{r}
            \arrow{d}[left]{\rho_{V,U}}
          & \coker^{\Pcat}(f_V)
            \arrow{d}[right]{\rho_{V,U}}
          \\
            \sheaf{G}(U)
            \arrow[two heads]{r}
          & \coker^{\Pcat}(f_U)
        \end{tikzcd}
      \]
      
      In constrast to kernels it does not necessarily hold that the cokernel~$\coker_{\Pcat}(f)$ is already a sheaf.
      
      \begin{examplenonum}
        Let~$X \defined \sphere^1$ and let~$\sheaf{F} \defined \sheaf{G} \defined \sheaf{C}^\infty$ be the sheaf of smooth \dash{real}{valued} functions on~$\sphere^1$, which is given by
        \[
            \sheaf{C}^\infty(U)
          = \{
              f \colon U \to \Real
            \suchthat
              \text{$f$ is smooth}
            \}
        \]
        for every open subset~$U \subseteq \sphere^1$ and has for all open subsets~$U \subseteq V \subseteq \sphere^1$ as restriction homomorphisms
        \[
                  \rho_{V,U}
          \colon  \sheaf{C}^\infty(V)
          \to     \sheaf{C}^\infty(U) \,,
          \quad   f
          \mapsto \restrict{f}{U}
        \]
        the (literal) restriction maps.

        Let us consider the homomorphism~$d \colon \sheaf{C}^\infty \to \sheaf{C}^\infty$ that is given by the derivative, i.e.\ by
        \[
                  d_U
          \colon  \sheaf{C}^\infty(U)
          \to     \sheaf{C}^\infty(U) \,,
          \quad   f
          \mapsto f'
        \]
        for every open subset~$U \subseteq \sphere^1$.
        If~$U \subsetneq \sphere^1$ is a proper open subset then~$U$ corresponds to an open subset of~$(0,1)$, hence there exists for every~$f \in \sheaf{C}^\infty(U)$ an antiderivative for~$f$ on~$U$.
        The homomorphism~$d_U \colon \sheaf{C}^\infty(U) \to \sheaf{C}^\infty(U)$ is hence surjective, and has therefore the cokernel
        \[
            \coker(d_U)
          = 0 \,.
        \]
        Consider on the other hand the open subset~$U = S^1$.
        On this open subset, the constant~\dash{$1$}{function}~$1 \in \sheaf{C}^\infty(\sphere^1)$ has no antiderivative because it lifts to the constant~\dash{$1$}{function}~$1 \in \sheaf{C}^\infty(\Real)$, whose antiderivatives~$x \mapsto x + c$ with~$c \in \Real$ are not periodic.
        The homomorphism~$d_{\sphere^1}$ does therefore have a nonvanishing cokernel
        \[
                \coker(f_{\sphere^1})
          \neq  0 \,.
        \]
        (One can actually identify the cokernel~$\coker(f_{\sphere^1})$ with the constant functions on~$\sphere^1$, so that~$\coker(f_{\sphere^1}) \cong \Real$.)
        Let~$\sphere^1 = U_1 \cup U_2$ be an open cover by proper open subsets~$U_1, U_2 \subsetneq \sphere^1$.
        If~$\coker(d)$ were a sheaf then it would follows from~$\coker(d_{U_1}) = 0$ and~$\coker(d_{U_2}) = 0$ by the separation axiom that also~$\coker(d_{\sphere^1}) = 0$.
        Hence~$\coker(d)$ is not a sheaf.
      \end{examplenonum}
      
    \item
      Let~$I \colon \Scat \to \Pcat$ be the inclusion functor, which is both fully faithful and additive.
      
      \begin{fact*}
        The inlcusion functor~$I$ has a left adjoint~$S \colon \Pcat \to \Scat$.
        The functor~$S$ is again additive and the adjunction is also additive, in the sense that the natural bijections
        \[
                  \varphi_{\sheaf{F},\sheaf{G}}
          \colon  \Scat(S(\sheaf{F}), \sheaf{G})
          \to     \Pcat(\sheaf{F}, I(\sheaf{G}))
        \]
        are isomorphisms of abelian groups.
      \end{fact*}
      
      \begin{definitionnonum}
        For a presheaf~$\sheaf{F}$ on~$X$, the sheaf~$S(\sheaf{F})$ is the \emph{sheafification}\index{sheafification} of~$\sheaf{F}$.
      \end{definitionnonum}
      
      Let~$\eta \colon \id_{\Pcat} \to I \circ S$ be the unit of the adjunction~$(S,I,\varphi)$.
      This adjunction then states the the sheafification has the following universal property:      
      There exists for every sheaf~$\sheaf{G}$ and every homomorphism of presheaves~$f \colon \sheaf{F} \to I(\sheaf{G})$ a unique homomorphism of sheaves~$\lambda \colon S(\sheaf{F}) \to \sheaf{G}$ which make the following triangle commute:
      \[
        \begin{tikzcd}[sep = large]
            \sheaf{F}
            \arrow{r}[above]{\eta_{\sheaf{F}}}
            \arrow{dr}[below left]{f}
          & IS(\sheaf{F})
            \arrow[dashed]{d}[right]{I(\lambda)}
          \\
            {}
          & I(\sheaf{G})
        \end{tikzcd}
      \]
      
      If a presheaf~$\sheaf{F}$ is already a sheaf, then it follows from the fully faithfulness of the inclusion~$I$ that the sheafification~$SI(\sheaf{F})$ of~$\sheaf{F}$ is just~$\sheaf{F}$ itself.
      
      To be more precise, the sheaf~$\sheaf{F}$ is also a sheafification of the presheaf~$I(\sheaf{F})$, in the sense that the identity~$\id_{I(\sheaf{F})} \colon I(\sheaf{F}) \to I(\sheaf{F})$ fullfills the same universal property as the counit~$\eta_{\sheaf{F}} \colon I(\sheaf{F}) \to ISI(\sheaf{F})$:
      Let~$\sheaf{G}$ be another sheaf and let~$f \colon I(\sheaf{F}) \to I(\sheaf{G})$ be a homomorphism of presheaves.
      Then, as~$I$ is fully faithful, there exist a unique homomorphism of sheaves~$\lambda \colon \sheaf{F} \to \sheaf{G}$ which makes the desired triangle
      \[
        \begin{tikzcd}[sep = large]
            I(\sheaf{F})
            \arrow{r}[above]{\id_{I(\sheaf{F})}}
            \arrow{dr}[below left]{f}
          & I(\sheaf{F})
            \arrow{d}[right]{I(\lambda)}
          \\
            {}
          & I(\sheaf{G})
        \end{tikzcd}
      \]
      commute.
      
      As both~$\eta \colon I(\sheaf{F}) \to ISI(\sheaf{F})$ and~$\id_{I(\sheaf{F})} \colon I(\sheaf{F}) \to I(\sheaf{F})$ have the same universal property, we can conclude that there exist a unique homomorphism of presheaves~$ISI(\sheaf{F}) \to I(\sheaf{F})$ which makes the triangle
      \[
        \begin{tikzcd}[column sep = small]
            {}
          & I(\sheaf{F})
            \arrow{dl}[above left]{\eta_{\sheaf{F}}}
            \arrow{dr}[above right]{\id_{I(\sheaf{F})}}
          & {}
          \\
            ISI(\sheaf{F})
            \arrow[dashed]{rr}
          & {}
          & I(\sheaf{F})
        \end{tikzcd}
      \]
      commute, and that this homomorphism is already an isomorphism.
      This homomorphism is just~$\eta_{\sheaf{F}}$ itself (as~$\eta_{\sheaf{F}}$ makes the above triangle commute) and so~$\eta_{\sheaf{F}}$ is an isomorphism (of presheaves, and hence also of sheaves).
      
      \begin{remark*}
        One can also express the above argumentation in the language of representable functors:
        It holds that by the naturality of~$\varphi$ and the fully faithfulness of~$I$ that
        \[
                \Scat(S(I(\sheaf{F})), -)
          \cong \Pcat(I(\sheaf{F}), I(-))
          \cong \Scat(\sheaf{F}, -) \,,
        \]
        and hence~$S(I(\sheaf{F})) \cong \sheaf{F}$ (because representing objects are unique up to isomorphism).
%       TODO: How these isomorphisms look like explicitely, i.e. that eta is an iso.
      \end{remark*}
      
      \begin{examplenonum}[Sheafificaton]
        \leavevmode
        \begin{enumerate}
          \item
            Let~$A$ be an abelian group.
            The~$\sheaf{C}_{X,A}$\emph{constant sheaf}\index{constant!sheaf}\index{sheaf!constant} is defined to be the sheafification of the constant presheaf~$\widetilde{\sheaf{C}}_{X,A}$.
            It is given by
            \[
                \sheaf{C}_{X,A}(U)
              = \{
                  f \colon U \to A
                \colon
                  \text{$f$ is locally constant}
                \} \,.
            \]
            for every open subset~$U \subseteq X$, and for all open subsets~$U \subseteq V \subseteq X$ the restriction homomorphism
            \[
                      \rho_{V,U}
              \colon  \sheaf{C}_{X,A}(V)
              \to     \sheaf{C}_{X,A}(U) \,,
              \quad   f
              \mapsto \restrict{f}{U}
            \]
            is the (literal) restriction map.
          \item
            Let~$\sheaf{F}$ is a sheaf and let~$\sheaf{F}' \subseteq \sheaf{F}$ be a subpresheaf, i.e.~$\sheaf{F}$ is a presheaf such that $\sheaf{F}'(U) \subseteq \sheaf{F}(U)$ is a subgroup for every open subset~$U \subseteq X$, and such that the square
            \[
              \begin{tikzcd}[sep = large]
                  \sheaf{F}'(V)
                  \arrow[hook]{r}
                  \arrow{d}[left]{\rho^{\sheaf{F}'}_{V,U}}
                & \sheaf{F}(V)
                  \arrow{d}[right]{\rho^\sheaf{F}_{V,U}}
                \\
                  \sheaf{F}'(U)
                  \arrow[hook]{r}
                & \sheaf{F}(U)
              \end{tikzcd}
            \]
            commutes for all open subsets~$U \subseteq V \subseteq X$.
            Then the sheafification~$S(\sheaf{F}')$ is given by
            \[
                S(\sheaf{F}')(U)
              = \left\{
                  s \in \sheaf{F}(U)
                \suchthat*
                  \begin{tabular}{@{}c@{}}
                    there exists an open cover~$\{U_i\}_{i \in I}$ of~$U$ \\
                    with~$\restrict{s}{U_i} \in \sheaf{F}'(U_i)$ for every~$i\in I$
                  \end{tabular}
                \right\} \,,
            \]
            and the restriction homomorphism
            \[
                      \rho_{V,U}^{S(\sheaf{F}')}
              \colon  S(\sheaf{F}')(V)
              \to     S(\sheaf{F}')(U)
            \]
            is for all open subsets~$U \subseteq V \subseteq X$ the restriction of the restriction homomorphism~$\rho_{V,U}^{\sheaf{F}} \colon \sheaf{F}(V) \to \sheaf{F}(U)$.
        \end{enumerate}
      \end{examplenonum}
      
      For the general construction of the sheafification functor one uses stalks.
      We will not do this here.
      
    \item
      Let~$f \colon \sheaf{F} \to \sheaf{G}$ be a homomorphism of sheaves.
      That the cokernel~$\coker_{\Pcat}(I(f))$ is in general not a sheaf can be fixed by applying the sheafification functor~$S$:
      
      \begin{claimnonum}
        The sheafification~$S(\coker_{\Pcat}(I(f)))$ is a cokernel of~$f$ in~$\Scat$.
      \end{claimnonum}
      
      \begin{proof}
        Let~$\eta \colon \coker_{\Pcat}(I(f)) \to IS( \coker_{\Pcat}(I(f)) )$ be the canonical homomorphism, and let~$c' \colon I(\sheaf{G}) \to \coker(I(f))$ be the morphism belonging to the cokernel~$\coker(I(f))$.
        The functor~$I$ is fully faithful, and so there exists for the composition
        \[
                          I(\sheaf{G})
          \xlongto{c'}    \coker_{\Pcat}(I(f))
          \xlongto{\eta}  IS(\coker_{\Pcat}(I(f)))
        \]
        a unique homomorphism of sheaves~$c \colon \sheaf{G} \to S(\coker_{\Pcat}(I(f)))$ such that the homomorphism~$I(c)$ is the above composition.
        Then~$c$ is a cokernel of~$f$:
        
        It holds that
        \[
            I(c \circ f)
          = I(c) \circ I(f)
          = c' \circ I(f)
          = 0
        \]
        because~$c'$ is a cokernel of~$I(f)$ and~$I$ is additive.
        Suppose that~$h \colon \sheaf{G} \to \sheaf{H}$ is a homomorphism of sheaves with~$h \circ f = 0$.
        Then
        \[
          I(h) \circ I(f) = I(h \circ f) = I(0) = 0
        \]
        and it follows from the universal property of the cokernel~$c'$ that there exists a unique homomorphism of presheaves~$\lambda' \colon I(\sheaf{H}) \to \coker_{\Pcat}(I(f))$ which makes the triangle
        \[
          \begin{tikzcd}[row sep = large]
              {}
            & I(\sheaf{H})
            \\
              I(\sheaf{G})
              \arrow{ur}[above left]{I(h)}
              \arrow{r}[below]{c'}
            & \coker_{\Pcat}(I(f))
              \arrow[dashed]{u}[right]{\lambda'}
          \end{tikzcd}
        \]
        commute. 
        It follows from the universal property of the sheafification that there exists a unique homomorphism of sheaves~$\lambda' \colon \sheaf{H} \to S(\coker_\Pcat(I(f)))$ which makes the triangle
        \[
          \begin{tikzcd}[row sep = large]
              \coker_\Pcat(I(f))
              \arrow{r}[above]{\eta}
              \arrow{dr}[below left]{\lambda'}
            & IS(\coker_\Pcat(I(f)))
              \arrow{d}[right]{I(\lambda)}
            \\
              {}
            & I(\sheaf{H})
          \end{tikzcd}
        \]
        commute.
        It holds that
        \[
            I(h)
          = \lambda' \circ c'
          = I(\lambda) \circ \eta \circ c'
          = I(\lambda) \circ I(c)
          = I(\lambda \circ c)
        \]
        and hence~$h = \lambda \circ c$ because~$f$ is fully faithful.
        
        The uniqueness of~$\lambda$ can be shown similarly.
%       TODO: Do this.
      \end{proof}
      
      \begin{remark*}
        One can express the above proof in the language of representable functors:
        It follows from the naturality of the adjunction~$\varphi$, the universal property of the cokernel~$\coker_{\Pcat}(f)$ and the fully faithfulness of the inclusion functor~$I$ that
        \begin{align*}
               {}&  \Scat(S(\coker_{\Pcat}(I(f))), -) \\
          \cong{}&  \Pcat(\coker_{\Pcat}(I(f)), I(-)) \\
          \cong{}&  \{
                      g \in \Pcat(I(\sheaf{G}), I(-))
                    \suchthat
                      g \circ I(f) = 0
                    \}  \\
          \cong{}&  \{
                      h \in \Scat(\sheaf{G},-)
                    \suchthat
                      I(h) \circ I(f) = 0
                    \}  \\
          \cong{}&  \{
                      h \in \Scat(\sheaf{G},-)
                    \suchthat
                      I(h \circ f) = 0
                    \}  \\
          \cong{}&  \{
                      h \in \Scat(\sheaf{G},-)
                    \suchthat
                      h \circ f = 0
                    \} \,.
        \end{align*}
        This shows that~$S(\coker_{\Pcat}(I(f)))$ represents the right kind of functor to make it a cokernel of~$f$.
%       TODO: Explain this in more detail at a previous spot.
      \end{remark*}
  \end{enumerate}
  
  \lecturend{12}
\end{example}







