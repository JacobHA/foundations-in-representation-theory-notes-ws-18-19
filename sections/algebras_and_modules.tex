\chapter{Algebras and Modules}


\begin{conventions}
  In this course we adhere to the following conventions:
  \begin{enumerate}
    \item
      All rings are unital, i.e.\ there exists for every ring~$A$ an element~$1 \in A$ with both~$1 \cdot a = a$ and~$a \cdot 1 = a$ for every~$a \in A$.
    \item
      If~$A$ and~$B$ are rings then every ring homomorphism~$f \colon A \to B$ respects the unit, i.e.\ it holds that~$f(1_A) = 1_B$.
  \end{enumerate}
\end{conventions}





\section{Algebras}


\begin{conventions}
  Throughout this lecture~$\kf$ denotes a commutative ring, which we will often additionally assume to be a field.
\end{conventions}


\begin{definition}
  A~\emph{{\kalg}}\index{k-algebra@{\kalg}} is a ring~$A$ together with the structure of a~{\kmod} on~$A$, such that the ring multiplication and the scalar multiplication are compatible in the sense that
  \begin{equation}
    \label{compatibility of multiplications}
      (\lambda a) b
    = \lambda (ab)
    = a (\lambda b)
  \end{equation}
  for all~$\lambda \in \kf$ and all~$a, b \in A$.
\end{definition}


\begin{definition}
  For~{\kalgs}~$A$ and~$B$ a map~$f \colon A \to B$ is a \emph{homomorphism of~{\kalgs}}\index{homomorphism!of k-algebras@homomorphism of~{\kalgs}}\index{k-algebra@{\kalg}!homomorphism of} if it is a ring homomorphism which is also~{\klin}.
\end{definition}


\begin{remark}
  Let~$A$ be a ring.
  Then
  \[
              \ringcenter(A)
    \defined  \{
                a \in A
              \suchthat
                \text{$ab = ba$ for every~$b \in A$}
              \}
  \]
  is a commutative subring of~$A$, the \emph{center}\index{center} of~$A$.
\end{remark}


\begin{remark}
  Let~$A$ be a ring.
  To give a~{\kalg} structure on~$A$ is the same as giving a ring homomorphism~$\varphi \colon \kf \to \ringcenter(A)$.
  More precisely:
  \begin{enumerate}
    \item
      If~$A$ is a~{\kalg} then we get a ring homomorphism~$\tilde{\varphi} \colon \kf \to A$ given by~$\lambda \mapsto \lambda \cdot 1_A$.
      This ring homomorphism satisfies~$\im(\tilde{\varphi}) \subseteq \ringcenter(A)$ and therefore restricts to a ring homomorphism~$\varphi \colon A \to \ringcenter(A)$.
    \item
      Let~$\varphi \colon \kf \to \ringcenter(A)$ be a ring homomorphism.
      Define~$\lambda \cdot a = \varphi(\lambda) \cdot a$ for all~$\lambda \in \kf$ and all~$a \in A$.
      This gives~$A$ the structure of a~{\kmod}.
      The compatibility condition~\eqref{compatibility of multiplications} is satisfied because~$\im(\varphi) \subseteq \ringcenter(A)$.
  \end{enumerate}
  These two constructions are mutually inverse.
  \begin{enumerate}[resume]
    \item
      Let~$A$ and~$B$ be~{\kalgs} and let~$f \colon A \to B$ be a homomorphisms of rings.
      Then~$f$ is a homomorphism of~{\kalgs} if and only if it is compatible with the ring homomorphisms~$\kf \to \ringcenter(A)$ and~$\kf \to \ringcenter(B)$ in the sense that the following diagram commutes:
      \[
        \begin{tikzcd}[column sep = small]
            A
            \arrow{rr}[above]{f}
          & {}
          & B
          \\
            \ringcenter(A)
            \arrow[hook]{u}
          & {}
          & \ringcenter(B)
            \arrow[hook]{u}
          \\
            {}
          & \kf
            \arrow{ul}
            \arrow{ur}
          & {}
        \end{tikzcd}
      \]
  \end{enumerate}
\end{remark}


\begin{example}
  \leavevmode
  \begin{enumerate}
    \item
      Let~$V$ be a~{\kmod} and consider ~$\End_\kf(V)$ with the multiplication
      \[
                \End_\kf(V) \times \End_\kf(V)
        \to     \End_\kf(V) \,,
        \quad   (f, g)
        \mapsto f \circ g \,.
      \]
      Then~$\End_\kf(V)$ is a ring and becomes a~{\kalg} via the ring homomorphim
      \[
                \tilde{\varphi}
        \colon  \kf
        \to     \End_\kf(V) \,,
        \quad   \lambda
        \mapsto \lambda \cdot \id_V \,,
      \]
      for which~$\im(\tilde{\varphi}) \subseteq \End_\kf(V)$.
      (If~$\kf$ is a field then~$\ringcenter(\End_\kf(V)) = \kf \cdot \id_V \cong \kf$.)
    \item
      Take~$V = \kf^n$ (the free~{\kmod} of rank~$n$).
      Then
      \begin{gather*}
              \End_\kf(V)
        \cong \mat{n}{\kf} \,,
      \shortintertext{and}
                  \triag{n}{\kf}
        \defined  \left\{
                    M \in \mat{n}{\kf}
                  \suchthat
                    \text{$M$ is upper triangular}
                  \right\}
      \end{gather*}
      is a subalgebra\index{subalgebra} of~$\mat{n}{\kf}$, i.e.\ it is both a subring and a {\ksmod} of~$\mat{n}{\kf}$.
    \item
      Let~$G$ be a group.
      We define the \emph{group algebra}\index{group algebra}\index{algebra!group}~$\kf[G]$ as follows:
      As a~{\kmod} we have that
      \begin{align*}
                  \kf[G]
         \defined \kf^{(G)}
        &\defined \text{free~{\kmod} with basis~$G$}  \\
        &=        \left\{
                    \sum_{g \in G} a_g g
                  \suchthat*
                    \begin{tabular}{@{}c@{}}
                      $a_g \in \kf$ for every~$g \in G$, \\
                      all but finitely many~$a_g$ vanish
                    \end{tabular}
                  \right\} \,.
      \end{align*}
      The multiplication of two elements~$x, y \in \kf[G]$ that are given by linear combinations~$x = \sum_{g \in G} a_g g$ and~$y = \sum_{g \in G} b_g g$ is given by
      \[
          x \cdot y
        = \sum_{g, h \in G} a_g b_h (gh)
        = \sum_{g \in G}
          \left(
            \sum_{\substack{h, h' \in G \\ h h' = g}} a_{h} b_{h'}
          \right)
          g \,.
      \]
      This multiplication is associative and {\kbilin}, and the unit of~$\kf[G]$ is given by~$1_{\kf[G]} = e$ (where~$e$ denotes the neutral element of~$G$).
  \end{enumerate}
\end{example}


\begin{definition}
  A \emph{quiver}\index{quiver}~$Q$ is a directed graph.
  Formally~$Q$ is a quadruple~$(Q_0, Q_1, s, t)$ consisting of two sets~$Q_0$,~$Q_1$ and two functions~$s, t \colon Q_1 \to Q_0$.
  \begin{itemize}
    \item
      The elements~$i \in Q_0$ are the \emph{vertices}\index{vertex} of~$Q$.
    \item
      The lements~$\alpha \in Q_1$ are the \emph{arrows}\index{arrow} of~$Q$.
    \item
      For an arrow~$\alpha \in Q_1$ the vertex~$s(\alpha)$ is the \emph{source}\index{source!of an arrow} of~$\alpha$.
    \item
      For an arrow~$\alpha \in Q_1$ the vertex~$t(\alpha)$ is the \emph{target}\index{target!of an arrow} of~$\alpha$.
  \end{itemize}
  An arrow~$\alpha \in Q_1$ can pictorially be represented as follows:
  \[
    \begin{tikzcd}
        s(\alpha)
        \arrow{r}[above]{\alpha}
      & t(\alpha)
    \end{tikzcd}
  \]
\end{definition}


\begin{example}
  \label{quivers first examples}
  \leavevmode
  \begin{enumerate}
    \item
      The quiver~$Q = (\{1\}, \emptyset, \emptyset, \emptyset)$ is given by a single vertex, labeled~$1$, and no arrows:% dont let footnote see this line break
      \footnote{The third and fourth elements of the quadrupel~$(\{1\}, \emptyset, \emptyset, \emptyset)$ refer to the empty function~$\emptyset \to \{1\}$.}
      \[
        \begin{tikzcd}
          1
        \end{tikzcd}
      \]
    \item
      The quiver~$Q = (\{1\}, \{\alpha\}, s, t)$ with (necessarily)~$s(\alpha) = t(\alpha) = 1$ is given by a single vertex, labeled~$1$, together with a single arrow, labeled~$\alpha$:
      \[
        \begin{tikzcd}
          1
          \arrow[loop]{r}[above]{\alpha}
        \end{tikzcd}
      \]
    \item
      The quiver~$Q = (\{1, 2\}, \{\alpha, \beta\}, s, t)$ with~$s(\alpha) = s(\beta) = 1$ and~$t(\alpha) = t(\beta) = 2$ is given by two vertices, labeled~$1$ and~$2$, which are connected via two parallel arrows going from~$1$ to~$2$, that are labeled~$\alpha$ and~$\beta$:
      \[
        \begin{tikzcd}
            1
            \arrow[shift left = 0.3em]{r}[above]{\alpha}
            \arrow[shift right = 0.3em]{r}[below]{\beta}
          & 2
        \end{tikzcd}
      \]
  \end{enumerate}
\end{example}


\begin{definition}
  Let~$Q$ be a quiver, assumed to be finite (i.e.\ both~$Q_0$ and~$Q_1$ are finite sets)\index{quiver!finite}.
  \begin{enumerate}
    \item
      Let~$\ell \in \Integer_{\geq 1}$.
      A \emph{path}\index{path!of length~$\geq 1$} in~$Q$ of length\index{length}~$\ell$ is a sequence~$\alpha_\ell \dotsm \alpha_1$ of arrows in~$Q$ such that~$t(\alpha_i) = s(\alpha_{i+1})$ for every~$i$.
      This may pictorially be represented as follows:
      \[
        \begin{tikzcd}
            \bullet
            \arrow{r}[above]{\alpha_1}
          & \bullet
            \arrow{r}[above]{\alpha_2}
          & \bullet
            \arrow{r}
          & \cdots
            \arrow{r}[above]{\alpha_\ell}
          & \bullet
        \end{tikzcd}
      \]
      The set of all paths of length~$\ell$ in~$Q$ is denoted by~$Q_\ell$.
      For a path~$p = \alpha_\ell \dotsm \alpha_1 \in Q_\ell$ its \emph{source}\index{source!of a path} is given by~$s(p) \defined s(\alpha_1)$ and its \emph{target}\index{target!of a path} is given by~$t(p) \defined t(\alpha_\ell)$.
      
      The set of paths of length~$0$\index{path!of length~$0$} in~$Q$ is formally defined as~$Q_0$, i.e.\ there exists for every~$i \in Q_0$ a unique path~$\varepsilon_i$ of length~$0$, and every path of length~$0$ is of this form.
      The path~$\varepsilon_i$ is the \enquote{lazy path}\index{path!lazy} at~$i$.
      We set~$s(\varepsilon_i) = i$,~$t(\varepsilon_i) = i$.
    \item
      Let~$p = \alpha_\ell \dotsm \alpha_1$ and~$q = \beta_k \dotsm \beta_1$ be paths in~$Q$ of lengths~$\ell, k \geq 1$.
      If~$t(p) = s(q)$ then the \emph{composition}\index{composition of paths} of~$p$ and~$q$ is defined as
      \[
          p \circ q
        = \beta_k \dotsm \beta_1 \alpha_\ell \dotsm \alpha_1 \,.
      \]
      The path~$p \circ q$ can pictorially be represented as follows:
      \[
        \begin{tikzcd}
            \bullet
            \arrow{r}[above]{\alpha_1}
          & \bullet
            \arrow{r}[above]{\alpha_2}
          & \cdots
            \arrow{r}[above]{\alpha_\ell}
          & \bullet
            \arrow[equal]{r}
          & \bullet
            \arrow{r}[above]{\beta_1}
          & \bullet
            \arrow{r}[above]{\beta_2}
          & \cdots
            \arrow{r}
            \arrow{r}[above]{\beta_k}
          & \bullet
        \end{tikzcd}
      \]
      % TODO: Add underbraces under the parts belonging to p and q.
      
      If~$p$ is a path in~$Q$ of length~$\ell \geq 0$ and~$i \in Q_0$ is a vertex then we set~$\varepsilon_i \circ p = p$ if~$t(p) = i$, as well as~$p \circ \varepsilon_i = p$ if~$s(p) = i$.
      
      In all other cases the composition of paths is not defined.
      
    \item
      We define~$Q_*$ to be the set of all paths in~$Q$, that is~$Q_* \defined \coprod_{\ell \geq 0} Q_\ell$.
      The \emph{path algebra}\index{path!algebra}\index{algebra!path}~$\kf Q$ of~$Q$ is defined as follows:
      As a {\kmod} we have that
      \[
                  \kf Q
        \defined  \kf^{(Q_*)}
        =         \left\{
                    \sum_{p \in Q_*} a_p p
                  \suchthat*
                    \begin{tabular}{@{}c@{}}
                      $a_p \in \kf$ for every~$p \in Q_*$ \\
                      all but finitely many~$a_p$ vanish
                    \end{tabular}
                  \right\} \,.
      \]
      The multiplication of two elements~$x, y \in \kf Q$ which are given by linear combinations~$x = \sum_{p \in Q_*} a_p p$ and~$y = \sum_{p \in Q_*} b_p p$ is given by
      \begin{gather*}
                  x \cdot y
        \defined \sum_{p, q \in Q_*} a_p b_q (p \cdot q) \,,
      \shortintertext{where}
                  p \cdot q
        \defined  \begin{cases}
                    p \circ q & \text{if~$s(p) = t(q)$},  \\
                    0         & \text{else} \,,
                  \end{cases}
      \end{gather*}
      for all paths~$p, q \in Q_*$.
      This multiplication is associative because composition of paths is associative and {\kbilin}, and the unit of~$\kf Q$ is given by~$1_{\kf Q} = \sum_{i \in Q_0} \varepsilon_i$.
  \end{enumerate}
\end{definition}


\begin{example}
  We determine the path algebras of the quivers from \cref{quivers first examples}.
  \begin{enumerate}
    \item
      For the quiver
      \[
        Q :
        \begin{tikzcd}
          1
        \end{tikzcd}
      \]
      its path algebra is given by~$\kf Q = k$, as it is \dash{one}{dimensional}.
    \item
      For the quiver
      \[
        Q :
        \begin{tikzcd}
            1
            \arrow[loop]{r}[above]{\alpha}
        \end{tikzcd}
      \]
      % TODO: Fix the vertical placement of Q in the above.
      its set of paths is given by~$Q_* = \{ \alpha^n \suchthat n \geq 0 \}$ with multiplication given by~$\alpha^n \cdot \alpha^m = \alpha^n \circ \alpha^m = \alpha^{n+m}$ for all~$n, m \geq 0$.
      It follows that~$\kf Q \cong \kf[t]$.
    \item
      For the quiver
      \[
        Q :
        \begin{tikzcd}
            1
            \arrow[shift left = 0.3em]{r}[above]{\alpha}
            \arrow[shift right = 0.3em]{r}[below]{\beta}
          & 2
        \end{tikzcd}
      \]
      its set of paths is given by~$Q_* = \{ \varepsilon_1, \varepsilon_2, \alpha, \beta \}$.
      Its path algebra is given by
      \[
          \kf Q
        = \kf \varepsilon_1 \oplus \kf \varepsilon_2 \oplus \kf \alpha \oplus \kf \beta \,,
      \]
      with multiplication being given on the basis elements given as follows:
      \[
        \begin{array}{r|cccc}
            \text{row $\cdot$ column}
          & \varepsilon_1
          & \varepsilon_2
          & \alpha
          & \beta
          \\
          \hline
            \varepsilon_1
          & \varepsilon_1
          & 0
          & 0
          & 0
          \\
            \varepsilon_2
          & 0
          & \varepsilon_2
          & \alpha
          & \beta
          \\
            \alpha
          & \alpha
          & 0
          & 0
          & 0
          \\
            \beta
          & \beta
          & 0
          & 0
          & 0
        \end{array}
      \]

  \end{enumerate}
\end{example}


\begin{lemma}
  Let~$\kf$ be a field and let~$A$ be a {\fd}~{\kalg}.
  Then there exists an injective homomorphism of~{\kalgs}~$\varphi \colon A \to \mat{n}{\kf}$ for~$n \defined \dim_\kf(A)$.
\end{lemma}


\begin{proof}
  Consider the~{\kalg}~$\End_\kf(A)$.
  It follows from choosing a \kdash{basis} of~$A$ that~$\End_\kf(A) \cong \mat{n}{\kf}$.
  It therefore sufficies to construct an injective homomorphism of~{\kalgs}~$\varphi \colon A \to \End_\kf(A)$.
  Let~$a \in A$.
  Then the map
  \[
            \varphi(a)
    \colon  A
    \to     A \,,
    \quad   b
    \mapsto ab
  \]
  is {\klin} by~\eqref{compatibility of multiplications}.
  The map~$\varphi \colon A \to \End_\kf(A)$ is the desired injective homomorphism of~{\kalgs}:
  \begin{itemize}
    \item
      The map~$\varphi$ is~{\klin} by~\eqref{compatibility of multiplications} and by the distributivity of the multiplication of~$A$.
    \item
      It holds for all~$a, a', b \in A$ that
      \[
          \varphi(a a')(b)
        = (a a') b
        = a (a' b)
        = \varphi(a)( \varphi(a')(b) )
        = (\varphi(a) \varphi(a'))(b) \,,
      \]
      which shows that~$\varphi(a a') = \varphi(a) \varphi(a')$, i.e.\ that~$\varphi$ is multiplicative.
    \item
      It holds that~$\varphi(1_A) = \id_A = 1_{\End_\kf(A)}$.
    \item
      It holds for~$a \in \ker(\varphi)$ that~$\varphi(a) = 0$ and therefore that
      \[
          0
        = \varphi(a)(1)
        = a \cdot 1
        = a \,,
      \]
      which shows that~$\varphi$ is injective.
  \end{itemize}
  This altogether shows claim.
\end{proof}


\boxline{End of lecture 1}


\begin{definition}
  Let~$A$ be a~{\kalg}.
  The \emph{opposite algebra}\index{opposite!algebra}\index{algebra!opposite}~$A^\op$ of~$A$ has the same underlying~{\kmod} as~$A$ but with multiplication given by
  \[
              a \cdot_{A^\op} b
    \defined  b \cdot_A a
  \]
  for all~$a, b \in A^\op$.
\end{definition}


\begin{example}
  Let~$Q$ be a quiver.
  The \emph{opposite quiver}\index{opposite!quiver}\index{quiver!opposite}~$Q^\op$ of~$Q$ results from~$Q$ by reversing the direction of all its arrows.
  More formally, if $Q = (Q_0, Q_1, s, t)$ then the opposite quiver~$Q^\op$ is given by~$Q^\op = (Q_0, Q_1, s^\op, t^\op)$ with
  \[
      s^\op(\alpha)
    = t(\alpha)
    \quad\text{and}\quad
      t^\op(\alpha)
    = s(\alpha) \,.
  \]
  This may also be expressed as~$(Q_0, Q_1, s, t)^\op = (Q_0, Q_1, t, s)$.
  It then holds that
  \[
    (\kf Q)^\op = \kf (Q^\op) \,.
  \]
\end{example}





\section{Modules}


\begin{definition}
  Let~$A$ be a~{\kalg}.
  A \emph{left~{\Amod}}~$M$\index{module}\index{left module|see {module}} is a~{\kmod}~$M$ together with a multiplication
  \[
            A \times M
    \to     M \,,
    \quad   (a,m)
    \mapsto am \,,
  \]
  such that the following conditions are satisfied:
  \begin{enumerate}[label={(L\arabic*})]
    \item
      $a(m_1 + m_2) = am_1 + am_2$,
    \item
      $(a_1 + a_2) m = a_1 m + a_2 m$,
    \item
      $a_1 (a_2 m) = (a_1 a_2) m$,
    \item
      $1 \cdot m = m$,
    \item
      $(\lambda a)m = \lambda (am) = a (\lambda m)$
  \end{enumerate}
  for all~$a, a_1, a_2 \in A$, all~$m, m_1, m_2 \in M$ and all~$\lambda \in \kf$
  That~$M$ is a left~{\Amod} is denoted by~$\lmod{A}{M}$.
  
  The notion of a \emph{right~{\Amod}}\index{right module|see{module}} is defined analogous.
  That~$M$ is a right~{\Amod} is denoted by~$\rmod{M}{A}$.
  Note that the axiom~(R3) for a right~{\Amod}~$M$ reads
  \[
      (m a_1) a_2
    = m (a_1 a_2)
  \]
  for all~$a_1, a_2 \in A$ and all~$m \in M$.
  Note also that a right~{\Amod} has the underlying structure of a right~{\kmod}, and that scalars~$\lambda \in k$ therefore also act from the right on~$M$.
\end{definition}


\begin{remark}
  For a~{\kmod}~$M$, the data of a right~{\Amod} structure on~$M$ is equivalent to that of a left~{\Aopmod} module structure on~$M$, i.e.\ right~{\Amods} are \enquote{the same} as left~{\Aopmods}.
\end{remark}


\begin{definition}
  Let~$A$ be a~{\kalg} and let~$M$ and~$N$ be two left~{\Amods}.
  A map~$f \colon M \to N$ is a \emph{homomorphism of~{\Amods}}\index{homomorphism!of modules}\index{module!homomorphism of} if is it~{\klin} and satisfies
  \[
      f(am)
    = a f(m)
  \]
  for all~$a \in A$ and all~$m \in M$.
  The set of~{\Amod} homomorphisms~$M \to N$ is denoted by~$\Hom_A(M,N)$.
  A homomorphism of~{\Amods}~$f \colon M \to N$ is an \emph{isomorphism}\index{isomorphism!of modules}\index{module!isomorphism of} if~$f$ is bijective.
  
  The notion of a homomorphism of right~{\Amods} and that of an isomorphim of right~{\Amods} is defined analogous.
\end{definition}


\begin{remark}
  Let~$A$ be a~{\kalg} and let~$M$ and~$N$ be two left~{\Amods}.
  \begin{enumerate}
    \item
      The set of homomorphisms~$\Hom_A(M,N)$ becomes a~{\kmod} when endowed with pointwise addition
      \[
                  (f + g)(m)
        \defined  f(m) + g(m)
      \]
      and pointwise scalar multiplication
      \[
                  (\lambda \cdot f)(m)
        \defined  \lambda \cdot f(m)
      \]
      for all~$f, g \in \Hom_A(M,N)$, all~$\lambda \in k$ and all~$m \in M$.
      That this scalar multiplication is {\welldef} follows from~$\lambda$ being central in~$A$.
    \item
      The homomorphism space~$\Hom_A(M,N)$ does in general carry neither the structure of a left~{\Amod} nor that of a right~{\Amod}.
    \item
      A map~$f \colon M \to N$ is an isomorphism of~{\Amods} if and only if there exists a homomorphism of left~{\Amods}~$g \colon N \to M$ with~$g \circ f = \id_M$ and~$f \circ g = \id_N$.
    \item
      Every homomorphism of left~{\Amods}~$f \colon M' \to M$ induces a~{\klin} map
      \[
                f^*
        \colon  \Hom_A(M, N)
        \to     \Hom_A(M', N) \,,
        \quad   h
        \mapsto h \circ f \,,
      \]
      and every homomorphism of left~{\Amods}~$g \colon N \to N'$ induces a~{\klin} map
      \[
                g_*
        \colon  \Hom_A(M, N)
        \to     \Hom_A(M, N') \,,
        \quad   g
        \mapsto g \circ h \,.
      \]
  \end{enumerate}
\end{remark}


\begin{remarkdefinition}
  Let~$A$ be a~{\kalg} and let~$M$,~$N$ be two left~{\Amods}.
  \begin{enumerate}
    \item
      A subset~$M' \subseteq M$ is a \emph{left~{\Asmod}}\index{submodule} of~$M$ if
      \begin{itemize}
        \item
          $0 \in M'$,
        \item
          $m'_1 + m'_2 \in M'$ for all~$m'_1, m'_2 \in M'$,
        \item
          $a m' \in M'$ for all~$a \in A$ and all~$m' \in M'$.
      \end{itemize}
    \item
      Let~$M' \subseteq M$ be a left~{\Asmod}.
      We can form the quotient~{\kmod}~$M/M'$, which becomes a left~{\Amod} via the scalar multiplication
      \[
          a \cdot (m + I)
        = (am) + I
      \]
      for all~$a \in A$ and all~$m + I \in M/M'$.
      The canonical projection
      \[
                \pi
        \colon  M
        \to     M/M' \,,
        \quad   m
        \mapsto m + I
      \]
      is a homomorphism of~{\Amods}.
    \item
      A \emph{left ideal}\index{ideal}\index{left ideal|see {ideal}} of~$A$ is a left~{\Asmod} of~$\lmod{A}{A}$.
      The notion a right ideal is defined analogous.
      
      Note that if~$I \subseteq A$ is a left ideal, then the quotient~$A/I$ does in general not inherit an~{\kalg} structure from~$A$.

      A subset~$I \subseteq A$ is a \emph{{\twosided} ideal}\index{ideal}\index{two-sided@{\twosided}} if it is both a left ideal and a right ideal.
      The quotient~$A/I$ then inherits from~$A$ the structure of a~{\kalg} with multiplication given by
      \[
                  (x + I) \cdot (y + I)
        \defined  xy + I
      \]
      for all~$x + I, y + I \in A/I$.
    \item
      Associated to every homomorphism of~{\Amods}~$f \colon M \to N$ are
      \begin{itemize}
        \item
          the \emph{kernel}\index{kernel}~$\ker(f) \defined \{m \in M \suchthat f(m) = 0\}$,
        \item
          the \emph{image}\index{image}~$\im(f) \defined \{f(m) \suchthat m \in M\}$,
        \item
          the \emph{cokernel}\index{cokernel}~$\coker(f) \defined N/\im(f)$,
        \item
          the \emph{coimage}\index{coimage}~$\coim(f) \defined M/\ker(f)$,
      \end{itemize}
      all of which are~{\Amods}.
      The homomorphism~$f$ factors uniquely as a composition of the canonical projection~$M \to \coim(f)$, followed by a homomorphism~$\tilde{f} \colon \coker(f) \to \im(f)$ and then by the inclusion~$\im(f) \to N$; and the induced homomorphism~$\tilde{f}$ is an isomorphism.
      This results in the following commutative diagram:
      \[
        \begin{tikzcd}
            M
            \arrow{r}[above]{f}
            \arrow[two heads]{d}
          & N
          \\
            \coim(f)
            \arrow{r}[above]{\exists!}[below]{\cong}
          & \im(f)
            \arrow[hook]{u}
        \end{tikzcd}
      \]
    \item
      Let~$M_i \subseteq M$ with~$i \in I$ be a family of left~{\Asmods}.
      Then the intersection~$\bigcap_{i \in I} M_i$ and the sum~$\sum_{i \in I} M_i$ are again~{\Asmods} of~$M$.
    \item
      For~$x \in M$ the subset~$Ax = \{ax \suchthat a \in A\}$ is the~\emph{{\Asmod} of~$M$ generated by~$x$}\index{generated submodule}\index{submodule!generated}.
      It is the smallest~{\Asmod} of~$M$ which contains~$x$.
      For any subset~$E \subseteq M$ the subset~$\sum_{x \in E} Ax$ of~$M$ is the \emph{{\Asmod} of~$M$ generated by~$E$}\index{generated submodule}\index{submodule!generated}.
      It is the smallest~{\Asmod} of~$M$ which contains~$E$, i.e.\ it holds that
      \[
          \sum_{x \in E} Ax
        = \bigcap_{\substack{\text{submodule $M' \subseteq M$} \\ \text{with $E \subseteq M'$}}} M' \,.
      \]
      The~{\Amod}~$M$ is \emph{finitely generated}\index{finitely generated}\index{module!finitely generated} if there exist finitely many~$x_1, \dotsc, x_n \in M$ with~$M = \sum_{i=1}^n A x_i$.
    \item
      Let~$M_i$ with~$i \in I$ be a family of~{\Amods}.
      Then the product~$\prod_{i \in I} M_i$ and the direct sum~$\bigdsum_{i \in I} M_i$ are again left~{\Amods}.
      For every~$j \in I$ both the projection
      \[
                \pi_j
        \colon  \prod_{i \in I} M_i
        \to     M_j \,,
        \quad   (x_i)_i
        \mapsto x_j
      \]
      and the inclusion
      \[
                \iota_j
        \colon  M_j
        \to     \bigdsum_{i \in I} M_i \,,
        \quad   x
        \mapsto ( \delta_{ij} x )_{i \in I}
      \]
      are homomorphism of left~{\Amods}.
    \item
      The left~{\Amod}~$M$ is finitely generated if and only if there exists for some~$n \in \Natural$ a surjective homomorphism of left~{\Amods}~$A^n \to M$, where~$A^n = \bigoplus_{i=1}^n A$.
      
      The left~{\Amod}~$M$ is \emph{finitely presented}\index{finitely presented}\index{presented, finitely} if there exists for some~$m, n \in \Natural$ an exact sequence of left~{\Amods}
      \[
            A^m
        \to A^n
        \to M
        \to 0 \,,
      \]
      i.e.~there exists a surjective homomorphism of~{\Amods}~$u \colon A^n \to M$ for which the kernel~$\ker(u)$ is again finitely generated.
      (Finitely presented~{\Amods} are in particular finitely generated.)
  \end{enumerate}
\end{remarkdefinition}


\begin{proposition}[Left exactness of~$\Hom$]
  \leavevmode
  \begin{enumerate}
    \item
      A sequence
      \[
        M_1
        \xlongrightarrow{f_1}
        M_2
        \xlongrightarrow{f_2}
        M_3
        \to 0
      \]
      of~{\Amods} is exact if and only if for every~{\Amod}~$N$ the induced sequence
      \[
        0
        \to
        \Hom_A(M_3, N)
        \xlongrightarrow{f_2^*}
        \Hom_A(M_2, N)
        \xlongrightarrow{f_1^*}
        \Hom_A(M_1, N)
      \]
      of~{\kmods} is exact.
    \item
      \label{left exactness of covariant hom}
      A sequence
      \[
        0
        \to
        N_1
        \xlongrightarrow{g_1}
        N_2
        \xlongrightarrow{g_2}
        N_3
      \]
      of~{\Amods} is exact if and only if for every~{\Amod}~$M$ the induced sequence
      \[
        0
        \to
        \Hom_A(M, N_1)
        \xlongrightarrow{(g_1)_*}
        \Hom_A(M, N_2)
        \xlongrightarrow{(g_2)_*}
        \Hom_A(M, N_3)
      \]
      of~{\kmods} is exact.
  \end{enumerate}
\end{proposition}


\begin{proof}
  We show only the implication \enquote{$\implies$} for part~\ref*{left exactness of covariant hom}, the rest will be done in the tutorials.
  
  To show that~$(g_1)_*$ is injective let~$h \in \Hom_A(M,N_1)$ with~$0 = (g_1)_*(h) = g_1 \circ h$.
  It then follows that~$\im(h) \subseteq \ker(g_1) = 0$ and therefore that~$h = 0$.
  
  The inclusion~$\im((g_1)_*) \subseteq \ker((g_2)_*)$ holds because
  \[
      (g_2)_* \circ (g_1)_*
    = (g_2 \circ g_1)_*
    = 0_*
    = 0 \,.
  \]
  
  To show the inclusion~$\ker((g_2)_*) \subseteq \im((g_1)_*)$ let~$h \in \ker(g_2)_*$.
  It follows from the injectivity of~$g_1$ that~$g_1 = \tilde{g}_1 \circ i$ for a unique isomorphism~$\tilde{g}_1 \colon N_1 \to \im(g_1)$ and the inclusion~$i \colon \im(g_1) \to N_2$.
  This results in the following commutative diagram:
  \[
    \begin{tikzcd}[row sep = large]
        0
        \arrow{rr}
      & {}
      & N_1
        \arrow{rr}[above]{g_1}
        \arrow{dr}[below left]{\exists! \tilde{g}_1}[above right]{\cong}
      & {}
      & N_2
        \arrow{rr}[above]{g_2}
      & {}
      & N_3
      \\
        {}
      & {}
      & {}
      & \im(g_1)
        \arrow[hook]{ur}[above left]{i}
        \arrow[equal]{rr}
      & {}
      & \ker(g_2)
        \arrow[hook']{ul}[above right]{i}
      & {}
    \end{tikzcd}
  \]
  It follows from~$0 = (g_2)_*(h) = g_2 \circ h$ that~$\im(h) \subseteq \ker(g_2)$ and therefore that~$h = i \circ \tilde{h}$ for a unique homomorphism~$\tilde{h} \colon M \to \ker(g_2)$.
  This results in the following commutative digram:
    \[
    \begin{tikzcd}[row sep = large]
        {}
      & {}
      & {}
      & {}
      & M
        \arrow[bend left]{rrd}{0}
        \arrow{d}{h}
        \arrow[bend left, dashed]{ddr}[near start, above right]{\exists! \tilde{h}}
      & {}
      & {}
      \\
        0
        \arrow{rr}
      & {}
      & N_1
        \arrow{rr}[above]{g_1}
        \arrow{dr}[below left]{\exists! \tilde{g}_1}[above right]{\cong}
      & {}
      & N_2
        \arrow{rr}[above]{g_2}
      & {}
      & N_3
      \\
        {}
      & {}
      & {}
      & \im(g_1)
        \arrow[hook]{ur}[above left]{i}
        \arrow[equal]{rr}
      & {}
      & \ker(g_2)
        \arrow[hook']{ul}[above right]{i}
      & {}
    \end{tikzcd}
  \]
  It follows for the homomorphism~$h' \defined \tilde{g}_1^{-1} \circ \tilde{h}$ that
  \[
      (g_1)_*(h')
    = g_1 \circ h'
    = i \circ \tilde{g}_1 \circ \tilde{g}_1^{-1} \circ \tilde{h}
    = i \circ \tilde{h}
    = h \,,
  \]
  which shows the claimed inclusion.
\end{proof}



\begin{proposition}
  Let~$A$ be a~{\kalg}.
  \begin{enumerate}
    \item
      To give a left~{\Amod} structure on a~{\kmod}~$V$ is equivalent to giving a homomorphism of~{\kalg}~$A \to \End_k(V)$.
    \item
      To give a right~{\Amod} structure on a~{\kmod}~$V$ is equivalent to giving a homomorphism of~{\kalg}~$A \to \End_k(V)^\op$.
  \end{enumerate}
\end{proposition}


\begin{proof}
  This will be an exercise on the second exercise sheet.
\end{proof}










\section{Representations of Quivers}


\begin{remarkdefinition}
  Let~$Q$ be a quiver.
  \begin{enumerate}
    \item
      A \emph{representation}~$X$\index{representation of a quiver}\index{quiver!representation} of~$Q$ consists of the following data:
      \begin{itemize}
        \item
          A~{\kmod}~$X_i$ for every vertex~$i \in Q_0$.
        \item
          For every arrow~$\alpha \in Q_1$ a~{\klin} map~$X_\alpha \colon X_{s(\alpha)} \to X_{t(\alpha)}$.
      \end{itemize}
  \end{enumerate}
  Let~$X$,~$Y$ and~$Z$ be representations of~$Q$.
  \begin{enumerate}[resume]
    \item
      A \emph{homomorphism}\index{homomorphism!of quiver representations}~$f \colon X \to Y$ is a tupel~$(f_i)_{i \in Q_0}$ of~{\klin} maps~$f_i \colon X_i \to Y_i$ such that the diagram
      \[
        \begin{tikzcd}
            X_{s(\alpha)}
            \arrow{r}[above]{X_\alpha}
            \arrow{d}[left]{f_{s(\alpha)}}
          & X_{t(\alpha)}
            \arrow{d}[right]{f_{t(\alpha)}}
          \\
            Y_{s(\alpha)}
            \arrow{r}[above]{Y_\alpha}
          & Y_{t(\alpha)}
        \end{tikzcd}
      \]
      commutes for every arrow~$\alpha \in Q_1$.
      
      If~$f \colon X \to Y$ and~$g \colon Y \to Z$ are homomorphisms of representations then the \emph{composition}~$g \circ f$ is the homomorphism~$X \to Z$ with components~$(g \circ f)_i = g_i \circ f_i$ for every~$i \in Q_0$.
      
      The \emph{identity homomorphism}\index{identity homomorphism}\index{homomorphism!identity} of the representation~$X$ is the homomorphism of representations~$\id_X \colon X \to X$ with~$(\id_X)_i = \id_{X_i}$ for every~$i \in Q_0$.
      It holds for every homomorphism of representations~$f \colon X \to Y$ that~${\id_Y} \circ f = f$ and~$f \circ {\id_X} = f$.
    \item
      A homomorphism~$f \colon X \to Y$ of representations is an~\emph{isomorphism}\index{isomorphism!of quiver representations} if there exists a homomorphism of representations~$g \colon Y \to X$ with~$g \circ f = \id_X$ and~$f \circ g = \id_Y$.
      The homomorphism~$f$ is an isomorphism if and only if the component~$f_i$ is an isomorphism for every~$i \in Q_0$.
  \end{enumerate}
\end{remarkdefinition}


\begin{example}
  \leavevmode
  \begin{enumerate}
    \item
      For the quiver~$Q = (\bullet)$ a representation of~$Q$ is the same a~{\kmod}~$V$.
      For two such representations~$V$ and~$W$, a homomorphism of representations~$V \to W$ is just a {\klin} map.
    \item
      For the quiver~$Q = (\begin{tikzcd} \bullet \arrow[loop right] \end{tikzcd})$ a representation of~$Q$ is the same as a pair~$(V,\varphi)$ consisting of a~{\kmod}~$V$ together with a~{\klin} endomorphism~$\varphi \colon V \to V$.
      
      Given two such representations~$(V, \varphi)$ and~$(W,\psi)$, a homomorphism of representations~$f \colon (V,\varphi) \to (W,\psi)$ is the same as a~{\klin} map~$f \colon V \to W$ with~$f \circ \varphi = \psi \circ f$.
    \item
      For the quiver~$Q = (\begin{tikzcd} 1 \arrow[shift left]{r}[above]{\alpha} \arrow[shift right]{r}[below]{\beta} & 2 \end{tikzcd})$ a representation of~$Q$ is the same as a quadruple~$(V_1,V_2,A_1,A_2)$ consisting of two~{\kmods}~$V_1$ and~$V_2$ and two~{\klin} maps~$A_1, A_2 \colon V_1 \to V_2$.
      
      Given two such representations~$(V_1, V_2, A_1, B_1)$ and~$(W_1, W_2, A_2, B_2)$, a homomorphism of representations~$f \colon (V_1, V_2, A_1, B_1) \to (W_1, W_2, A_2, B_2)$ is the same as a pair~$(f_1, f_2)$ of~{\klin} maps~$f_1 \colon V_1 \to W_1$ and~$f \colon V_2 \to W_2$ with~$f_2 A_1 = A_2 f_1$ and~$f_2 B_1 = B_2 f_1$.
  \end{enumerate}
\end{example}


\boxline{End of lecture 2}




