\section{Functors}


\begin{definition}
  Let~$\Ccat$ and~$\Dcat$ be categories.
  A \emph{functor}\index{functor}~$F \colon \Ccat \to \Dcat$ consists of the following data:
  \begin{itemize}
    \item
      A map (of classes)~$F \colon \Ob(\Ccat) \to \Ob(\Dcat)$, $X \mapsto F(X)$.
    \item
      For any two objects~$X$ and~$Y$ of $\Ccat$ a map
      \[
                \Ccat(X, Y)
        \to     \Dcat(F(X), F(Y)) \,,
        \quad   f
        \mapsto F(f) \,.
      \]
  \end{itemize}
  These data are subject to the following conditions:
  \begin{enumerate}[label=(F\arabic*)]
    \item
      It holds for every object~$X$ of~$\Ccat$ that~$F(\id_X) = \id_{F(X)}$.
    \item
      It holds for any two composable morphisms~$f \colon X \to Y$ and~$g \colon Y \to Z$ in~$\Ccat$ that
      \[
          F(g \circ f)
        = F(g) \circ F(f) \,.
      \]
  \end{enumerate}
\end{definition}


\begin{notation*}
  The application of a functor~$F \colon \Ccat \to \Dcat$ to an object~$X \in \Ob(\Ccat)$ or a morphisms~$f \colon X \to X'$ in~$\Ccat$ is often written without parentheses as~$FX$, resp.\ as~$Ff$.
\end{notation*}


\begin{remark}
  What we call a \enquote{functor} is sometimes called a \enquote{covariant functor}\index{functor!covariant}\index{covariant functor|see {functor}}.
  A \emph{contravariant functor}~$G \colon \Ccat \to \Dcat$\index{functor!contravariant}\index{contravariant functor|see {functor}} is a (covariant) functor~$G \colon \Ccat^\op \to \Dcat$.
  This means in terms of the category~$\Ccat$ that~$G$ assigns to every object~$X \in \Ob(\Ccat)$ an object~$G(X) \in \Dcat$ and to every morphism~$f \colon X \to Y$ in~$\Ccat$ a morphism~$G(f) \colon G(Y) \to G(X)$ in~$\Dcat$, in such a way that
  \begin{itemize}
    \item
      $G(\id_X) = \id_{G(X)}$ for every~$X \in \Ccat$, and
    \item
      $G(g \circ f) = G(f) \circ G(g)$ for every pair of composable morphisms~$f \colon X \to Y$ and~$g \colon Y \to Z$ in~$\Ccat$.
  \end{itemize}
\end{remark}


\begin{remark}
  \leavevmode
  \begin{enumerate}
    \item
      Let~$\Ccat$ be a category.
      The \emph{identity functor}~$\Id_{\Ccat} \colon \Ccat \to \Ccat$\index{identity!functor} is given by
      \[
                \Id_\Ccat
        \colon  \left\{
                  \begin{aligned}
                    X &\mapsto  X \,, \\
                    f &\mapsto  f \,.
                  \end{aligned}
                \right.
      \]
    \item
      If~$\Ccat$,~$\Dcat$ and~$\Ecat$ are categories and~$F \colon \Ccat \to \Dcat$ and~$G \colon \Dcat \to \Ecat$ are functors, then they can be composed to a functor~$G \circ F \colon \Ccat \to \Ecat$ given by
      \[
                (F \circ G)
        \colon  \left\{
                  \begin{aligned}
                    X &\mapsto  G(F(X)) \,, \\
                    f &\mapsto  G(F(f)) \,.
                  \end{aligned}
                \right.
      \]
  \end{enumerate}
\end{remark}


\begin{example}
  \label{examples for functors}
  \leavevmode
  \begin{enumerate}
    \item
      We can define two functors which assign to each set~$X$ its power set:
      
      We define a functor~$P_* \colon \Set \to \Set$ which assigns to each set~$X$ its power set~$\power(X)$, and to each map~$f \colon X \to Y$ the induced map
      \[
                P_*(f)
        \colon  \power(X)
        \to     \power(Y) \,,
        \quad   A
        \mapsto f(A) \,.
      \]
      We can also define a (contravariant) functor~$P^* \colon \Set^\op \to \Set$ which again assigns to each set~$X$ its power set~$\power(X)$, but to each map~$f \colon X \to Y$ the induced map
      \[
                P^*(f)
        \colon  \power(Y)
        \to     \power(X) \,,
        \quad   B
        \mapsto f^{-1}(B) \,.
      \]
    \item
      We have two functors between the categories~$\kAlg$ and~$\Group$:
      
      In the one direction there exists a functor~$(-)^\times \colon \kAlg \to \Group$ which assigns to each~{\kalg}~$A$ its group of units~$A^\times$ and to every homomorphism of~{\kalgs}~$f \colon A \to B$ its induces group homomorphism~$f^\times \colon A^\times \to B^\times$.
      
      In the other direction there exists a functor~$\kf[-] \colon \Group \to \kAlg$ which assigns to each group~$G$ its group algebra~$\kf[G]$, and to each group homomorphism~$\varphi \colon G \to H$ the induced homorphism of~{\kalgs}~$\kf[\varphi] \colon \kf[G] \to \kf[H]$, i.e.\ the unique homomorphism of~{\kalgs}~$\kf[G] \to \kf[H]$ which makes the square
      \[
        \begin{tikzcd}
            G
            \arrow{r}[above]{\varphi}
            \arrow[hook]{d}
          & H
            \arrow[hook]{d}
          \\
            \kf[G]
            \arrow{r}[above]{\kf[\varphi]}
          & \kf[H]
        \end{tikzcd}
      \]
      commute.
      (Recall from the first exercise sheet that for every~{\kalg}~$A$, every group homomorphism~$G \to A^\times$ extends uniquely to a homomorphism of~{\kalgs}~$\kf[G] \to A$.
      By applying this to the composition~$G \to H \inclusion k[H]^\times$ it follows that there exists a unique homomorphism~$\kf[G] \to \kf[H]$ which makes the above square commute.)
    \item
      The functor
      \[
                V
        \colon  \Group
        \to     \Set \,,
        \quad   \left\{
                  \begin{aligned}
                              G
                    &\mapsto  (\text{$G$ as a set}) \,,
                    \\
                              f
                    &\mapsto  f
                  \end{aligned}
                \right.
      \]
      is the \emph{forgetful functor}\index{forgetful functor}\index{functor!forgetful}.
      More generally, we call every functor \emph{forgetful} if it forgets part of the structure.
      We have for example forgetful functors~$\Top \to \Set$ and~$\Modl{A} \to \Modl{\kf}$, where~$A$ is a~{\kalg}.
  \end{enumerate}
\end{example}


\begin{example}
  Let~$\Ccat$ be a category.
  \begin{enumerate}
    \item
      Every object~$X \in \Ob(\Ccat)$ gives rise to a functor
      \[
                h^X
        \colon  \Ccat
        \to     \Set \,,
        \quad   \left\{
                  \begin{aligned}
                              Y
                    &\mapsto  \Ccat(X,Y) \,,
                    \\
                              \left( Y \xlongto{f} Y' \right)
                    &\mapsto  \left( \Ccat(X,Y) \xlongto{f_*} \Ccat(X,Y') \right) \,,
                  \end{aligned}
                \right.
      \]
      where the induced map~$f_* \colon \Ccat(X,Y) \to \Ccat(X,Y')$ is given by~$f_*(g) = f \circ g$ for every~$g \in \Ccat(X,Y)$.
      The functor~$h^X$ is also denoted by~$\Ccat(X,-)$.
    \item
      Every object~$X \in \Ob(\Ccat)$ gives rise to a (contravariant) functor
      \[
                h_X
        \colon  \Ccat^\op
        \to     \Set \,,
        \quad   \left\{
                  \begin{aligned}
                              Y
                    &\mapsto  \Ccat(Y,X) \,,
                    \\
                              \left( Y \xlongto{f} Y' \right)
                    &\mapsto  \left( \Ccat(Y',X) \xlongto{f^*} \Ccat(Y,X) \right) \,,
                  \end{aligned}
                \right.
      \]
      where the induced map~$f^* \colon \Ccat(Y',X) \to \Ccat(Y,X)$ is given by~$f^*(g) = g \circ f$ for every~$g \in \Ccat(Y',X)$.
      The functor~$h_X$ is also denoted by~$\Ccat(-,X)$.
  \end{enumerate}
\end{example}


\begin{warning*}
  Images of functors are not necessarily subcategories.
\end{warning*}


\begin{definition}[label=properties of functors]
  Let~$F \colon \Ccat \to \Dcat$ be a functor.
  \begin{enumerate}
    \item
      The functor~$F$ is \emph{faithful}\index{faithful}\index{functor!faithful} if the induced map~$\Ccat(X,Y) \to \Dcat(F(X),F(Y))$,~$f \mapsto F(f)$ is injective for all~$X, Y \in \Ccat$.
    \item
      The functor~$F$ is \emph{full}\index{full}\index{functor!full} if the induced map~$\Ccat(X,Y) \to \Dcat(F(X),F(Y))$,~$f \mapsto F(f)$ is surjective for all~$X, Y \in \Ccat$.
    \item
      The functor~$F$ is \emph{fully faithful}\index{fully faithful}\index{functor!fully faithful} if it is both full and faithful, i.e.\ if the induced map~$\Ccat(X,Y) \to \Dcat(F(X),F(Y))$,~$f \mapsto F(f)$ is injective for all~$X, Y \in \Ccat$.
  \end{enumerate}
\end{definition}


\begin{example*}
  For every subcategory~$\Scat$ of a category~$\Ccat$ there exists an inclusion functor~$I \colon \Scat \to \Ccat$ which is given on objects by the inclusion~$\Ob(\Scat) \inclusion \Ob(\Ccat)$ on morphisms by the inclusion~$\Scat(X,Y) \inclusion \Ccat(X,Y)$ for any two objects~$X, Y, \in \Ob(\Scat)$.
  This inclusion functor~$I$ is faithful, and it is full if and only if~$\Scat$ is full as a subcategory of~$\Ccat$.
\end{example*}




