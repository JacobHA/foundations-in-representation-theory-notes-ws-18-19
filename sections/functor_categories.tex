\section{Functor Categories}


\begin{definition}
  Let~$\Ccat$ and~$\Dcat$ be two categories.
  The \emph{functor category}\index{functor!category}\index{category!functor}~$\Fun(\Ccat, \Dcat)$ has as objects
  \[
              \Ob(\Fun(\Ccat, \Dcat))
    \defined  \{
                \text{functors~$F \colon \Ccat \to \Dcat$}
              \} \,,
  \]
  and for any two functors~$F, G \colon \Ccat \to \Dcat$ their morphism set~$\Fun(\Ccat, \Dcat)(F,G)$ is given by
  \[
              \Fun(\Ccat, \Dcat)(F,G)
    \defined  \{
                \text{natural transformations~$\eta \colon F \to G$}
              \} \,.
  \]
  The composition of morphisms in~$\Fun(\Ccat, \Dcat)$ is the composition of natural transformations.
\end{definition}


\begin{remark}
  We’re running into \dash{set}{theoretic} issues again:
  If~$\Ccat$ and~$\Dcat$ are two categories for our fixed universe~$U$ (i.e.\~$\Ob(\Ccat), \Ob(\Dcat) \subseteq U$) then~$\Ob(\Fun(\Ccat, \Dcat))$ might not be a subset of~$U$.
  The solution to this problem is to choose another, larger universe~$V$ with~$U \in V$.
  Then~$\Ob(\Fun(\Ccat, \Dcat)) \subseteq V$, so~$\Fun(\Ccat, \Dcat)$ becomes a category with respect to the universe~$V$.
\end{remark}


\begin{example}
  Let~$Q$ be a quiver and consider the category~$\Fun(\Path(Q),\Modl{\kf})$.
  
  Every functor~$V \in \Ob(\Fun(\Path(Q),\Modl{\kf}))$ gives rise to a representation~$F(V)$ of~$Q$ over~$\kf$ with
  \[
              F(V)_i
    \defined  V(i)
  \]
  for every~$i \in Q_0 = \Ob(\Path(Q))$ and
  \[
              F(V)_\alpha
    \defined  V(\alpha)
    \colon    F(V)_i
    \to       F(V)_j
  \]
  for every arrow~$\alpha$ from~$i$ to~$j$ in~$Q_0$ (and thus morphisms from~$i$ to~$j$ in~$\Path(Q)$).
  In this way we obtain a functor
  \[
            F
    \colon  \Fun(\Path(Q), \Modl{\kf})
    \to     \Rep{\kf}{Q}  \,.
  \]
  
  Conversely, let~$X$ be a representation of~$Q$ over~$\kf$.
  We can use~$X$ to define a functor~$G(X) \colon \Path(Q) \to \Modl{\kf}$ via
  \[
            G(X)
    \colon  \left\{
              \begin{aligned}
                          i
                &\mapsto  X_i \,, \\
                          (p = \alpha_\ell \dotsm \alpha_1)
                &\mapsto  X_{\alpha_\ell} \circ \dotsb \circ X_{\alpha_1} \,.
              \end{aligned}
            \right.
  \]
  This construction yields a functor
  \[
            G
    \colon  \Rep{\kf}{Q}
    \to     \Fun(\Path(Q), \Modl{\kf}) \,.
  \]
  It can now be checked that~$G \circ F = \Id_{\Fun(\Path(Q), \Modl{\kf})}$ and~$F \circ G = \Id_{\Rep{\kf}{Q}}$, which shows that~$\Rep{\kf}{Q} \simeq \Fun(\Path(Q), \Modl{\kf})$.
\end{example}


\begin{example*}
  \label{morphism category is a functor category}
  Let~$\Ccat$ be a category and let~$2$ denote the category that consists of two objects~$0$ and~$1$, and in which there exists precisely one \dash{non}{idenity} morphism, namely~$\alpha \colon 0 \to 1$.
  The category~$2$ may be visualized as follows:
  \[
    \begin{tikzcd}
        1
        \arrow{r}[above]{\alpha}
      & 2
    \end{tikzcd}
  \]
  
  A functor~$F \colon 2 \to \Ccat$ is the same as a choice of objects~$F(1), F(2) \in \Ob(\Ccat)$ together with a morphism~$F(\alpha) \colon F(1) \to F(2)$.
  Given two such functors~$F, G \colon 2 \to \Ccat$, a morphism~$\eta \colon F \to G$ consists of two morphisms~$\eta_1 \colon F(1) \to G(1)$ and~$\eta_2 \colon F(2) \to G(2)$ that make the square
  \[
    \begin{tikzcd}
        F(1)
        \arrow{r}[above]{\eta_1}
        \arrow[dashed]{d}[left]{F(\alpha)}
      & F(2)
        \arrow[dashed]{d}[right]{G(\alpha)}
      \\
        G(1)
        \arrow{r}[above]{\eta_2}
      & G(2)
    \end{tikzcd}
  \]
  commute.
  
  We see from this explicit description of the functor category~$\Fun(2, \Ccat)$ that it is equivalent to the morphism category~$\Mor(\Ccat)$.
\end{example*}


\begin{definition}
  Let~$\Ccat$ and~$\Dcat$ be two categories.
  For every object~$X \in \Ob(\Ccat)$ the \emph{evaluation}\index{evaluation}\index{functor!evaluation} at~$X$ is the functor~$\ev_X \colon \Fun(\Ccat, \Dcat) \to \Dcat$ given by
  \[
            \ev_X
    \colon  \left\{
              \begin{aligned}
                          F
                &\mapsto  F(X) \,,
                \\
                          (F \xto{\eta} G)
                &\mapsto  (\eta_X \colon F(X) \to G(X)) \,.
              \end{aligned}
            \right.
  \]
\end{definition}


\begin{remark}
  Let~$\Ccat$ and~$\Dcat$ be two categories.
  If~$f \colon X \to Y$ is a morphism in~$\Ccat$ then we get an induced natural transformation~$\ev_f \colon \ev_X \to \ev_Y$ given by
  \[
              (\ev_f)_F
    \defined  F(f)
    \colon    F(X)
    \to       F(Y)
  \]
  for every functor~$F \in \Ob(\Fun(\Ccat, \Dcat))$.
  This is indeed a natural transformation:
  Let~$F, G \in \Ob(\Fun(\Ccat, \Dcat))$ be functors and let~$\eta \colon F \to G$ be a natural transformation between them.
  We then have the following diagram:
  \[
    \begin{tikzcd}
        \ev_X(F)
        \arrow{rrr}[above]{\ev_X(\eta)}
        \arrow{ddd}[left]{(\ev_f)_F}
        \arrow[equal]{dr}
      & {}
      & {}
      & \ev_X(G)
        \arrow{ddd}[right]{(\ev_f)_G}
        \arrow[equal]{dl}
      \\
        {}
      & F(X)
        \arrow{r}[above]{\eta_X}
        \arrow{d}[left]{F(f)}
      & G(X)
        \arrow{d}[right]{G(f)}
      & {}
      \\
        {}
      & F(Y)
        \arrow{r}[below]{\eta_Y}
      & G(Y)
      & {}
      \\
        \ev_Y(F)
        \arrow{rrr}[below]{\ev_Y(\eta)}
        \arrow[equal]{ur}
      & {}
      & {}
      & \ev_Y(G)
        \arrow[equal]{ul}
    \end{tikzcd}
  \]
  The inner square commutes because~$\eta$ is a natural transformation, and so it follows that the outer square commutes.
  
  It also holds that~$\ev_{\id_X} = \id_{\ev_X}$ and that~$\ev_{f \circ g} = \ev_f \circ \ev_g$ for every two composable morphisms~$f \colon X \to Y$ and~$g \colon Y \to Z$ in~$\Ccat$.
  We hence obtain a functor
  \[
            \ev
    \colon  \Ccat
    \to     \Fun( \Fun(\Ccat, \Dcat), \Dcat ) \,.
  \]

\end{remark}




