\section{Additive Categories}


\begin{definition}
  A \emph{preadditive~category}\index{pre-additive category}\index{category!pre-additive} is a category~$\Acat$ together with the structure of an abelian group on~$\Acat(X,Y)$ for all~$X, Y \in \Ob(\Acat)$ such that the composition in~$\Acat$ is~{\Zbilin}, i.e.\ such that
  \[
    k \circ (g + h) = k \circ g + k \circ h
    \quad\text{and}\quad
    (g + h) \circ f = g \circ f + h \circ f
  \]
  for all morphisms~$f \colon W \to X$,~$g, h \colon X \to Y$ and~$k \colon Y \to Z$ in~$\Acat$.
\end{definition}


\begin{remark}
  \leavevmode
  \begin{enumerate}
    \item
      Preddative categories are also known as~\emph{\dash{$\Ab$}{categories}}\index{Ab-category@$\Ab$-category}\index{category!Ab-@$\Ab$-} (where~$\Ab$ denotes the category of abelian groups).
      ()
    \item
      If~$\kf$ is a commutative ring then one can similarly define the notion of a \emph{{\preklin} category}\index{pre-k-linear category@{\preklin} category}\index{category!pre-k-linear@{\preklin}} (also known as~\emph{\dash{$\Modl{\kf}$}{category}}\index{k-Mod-category@$\Modl{\kf}$-category}\index{category!k-Mod-@$\Modl{\kf}$-})~$\Ccat$.
      Every~$\Ccat(X,Y)$ is then endowed with the structure of a~{\module{$\kf$}} and the composition is~{\kbilin}.
  \end{enumerate}
\end{remark}


\begin{remark*}
  In the language of enriched category theory, an~\dash{$\Ab$}{category} is precisely a category that is enriched over the monoidal category~$(\Ab,{\tensor_\Integer})$, and a \dash{$\Modl{\kf}$}{category} is precisely a category that is enriched over the monoidal category~$(\Modl{\kf}, {\tensor_\kf})$.
\end{remark*}


\begin{example}
  \leavevmode
  \begin{enumerate}
    \item
      The category~$\Ab = \Modl{\Integer}$ is preadditive.
    \item
      If~$A$ is a~{\kalg} then the categories~$\Modl{A}$ and~$\Modr{A}$ are {\preklin}.
    \item
      If~$R$ is a ring then we may think about~$R$ as a preadditive category~$\Rcat$ consisting of a single object~$\Ob(\Rcat) = \{ \ast \}$ with~$\Rcat(\ast,\ast) = R$.
      The composition in~$\Rcat$ is given by the multiplication of~$R$, i.e.\ by~$g \circ f = gf$ for all~$f, g \in R$, and the addition of morphisms is the addition in~$R$.
      
      One can similarly regard every~{\kalg}~$A$ as a {\preklin} category~$\Acat$ that consists of a single object~$\Ob(\Acat) = \{\ast\}$ with~$\Acat(\ast,\ast) = A$.
    \item
      Let~$\Ccat$ be any category and let~$\Acat$ be a preadditive category.
      Then the functor category~$\Fun(\Ccat, \Acat)$ is again preadditive:
      For any two natural transformations~$\eta, \zeta \colon F \to G$ between functors~$F, G \in \Ob(\Fun(\Ccat, \Acat))$, their sum~$\eta + \zeta$ is at an object~$X \in \Ob(\Ccat)$ given by
      \[
          (\eta + \zeta)_X
        = \eta_X + \zeta_X \,.
      \]
      This defines again a natural transformation~$\eta + \zeta \colon F \to G$.
      Indeed, for every morphism~$f \colon X \to X'$ in~$\Ccat$ the square
      \[
        \begin{tikzcd}[sep = large]
            F(X)
            \arrow{r}[above]{F(f)}
            \arrow{d}[left]{\eta_X + \zeta_X}
          & F(X')
            \arrow{d}[right]{\eta_{X'} + \zeta_{X'}}
          \\
            G(X)
            \arrow{r}[above]{G(f)}
          & G(X')
        \end{tikzcd}
      \]
      commutes because
      \begin{align*}
            (\eta_{X'} \circ \zeta_{X'}) \circ F(f)
        &=  \eta_{X'} \circ F(f) + \zeta_{X'} \circ F(f)  \\
        &=  G(f) \circ \eta_X + G(f) \circ \zeta_X
         =  G(f) \circ (\eta_X + \zeta_X) \,.
      \end{align*}
      
      We find similarly that for every category~$\Ccat$ and every {\preklin} category~$\Acat$ the functor category~$\Fun(\Ccat, \Acat)$ is again~{\preklin}. 
  \end{enumerate}
\end{example}


\begin{remark*}
  \leavevmode
  \begin{enumerate}
    \item
      A preadditive category is the same as a pre\nobreakdash-$\Integer$\nobreakdash-linear category.
    \item
      If~$\Acat$ is a preadditive (resp.\ {\preklin}) category, then the opposite category~$\Acat^\op$ is again preadditive (resp.\ {\preklin}) with the same addition (resp.~{\module{$\kf$}} structure) of morphisms.
  \end{enumerate}
\end{remark*}


\begin{definition}
  Let~$F \colon \Acat \to \Bcat$ be a functor between categories~$\Acat$ and~$\Bcat$.
  \begin{enumerate}
    \item
      If~$\Acat$ and~$\Bcat$ are preaddive categories then the functor~$F$ is \emph{additive}\index{additive!functor}\index{functor!additive} if
      \[
          F(f + g)
        = F(f) + F(g)
      \]
      for all morphisms~$f, g \colon X \to Y$ in~$\Acat$, i.e.\ if the map
      \[
                    \Acat(X, Y)
        \xlongto{F} \Bcat(F(X), F(Y))
      \]
      is a group homorphism for all~$X, Y \in \Ob(\Acat)$.
    \item
      If~$\Acat$ and~$\Bcat$ are {\preklin} categories then the functor~$F$ is~\emph{{\klin}}\index{k-linear@$\kf$-linear!functor}\index{functor!k-linear@$\kf$-linear} if
      \[
        F(f + g) = F(f) + F(g)
        \quad\text{and}\quad
        F(\lambda f) = \lambda F(f)
      \]
      for all morphisms~$f, g \colon X \to Y$ in~$\Acat$ and scalars~$\lambda \in \kf$, i.e.\ if the map
      \[
                    \Acat(X, Y)
        \xlongto{F} \Bcat(F(X), F(Y))
      \]
      is~{\klin} for all~$X, Y \in \Ob(\Acat)$.%
      \footnote{The notion of a~{\klin} functor was not introduced in the lecture.}
  \end{enumerate}
\end{definition}


\begin{lemma}
  \label{inital terminal zero}
  Let~$\Acat$ be a preadditive category.
  \begin{enumerate}
    \item
      For any object~$X \in \Ob(\Acat)$ the following conditions are equivalent:
      \begin{enumerate}
        \item
          The object~$X$ is inital in~$\Acat$.
        \item
          The object~$X$ is terminal in~$\Acat$.
        \item
          The object~$X$ is a zero object for~$\Acat$.
        \item
          It holds that~$\id_X = 0_{\Acat(X,X)}$.
        \item
          The abelian group~$\Acat(X,X)$ consists of only a single element.
      \end{enumerate}
    \item
      Suppose that the category~$\Acat~$ has a zero object.
      Then it holds for any two objects~$X, Y \in \Ob(\Acat)$ that~$0_{X,Y} = 0_{\Acat(X,Y)}$.
  \end{enumerate}
\end{lemma}


\begin{proof}
  This is Exercise~3 on Exercise~sheet~5.
\end{proof}





\lecturend{9}


\begin{definition}
  Let~$\Acat$ be a preadditive category and let~$X_1, \dotsc, X_n \in \Ob(\Acat)$ be objects, where~$n \in \Integer_{\geq 0}$.
  A \emph{biproduct}\index{biproduct} of~$X_1, \dotsc, X_n$ is a triple~$(X, (p_1, \dotsc, p_n), (c_1, \dotsc, c_n))$ consisting of an object~$X \in \Ob(\Acat)$ together with morphisms~$p_i \colon X \to X_i$ and morphisms~$c_i \colon X_i \to X$ in~$\Acat$, such that
  \begin{equation}
    \label{no abuse of notation}
      p_j c_i
    = \begin{cases}
        \id_{X_i}     & \text{if~$i = j$}     \,, \\
        0_{X_i, X_j}  & \text{if~$i \neq j$}  \,,
      \end{cases}
  \end{equation}
  for all~$i,j = 1, \dots, n$, and
  \[
      \sum_{i=1}^n c_i p_i
    = {\id_X} \,.
  \]
% Add notation.
\end{definition}


\begin{remark*}
  In the lecture, the formula~\ref{no abuse of notation} was instead written as
  \[
      p_j c_i
    = \delta_{ij} \id_{X_i} \,.
  \]
  This is an abuse of notation:
  For~$j \neq i$ the composition~$p_j c_i$ is a morphism~$X_i \to X_j$, whereas~$\delta_{ij} \id_{X_i} = 0 \cdot \id_{X_i} = 0_{X_i, X_i}$ is the zero morphism~$X_i \to X_i$.
  The author tries to avoid this abuse of notation, but will still sometimes write~$p_j c_i = \delta_{ij}$ as an abbreviation for~\eqref{no abuse of notation}.
\end{remark*}


\begin{remarknonum}
  For a preadditive category~$\Acat$, a biproduct of an empty family of objects in~$\Acat$ is the same as a zero object of~$\Acat$.
\end{remarknonum}


\begin{lemma}
  \label{product coproduct biproduct}
  Let~$\Acat$ be a preadditive category, let~$X_1, \dotsc, X_n \in \Acat$ where~$n \in \Integer_{\geq 0}$.
  \begin{enumerate}
    \item
      If~$(X, (p_1, \dotsc, p_n), (c_1, \dotsc, c_n))$ is a biproduct of~$X_1, \dotsc, X_n$ then~$(X, (p_1, \dotsc, p_n))$ is a product of~$X_1, \dotsc, X_n$ and~$(X, (c_1, \dotsc, c_n))$ is a coproduct of~$X_1, \dotsc, X_n$.
    \item
      \label{products into biproducts}
      Suppose that~$(X, (p_1, \dotsc, p_n))$ is a product of~$X_1, \dotsc, X_n$.
      Then there exist for every~$i = 1, \dotsc, n$ a unique morphism~$c_i \colon X_i \to X$ with~$p_j c_i = \delta_{ij} \id_{X_i}$ for every~$j = 1, \dotsc, n$.
      The triple~$(X, (p_1, \dotsc, p_n), (c_1, \dotsc, c_n))$ is then a biproduct of~$X_1, \dotsc, X_n$.
    \item
      Dually, suppose that~$(X, (c_1, \dotsc, c_n))$ is a coproduct of~$X_1, \dotsc, X_n$.
      Then there exist for every~$i = 1, \dotsc, n$ a unique morphism~$p_i \colon X \to X_i$ with~$p_i c_j = \delta_{ij} \id_{X_i}$ for every~$j = 1, \dotsc, n$.
      The triple~$(X, (p_1, \dotsc, p_n), (c_1, \dotsc, c_n))$ is then a biproduct of~$X_1, \dotsc, X_n$.
  \end{enumerate}
\end{lemma}


\begin{proof}
  For security reasons we consider the case~$n = 0$ separately:
  The product over the empty family is a final object of~$\Acat$, the coproduct over the empty family is an initial object of~$\Acat$, and the biproduct over the empty family is a zero object of~$\Acat$.
  The statements therefore follow for~$n = 0$ from \cref{inital terminal zero}.
  In the following we consider the case~$n \geq 1$.
  \begin{enumerate}
    \item
      It sufficies by duality to show that~$(X, (c_i)_i)$ is a coproduct for~$X_1, \dotsc, X_n$.
      Let~$(D, (d_i)_i)$ be another pair consisting of an object~$D \in \Ob(\Acat)$ and morphisms~$d_i \colon X_i \to D$.
      We need to show that there exists a unique morphism~$\mu \colon X \to D$ that makes the triangle
      \[
        \begin{tikzcd}[sep = large]
            X_i
            \arrow{r}[above]{c_i}
            \arrow{dr}[below left]{d_i}
          & X
            \arrow[dashed]{d}[right]{\mu}
          \\
            {}
          & D
        \end{tikzcd}
      \]
      commute for every~$i = 1, \dotsc, n$. 
      If such a morphism~$\mu$ exists then
      \[
          \mu
        = \mu \id_X
        = \mu \sum_{i=1}^n c_i p_i
        = \sum_{i=1}^n \mu c_i p_i
        = \sum_{i=1}^n d_i p_i \,,
      \]
      which shows that~$\mu$ is unique.
      If we define on the other hand~$\mu \defined \sum_{i=1}^n d_i p_i$ then
      \[
          \mu c_i
        = \sum_{j=1}^n d_j \underbrace{p_j c_i}_{= \delta_{ij}}
        = d_i \,,
      \]
      which shows the existence of~$\mu$.
    \item
      By the universal property of the product there exists for every~$i = 1, \dotsc, n$ a unique morphism~$c_i \colon X_i \to X$ that makes for all~$j \neq i$ the triangles
      \[
        \begin{tikzcd}[sep = large]
            X_i
            \arrow{dr}[above right]{\id_{X_i}}
            \arrow[dashed]{d}[left]{c_i}
          & {}
          \\
            X
            \arrow{r}[below]{p_i}
          & X_i
        \end{tikzcd}
        \qquad\text{and}\qquad
        \begin{tikzcd}[sep = large]
            X_i
            \arrow{dr}[above right]{0}
            \arrow[dashed]{d}[left]{c_i}
          & {}
          \\
            X
            \arrow{r}[below]{p_j}
          & X_j
        \end{tikzcd}
      \]
      commute.
      This means that~$p_j c_i = \delta_{ij}$ for all~$i, j = 1, \dotsc, n$.
      
      We now show that~$\sum_{i=1}^n c_i p_i = \id_X$.
      Indeed, we find for every~$j = 1, \dotsc, n$ that
      \[
          p_j \circ \sum_{i=1}^n c_i p_i
        = \sum_{i=1}^n \underbrace{ p_j c_i }_{= \delta_{ij}} p_i
        = p_j \,.
      \]
      That shows that for every~$j = 1, \dotsc, n$ the triangle
      \[
        \begin{tikzcd}
            X
            \arrow{dr}[below left]{p_j}
            \arrow[dashed]{rr}[above]{\sum_{i=1}^n c_i p_i}
          & {}
          & X
            \arrow{dl}[below right]{p_j}
          \\
            {}
          & X_j
          &
        \end{tikzcd}
      \]
      commutes.
      But it follows from the uniqueness of products up to unique isomorphism that there exist a \emph{unique} morphism~$X \to X$ that makes this triangle commute.
      The identity~$\id_X \colon X \to X$ also makes the above triangle commute, and so it follows that~$\sum_{i=1}^n c_i p_i = \id_X$.
    \item
      This can be shown dually to part~\ref*{products into biproducts}.
    \qedhere
  \end{enumerate}
\end{proof}


\begin{remark}
  It follows from \cref{product coproduct biproduct} that for a preadditive category~$\Acat$ the following are equivalent:
  \begin{enumerate}
    \item
      $\Acat$ has finite products.
    \item
      $\Acat$ has finite coproducts.
    \item
      $\Acat$ has finite biproducts.
  \end{enumerate}
\end{remark}


\begin{definition}
  A preadditve (or {\preklin}) category~$\Acat$ is \emph{additive}\index{additive!category}\index{category!additive} (resp.~{\klin}\index{k-linear@$\kf$-linear!category}\index{category!k-linear@$\kf$-linear}) if it has finite biproducts (and thus equivalently finite products, and equivalently finite coproducts).
\end{definition}


\begin{remarknonum}
  Additive (and~{\klin}) categories have zero objects, as these are the biproducts of empty family of objects.
\end{remarknonum}


\begin{remark*}
  A category~$\Acat$ is additive (resp.~{\klin}) if and only if its dual category~$\Acat^\op$ is additive (resp.\ {\klin}).
\end{remark*}


\begin{remark*}
  In a preadditive category~$\Acat$ one can express morphisms between biproducts as matrices:
  Let~$X_1, \dotsc, X_n$ and~$Y_1, \dotsc, Y_m$ be two families of objects in~$\Acat$ whose biproducts~$X_1 \oplus \dotsb \oplus X_n$ and~$Y_1 \oplus \dotsb \oplus Y_m$ exist, and denote the associated morphisms by
  \begin{align*}
    c_i \colon X_i \to X_1 \oplus \dotsb \oplus X_n
    \quad&\text{and}\quad
    p_i \colon X_1 \oplus \dotsb \oplus X_n \to X_i \,,
  \shortintertext{and}
    d_i \colon Y_i \to Y_1 \oplus \dotsb \oplus Y_m
    \quad&\text{and}\quad
    q_i \colon Y_1 \oplus \dotsb \oplus Y_m \to Y_i \,.
  \end{align*}
  
  \begin{enumerate}
    \item
      Suppose first that we are given a morphism
      \[
                f
        \colon  X_1 \oplus \dotsb \oplus X_n
        \to     Y_1 \oplus \dotsb \oplus Y_m
      \]
      in~$\Acat$.
      It then follows from the calculation
      \begin{align*}
            f
        &=  {\id_{Y_1 \oplus \dotsb \oplus Y_m}} \circ f \circ {\id_{X_1 \oplus \dotsb \oplus X_n}} \\
        &=  \left( \sum_{i=1}^n d_i q_i \right) \circ f \circ \left( \sum_{j=1}^m c_j p_j \right)
        =  \sum_{i=1}^n \sum_{j=1}^m d_i (q_i \circ f \circ c_j) p_j \,.
      \end{align*}
      that the morphism~$f$ is unique determined by the compositions~$q_i \circ f \circ c_j$.
      We will refer to the composition
      \[
                  [f]_{ij}
        \defined  q_i \circ f \circ c_j
      \]
      as the~\dash{$(i,j)$}{th} component of~$f$.
      The above calculation shows that the morphism~$f$ can be retrieved from its components via the formula
      \[
        f = \sum_{i=1}^n \sum_{j=1}^m d_i [f]_{ij} p_j \,.
      \]
      To better visualize the relation between~$f$ and its components, we may arrange these components in the form of an~\dash{$(m \times n)$}{matrix}
      \[
        \begin{bmatrix}
          f_{11}  & \cdots  & f_{1n}  \\
          \vdots  & \ddots  & \vdots  \\
          f_{m1}  & \cdots  & f_{mn}
        \end{bmatrix} \,.
      \]
      This is the \emph{representing matrix of~$f$} and is denoted by~$[f]$.
      (Note that the~\dash{$(i,j)$}{th} entry of the matrix~$[f]$ is precisely~$[f]_{ij}$.)
    \item
      Let on the other hand~$g_{ij} \colon X_j \to Y_i$ for~$i = 1, \dotsc, m$ and~$j = 1, \dotsc, n$ be a collection of morphisms.
      We can then define a morphism~$g \colon X \to Y$ via
      \[
                  g
        \defined  \sum_{i=1}^m \sum_{j=1}^n d_i g_{ij} p_j \,.
      \]
      The components~$[g]_{ij}$ of the morphism~$g$ are for all~$i = 1, \dotsc, m$ and~$j = 1, \dotsc, n$ given by
      \begin{align*}
            [g]_{ij}
        =  q_i \circ g \circ c_j
        &=  q_i \circ \left( \sum_{i'=1}^m \sum_{j'=1}^n d_{i'} g_{i'j'} p_{j'} \right) \circ c_j \\
        &=  \sum_{i=1}^m \sum_{j=1}^n
            \underbrace{q_i d_{i'}}_{= \delta_{i,i'}} g_{i'j'} \underbrace{p_{j'} c_j}_{= \delta_{j',j}}
        =  g_{ij} \,.
      \end{align*}
      The components~$[g]_{ij}$ of~$g$ are hence the morphisms~$g_{ij}$ that we started with.
    \item
      This shows overall that we have constructed a bijection
      \begin{align*}
          \Acat(X_1 \oplus \dotsb \oplus X_n, Y_1 \oplus \dotsb \oplus Y_m)
        &\longleftrightarrow
          \left\{
            \begin{bsmallmatrix}
              g_{11}  & \cdots  & g_{1n}  \\
              \vdots  & \ddots  & \vdots  \\
              g_{m1}  & \cdots  & g_{mn}
            \end{bsmallmatrix}
          \suchthat*
            g_{ij} \in \Acat(X_j, Y_i)
          \right\}  \,,
        \\
          f
        &\longmapsto
          [f] \,,
        \\
          \sum_{i=1}^m \sum_{j=1}^n q_i g_{ij} c_j
        =
          g
        &\longmapsfrom
          \begin{bsmallmatrix}
              g_{11}  & \cdots  & g_{1n}  \\
              \vdots  & \ddots  & \vdots  \\
              g_{1n}  & \cdots  & g_{mn}
            \end{bsmallmatrix}  \,.
      \end{align*}
      This way of representing morphisms between biproducts as matrices is compatible with both sums and composition of morphisms, and if~$\Acat$ is~{\preklin} then also with scalar multiplication of morphisms:
      \begin{itemize}
        \item
          Let~$f_1, f_2 \colon X_1 \oplus \dotsb \oplus X_n \to Y_1 \oplus \dotsb \oplus Y_m$ be two parallel morphisms in~$\Acat$.
          Then the morphism~$f_1 + f_2$ has for all~$i = 1, \dotsc, m$ and~$j = 1, \dotsc, n$ the components
          \[
              [f_1 + f_2]_{ij}
            = q_i \circ (f_1 + f_2) \circ c_j
            = q_i \circ f_1 \circ c_j + q_i \circ f_2 \circ c_j
            = [f_1]_{ij} + [f_2]_{ij} \,.
          \]
          This shows that indeed
          \[
              [f_1 + f_2]
            = [f_1] + [f_2] \,.
          \]
        \item
          Let~$Z_1, \dotsc, Z_l$ be objects in~$\Acat$ whose biproduct~$Z_1 \oplus \dotsb \oplus Z_l$ exists in~$\Acat$, and let
          \[
            e_i \colon Z_i \to Z_1 \oplus \dotsb \oplus Z_l
          \quad\text{and}\quad
            r_i \colon Z_1 \oplus \dotsb \oplus Z_l \to Z_i \,,
          \]
          be the associated morphisms.
          It then holds for any two composable morphisms
          \[
              X_1 \oplus \dotsb \oplus X_n
            \xlongto{f}
              Y_1 \oplus \dotsb \oplus Y_m
            \xlongto{g}
              Z_1 \oplus \dotsb \oplus Z_l
          \]
          in~$\Acat$ that
          \[
              [g \circ f]
            = [g] \cdot [f]
          \]
          where the product on the right hand side is taken in the naive way.
          Indeed the composition~$g \circ f$ has for all~$i = 1, \dotsc, l$ and~$k = 1, \dotsc, n$ the components
          \begin{align*}
                [g \circ f]_{ik}
            &=  r_i \circ (g \circ f) \circ c_k
            =  r_i \circ g \circ \id_Y \circ f \circ c_k \\
            &=  r_i \circ g \circ \left( \sum_{j=1}^m d_j q_j \right) \circ f \circ c_k \\
            &=  \sum_{j=1}^m (r_i \circ g \circ d_j) \circ (q_j \circ f \circ c_k)
            =  \sum_{j=1}^m [g]_{ij} [f]_{jk} \,.
          \end{align*}
          The resulting term~$\sum_{j=1}^m [g]_{ij} [f]_{jk}$ is precisely the~\dash{$(i,k)$}{th} entry of the matrix product~$[g] \cdot [f]$.
        \item
          If~$\Acat$ also~{\preklin} then let~$f \colon X_1 \oplus \dotsb \oplus X_n \to Y_1 \oplus \dotsb \oplus Y_m$ be a morphism in~$\Acat$ and let~$\lambda \in \kf$ be a scalar.
          Then the morphism~$\lambda f$ has for all~$i = 1, \dotsc, m$ and~$j = 1, \dotsc, n$ the components
          \[
              [\lambda f]_{ij}
            = q_i \circ (\lambda f) \circ c_j
            = \lambda (q_i \circ f \circ c_j)
            = \lambda [f]_{ij} \,.
          \]
          This shows that indeed
          \[
              [\lambda f]
            = \lambda [f] \,.
          \]
      \end{itemize}
    \item
      In the following we will notationally often not distinguish between the morphism~$f \colon X_1 \oplus \dotsb \oplus X_n \to Y_1 \oplus \dotsb \oplus Y_m$ and its matrix representation.
      So instead of
      \[
          [f]
        = \begin{bmatrix}
            f_{11}  & \cdots  & f_{1n}  \\
            \vdots  & \ddots  & \vdots  \\
            f_{m1}  & \cdots  & f_{mn}
          \end{bmatrix}
      \]
      (where~$f_{ij} = [f]_{ij}$ is the~\dash{$(i,j)$}{th} component of~$f$) we will just write
      \[
          f
        = \begin{bmatrix}
            f_{11}  & \cdots  & f_{1n}  \\
            \vdots  & \ddots  & \vdots  \\
            f_{m1}  & \cdots  & f_{mn}
          \end{bmatrix} \,.
      \]
      If one of the morphisms~$f_{ij}$ is the identity~$\id_Z$ of some object~$Z$ (that is then necessarily given by~$X_j = Z = Y_i$) then we will often just write the corresponding matrix entry as~$1$ instead of~$\id_Z$.
    \item
      We finish this remark by pointing out that the morphisms~$c_i \colon X_i \to X_1 \oplus \dotsb \oplus X_n$ and~$p_i \colon X_1 \oplus \dotsb \oplus X_n \to X_i$ are by these notional conventions given by the matrices
      \[
          c_i
        = \begin{bsmallmatrix}
            {} \\0 \\ \vdots \\ 0 \\ 1 \\ 0 \\ \vdots \\ 0 \\ {}
          \end{bsmallmatrix}
        \qquad\text{and}\qquad
          p_i
        = \begin{bsmallmatrix}
            0 & \cdots & 0 & 1 & 0 & \cdots & 0
          \end{bsmallmatrix} \,.
      \]
  \end{enumerate}
\end{remark*}


\begin{remark}
  \label{sum via category structure}
  Let$~\Acat$ be an additive category.
  For any objects~$X \in \Ob(\Acat)$ we can define the \emph{diagonal \textup(morphism\textup)}\index{diagonal morphism}
  \[
            \diag_X
    \colon  X
    \to     X \oplus X \,,
  \]
  by using the universal property of the product for~$X \oplus X$, as the unique morphism~$X \to X \oplus X$ that makes the diagram
  \[
    \begin{tikzcd}[sep = large]
        {}
      & X
        \arrow[dashed]{d}[right]{\diag_X}
        \arrow{dl}[above left]{\id_X}
        \arrow{dr}[above right]{\id_X}
      & {}
      \\
        X
      & X \oplus X
        \arrow{l}[below]{p_1}
        \arrow{r}[below]{p_2}
      & X
    \end{tikzcd}
  \]
  commute.
  This means that
  \[
    p_1 \circ \diag_X = \id_X
    \quad\text{and}\quad
    p_2 \circ \diag_X = \id_X \,,
  \]
  so the morphism~$\diag_X$ can be written is matrix form as
  \[
      \diag_X
    = \begin{bmatrix}
        1 \\ 1
      \end{bmatrix} \,.
  \]
  We can dually define the \emph{codiagonal \textup(morphism\textup)}\index{codiagonal morphism}
  \[
            \codiag_X
    \colon  X \oplus X
    \to     X
  \]
  by using the universal property of the coproduct for~$X \oplus X$, as the unique morphism~$X \oplus X \to X$ that makes the diagram
  \[
    \begin{tikzcd}[sep = large]
        X
        \arrow{dr}[below left]{\id_X}
      & X \oplus X
        \arrow[dashed]{d}[right]{\codiag_X}
        \arrow{l}[above]{c_1}
        \arrow{r}[above]{c_2}
      & X
        \arrow{dl}[below right]{\id_X}
      \\
        {}
      & X
      & {}
    \end{tikzcd}
  \]
  commute.
  This means that
  \[
    \codiag_X \circ c_1 = \id_X
    \quad\text{and}\quad
    \codiag_X \circ c_2 = \id_X \,,
  \]
  so the morphism~$\codiag_X$ can be written in matrix form as
  \[
      \codiag_X
    = \begin{bmatrix}
        1 & 1
      \end{bmatrix} \,.
  \]
  
  (Note that for~$\Acat = \Modl{A}$, where~$A$ is a~{\kalg}, the diagonal~$\diag_X$ is the usual diagonal map~$\diag_X(x) = (x,x)$, and the codiagonal~$\codiag_X$ is the addition~$\codiag_X(x_1, x_2) = x_1 + x_2$.)
  
  We can now describe the sum~$f + g$ of two parallel morphisms~$f, g \colon X \to Y$ in~$\Acat$ as the compositions
  \begin{equation}
    \label{composition for sum}
      X
    \xlongto{\diag_X}
      X \oplus X
    \xlongto{\begin{bsmallmatrix} f & 0 \\ 0 & g \end{bsmallmatrix}}
      Y \oplus Y
    \xlongto{\codiag_Y}
      Y \,.
  \end{equation}
  Indeed, we find by matrix multiplication that
  \[
      \codiag_Y
      \circ
      \begin{bmatrix}
        f & 0 \\
        0 & g
      \end{bmatrix}
      \circ
      \diag_X
    = \begin{bmatrix}
        1 & 1
      \end{bmatrix}
      \begin{bmatrix}
        f & 0 \\
        0 & g
      \end{bmatrix}
      \begin{bmatrix}
        1 \\
        1
      \end{bmatrix}
    = f + g \,.
  \]
  By using that
  \[
    \begin{bmatrix}
      1 & 1
    \end{bmatrix}
    \begin{bmatrix}
      f & 0 \\
      0 & g
    \end{bmatrix}
    =
    \begin{bmatrix}
      f & g
    \end{bmatrix}
    \quad\text{and}\quad
    \begin{bmatrix}
      f & 0 \\
      0 & g
    \end{bmatrix}
    \begin{bmatrix}
      1 \\
      1
    \end{bmatrix}
    =
    \begin{bmatrix}
      f \\
      g
    \end{bmatrix}
  \]
  we can also rewrite the composition~\eqref{composition for sum} as
  \[
      X
    \xlongto{\diag_X}
      X \oplus X
    \xlongto{\begin{bsmallmatrix} f & g \end{bsmallmatrix}}
      Y
    \qquad\text{or}\qquad
      X
    \xlongto{\begin{bsmallmatrix} f \\ g \end{bsmallmatrix}}
      Y \oplus Y
    \xlongto{\codiag_Y}
      Y \,.
  \]
  
  This shows that the addition of~$\Acat$ can be retrieved from the categorical structure of~$\Acat$.
  It follows that an arbitrary category~$\Acat$ can be made into an additive category in at most one way.
  We can therefore regard \enquote{being additive} as a property of a category.
\end{remark}


\begin{definition}
  Let~$F \colon \Ccat \to \Dcat$ be a functor between arbitrary categories~$\Ccat$ and~$\Dcat$.
  \begin{enumerate}
    \item
      The functor~$F$ \emph{respects \textup(finite\textup) products}\index{functor!respects!products} if it holds for every (finite) family~$(X_i)_{i \in I}$ of objects~$X_i \in \Ob(\Ccat)$ and every product~$(P, (p_i)_{i \in I})$ of this family that the pair~$(F(P), (F(p_i))_{i \in I})$ is a product of the family~$(F(X_i))_{i \in I}$.
    \item
      The functor~$F$ \emph{respects \textup(finite\textup) coproducts}\index{functor!respects!coproducts} if it holds for every (finite) family~$(X_i)_{i \in I}$ of objects~$X_i \in \Ob(\Ccat)$ and every coproduct~$(C, (c_i)_{i \in I})$ of this family that the pair~$(F(C), (F(c_i))_{i \in I})$ is a coproduct of the family~$(F(X_i))_{i \in I}$.
  \end{enumerate}
  Suppose now that~$F \colon \Acat \to \Bcat$ is a functor between preadditive categories~$\Acat$ and~$\Bcat$.
  \begin{enumerate}[resume]
    \item
      The functor~$F$ \emph{respects biproducts}\index{functor!respects!biproducts} if it holds for all objects~$X_1, \dotsc, X_n \in \Ob(\Acat)$ (where~$n \geq 0$) and every biproduct~$(X, (p_1, \dotsc, p_n), (c_1, \dotsc, c_n))$ of these objects that the triple~$(F(X), (F(p_1), \dotsc, F(p_n)), (F(c_1), \dotsc, F(c_n)))$ is a biproduct of the objects~$F(X_1), \dotsc, F(X_n)$.
  \end{enumerate}
\end{definition}


\begin{theorem}
  \label{characterizations of additive functors}
  Let~$F \colon \Acat \to \Bcat$ be a functor between preadditive categories~$\Acat$ and~$\Bcat$.
  \begin{enumerate}
    \item
      \label{additive preserves biproducts}
      If the functor~$F$ is additive then it respects biproducts (and hence also finite products and finite coproducts).
    \item
      If the categories~$\Acat$ and~$\Bcat$ are already additive, then the following conditions on~$F$ are equivalent:
      \begin{enumerate}
        \item
          \label{is additive}
          $F$ is additive.
        \item
          \label{respects biproducts}
          $F$ respects biproducts.
        \item
          \label{respects finite products}
          $F$ respects finite products.
        \item
          \label{respects finite coproducts}
          $F$ respects finite coproducts.
      \end{enumerate}
  \end{enumerate}
\end{theorem}


\begin{proof}
  \leavevmode
  \begin{enumerate}
    \item
      Let~$n \geq 0$, let~$X_1, \dotsc, X_n \in \Ob(\Acat)$ and let~$(X, (p_i)_i, (c_i)_i)$ be a biproduct of~$X_1, \dotsc, X_n$.
      
      Once again we consider the case~$n = 0$ separately:
      The biproduct~$X$ is then a zero object of~$\Acat$.
      It follows that $\id_X = 0_{X,X}$ by \cref{inital terminal zero}, and hence
      \[
          \id_{F(X)}
        = F( \id_X )
        = F( 0_{X,X} )
        = 0_{F(X), F(X)} \,,
      \]
      by the additivity of~$F$.
      This shows that~$F(X)$ is a zero object for~$\Bcat$.
      
      Let now~$n \geq 1$.
      We then calculate that
      \begin{gather*}
          F(p_j)  F(c_i)
        = F(p_j c_i)
        = F
          \left(
              \begin{cases}
                \id_{X_i}     & \text{if~$i = j$} \\
                0_{X_i, X_j}  & \text{if~$i \neq j$}
              \end{cases}
          \right)
        = \begin{cases}
            \id_{F(X_i)}        & \text{if~$i = j$}     \,, \\
            0_{F(X_i), F(X_j)}  & \text{if~$i \neq j$}  \,,
          \end{cases}
      \intertext{and}
          \sum_{i=1}^n F(c_i) F(p_i)
        = \sum_{i=1}^n F(c_i p_i)
        = F\left( \sum_{i=1}^n c_i p_i \right)
        = F( \id_X )
        = \id_{F(X)} \,.
      \end{gather*}
    \item
      \begin{description}
        \item[\ref*{is additive}~$\implies$~\ref*{respects biproducts}]
          This has been shown in part~\ref*{additive preserves biproducts}.
          
        \item[\ref*{respects biproducts}~$\implies$~\ref*{respects finite products}]
          Let~$(X, (p_i)_i)$ be a product of the objects~$X_1, \dotsc, X_n$.
          It follows from \cref{product coproduct biproduct} unique morphism~$c_i \colon X_i \to X$ such that the triple~$(X, (p_i)_i, (c_i)_i)$ is a biproduct for~$X_1, \dotsc, X_n$.
          It follows that the triple~$(F(X), (F(p_i))_i, (F(c_i))_i)$ is a biproduct for the objects~$F(X_1), \dotsc, F(X_n)$ because the functor~$F$ preserves biproducts.
          This entails that the tuple~$(F(X), (F(p_i))_i)$ is a product of these objects.
        
        \item[\ref*{respects finite products}~$\implies$~\ref*{respects biproducts}]
          It follows from~$F$ respecting products that~$F$ respects terminal objects, because a terminal object is the same as an empty product.
          Hence~$F(0) = 0$.
          It follows that~$F(0_{X,Y}) = 0_{F(X), F(Y)}$ for any two objects~$X, Y \in \Ob(\Acat)$.
          Indeed, by applying the functor~$F$ to the commutative triangle
          \[
            \begin{tikzcd}[column sep = small]
                X
                \arrow{rr}[above]{0_{X,Y}}
                \arrow{dr}
              & {}
              & Y
              \\
                {}
              & 0
                \arrow{ur}
              & {}
            \end{tikzcd}
          \]
          we get the following commutative triangle:
          \[
            \begin{tikzcd}[column sep = small]
                F(X)
                \arrow{rr}[above]{F(0_{X,Y})}
                \arrow{dr}
              & {}
              & F(Y)
              \\
                {}
              & 0
                \arrow{ur}
              & {}
            \end{tikzcd}
          \]
          The commutativity of this triangle shows that the morphism~$F(0_{X,Y})$ factors through the zero object~$0$, which is precisly what is means for~$F(0_{X,Y})$ to be the zero morphism.
          
          Let now~$(X, (p_i)_i, (c_i)_i)$ be a biproduct of some objects~$X_1, \dotsc, X_n \in \Ob(\Acat)$.
          Then~$(X, (p_i)_i)$ is a product of~$X_1, \dotsc, X_n$, and it follows that~$(F(X), (F(p_i))_i)$ is a product of~$F(X_1), \dotsc, F(X_n)$ because~$F$ respects finite products.
          We find for the morphisms~$F(c_i) \colon F(X_i) \to F(X)$ that
          \[
              F(p_i) \circ F(c_i)
            = F(p_i \circ c_i)
            = F( \id_{X_i} )
            = \id_{F(X_i)}
          \]
          and we also find for~$j \neq i$ that
          \[
              F(p_j) \circ F(c_i)
            = F(p_j \circ c_i)
            = F(0_{X_i, X_j})
            = 0_{F(X_i), F(X_j)} \,.
          \]
          It follows from \cref{product coproduct biproduct} that the triple~$(F(X), (F(p_i))_i, (F(c_i))_i)$ is a biproduct of~$F(X_1), \dotsc, F(X_n)$.
          
        \item[\ref*{respects biproducts}~$\iff$~\ref*{respects finite coproducts}]
          This can be shown dually to the equivalence of~\ref*{respects biproducts} and~\ref*{respects finite products}.
        
        \item[\ref*{respects biproducts}~$\implies$~\ref*{is additive}]
          We find as in the implication \ref*{respects finite products}~$\implies$~\ref*{respects biproducts} that~$F(0) = 0$ and that consequently~$F(0_{X,Y}) =  0_{F(X), F(Y)}$ for all~$X, Y \in \Ob(\Acat)$.
          It also follows from the already proven implications~\ref*{respects biproducts}~$\implies$~\ref*{respects finite products} and \ref*{respects biproducts}~$\implies$~\ref*{respects finite coproducts} that~$F$ respects products and coproducts.
          We hence find that
          \[
              F(\diag_X)
            = \diag_{F(X)}
            \quad\text{and}\quad
              F(\codiag_Y)
            = \codiag_{F(Y)}
          \]
          for all~$X, Y \in \Ob(\Acat)$.
          
          Let~$f, g \colon X \to Y$ be two parallel morphisms in~$\Acat$.
          We can then describe their sum~$f+g$ as the composition
          \begin{equation}
            \label{sum as composition}
              f+g
            \colon
              X
            \xlongto{\diag_X}
              X \oplus X
            \xlongto{\begin{bsmallmatrix} f & 0 \\ 0 & g \end{bsmallmatrix}}
              Y \oplus Y
            \xlongto{\codiag_Y}
              Y \,.
          \end{equation}
          It follows from~$F$ preserving coproducts and products, as well as identities and zero morphisms, that
          \[
            F
            \left(
              \begin{bmatrix}
                f & 0 \\
                0 & g
              \end{bmatrix}
            \right)
            =
            \begin{bmatrix}
              F(f)  & 0     \\
              0     & F(g)
            \end{bmatrix} \,.
          \]
          By using that $F(\diag_X) = \diag_{F(X)}$ and~$F(\codiag_Y) = \codiag_{F(Y)}$ we altogether find that applying the functor~$F$ to the compositon~$\eqref{sum as composition}$ exhibits the morphism~$F(f+g)$ as the composition
          \[
              F(f+g)
            \colon
              F(X)
            \xlongto{\diag_{F(X)}}
              F(X) \oplus F(X)
            \xlongto{\begin{bsmallmatrix} F(f) & 0 \\ 0 & F(g) \end{bsmallmatrix}}
              F(Y) \oplus F(Y)
            \xlongto{\codiag_{F(Y)}}
              Y \,.
          \]
          This composition is precisely~$F(f) + F(g)$, so~$F(f + g) = F(f) + F(g)$.
        \qedhere
      \end{description}
  \end{enumerate}
\end{proof}




