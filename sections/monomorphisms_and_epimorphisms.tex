\section{Monomorphisms and Epimorphisms}


\begin{definition}
  Let~$f \colon X \to Y$ be a morphism in a category~$\Ccat$.
  \begin{enumerate}
    \item
      The morphism~$f$ is a \emph{monomorphism}\index{monomorphism}\index{morphism!mono-} if it follows for every pair of parallel morphisms~$u, v \colon W \to X$ in~$\Ccat$ from~$f \circ u = f \circ v$ that already~$u = v$.
    \item
      The morphism~$f$ is an \emph{epimorphism}\index{epimorphism}\index{morphism!epi-} if it follows for every pair of parallel morphisms~$u, v \colon Y \to Z$ in~$\Ccat$ from~$u \circ f = v \circ f$ that already~$u = v$.
  \end{enumerate}
\end{definition}


\begin{remark}
  Let~$f \colon X \to Y$ and~$g \colon Y \to Z$ be composable morphisms in a category~$\Ccat$.
  \begin{enumerate}
    \item
      If~$f$ is an isomorphism then it is both a monomorphism and an epimorphism.
    \item
      If both~$f$ and~$g$ are monomorphisms then their composition~$g \circ f$ is again a monomorphism.
      If both~$f$ and~$g$ are epimorphisms then their composition~$g \circ f$ is again an epimorphism.
    \item
      If the composition~$g \circ f$ is a monomorphism then~$f$ is a monomorphism.
      If the composition~$g \circ f$ is an epimorphism then~$g$ is an epimorphism.
    \item
      The morphism~$f$ is a monomorphism (in~$\Ccat$) if and only if it is an epimorphism in~$\Ccat^\op$.
  \end{enumerate}
\end{remark}


\begin{example}
  We give examples of monomorphisms.
  \begin{enumerate}
    \item
      In the category~$\Set$ the monomorphisms are precisely the injective maps.
      The same holds for the categories~$\Modl{A}$,~$\Group$,~$\Ring$,~$\CommRing$,~$\kAlg$,~$\kCommAlg$,~$\Top$.
    \item
      If~$Q$ is a quiver then in its path category~$\Path(Q)$ every morphism is a monomorphism:
      Let~$p = \alpha_\ell \dotsm \alpha_1$ be a morphism in~$Q$, i.e.\ a path in~$Q$.
      If~$u = u_r \dotsm u_1$ and~$v = v_s \dotsm v_1$ are morphisms in~$\Path(Q)$, i.e.\ paths in~$Q$, with~$s(u) = s(v)$ and~$t(u) = t(v) = s(p)$ then the equality~$p \circ u = p \circ v$ means that
      \[
          \alpha_\ell \dotsm \alpha_1 u_r \dotsm u_1
        = \alpha_\ell \dotsm \alpha_1 v_s \dotsm v_1 \,.
      \]
      It then follows that~$r = s$ and~$u_i = v_i$ for all~$i = 1, \dotsc, r$.
    \item
      Let~$\Conn_*$ be the category of pointed, connected topological spaces:
      The objects of~$\Conn_*$ are pairs~$(X, x_0)$ consisting of a connected topological space~$X$ and a base point~$x_0 \in X$.
      A morphism~$f \colon (X, x_0) \to (Y, y_0)$ is a continuous map~$f \colon X \to Y$ with~$f(x_0) = y_0$.
      The morphism~$f \colon (\Real, 0) \to (S^1, 1)$ with~$f(x) = e^{2 \pi i x}$ is then a monomorphism.
  \end{enumerate}
\end{example}


% TODO: Add example Q -> Q/Z in divisible abelian groups for non-injective mono.


\begin{example}
  We give examples for epimorphisms.
  \begin{enumerate}
    \item
      In the category~$\Set$ a morphism is an epimorphism if and only if it surjective.
      The same holds for the category~$\Group$ of groups.%
      \footnote{This is not as evident as one may suspect at first glance.}
    \item
      If~$Q$ is a quiver then in its path category~$\Path(Q)$ every morphism in an epimorphism.
    \item
      Let~$\Haus$ be the category of Hausdorff topological spaces (where morphisms are just continuous maps).
      A morphism~$f \colon X \to Y$ in~$\Haus$ is an epimorphism if and only if it has dense image.
    \item
      Let~$A$ be a commutative ring and let~$S \subseteq A$ be a multiplicative subset.
      Then the canonical map~$f \colon A \to S^{-1} A$,~$a \mapsto a/1$ is an epimorphism:
      If~$u,v \colon S^{-1} A \to B$ are two ring homomorphisms with~$u \circ f = v \circ f$ then~$u(a/1) = v(a/1)$ for every~$a \in A$.
      It then follows for every fraction~$a/s \in S^{-1} A$ that
      \[
          u\left( \frac{a}{s} \right)
        = u\left( \frac{a}{1} \right) u\left( \frac{s}{1} \right)^{-1}
        = v\left( \frac{a}{1} \right) v\left( \frac{s}{1} \right)^{-1}
        = v\left( \frac{a}{s} \right) \,,
      \]
      which then shows that~$u = v$.
  \end{enumerate}
\end{example}


% TODO: Add a proof that the epimorphisms in Grp are precisely the surjective homomorphisms.


\begin{remark*}
  Let~$f \colon X \to Y$ be a morphism in a category~$\Ccat$.
  \begin{enumerate}
    \item
      The morphism~$f$ is a monomorphism if and only if for every object~$W \in \Ob(\Ccat)$ the induced map
      \[
        f_*
        \colon
        \Ccat(W,X)
        \to
        \Ccat(W,Y)
      \]
      is injective.
    \item
      The morphism~$f$ is an epimorphism if and only if for every object~$Z \in \Ob(\Ccat)$ the induced map
      \[
        f^*
        \colon
        \Ccat(Y,Z)
        \to
        \Ccat(X,Z)
      \]
      is injective.
  \end{enumerate}
\end{remark*}




